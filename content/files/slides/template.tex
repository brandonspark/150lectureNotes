% Jacob Neumann

% DOCUMENT CLASS AND PACKAGE USE
    \documentclass[aspectratio=54]{beamer}
 
    % Establish the colorlambda boolean, to control whether the lambda is solid color (true), or the same as the picture (false)
    \newif\ifcolorlambda
    \colorlambdafalse % DEFAULT: false
    
    % Use auxcolor for syntax highlighting
    \newif\ifuseaux
    \useauxfalse % DEFAULT: false
   
    % Load color settings
    \usepackage{SETTINGS}

    \usepackage{lectureSlides}
    %%%%%%%%%%%%%%%%%%%%%%%%%%%%%%%%%%%%%%%%%| <----- Don't make the title any longer than this
    \title{Test Slides} % TODO
    \subtitle{Awesome slides with an awesome subtitle} % TODO
    \date{01 January 2020} % TODO
    \author{Author's Name} % TODO

    \graphicspath{ {./img/} }
    % DONT FORGET TO PUT [fragile] on frames with codeblocks, specs, etc.
        %\begin{frame}[fragile]
        %\begin{codeblock}
        %fun fact 0 = 1
        %  | fact n = n * fact(n-1)
        %\end{codeblock}
        %\end{frame}

    % INCLUDING codefile:
        % 1. In some file under code/NN (where NN is the lecture id num), include:
    %       (* FRAGMENT KK *)
    %           <CONTENT>
    %       (* END KK *)
    
    %    Remember to not put anything on the same line as the FRAGMENT or END comment, as that won't be included. KK here is some (not-zero-padded) integer. Note that you MUST have fragments 0,1,...,KK-1 defined in this manner in order for fragment KK to be properly extracted.
        %  2. On the slide where you want code fragment K
                % \smlFrag[color]{KK}
        %     where 'color' is some color string (defaults to 'white'. Don't use presentColor.
    %  3. If you want to offset the line numbers (e.g. have them start at line 5 instead of 1), use
                % \smlFragOffset[color]{KK}{5}

\begin{document}

% Make it so ./mkWeb works correctly
    \ifweb
       \renewcommand{\pause}{}
    \fi


% SOLID COLOR TITLE (see SETTINGS.sty)
{
\setbeamercolor{background canvas}{bg=white}
\begin{frame}[plain]
    \colorlambdatrue
    \titlepage
\end{frame}
}
{
\setbeamercolor{background canvas}{bg=white}
\begin{frame}[plain]
    \colorlambdafalse
    \pictureTitle{stars}
\end{frame}
}

\begin{frame}
    \frametitle{\texttt{\textbackslash smlFrag\{N\}} example}
    The following is produced with \texttt{\textbackslash smlFrag\{0\}}. See \texttt{content/code/template/code.sml}. \pause

    \smlFrag{0}
\end{frame}

\begin{frame}[fragile]
    \frametitle{\texttt{codeblock} example}
    Note the \texttt{[fragile]} on this frame.\pause

    \begin{codeblock}
       fun pseudocode (m : inputType) : out = 
                 ... 
             pseudocode(m-1) 
                 ... 
    \end{codeblock}
\end{frame}

\begin{frame}
	\begin{center} Thank you! \end{center}
\end{frame}


\end{document}

