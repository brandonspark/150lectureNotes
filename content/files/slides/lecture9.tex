% Jacob Neumann

% DOCUMENT CLASS AND PACKAGE USE
    \documentclass[aspectratio=169]{beamer}
 
    % Establish the colorlambda boolean, to control whether the lambda is solid color (true), or the same as the picture (false)
    \newif\ifcolorlambda
    \colorlambdafalse % DEFAULT: false
    
    % Use auxcolor for syntax highlighting
    \newif\ifuseaux
    \useauxfalse % DEFAULT: false
   
    % Color settings
    \useauxtrue
    
    \newcommand{\auxColor}{ECD37A}     % the color of note boxes and stuff
    \newcommand{\presentColor}{32A2A5} % the primary color of the slide borders
    \newcommand{\bgColor}{d1eff0}      % the color of the background of the slide
    \newcommand{\darkBg}{8b98ad}
    \newcommand{\lambdaColor}{\auxColor}
  
    \colorlambdatrue

    \usepackage{comment} % comment blocks
    \usepackage{soul} % strikethrough
    \usepackage{listings} % code
    \usepackage{makecell}
    \usepackage{url}

    \setbeamertemplate{itemize items}[circle]
    % \setbeameroption{show notes on second screen=right}

    \usepackage{lectureSlides}
    %%%%%%%%%%%%%%%%%%%%%%%%%%%%%%%%%%%%%%%%%| <----- Don't make the title any longer than this
    \title{Title} % TODO
    \subtitle{Awesome slides with an awesome subtitle} % TODO
    \date{01 January 2020} % TODO
    \author{Brandon Wu} % TODO

    \graphicspath{ {./img/} }
    % DONT FORGET TO PUT [fragile] on frames with codeblocks, specs, etc.
        %\begin{frame}[fragile]
        %\begin{codeblock}
        %fun fact 0 = 1
        %  | fact n = n * fact(n-1)
        %\end{codeblock}
        %\end{frame}

    % INCLUDING codefile:
        % 1. In some file under code/NN (where NN is the lecture id num), include:
    %       (* FRAGMENT KK *)
    %           <CONTENT>
    %       (* END KK *)
    
    %    Remember to not put anything on the same line as the FRAGMENT or END comment, as that won't be included. KK here is some (not-zero-padded) integer. Note that you MUST have fragments 0,1,...,KK-1 defined in this manner in order for fragment KK to be properly extracted.
        %  2. On the slide where you want code fragment K
                % \smlFrag[color]{KK}
        %     where 'color' is some color string (defaults to 'white'. Don't use presentColor.
    %  3. If you want to offset the line numbers (e.g. have them start at line 5 instead of 1), use
                % \smlFragOffset[color]{KK}{5}

\begin{document}

% Make it so ./mkWeb works correctly
\ifweb
    \renewcommand{\pause}{}
\fi

\setbeamertemplate{itemize items}[circle]

% SOLID COLOR TITLE (see SETTINGS.sty)
{
\begin{frame}[plain]
    \colorlambdatrue
    \titlepage
\end{frame}
}

\begin{frame}[fragile]
  \frametitle{Lesson Plan}

  \tableofcontents
\end{frame}

\begin{frame}[fragile]
  \frametitle{Last time}

  Last time, we learned about the concept of \term{parametric polymorphism}.

  We saw how polymorphic types could be parameterized by \term{type variables}, which
  allowed values of polymorphic type to be used in generic ways, by instantiating them
  at different types at each use site.

  We also saw how we could define our own polymorphic datatypes, and rederived the true
  definition of \code{option} and \code{list} from it.

  We then experimented with polymorphic sorting, where we could create a sorting function
  on lists which operated generically in the type of the list's elements, by providing
  a comparison function as another input to the sorting function. 
\end{frame}

\sectionSlide{1}{Higher-Order Functions}

\begin{frame}[fragile]
  \frametitle{Family First}

  We saw how, with polymorphic types and polymorphic functions, we have a certain concept
  of parameterization. A polymorphic type denoted not just a single type, but 
  \textit{many} different types. Similarly, a polymorphic function denoted not a single
  function, but a template for many functions of different types. 

  Each of these functions essentially did the same thing, however\footnotemark. In this
  lecture, we'll explore \term{higher-order functions}, which allows us to capture 
  common design patterns in code to create families of functions which all do different,
  but related, things. 

  \footnotetext[1]{They had to, in order to be polymorphically generalizable!}
\end{frame}

\begin{frame}[fragile]
  \frametitle{Polymorphic Sorting}

  In the last lecture, we proposed a polymorphic sorting function that took in a 
  comparison function. It looked like:

  \begin{codeblock}
    fun sort (cmp : 'a * 'a -> order, L : 'a list) = (* ... *)
  \end{codeblock}

  This \code{sort} function is itself a higher-order function!
\end{frame}

\begin{frame}[fragile]
  \frametitle{First-Order Functions}

  When writing functions so far this semester, we've mostly looked at functions which take
  in tuple values, and return values of base type, or other slightly more interesting types
  like \code{option}s, \code{list}s, or \code{tree}s.

  These functions are \term{first-order}. They are functions in the classic, intuitive sense.

  But it is possible for a function to return a function -- or for that matter, take in a
  function, as well.

  \defBox{\, A \term{higher-order function} is a function which takes in functions or
  returns functions.}
\end{frame}

\begin{frame}[fragile]
  \frametitle{\code{sort} is Higher-Order}

  \code{sort} is thus a higher-order function, due to taking in the comparison function
  \code{cmp}.

  We could write code like so:

  \begin{codeblock}
    fun intCompare (x, y) = 
      if x < y then LESS 
      else if x = y then EQUAL 
      else GREATER

    val sorted = sort (intCompare, [4, 3, 1, 2])
  \end{codeblock}
\end{frame}

\begin{frame}[fragile]
  \frametitle{\code{sort} is Higher-Order}

  This is kind of verbose, though. Do we need to do a \code{fun} declaration every
  time that we want to specify our comparison function?

  Luckily, the answer is no. Recall lambda expressions, which are anonymous function
  values. We haven't used them extensively, but it turns out something they are very
  useful for is making quick functions as arguments to other functions.

  Suppose we wanted to sort a list in descending order. Then we might write:

  \begin{codeblock}
    val sorted = sort (fn (x, y) => if x < y then GREATER else LESS, [4, 3, 1, 2])
  \end{codeblock}
\end{frame}

\begin{frame}[fragile]
  \frametitle{Returning Functions}

  We can also return functions, not just take them in. 

  \defBox{\, We call a function which returns another function \term{curried}.}

  Suppose we have our \code{add} function.

  \begin{codeblock}
    (* add : int * int -> int *)
    fun add (x, y) = x + y
  \end{codeblock}

  What if instead of taking in a tuple of both \code{int}s to be added,
  we returned a lambda expression which took in the second?

  \begin{codeblock}
    (* cadd : int -> int -> int *)
    fun cadd x = fn y => x + y
  \end{codeblock}

  \noteBox{\, Type arrows are \textit{right-associative}. This means the type
  \code{int -> int -> int} means the same as \code{int -> (int -> int)}.}
\end{frame}

\begin{frame}[fragile]
  \frametitle{Syntactic Sugar}

  It's a little bit of a pain to have to explicitly write out the \code{fn y => x + y} 
  that we return, however.

  SML helps, by having some syntactic sugar for writing curried functions. 

  The following two definitions are equivalent:
  \begin{codeblock}
    (* cadd : int -> int -> int *)
    fun cadd x = fn y => x + y
    fun cadd x y = x + y
  \end{codeblock}

  In general, you can always add more arguments (separated by spaces), which SML will
  understand to mean "return a function which takes in that argument as a parameter".
  This generalizes to many arguments, too.
\end{frame}

\begin{frame}[fragile]
  \frametitle{Using Curried Functions}

  Here, we say that \code{cadd} is a \textit{curried} form of \code{add}!
  It differs from \code{add} in usage in that arguments are passed in one by one.

  \begin{codeblock}
    val 2 = add (1, 1)
    val 2 = add 2 2 
  \end{codeblock}

  Note that function application is \textit{left-associative}, meaning that 
  \code{add 2 2} is the same as \code{(add 2) 2}.

  Seems these functions do the same thing. Are they extensionally equivalent?
\end{frame}

\begin{frame}[fragile]
  \frametitle{Correspondence of \code{add} and \code{cadd}}

  The answer is no! \code{add} and \code{cadd} don't even have the same type. 
  Two expressions can only be extensionally equivalent if they have the same type.

  So \code{add} and \code{cadd} aren't extensionally equivalent, but they do
  "essentially the same thing".

  The main advantage of \code{cadd} in this scenario is that it can take in its
  arguments separately. We will see later why this is a virtue, but first we have
  some more HOFs to learn about.
\end{frame}

\sectionSlide{2}{Higher-Order Function Examples}

\begin{frame}[fragile]
  \frametitle{Abstraction}

  Abstraction is the name of the game in computer science.

  We abstracted away bits and bytes so that we could think about data and programs.
  We abstracted away unrestricted control flow for structured constructs so that
  we could better reason about thoser programs, and we added specifications and types
  so that we could better communicate what our programs do.

  With higher-order functions, we can abstract code over code itself. We've seen this
  once, by writing a \code{sort} function which varies depending on the code of the 
  corresponding \code{cmp} function.

  This prevents us from having to rewrite multiple \code{sort} functions with the 
  same "core logic". Let's see some examples of other HOFs which reduce common logic.
\end{frame}

\begin{frame}[fragile]
  \frametitle{Transforming Lists}

  Sometimes we're interested in applying some transformation to every element of a list.

  \begin{codeblock}
    fun incrementAll [] = []
      | incrementAll (x::xs) = (x + 1) :: xs

    fun toStringAll [] = [] 
      | toStringAll (x::xs) = Int.toString x :: xs 
  \end{codeblock}

  If we take away the operation that we perform to each element \code{x}, the underlying
  function looks exactly the same! 

  We will capture this phenomenon with a HOF called \code{map}.
\end{frame}

\begin{frame}[fragile]
  \frametitle{Map}

  \begin{codeblock}
    (* map : ('a -> 'b) -> 'a list -> 'b list *)
    (* REQUIRES: true *)
    (* ENSURES: map f [x1, ..., xn] evaluates to [f x1, ..., f xn] *)
    fun map (f : 'a -> 'b) ([] : 'a list) : 'b list = []
      | map f (x::xs) = f x :: map f xs
  \end{codeblock}

  Then we obtain that \code{incrementAll} $\eeq$ \code{map (fn x => x + 1)}, 
  and \code{toStringAll} $\eeq$ \code{map Int.toString}.
\end{frame}

\begin{frame}[fragile]
  \frametitle{Partial Application}

  The main strength of currying is in \term{partial application}.

  \defBox{\, \term{Partial application} is the act of applying \textit{some} of the
  curried arguments to a curried function, but not all.}

  Partial application lets us obtain increasingly-specific instances of a higher-order
  function, which acts as a \textit{template} for a family of functions that all behave
  the same.

  In this case, \code{map} is the general design for a family of functions that entail
  transforming elements of a list, and \code{incrementAll} and \code{toStringAll} are
  concrete instances of this design! So instead, we could write: 

  \begin{codeblock}
    val incrementAll = map (fn x => x + 1)
    val toStringAll = map Int.toString 
  \end{codeblock}
\end{frame}

\begin{frame}[fragile]
  \frametitle{Filter}

  Another common pattern is keeping only the elements of a list that satisfy
  some predicate. This predicate might vary depending on the use case, but 
  the overall pattern remains the same.

  This leads us to a HOF named \code{filter}.

  \begin{codeblock}
    (* filter : ('a -> bool) -> 'a list -> 'a list *)
    (* REQUIRES: true *)
    (* ENSURES: filter p xs evaluates to all elements x in xs s.t.
     * p x evaluates to true *)
    fun filter (p : 'a -> bool) ([] : 'a list) : 'a list = []
      | filter p (x::xs) =
          if p x then
            x :: filter p xs
          else
            filter p xs
  \end{codeblock}
\end{frame}

\begin{frame}[fragile]
  \frametitle{Filter}

  So, for instance, we have that:

  \begin{codeblock}
    val isEven = fn x => x mod 2 = 0
    val keepEvens = filter isEven 
    val [1, 3] = keepEvens [1, 2, 3, 4] 
    val keepOdds = filter (fn x => not (isEven x))
    val [2, 4] = keepOdds [1, 2, 3, 4]
  \end{codeblock}

  It's a little ugly to have to write \code{fn x => not (isEven x)}, though.
\end{frame}

\begin{frame}[fragile]
  \frametitle{Function Composition}

  Something you learn about early on in mathematics is \textit{function composition}.

  In SML, functions are meant to be closer to their mathematical counterparts. We
  can define a notion of function composition for SML functions too!

  We want a function which takes in two functions and essentially strings them
  together. We don't know what their types are, so we will simply call the first
  one \code{'a -> 'b}, and the second \code{'b -> 'c}. Whatever the first one 
  returns, the second one needs to take as input.
\end{frame}

\begin{frame}[fragile]
  \frametitle{Compose}

  Our output will be a function which takes in an input, passes it into the
  first function, and then passes the result of that into the second function.

  Given \code{'a -> 'b} and \code{'b -> 'c}, that function's type must be
  \code{'a -> 'c}. Let's write it.

  \begin{codeblock}
    fun compose (g : 'b -> 'c, f : 'a -> 'b) : 'a -> 'c =
      fn x => g (f x)
  \end{codeblock}

  We take the function arguments in reverse, to be more in line with the
  mathematical notation, writing $f \o g$ to be the composition of $f$
  with $g$, such that $g$ is applied first.

  In SML, we also have \code{o} defined as the infix composition operator.  
\end{frame}

\begin{frame}[fragile]
  \frametitle{Compose}

  So now, instead of writing \code{fn x => not (isEven x)}, we can write:

  \begin{codeblock}
    val isOdd = not o isEven
  \end{codeblock}

  Much more terse!
\end{frame}

\begin{frame}[fragile]
  \frametitle{What's in a List?}

  There is one final pattern that is common to working with lists.

  Oftentimes, we are interested in "summarizing" the data in a list. We are 
  interested in iterating over the elements of a list, producing some value which
  changes for every element that we see.

  This takes the form of, for instance, summing all the elements of an \code{int list},
  or concatenating all the strings in a \code{string list}.
\end{frame}

\begin{frame}[fragile]
  \frametitle{List Traversing Examples}

  For instance:

  \begin{codeblock}
    fun sum [] = 0
      | sum (x::xs) = x + sum xs

    fun concat [] = ""
      | concat (x::xs) = x ^ concat xs

    fun flatten [] = []
      | flatten (x::xs) = x @ flatten xs
  \end{codeblock}

  These all look kind of similar!
\end{frame}

\begin{frame}[fragile]
  \frametitle{Transforming Lists}

  All of these functions have a common root, in that they have some "initial value"
  that is returned upon the empty list, and which is otherwise transformed by some
  common operation, in conjunction with each element of the list.

  In a sense, it looks very similar to this common pattern in other programming
  languages:
  
  \begin{lstlisting}[language=Python]
    acc = default 

    for x in xs:
      acc = f(x)
  \end{lstlisting}

  Let's write it in SML!
\end{frame}

\begin{frame}[fragile]
  \frametitle{Folding}

  We call this process \term{folding}. We will implement a function, \code{foldl}, which
  involves traversing the list from left to right, and transforming an accumulator
  value.

  The type of this function will be \code{('a * 'b -> 'b) -> 'b -> 'a list -> 'b}.

  These can be broken down into four parts:
  \begin{itemize}
    \item \code{'a * 'b -> 'b} - the "transforming function", which acts upon the 
    accumulator and each fresh value of the list
    \item \code{'b} - the "default value" which serves as the initial accumulator
    \item \code{'a list} - the list to be "folded"
    \item \code{'b} - the final value to be returned, of the same type as the accumulator
  \end{itemize}
\end{frame}

\begin{frame}[fragile]
  \frametitle{Folding, Left}

  \begin{codeblock}
    (* foldl : ('a * 'b -> 'b) -> 'b -> 'a list -> 'b *)
    (* REQUIRES: true *)
    (* ENSURES: foldl f z [x1, ..., xn] evaluates to 
     * f (xn, ... f (x2, f (x1, z)) ... ) *)
    fun foldl f z [] = z
      | foldl f z (x::xs) = foldl f (f (x, z)) xs  
  \end{codeblock}

  Essentially, when we run out of elements in the list, we simply return
  what we have accumulated so far.

  Otherwise, we update our accumulator and keep recursing through the list. 
\end{frame}

\begin{frame}[fragile]
  \frametitle{Folding, Right}

  That's not the only way to fold a list, however. What if we want to fold
  a list from right to left? Let's implement \code{foldr}, of the same type.

  \begin{codeblock}
    (* foldl : ('a * 'b -> 'b) -> 'b -> 'a list -> 'b *)
    (* REQUIRES: true *)
    (* ENSURES: foldl f z [x1, ..., xn] evaluates to 
     * f (x1, f (x2, ... f (xn, z) ... )) *)
    fun foldr f z [] = z
      | foldr f z (x::xs) = f (x, foldr f z xs)
  \end{codeblock}
\end{frame}

\begin{frame}[fragile]
  \frametitle{A Difference in Folds}

  The main way to remember how to implement the two folds is in when we 
  combine with the first element, \code{x}.

  \begin{codeblock}
    fun foldl f z [] = z
      | foldl f z (x::xs) = foldl f (f (x, z)) xs  

    fun foldr f z [] = z
      | foldr f z (x::xs) = f (x, foldr f z xs)
  \end{codeblock}

  In \code{foldl}, due to eager evaluation, the first thing that happens is we
  apply \code{f} to \code{x}! This corresponds to going \textit{left-to-right},
  as we want to transform the first element first.

  In \code{foldr}, due to eager evaluation, we do the application of \code{f}
  to \code{x} \textit{last}. For similar reasons, this corresponds to going
  right-to-left.
\end{frame}

\begin{frame}[fragile]
  \frametitle{Fold Examples}

  So now, how do we use \code{foldl} and \code{foldr} to implement
  \code{sum}, \code{concat}, and \code{flatten}?

  The simple way is simple to visualize the accumulator changing, by 
  applying the function to the elements and accumulator, going left to
  right or right to left. Recall that the transform function \code{f}
  always takes the accumulator as its second argument.

  \begin{codeblock}
    val sum = foldl (op+) 0 
    val concat = foldr (op^) ""
    val flatten = foldr (op@) [] 
  \end{codeblock}
\end{frame}

\begin{frame}[fragile]
  \frametitle{Stepping Examples}

  Let's try \code{foldr (op^) "" ["I", "LOVE", "150"]}.

  \begin{align*}
    &= \code{foldr (op^) "" ["I", "LOVE", "150"]} \\
    &= \code{"I" ^ foldr (op^) "" ["LOVE", "150"]} \\ 
    &= \code{"I" ^ ("LOVE" ^ foldr (op^) "" ["150"])} \\ 
    &= \code{"I" ^ ("LOVE" ^ ("150" ^ foldr (op^) "" []))} \\ 
    &= \code{"I" ^ ("LOVE" ^ ("150" ^ ""))} \\ 
    &= \code{"ILOVE150"}
  \end{align*}
\end{frame}

\begin{frame}[fragile]
  \frametitle{A Natural Fold}

  It is said that \code{foldr} is the "natural" fold\footnotemark. We end up with
  all the elements in order, merely joined by the transformation function.

  Let's try \code{foldr (op^) "" ["I", "LOVE", "150"]}.

  \begin{align*}
    &= \code{foldr (op^) "" ["I", "LOVE", "150"]} \\
    &= \code{"I" ^ foldr (op^) "" ["LOVE", "150"]} \\ 
    &= \code{"I" ^ ("LOVE" ^ foldr (op^) "" ["150"])} \\ 
    &= \code{"I" ^ ("LOVE" ^ ("150" ^ foldr (op^) "" []))} \\ 
    &= \code{"I" ^ ("LOVE" ^ ("150" ^ ""))} \\ 
    &= \code{"ILOVE150"}
  \end{align*}

  \footnotetext[2]{For more, consult Frank Pfenning's excellent document \url{http://www.cs.cmu.edu/~me/courses/15-150-Spring2020/lectures/10/origami.pdf}}
\end{frame}

\begin{frame}[fragile]
  \frametitle{Reversing Things}

  For instance, let's try \code{foldl (op::) [] [1, 2, 3]}. What do you expect to happen?

  \begin{align*}
    &= \code{foldl (op::) [] [1, 2, 3]} \\ 
    &= \code{foldl (op::) [1] [2, 3]} \\ 
    &= \code{foldl (op::) [2, 1] [3]} \\ 
    &= \code{foldl (op::) [3, 2, 1] []} \\ 
    &= \code{[3, 2, 1]}
  \end{align*}

  We see that we end up with, essentially, \code{3 :: 2 :: 1 :: []}. This 
  applied the transformation function to each element, but in reverse!

  If we look at the specification of \code{foldl}, though, this is exactly what it purported
  to do. We expected to see \code{f (xn, ... f (x2, f (x1, z)) ... )}. 

  So \code{rev} can be reimplemented as \code{foldl (op::) []}!
\end{frame}

\sectionSlide{3}{Mathematics of Higher-Order Functions}

\begin{frame}[fragile]
  \frametitle{Equivalence}

  At this point, we might be wondering about the implications for mathematical
  analysis of functions when higher-order functions get involved. In particular,
  how does equivalence get affected?

  Recall our definition for extensional equivalence on functions, for \code{f : t1 -> t2}
  and \code{g : t1 -> t2}. We require that, for all values \code{x : t1}, that
  \code{f x} $\eeq$ \code{g x}. 

  It turns out, no extra machinery is necessary! We already have a definition for 
  when functions should be $\eeq$ (which is exactly the above), meaning that it's
  no problem if \code{t2} is a function type. 
\end{frame}

\begin{frame}[fragile]
  \frametitle{Equivalence}

  It might seem a little different, however, when dealing with when \code{t1} is 
  a function type. How do we reason about if two function values are the "same"?
  
  Is \code{fn (x, y) => x + y} the same as \code{fn (y, x) => y + x}? 

  What about \code{fn x => x + x} versus \code{fn x => 2 * x}?

  Fortunately, it doesn't matter. While we specified that, for all values, \code{f}
  and \code{g} behave the same, because of referential transparency, we identify values
  by whether or not they are extensionally equivalent. So this is equivalent to saying:

  If, for \code{x} $\eeq$ \code{y}, then \code{f x} $\eeq$ \code{g y}. This goes
  both ways, so we get that two extensionally equivalent HOFs behave the same on
  extensionally equivalent arguments.
\end{frame}


\begin{frame}[fragile]
  \frametitle{Equivalence Examples}

  So we know that, for instance, as a consequence of the fact that \code{fn x => 2 * x}
  and \code{fn x => x + x} are equivalent, then \code{map (fn x => 2 * x)} and 
  \code{map (fn x => x + x)} must be equivalent as well.

  Even though HOFs generalize over other arbitrary code, we have a mathematical guarantee
  that "equals-for-equals" still holds! HOFs then, in a sense, still act generically over
  their inputs, in a way that respects extensioanl equivalence.

  A free consequence that comes out of this: easy refactoring. Code can be easily
  updated without fear of breaking a higher-order codebase, so long as simple
  mathematical equivalences are respected.\footnotemark

  \footnotetext[3]{This is \textit{really useful}.}
\end{frame}

\begin{frame}[fragile]
  \frametitle{Hierarchies}

  In this lecture, we saw how we could write \term{higher-order functions}, which were
  functions which generalized over other functions, possibly being able to be specialized
  to arbitrary precision by splitting up arguments into curried form.

  With functions like \code{map} and \code{foldl}, we can think of them as defining
  a hierarchy of functions, all of which are defined from a common ancestors. Thus, the
  descendants of \code{map} might be \code{incrementAll} and \code{toStringAll}, 
  and the descendants of \code{foldl} might be \code{sum}, \code{concat}, and \code{flatten}.

  With this hereditary understanding of functions, we can abstract away even design patterns
  in code, reducing boilerplate and overall achieving a more holistic understanding.
\end{frame}

\begin{frame}[plain]
	\begin{center} Thank you! \end{center}
\end{frame}


\end{document}

