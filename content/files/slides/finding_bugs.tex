% Jacob Neumann

% DOCUMENT CLASS AND PACKAGE USE
\documentclass[aspectratio=169, handout]{beamer}

% Establish the colorlambda boolean, to control whether the lambda is solid color (true), or the same as the picture (false)
\newif\ifcolorlambda
\colorlambdafalse % DEFAULT: false

% Use auxcolor for syntax highlighting
\newif\ifuseaux
\useauxfalse % DEFAULT: false

% Color settings
\useauxtrue

\newcommand{\auxColor}{e0b45a}     % the color of note boxes and stuff
\newcommand{\presentColor}{08A679} % the primary color of the slide borders
\newcommand{\bgColor}{e1f2ea}      % the color of the background of the slide
\newcommand{\darkBg}{8b98ad}
\newcommand{\lambdaColor}{\auxColor}

\colorlambdatrue

\usepackage[pro]{fontawesome5}
\usepackage{comment} % comment blocks
\usepackage{soul} % strikethrough
\usepackage{listings} % code
\usepackage{makecell}
\usepackage{tcolorbox}
\usepackage{amssymb}% http://ctan.org/pkg/amssymb
\usepackage{pifont}% http://ctan.org/pkg/pifont
\usepackage[outline]{contour}
\usepackage{ stmaryrd }

\setbeamertemplate{itemize items}[circle]
% \setbeameroption{show notes on second screen=right}

\usepackage{lectureSlides}
%%%%%%%%%%%%%%%%%%%%%%%%%%%%%%%%%%%%%%%%%| <----- Don't make the title any longer than this
\title{Finding Bugs and Scaling Your Security Program} % TODO
\subtitle{Using Semgrep to take your security game to the next level} % TODO
\date{16 June 2024} % TODO
\author{Brandon Wu} % TODO


\usetikzlibrary{patterns}

\def\checkmark{\tikz\fill[green, scale=0.4](0,.35) -- (.25,0) -- (1,.7) -- (.25,.15) -- cycle;}
\contourlength{.08em}% default is 0.03em
\newcommand{\cmark}{{\color{green!80!black}\ding{51}}}
\newcommand{\xmark}{{\color{red}\ding{55}}}

\newcommand{\exerciseHeader}[2]{
  \begin{minipage}[t][0.4in][t]{\textwidth}
    \Large \term{Exercise:} #1
  \end{minipage}
  \begin{minipage}[t][2.6in][t]{\textwidth}
    #2
  \end{minipage}
}

\newenvironment{exercisePage}[1]{
  \begin{minipage}[t][0.4in][t]{\textwidth}
    \Large \term{Exercise:} #1
  \end{minipage}
  \begin{minipage}[t][2.6in][t]{\textwidth}
}{
  \end{minipage}
}

\newenvironment{featurePage}[1]{
  \begin{minipage}[t][0.4in][t]{\textwidth}
    \Large \term{Feature:} #1
  \end{minipage}
  \begin{minipage}[t][2in][t]{\textwidth}
}{
  \end{minipage}
}

\newcommand*\circled[1]{\tikz[baseline=(char.base)]{
            \node[shape=circle,draw,inner sep=2pt] (char) {#1};}}

\newcommand\blfootnote[1]{
    \begingroup
    \renewcommand\thefootnote{}\footnote{#1}
    \addtocounter{footnote}{-1}
    \endgroup
}

\newcommand\solutionLink[1]{
  \blfootnote{\color{blue}\href{#1}{Solution}}
}

\newcommand\ellipsisLink[1][]{{\color{blue}\href{https://semgrep.dev/docs/writing-rules/pattern-syntax\#ellipsis-operator}{ellipsis}}}
\newcommand\metavariableLink[1][]{{\color{blue}\href{https://semgrep.dev/docs/writing-rules/pattern-syntax\#metavariables}{metavariable}}}
\newcommand\patternNotLink[1][]{{\color{blue}\href{https://semgrep.dev/docs/writing-rules/rule-syntax\#pattern-not}{not}}}
\newcommand\patternAllLink[1][]{{\color{blue}\href{https://semgrep.dev/docs/writing-rules/rule-syntax\#patterns}{all}}}
\newcommand\patternAnyLink[1][]{{\color{blue}\href{https://semgrep.dev/docs/writing-rules/rule-syntax\#pattern-either}{any}}}
\newcommand\literalEllipsisLink[1][]{{\color{blue}\href{https://semgrep.dev/docs/writing-rules/pattern-syntax\#strings}{literal ellipsis}}}
\newcommand\literalMetavariableLink[1][]{{\color{blue}\href{https://semgrep.dev/docs/writing-rules/pattern-syntax\#literal-metavariables}{literal metavariable}}}
\newcommand\focusLink[1][]{{\color{blue}\href{https://semgrep.dev/docs/writing-rules/rule-syntax\#focus-metavariable}{focus}}}
\newcommand\metavariableRegexLink[1][]{{\color{blue}\href{https://semgrep.dev/docs/writing-rules/rule-syntax\#metavariable-regex}{metavariable regex}}}
\newcommand\metavariableComparisonLink[1][]{{\color{blue}\href{https://semgrep.dev/docs/writing-rules/rule-syntax\#metavariable-comparison}{metavariable comparison}}}

% \renewcommand\blfootnote[1]{}

\definecolor{semgrepGreen}{HTML}{2ACFA6}

\graphicspath{ {./img/} }


\newcommand{\dollar}{\mbox{\textdollar}}


    % DONT FORGET TO PUT [fragile] on frames with codeblocks, specs, etc.
        %\begin{frame}[fragile]
        %\begin{codeblock}
        %fun fact 0 = 1
        %  | fact n = n * fact(n-1)
        %\end{codeblock}
        %\end{frame}

    % INCLUDING codefile:
        % 1. In some file under code/NN (where NN is the lecture id num), include:
    %       (* FRAGMENT KK *)
    %           <CONTENT>
    %       (* END KK *)

    %    Remember to not put anything on the same line as the FRAGMENT or END comment, as that won't be included. KK here is some (not-zero-padded) integer. Note that you MUST have fragments 0,1,...,KK-1 defined in this manner in order for fragment KK to be properly extracted.
        %  2. On the slide where you want code fragment K
                % \smlFrag[color]{KK}
        %     where 'color' is some color string (defaults to 'white'. Don't use presentColor.
    %  3. If you want to offset the line numbers (e.g. have them start at line 5 instead of 1), use
                % \smlFragOffset[color]{KK}{5}
\colorlet{bgBlue}{hlBlue!130}
\colorlet{bgOrange}{hlOrange}
\colorlet{fgBlue}{fgCodeBlue}
\colorlet{fgGreen}{fgCodeGreen}
\colorlet{fgRed}{fgCodeRed}
\colorlet{fgOrange}{fgCodeOrange}

\colorlet{codeBackground}{background_color}

\definecolor{bgPurple}{HTML}{dbc7ff}
\definecolor{bgRed}{HTML}{fcaea9}
\definecolor{bgYellow}{HTML}{fffeb5}
\definecolor{bgGreen}{HTML}{b4ffb3}
\definecolor{bgGray}{HTML}{bfbfbf}

\definecolor{fgOrange}{HTML}{fcae4e}
\definecolor{fgYellow}{HTML}{ffbe0a}

\definecolor{unknownPink}{HTML}{e0fff6}

\definecolor{sourceColor}{HTML}{8fc4f2}
\definecolor{sinkColor}{HTML}{f77352}

\tikzset{
  every path/.style={line width=0.25mm},
  tok/.style={
    font=\large,
    align=center,
    minimum height=3cm,
  },
  belowtok/.style={
    font=\large,
    align=center,
    minimum height=3cm,
  },
  astNode/.style={
    draw,
    inner sep=2pt,
    line width=0.4mm,
    minimum size=0.6cm,
    font=\large,
    align=center,
  },
  hex/.style={
    astNode,
    signal,
    signal to=east and west,
    signal pointer angle=130,
    fill=bgPurple,
  },
  box/.style={
    astNode,
    rectangle,
    fill=green!20!white,
  },
  bbox/.style={
    astNode,
    rectangle,
    fill=bgBlue,
  },
  circ/.style={
    astNode,
    circle,
    minimum size=0.7cm,
    fill=bgYellow
  },
  bhex/.style={
    astNode,
    signal,
    inner sep=0.5pt,
    signal to=east and west,
    signal pointer angle=130,
    fill=bgBlue,
    minimum width=0.8cm,
  },
  unknown/.style={hex, draw=black, preaction={fill, unknownPink}, pattern=north east lines, pattern color=gray},
  unknown2/.style={unknown, preaction={fill, bgYellow}},
  highlight/.style={draw=fgYellow, line width=0.55mm},
  highlight2/.style={draw=magenta, line width=0.55mm},
  highlight3/.style={draw=blue, line width=0.55mm},
  highlight4/.style={draw=orange, line width=0.55mm},
  highlight5/.style={draw=green!70!white, line width=0.55mm},
  hlbg/.style={fill=bgYellow},
  inline/.style={anchor=base, baseline},
  hledge/.style={thick, red},
  between/.style args={#1 and #2}{
    at = ($(#1)!0.5!(#2)$)
  },
}


\begin{document}

% Make it so ./mkWeb works correctly
\ifweb
    \renewcommand{\pause}{}
\fi

\setbeamertemplate{itemize items}[circle]

\begin{comment}

% SOLID COLOR TITLE (see SETTINGS.sty)
{
\begin{frame}[plain]
    \colorlambdatrue
    \titlepage
\end{frame}
}

\begin{frame}[fragile]
  \frametitle{Lesson Plan}

  \tableofcontents
\end{frame}

\sectionSlide{1}{Introduction}

% TODO: community slack link
% https://semgrep.slack.com/archives/C0770SP7WTZ

\begin{frame}[fragile]
  \frametitle{Staff}

  \begin{center}
    \begin{minipage}{0.65\textwidth}
      My name is Brandon Wu, and I am a program analysis engineer working at {\color{semgrepGreen}\href{https://semgrep.dev/}{Semgrep}}.

      \vspace{10pt}

      I have been working at Semgrep for two years now, and I was educated in computer science
      at Carnegie Mellon University, where I previously lectured on the subject of functional
      programming.

      \vspace{10pt}

      Semgrep is a software security startup and application security platform that
      helps developers find and fix security vulnerabilities in their code, at minimal
      friction to their workflow.
    \end{minipage}
    \hspace{\fill}
    \begin{minipage}{0.30\textwidth}
      \begin{center}
        \includegraphics[scale=0.4]{me_smaller.jpg}

        \vspace{5pt}

        \includegraphics[height=0.35cm]{twitter.png}
        {\color{blue}\href{https://twitter.com/onefiftyman}{@onefiftyman}}

        \vspace{5pt}

        \includegraphics[height=0.35cm]{linkedin.png}
        {\color{blue}\href{https://www.linkedin.com/in/brandon-wu-79935116b/}{LinkedIn}}

        % TODO: punch up with funny link
      \end{center}
    \end{minipage}
  \end{center}
\end{frame}

\begin{frame}[fragile]
  \frametitle{Introductions}

  Some questions to get us started:

  \vspace{5pt}

  \begin{itemize}
    \item Is everyone here a security professional?
    \item Who here has heard of Semgrep before?
    \item Has anyone here written a Semgrep rule before?\footnote{"No" is an acceptable answer.}
    \item Did anyone come to a previous Semgrep training?
  \end{itemize}
\end{frame}

\sectionSlide{2}{Static Analysis Essentials}

\begin{frame}[fragile]
  \frametitle{Endgame}

  We're here because we are interested in \term{software security}.

  \vspace{5pt}

  Broadly, we are interested in categories of undesirable behaviors in programs, and
  in particular, we are interested in minimizing them wherever possible. These may
  take the form of:

  \vspace{5pt}

  \begin{itemize}
    \item \textbf{logic errors} - passing an ill-typed argument to a function
    \item \textbf{code smells} - using a deprecated utility function instead of a more
    up-to-date one
    \item \textbf{security vulnerabilities} - SQL injection, cross-site scripting, broken authentication
  \end{itemize}

  \vspace{5pt}

  All of these are fair game for things that we want to find and fix. Our weapon of
  choice for doing so is \term{static analysis}.
\end{frame}

\begin{frame}[fragile]
  \frametitle{What is static analysis?}

  \defBox{}{\term{Static analysis} is the art of analyzing programs without ever running them.}

  \vspace{5pt}

  This is in contrast to \term{dynamic analysis}, which involves running the program and observing
  its behavior at runtime. While an interesting field in its own right, dynamic analysis
  has its own disadvantages, so it is out of scope for this workshop.

  \vspace{10pt}

  There are many static analysis tools out there which work for different languages, and
  have varying degrees of effectiveness. Semgrep is one such tool.

  \vspace{5pt}

  Other tools may include Checkmarx, SonarQube, Coverity, CodeQL, etc.
\end{frame}

% TODO: more explanation of static analysis?

\begin{frame}[fragile]
  \frametitle{Goals in Static Analysis}

  Static analysis should be:

  \vspace{5pt}

  \begin{enumerate}
      \item \textbf{fast} - Scanning code with a static analysis solution should run
      quickly, so that developers can get feedback and iterate as quickly as possible.
      \item \textbf{fitting} - Static analysis should be customized to \textit{your}
      use case. Teams, codebases, and organizations are \textit{not} one-size-fits-all.
      You should be able to scan your code in a \textbf{way that works for you}.
      \item \textbf{friendly} - Static analysis tools should be simple to use, for
      both developers and security engineers. A tool which is not understood is a tool
      which will be disabled.
  \end{enumerate}

  \vspace{\fill}

  These are all things that Semgrep aims to achieve.
\end{frame}

\sectionSlide{3}{Basics: Conceptual Fundamentals}

\begin{frame}[fragile]
  \frametitle{What is Semgrep?}

  Here's some things about Semgrep that are cool:
  \begin{itemize}
    \item {\color{auxColor}\faBookOpen} \, \textbf{open-source}, with community-contributed language support and rules
    \item {\color{auxColor}\faPen} \, \textbf{customizable} - Semgrep scans for code patterns that you define, and provide
    as an argument to the tool. This means that Semgrep can be customized to your code
    base's idioms, and tuned to your organization's security needs.
    \item {\color{auxColor}\faFont} \, \textbf{multilingual} - Semgrep supports over 25+ languages
    \item {\color{auxColor}\faTimes} \, \textbf{no building required}. Semgrep only needs source code to run, and doesn't
    require that the code it is analyzing can build
    \item {\color{auxColor}\faLightbulb} \, \textbf{No DSL}. Semgrep patterns look like code in the language you're targeting,
    meaning no hours of browsing documentation and API references. All you need to be able
    to do is read and write code.\footnote{This is really a killer, and will form the basis
    of this workshop as we move into Semgrep rule-writing.}
  \end{itemize}
\end{frame}

\begin{frame}[fragile]
  \frametitle{Useful references}

  Here are some useful Semgrep-related references, when it comes to utilizing the platform:

  \vspace{5pt}

  \begin{itemize}
    \item The {\color{blue}\href{https://formulae.brew.sh/formula/semgrep}{Semgrep brew formula}}, where you can
    pull the Semgrep CLI to your computer via \code{brew install semgrep}
    \item The {\color{blue}\href{https://marketplace.visualstudio.com/items?itemName=Semgrep.semgrep}{Semgrep VS Code Extension}},
    which allows developers to scan code directory from their IDE, \textbf{before it is committed}.
    \item The {\color{blue}\href{https://semgrep.dev/r}{Semgrep Rule Registry}}, which contains
    all the pre-written rule packs from Semgrep's security research team, as well as community-contributed
    rules
    \item The {\color{blue}\href{https://semgrep.dev/docs/}Semgrep docs page}, which contains
    information on how to run Semgrep in your CI/CD pipeline, how to write rules, and other
    useful information\footnote{This is mostly here as a reference for \textit{after} the workshop. Leave
    that link blue for now.}
  \end{itemize}
\end{frame}

\begin{frame}[fragile]
  \frametitle{What is Semgrep?}

  \begin{minipage}[t][0.5in][t]{\textwidth}
    \begin{center}
      \begin{tikzpicture}[every node/.style={node distance=2in}]
        \node[box, minimum width=2cm] (A) {code};
        \node[box, right of=A, minimum width=2cm] (B) {AST};
        \node[box, right of=B, minimum width=2cm] (C) {analysis};

        \draw[-stealth] (A) -- (B);
        \draw[-stealth] (B) -- (C);
      \end{tikzpicture}
    \end{center}
  \end{minipage}
  \begin{minipage}[t][1.1in][t]{\textwidth}
    \begin{center}
    \begin{minipage}{0.32\textwidth}
      \begin{center}
        \begin{pythoncodeblock}
          b = 1
          def foo(a):
            return a + b
        \end{pythoncodeblock}
      \end{center}
    \end{minipage}
    \begin{minipage}{0.32\textwidth}
      \begin{center}
      \begin{tikzpicture}
        [level distance=8mm,
        level 1/.style={sibling distance=16mm},
        level 2/.style={sibling distance=8mm},
        level 3/.style={sibling distance=4mm},
        every node/.style={circle, inner sep=4pt, draw=black!80, thick, fill=black!20!white},
        ]
        \node {}
          child{node {}
            child{node {}}
            child{node {}
              child{node{}}
              child[missing]
            }
          }
          child{node {}
            child{node{}
              child[missing]
              child[missing]
            }
            child[missing]
        };
      \end{tikzpicture}
    \end{center}
    \end{minipage}
    \begin{minipage}{0.32\textwidth}
      \begin{center}
        \begin{itemize}
          \item pattern matching
          \item IL translation
          \item taint analysis
          \item constant propagation
        \end{itemize}
      \end{center}
    \end{minipage}
    \end{center}
  \end{minipage}
\end{frame}

\begin{frame}[fragile]
  \frametitle{The Lifecycle of a Program}

  Consider the simple program seen on the previous page. To analyze it, we could
  look at the text directly, or we could notice that the shape of the program forms
  a kind of hierarchical structure.

  \vspace{5pt}

  For instance, we note that the program is made of two statements, which could be
  regarded as the children of a parent node.

  \vspace{8pt}

  \begin{minipage}[t][1in][t]{\textwidth}
    \begin{minipage}{0.3\textwidth}
      \begin{pythoncodeblock}
        `b = 1`
        |lr|def foo(a):|lr|
        |lr|  return a + b|lr|
      \end{pythoncodeblock}
    \end{minipage}
    \begin{minipage}{0.65\textwidth}
      \begin{center}
      \begin{tikzpicture}[level 1/.style={sibling distance=3cm}]
        \node[box, fill=bgGray] {\code{statements}}
          child{node[box, fill=hlYellow] {
            \code{b = 1}
          }}
          child{node [box, align=left, fill=hlLightRed] {
            \code{def foo(a):} \\
            \;\;\code{  return a + b}
          }}
        ;
      \end{tikzpicture}
    \end{center}
    \end{minipage}
  \end{minipage}

  \vspace{8pt}

  \customBox{Important}{\, We say that the AST nodes of the assignment to \code{b} and the definition of
  the function \code{foo} have a \term{range}, which is the contiguous span of
  text in the source file that it covers. These are highlighted in yellow and red.}
\end{frame}


\begin{frame}[fragile]
  \frametitle{The Lifecycle of a Program}

  But, this isn't the farthest granularity we can go. We see that the assignment to
  \code{b = 1}, for instance, is made up of parts also. We could regard the identifier
  \code{b} and the literal \code{1} as children of yet another node, this time for
  an assignment.

  \vspace{5pt}

  Our new tree will look like a more detailed version of the previous one:

  \vspace{8pt}

  \begin{minipage}[t][1.4in][t]{\textwidth}
    \begin{minipage}{0.3\textwidth}
      \begin{pythoncodeblock}
        |g|b|g| = |b|1|b|
        def |o|foo|o|(|lp|a|lp|):
          return |lb|a + b|lb|
      \end{pythoncodeblock}
    \end{minipage}
    \begin{minipage}{0.65\textwidth}
      \begin{center}
      \begin{tikzpicture}[level distance=10mm, level 1/.style={sibling distance=4cm}, level 2/.style={sibling distance=1.5cm}]
        \node[box, fill=bgGray] {\code{statements}}
          child{node[box, fill=bgGray, draw=hlYellow] {\code{assignment}}
            child{node[hex, fill=hlGreen, draw=hlYellow] {
              \code{b}
            }}
            child{node[hex, fill=hlBlue, draw=hlYellow] {
              \code{1}
            }}
          }
          child{node[box, fill=bgGray, draw=hlLightRed] {\code{function}}
            child{node[hex, fill=hlOrange, draw=hlLightRed]{\code{foo}}}
            child{node[hex, fill=hlLightPurple, draw=hlLightRed] {\code{a}}}
            child{node[box, fill=bgGray, xshift=0.5cm, draw=hlLightRed] {\code{return}}
              child{node[hex, fill=hlLightBlue, draw=hlLightRed] {\code{a + b}}}
            }
          }
        ;
        \end{tikzpicture}
      \end{center}
    \end{minipage}
  \end{minipage}

  \vspace{8pt}

  Keep in mind that the old assignment node still exists! It just comprises the entire
  subtree of the assignment, which is now highlighted in yellow. The same is true of the
  function node.
\end{frame}


\begin{frame}[fragile]
  \frametitle{What is Semgrep?}

  % TODO: label nodes
  \vspace{5pt}
  \begin{minipage}[t][1.2in][t]{\textwidth}
    \begin{minipage}{0.3\textwidth}
      \begin{pythoncodeblock}
        `b = 1`
        |lr|def foo(a):|lr|
        |lr|  return a + b|lr|
      \end{pythoncodeblock}
    \end{minipage}
    \begin{minipage}{0.65\textwidth}
      \begin{center}
      \begin{tikzpicture}[level 1/.style={sibling distance=3cm}]
        \node[box, fill=bgGray] {\code{statements}}
          child{node[box, fill=hlYellow] {
            \code{b = 1}
          }}
          child{node [box, align=left, fill=hlLightRed] {
            \code{def foo(a):} \\
            \;\;\code{  return a + b}
          }}
        ;
      \end{tikzpicture}
    \end{center}
    \end{minipage}
  \end{minipage}
  \hrule
  \vspace{15pt}
  \begin{minipage}[t][1.8in][t]{\textwidth}
    \begin{minipage}{0.3\textwidth}
      \begin{pythoncodeblock}
        |g|b|g| = |b|1|b|
        def |o|foo|o|(|lp|a|lp|):
          return |lb|a + b|lb|
      \end{pythoncodeblock}
    \end{minipage}
    \begin{minipage}{0.65\textwidth}
      \begin{center}
      \begin{tikzpicture}[level distance=10mm, level 1/.style={sibling distance=4cm}, level 2/.style={sibling distance=1.5cm}]
        \node[box, fill=bgGray] {\code{statements}}
          child{node[box, fill=bgGray, draw=hlYellow] {\code{assignment}}
            child{node[hex, fill=hlGreen, draw=hlYellow] {
              \code{b}
            }}
            child{node[hex, fill=hlBlue, draw=hlYellow] {
              \code{1}
            }}
          }
          child{node[box, fill=bgGray, draw=hlLightRed] {\code{function}}
            child{node[hex, fill=hlOrange, draw=hlLightRed]{\code{foo}}}
            child{node[hex, fill=hlLightPurple, draw=hlLightRed] {\code{a}}}
            child{node[box, fill=bgGray, xshift=0.5cm, draw=hlLightRed] {\code{return}}
              child{node[hex, fill=hlLightBlue, draw=hlLightRed] {\code{a + b}}}
            }
          }
        ;
        \end{tikzpicture}
      \end{center}
    \end{minipage}
  \end{minipage}
\end{frame}

\begin{frame}[fragile]
  \frametitle{Semgrep Patterns}

  Semgrep is about \textbf{matching nodes} and \textbf{obtaining ranges}.

  \vspace{\fill}

  Suppose we were interested in matching calls to \code{print} in Python. We could
  be concerned with the following program:

  \vspace{5pt}

  \begin{pythoncodeblock}
    # this line will print
    print("Hello, world!")

    isprinted(true)

    print("printing")
  \end{pythoncodeblock}

  \vspace{\fill}

  We could use \code{grep} or \code{ctrl-f} to find these calls, but they won't
  work properly!
\end{frame}


\begin{frame}[fragile]
  \frametitle{Semgrep Patterns}

  By \code{grep}ping for \code{print}, we would get the following:

  \vspace{5pt}

  % FRAGILE
  \begin{codeblock}
    # this line will |lr|print|lr|
    |lr|print|lr|('Hello, world!')

    is|lr|print|lr|ed(true)

    |lr|print|lr|('|lr|print|lr|ing now')
  \end{codeblock}

  \vspace{\fill}

  We were interested in finding \textbf{calls} to \code{print}. We see
  that we have three matches we didn't intend -- three \term{false positives}.

  \vspace{\fill}

  The reason is that \code{grep} is not aware of whether each contiguous \code{print}
  is a call or not. It just operates purely based on characters. Put another way,
  \code{grep} is not \term{semantic}.
\end{frame}

\begin{frame}[fragile]
  \frametitle{Semgrep Patterns}

  Semgrep stands for \term{semantic grep}. It is a code-searching tool that is aware
  of the underlying structure of the code -- that is, the AST.

  \vspace{\fill}

  Consider the following example:

  \vspace{5pt}

__compare(
<<<
  print
>>>,
<<<
    # this line will print
    print("Hello, world!")
    isprinted(true)
    print("printing")
>>>
)
\end{frame}

\begin{frame}[fragile]
  \frametitle{Semgrep Patterns}

  With Semgrep, we will match the two calls to \code{print}, and nothing else.

  \vspace{\fill}

__compare(
<<<
  print
>>>,
<<<
    # this line will print
    `print`("Hello, world!")
    isprinted(true)
    `print`("printing")
>>>
)

  \vspace{\fill}

  Note that it is not simply a matter of ignoring the comments and string literals -- we
  also do not match the identifier \code{isprinted}, because it is not the same as the
  identifier \code{print}.
\end{frame}

\begin{frame}[fragile]
  \frametitle{Semgrep Patterns}

  \begin{center}
    \begin{minipage}{0.23\textwidth}

  __pattern(<<<
    print
  >>>)

    \end{minipage}
    \hspace{10pt}
    \begin{minipage}{0.72\textwidth}

  __target(
  <<<
      # this line will print
      print("Hello, world!")
      isprinted(true)
      print("printing")
  >>>)

    \end{minipage}
  \end{center}

  \vspace{\fill}

  \begin{minipage}{0.3\textwidth}
      \begin{center}
      \begin{tikzpicture}
        \node[hex] {\code{print}}
          child[missing]
          child[missing]
        ;
        \end{tikzpicture}
      \end{center}
  \end{minipage}
  \begin{minipage}{0.65\textwidth}
      \begin{center}
      \begin{tikzpicture}[level distance=10mm, level 1/.style={sibling distance=3.5cm}, level 2/.style={sibling distance=1.5cm}]
        \node[box, fill=bgGray] {\code{statements}}
          child{node[box, fill=bgGray] {\code{call}}
            child{node[hex] {\code{print}}}
            child{node[hex, fill=hlBlue, yshift=-1cm, highlight] {\code{"Hello, world!"}}}
          }
          child{node[box, fill=bgGray] {\code{call}}
            child{node[hex, xshift=-0.6cm] {\code{isprinted}}}
            child{node[hex, xshift=0.2cm, fill=hlGreen] {\code{true}}}
          }
          child{node[box, fill=bgGray] {\code{call}}
            child{node[hex] {\code{print}}}
            child{node[hex, fill=hlBlue, yshift=-1cm] {\code{"printing"}}}
          }
        ;
        \end{tikzpicture}
      \end{center}
  \end{minipage}
\end{frame}

\begin{frame}[fragile]
  \frametitle{Semgrep Patterns}

  \begin{center}
    \begin{minipage}{0.23\textwidth}

  __pattern(<<<
    print
  >>>)

    \end{minipage}
    \hspace{10pt}
    \begin{minipage}{0.72\textwidth}

  __target(
  <<<
      # this line will print
      `print`("Hello, world!")
      isprinted(true)
      `print`("printing")
  >>>)

    \end{minipage}
  \end{center}

  \vspace{\fill}

  \begin{minipage}{0.3\textwidth}
      \begin{center}
      \begin{tikzpicture}
        \node[hex, highlight] {\code{print}}
          child[missing]
          child[missing]
        ;
        \end{tikzpicture}
      \end{center}
  \end{minipage}
  \begin{minipage}{0.65\textwidth}
      \begin{center}
      \begin{tikzpicture}[level distance=10mm, level 1/.style={sibling distance=3.5cm}, level 2/.style={sibling distance=1.5cm}]
        \node[box, fill=bgGray] {\code{statements}}
          child{node[box, fill=bgGray] {\code{call}}
            child{node[hex, highlight] {\code{print}}}
            child{node[hex, fill=hlBlue, yshift=-1cm] {\code{"Hello, world!"}}}
          }
          child{node[box, fill=bgGray] {\code{call}}
            child{node[hex, xshift=-0.6cm] {\code{isprinted}}}
            child{node[hex, xshift=0.2cm, fill=hlGreen] {\code{true}}}
          }
          child{node[box, fill=bgGray] {\code{call}}
            child{node[hex, highlight] {\code{print}}}
            child{node[hex, fill=hlBlue, yshift=-1cm] {\code{"printing"}}}
          }
        ;
        \end{tikzpicture}
      \end{center}
  \end{minipage}
\end{frame}

\begin{frame}[fragile]
  \frametitle{Semgrep Patterns}

  Semgrep works by doing three things:
  \begin{enumerate}
    \item parsing the \term{pattern} to an AST
    \item parsing the \term{target} to an AST
    \item \term{matching} both against each other
  \end{enumerate}

  % TODO? add a diagram

  \vspace{\fill}

  Since there are only two nodes on the right-hand side which look like the
  \tikz[inline] \node[hex]{\code{print}}; node, they are the only ones which
  end up being printed.

  \vspace{\fill}

  Note that the \tikz[inline] \node[hex]{\code{isprinted}}; node is \textbf{not}
  matched, because it contains different text!
\end{frame}

\sectionSlide{4}{Basics: Patterns}

\begin{frame}[fragile]
  \begin{exercisePage}{Try it for yourself!}
    Try writing the pattern we just saw within the Semgrep playground. {\color{blue}\href{https://semgrep.dev/playground/r/lBU4AY2/semgrep.print-example?editorMode=structure}{Link}}
    \solutionLink{https://semgrep.dev/playground/r/X5UQ0p0/semgrep.print-example-solution}

    \vspace{\fill}

    {\small\begin{pythoncodeblock}
      # this line will print
      # ruleid: print-example
      print("Hello, world!")
      isprinted(true)
      # ruleid: print-example
      print("printing")
    \end{pythoncodeblock}
    }

    \vspace{\fill}

    \noteBox{}{
    In the Playground page, you will see a couple of tabs, notably ones labeled \code{advanced}
    and \code{taint}. Leave those alone for now.
    }
  \end{exercisePage}
\end{frame}

\begin{frame}[fragile]
  \frametitle{Test Annotations}

  It's worth expanding on the strange comments that are present in the target source.

  \vspace{\fill}

  \customBox{Feature}{\, In the Semgrep playground, you can add \term{test annotations} to your target,
  which will tell the Playground whether a match is expected or not on the next line:}

  \vspace{\fill}

  \begin{pythoncodeblock}
    # ruleid: rule-id-here
    match_expected()
    # ok: rule-id-here
    match_unexpected()
  \end{pythoncodeblock}

  \vspace{\fill}

  This is useful for verifying that your rule is working as expected. We must
  obtain a match on line 2, and no match on line 4, or else the Playground will throw
  an error and tell you that tests are failing. This is also useful for conveying
  what you intend your rule to do!
\end{frame}

\begin{frame}[fragile]
  \frametitle{Test Annotations}

  Our solution has a bug in it, though! We only want to match \textbf{calls} to the
  \textbf{print} function, not the identifier \code{print} itself.

  \vspace{5pt}

  This causes a bug in the following example:

  \vspace{\fill}

__compare_30(
<<<
  print
>>>,
<<<
  # this code should be matched...
  `print`("a call to print")
  # but we also produce a match here!
  alias = `print`
>>>
)

  \vspace{\fill}

  Recall the mantra of Semgrep: \textbf{patterns look like code}. To write a pattern
  which matches call expressions, we will write a pattern which itself looks like a
  Python call.
\end{frame}

\begin{frame}[fragile]
  \frametitle{Ellipses}

  To do this, we will need to somehow vary our pattern over any argument to the
  call. We do this with a special Semgrep syntax, the \term{ellipsis}.

  \vspace{\fill}

  \customBox{Feature}{\, The \term{ellipsis} is a placeholder in a Semgrep pattern which matches
    zero or more occurrences of some element, for instance parameters, arguments,
    and statements.
  }

  \vspace{\fill}

__compare_30(
<<<
  print(|lg|...|lg|)
>>>,
<<<
  # this code should be matched...
  `print("a call to print")`
  # now, no more match here:
  alias = print
>>>
)

  \vspace{\fill}

  \noteBox{}{The range has changed! Instead of matching just the identifier \code{print},
  our match now covers the entire call.}
\end{frame}

\begin{frame}[fragile]
  \frametitle{Ellipses}

  When put as arguments to a function call, ellipses allow you to match \textit{any number}
  of arguments, even 0.

  \vspace{\fill}

  Behold the following example showcasing the mighty ellipsis, paired with an equally
  mighty haiku\footnote{The slide author would like you to know this was not generated by GPT.}:

  \vspace{\fill}

__compare_30(
<<<
  print(|lg|...|lg|)
>>>,
<<<
  # no matter the count
  # when up against arguments
  # ellipsis matches
  `print("one")`
  `print()`
  `print("two", "args")`
>>>
)
\end{frame}

% THINK: important?
\begin{frame}[fragile]
  \frametitle{Nodes and Ranges}

  It's worth clarifying why it is that Semgrep is producing these matches!

  \vspace{\fill}

  When compiled to their respective ASTs, the pattern and target look something like this:

  \vspace{\fill}

  \begin{minipage}{0.3\textwidth}
    \begin{center}
    \begin{tikzpicture}[level 1/.style={sibling distance=2cm}]
      \node[box, fill=bgGray] {\code{call}}
        child{node[hex] {\code{print}}}
        child{node[circ, unknown] {\code{...}}}
      ;
      \end{tikzpicture}
    \end{center}
  \end{minipage}
  \begin{minipage}{0.65\textwidth}
    \begin{center}
    \begin{tikzpicture}[level distance=10mm, level 1/.style={sibling distance=3cm}, level 2/.style={sibling distance=1.5cm}]
      \node[box, fill=bgGray] {\code{statements}}
        child{node[box, fill=bgGray, highlight] {\code{call}}
          child{node[hex, highlight] {\code{print}}}
          child{node[hex, fill=hlBlue, yshift=-1cm, highlight] {\code{"one"}}}
        }
        child{node[box, fill=bgGray, highlight, xshift=-1cm] {\code{call}}
          child{node[hex, highlight] {\code{print}}}
        }
        child{node[box, fill=bgGray, highlight, xshift=-0.3cm] {\code{call}}
          child{node[hex, highlight] {\code{print}}}
          child{node[hex, fill=hlBlue, yshift=-1cm, highlight, xshift=-0.5cm] {\code{"two"}}}
          child{node[hex, fill=hlBlue, yshift=-1cm, highlight] {\code{"args"}}}
        }
      ;
      \end{tikzpicture}
    \end{center}
  \end{minipage}

  \vspace{\fill}

  Here, we produce three matches, because the right tree has three subtrees which
  look the same as the left tree. They look slightly different, but the ellipsis
  node has special rules, and is able to match the rest of the call.
\end{frame}

\begin{frame}[fragile]
  \frametitle{Nodes and Ranges}

  It is also worth clarifying another point: the reason why the match has
  expanded from just \code{print} to the entire call is because we match the
  entire \code{call} node, which has a range comprising the entire call,
  including the parentheses.

  \vspace{\fill}

  You don't need to be able to picture these trees in your head, you just need to know
  that ultimately, all we are doing is matching nodes which have certain ranges.

  \vspace{\fill}

  \begin{center}
    \includegraphics[width=0.5\textwidth]{nodes_and_ranges.png}
  \end{center}
\end{frame}

\begin{frame}[fragile]
  \begin{exercisePage}{Marking banned, deprecated, or dangerous functions}
    Let's try an example out using the ellipsis. {\color{blue}\href{https://semgrep.dev/playground/r/j2Ug0p4/semgrep.python-exec}{Link to the playground}}.
    \solutionLink{https://semgrep.dev/playground/r/10UOwnr/semgrep.python-exec-solution}

    \vspace{5pt}

    {\footnotesize\begin{codeblock}[language=python]
      import exec as definitely_safe_function

      # ruleid: python-exec
      definitely_safe_function(user_input)
      # ruleid: python-exec
      exec()
      # ruleid: python-exec
      exec("ls")
    \end{codeblock}
    }

    \vspace{\fill}

    Keywords: \ellipsisLink
  \end{exercisePage}
\end{frame}

\begin{frame}[fragile]
  \begin{exercisePage}{Dangerous keyword arguments}
    Let's try an example out using the ellipsis. {\color{blue}\href{https://semgrep.dev/playground/r/j2Ug0p4/semgrep.python-exec}{Link to the playground}}.
    \solutionLink{https://semgrep.dev/playground/r/10UOwnr/semgrep.python-exec-solution}

    \vspace{5pt}

    {\footnotesize\begin{codeblock}[language=python]
      import exec as definitely_safe_function

      # ruleid: python-exec
      definitely_safe_function(user_input)
      # ruleid: python-exec
      exec()
      # ruleid: python-exec
      exec("ls")
    \end{codeblock}
    }

    \vspace{\fill}

    Keywords: \ellipsisLink
  \end{exercisePage}
\end{frame}

\begin{frame}[fragile]
  \frametitle{Metavariables}

  Sometimes, we are only interested in a single argument, though.

  In these cases, ellipsis is too permissive.

  \vspace{\fill}

  We need a construct which denotes a single element in the program, kind of like
  a placeholder. We will call this a \term{metavariable}.

  \vspace{\fill}

  \customBox{Feature}{\, The \term{metavariable} is a placeholder in a Semgrep
    pattern which matches exactly one occurrence of some element, for instance
    an argument, expression, or identifier.
  }

  \vspace{\fill}

  The syntax for a metavariable is a dollar sign prefixing a capital identifier,
  for instance \code{$\dollar$X} or \code{$\dollar$EXPR}.
\end{frame}

\begin{frame}[fragile]
  \frametitle{Metavariables}

  Using the ellipsis and metavariable we learned about earlier, we can try to write
  a pattern which matches any Python function.

  \vspace{\fill}

  __compare(
<<<
  def |lg|$FUNC|lg|(|lg|...|lg|):
    |lg|...|lg|
>>>,
<<<
  `def foo():`
  `  pass`

  `def bar(a, b):`
  `  return a + b`
>>>
  )
  __useless($)
\end{frame}

\begin{frame}[fragile]
  \begin{exercisePage}{Enforcing secure cookies}
    Let's use our newfound knowledge to tackle a real use case -- determining if
    a cookie's \code{setSecure} method was called with a \code{false} value.
     {\color{blue}\href{https://semgrep.dev/playground/r/6JUvwkA/semgrep.java-secure-cookies}{Link to the playground}}.
    \solutionLink{https://semgrep.dev/playground/r/yyURWzl/semgrep.java-secure-cookies-solution}

    \vspace{5pt}

    Remember: Semgrep patterns look like code!

    \vspace{\fill}

    Keywords: \metavariableLink
  \end{exercisePage}
\end{frame}

\begin{frame}[fragile]
  \begin{exercisePage}{Enforcing secure named arguments}
    Finally, let's use the ellipsis operator to look at examples of functions
    called in an improper way.

    \vspace{\fill}

    In particular, we are concerned with the Python \code{requests.get} function,
    which is a potential security concern if called with \code{verify=False}.
    {\color{blue}\href{https://semgrep.dev/playground/r/r6UyJXW/semgrep.requests-get-verify}{Link to the playground}}.
    \solutionLink{https://semgrep.dev/playground/r/NbU3pEw/semgrep.requests-get-verify-solution}

    \vspace{\fill}

    \begin{pythoncodeblock}
      # ruleid: requests-get-verify
      requests.get("https://example.com", verify=False)
    \end{pythoncodeblock}

    \vspace{\fill}

    Keywords: \ellipsisLink
  \end{exercisePage}
\end{frame}

\end{comment}

\sectionSlide{5}{Basics: Pattern Operators}

\begin{frame}[fragile]
  \frametitle{Semgrep Rules}

  \begin{minipage}{0.65\textwidth}
    Now that we have seen basic Semgrep patterns, we can go on to constructing actual Semgrep rules.

    \vspace{5pt}

    A Semgrep rule is a YAML file which contains a Semgrep pattern, and a few other things:
    {\small\begin{itemize}
      \setlength\itemsep{-0.15em}
      \item a \term{message}, which is a string which will be displayed when a match is found
      \item a \term{severity} (\code{INFO}, \code{WARNING}, or \code{ERROR}), which describes
      how severe findings from the rule are
      \item a \term{rule id}, which is the name of the rule
      \item a \term{languages} field, which specifies the languages the rule is written for
    \end{itemize}
    }

    \vspace{\fill}

    We will be using the Playground's
    Structure Mode, which will generate the YAML for us from an interface. Note the
    pattern in {\color{fgBlue}blue} and other fields in {\color{fgRed}red}.
  \end{minipage}
  \hspace{\fill}
  \begin{minipage}{0.32\textwidth}
    \begin{center}
      \includegraphics[width=\textwidth]{structure.png}
    \end{center}
  \end{minipage}
\end{frame}

% TODO: mention that in the slides the rules are written in YAML,
% but we will be using Structure Mode
\begin{frame}[fragile]
  \frametitle{Pattern Operators}

  Sometimes, when we are trying to match something, we are interested in combining
  multiple matches in some way.

  \vspace{\fill}

  Usually, this we want to match something which is \textbf{one of
  several things}, or \textbf{all of several things}.

  \vspace{\fill}

  \customBox{Feature}{\, The \code{all} and \code{any} operators are patterns which
  combine other patterns. \code{all} matches ranges where all of the children patterns
  match, and \code{any} matches ranges where at least one of the children patterns match.}

  \vspace{\fill}

__rule(
<<<
    any:
      - "exec(...)"
      - "print(...)"
>>>)

  \vspace{\fill}

  This pattern matches \textbf{both} calls to \code{exec} and calls to \code{print}.
\end{frame}

\begin{frame}[fragile]
  \frametitle{Pattern Operators}

  We can use the corresponding \code{all} operator to match patterns when multiple
  patterns match.

  \vspace{\fill}

__compare_rule(
<<<
  all:
    - "exec(...)"
    - "print(...)"
>>>,
<<<
  # no matches here
  print("hi")
  exec("ls")
>>>
)

  \vspace{\fill}

  For instance, we could write the above rule, but it's not very useful -- \textbf{it will
  match nothing}, because there is no way for a range to be both a call to
  \code{exec} \textit{and} a call to \code{print}.
\end{frame}

\begin{frame}[fragile]
  \frametitle{Pattern Operators}

  Instead, the \code{all} operator is most useful when combined with \code{not}.

  \vspace{\fill}

  \customBox{Feature}{\, The \code{not} operator, when placed under an \code{all}, means
  the result should be anything which \textit{does not} match the pattern under the
  \code{not}.}

  \vspace{\fill}

  With this, we can express complementary logic, such as "all calls which are not
  the print function".

__compare_rule(
<<<
  all:
    - $FUNC(...)
    - not: print(...)
>>>,
<<<
  print("hi")
  `foo()`
  `exec("ls")`
>>>
)
  __useless($)
\end{frame}

\begin{frame}[fragile]
  \begin{exercisePage}{Order of API calls must be enforced}

  Let's do another example. {\color{blue}\href{https://semgrep.dev/playground/r/QrUWj3A/semgrep.find-unverified-transactions-exercise}{Link to the playground}}.
  \solutionLink{https://semgrep.dev/playground/r/4bURwNG/semgrep.find-unverified-transactions-exercise-solution}

  \vspace{\fill}

  In this example, we are working in a financial trading application where we need to
  ensure that \code{verify_transaction} is called on any transaction, before we call
  \code{make_transaction}.

  \vspace{\fill}

  \begin{codeblock}[language=java]
    // ruleid: find-unverified-transactions-exercise
    public void bad(Transaction t) {
      make_transaction(t);
    }
  \end{codeblock}

  \vspace{\fill}

  Keywords: \ellipsisLink, \patternNotLink, \patternAllLink
  \end{exercisePage}
\end{frame}

% TODO: add an exercise exhibiting both `all` and `any` and `not`


\begin{frame}[fragile]
  \begin{exercisePage}{Hardcoded secrets detection}

  Let's do another example. {\color{blue}\href{https://semgrep.dev/playground/r/QrUWj3A/semgrep.find-unverified-transactions-exercise}{Link to the playground}}.
  \solutionLink{https://semgrep.dev/playground/r/4bURwNG/semgrep.find-unverified-transactions-exercise-solution}

  \vspace{\fill}

  We want to catch initializations of a \code{PasswordAuthenticator}, imported
  from a Python package \code{couchbase_core.cluster}. In particular, we are interested in
  when the password passed to the function is a hardcoded literal string (but not the empty string).

  \vspace{\fill}

  \customBox{Feature}{\, A pattern of the form \code{"..."}, with the quotes, is a
  \term{literal ellipsis}, and will match all instances of literal strings.}

  \vspace{\fill}

  \begin{codeblock}
    # ruleid: python-couchbase-hardcoded-secret
    PasswordAuthenticator('username', "password")
  \end{codeblock}

  \vspace{\fill}

  Keywords: \literalEllipsisLink, \patternNotLink, \patternAllLink
  \end{exercisePage}
\end{frame}

\begin{frame}[fragile]
  \frametitle{Pattern Operators}

  So far, we have discussed four pattern operators in \code{pattern}, \code{not},
  \code{all}, and \code{any}.

  \vspace{\fill}

  From the dropdown, we can see there are two more, \code{inside} and \code{regex}.
  These form the fundamental six pattern operators which are used to produce ranges.

  \vspace{\fill}

  \customBox{Feature}{\, The \code{inside} operator is used for when you want to
  specify that a match should be found only when it is a subrange of the pattern
  under the \code{inside}.}

  \vspace{\fill}

  {\small
  __compare_rule(
<<<
    all:
      - inside: |
          def $FUNC(...):
            ...
      - pattern: |
          structlog.get_logger()
>>>,
<<<
  logger = structlog.get_logger()

  def foo():
    logger = `structlog.get_logger()`
>>>)
  }
  __useless($)
\end{frame}

\begin{frame}[fragile]
  \frametitle{Pattern Operators}

  \customBox{Feature}{\, The \code{regex} operator matches ranges of \textbf{text}
  which match the regular expression.}

  \vspace{\fill}

  The \code{regex} operator is the sole exception when it comes to patterns matching
  AST nodes to obtain ranges. \code{regex} is like \code{grep}, and matches
  purely the text. It's useful as an escape hatch, but usually isn't the best solution.

  \vspace{\fill}

  The following rule matches all instances of the string "import" within functions.

  {\small
  __compare_rule(
<<<
    all:
      - regex: "import"
      - inside: |
          def $FUNC(...):
            ...
>>>,
<<<
  import structlog
  import typing

  def foo():
    `import` re
    `import`ed_function = re.match
>>>)
  }
  __useless($)
\end{frame}

\sectionSlide{6}{Advanced: Conditions and Focusing}

\begin{frame}[fragile]
  \frametitle{A Note on Presentation}

  Now that we've seen all the fundamental pattern operators, we can move on to
  \term{conditions} and \term{focusing}, which let us alter and constrain
  existing ranges.

  \vspace{\fill}

  Sometimes we don't care about just producing a match, we care about the specific
  range that the match covers. For instance, we may want to match the name of a
  function, rather than the function entirely.

  \vspace{\fill}

  \customBox{Feature}{\, The \code{focus} operator "focuses" a range to that of
  a metavariable it contains, making its range the same as what the metavariable
  matches.}

  \vspace{\fill}

__compare_rule(
<<<
  pattern: |
    def $FUNC(...):
      ...
  where:
    - focus: $FUNC
>>>,
<<<
  def `foo`():
    pass

  def `bar`():
    pass
>>>
)
\end{frame}

\begin{frame}[fragile]
  \frametitle{A Note on Presentation}

  To add a \code{focus} in the editor, click the filter button.

  \vspace{\fill}

  \begin{center}
    \includegraphics[width=0.5\textwidth]{constraint.png}
  \end{center}

  \vspace{\fill}

  This will open a new row where you can enter the metavariable you
  want to focus on.

  \vspace{\fill}

  \begin{center}
    \includegraphics[width=0.5\textwidth]{focus.png}
  \end{center}
\end{frame}

% TODO: maybe better in place of password auth example
\begin{frame}[fragile]
  \begin{exercisePage}{Non-index routes should check fetch metadata}

  Let's do another example. {\color{blue}\href{https://semgrep.dev/playground/r/0oULYkB/semgrep.secfetch-metadata}{Link to the playground}}.
  \solutionLink{https://semgrep.dev/playground/r/KxUv69E/semgrep.secfetch-metadata-solution}

  \vspace{\fill}

  In this example, we are concerned with routes in a web application which do not
  check fetch metadata. For routes which are not index routes, we want to ensure
  that it has a parameter of type \code{SecFetchMetadata}.

  \vspace{\fill}

  Keywords: \ellipsisLink, \patternNotLink, \focusLink
  \end{exercisePage}
\end{frame}

\begin{frame}[fragile]
  \frametitle{Pattern Operators}

  Sometimes we want to constrain our matches to only ones which fulfill some conditions.
  Usually this is based on specific parts of the pattern, such as metavariables.

  \vspace{\fill}

  \customBox{Feature}{\, \term{Metavariable conditions} are conditions that may be
  placed on a pattern, which filter out matches whose metavariables do not fulfill
  the condition. They come in three varieties:}
  \begin{itemize}
    \item metavariable \code{pattern} constraints, which specify that a metavariable
    must contain a specified pattern inside of it
    \item metavariable \code{regex} constraints, which specify that a metavariable
    must match a specific regex
    \item metavariable \code{analyzer}s, which are specific kinds of validators
    that check for things such as high-entropy strings and ReDoS regexes
  \end{itemize}
\end{frame}

\begin{frame}[fragile]
  \frametitle{Pattern Operators}

  To add a metavariable condition, we use the same drop-down that we would
  select a \code{focus} from:

  \vspace{\fill}

  \begin{center}
    \includegraphics[width=0.35\textwidth]{conditions.png}
  \end{center}

  \vspace{\fill}

  Switching to \code{metavariable} will default to a metavariable \code{pattern}
  constraint. The drop-down below will let you select from \code{pattern},
  \code{regex}, and \code{analyzer}.

  \vspace{\fill}

  \begin{center}
    \includegraphics[width=0.5\textwidth]{metavariable_pattern.png}
  \end{center}
\end{frame}

% TODO: edit solutions to be in new syntax
\begin{frame}[fragile]
  \begin{exercisePage}{Enforcing controller coventions}

  Let's do another example. {\color{blue}\href{https://semgrep.dev/playground/r/YGUpKG9/semgrep.ruby-inherit-appcontroller-fix-annotations}{Link to the playground}}.
  \solutionLink{https://semgrep.dev/playground/r/zdUK0g2/semgrep.ruby-inherit-appcontroller-solution}

  \vspace{\fill}

  We want to catch Ruby controllers which do not inherit from \code{ApplicationController}
  or \code{ActionController::Base}. The catch is that we don't necessarily know which
  Ruby classes are controllers, so we need to use regex on the name of the class to
  infer this.

  \vspace{\fill}

  \begin{codeblock}[language=ruby]
    # Doesn't inherit from ApplicationController: warn!
    class BarController < OtherController
    end
  \end{codeblock}

  \vspace{\fill}

  Keywords: \metavariableRegexLink, \patternNotLink
  \end{exercisePage}
\end{frame}

\begin{frame}[fragile]
  \frametitle{Pattern Operators}

  The final condition operator that we haven't talked about is \code{comparison},
  which lets you verify that metavariables satisfy some programmatic condition.

  \vspace{\fill}

  \customBox{Feature}{\, The \code{comparison} operator lets you write a simple
  Python expression, and only produces matches whose metavariables satisfy the
  expression.}

  \vspace{\fill}

  % TODO: change syntax highlighting in target to terraform
{\footnotesize
__compare_rule(
<<<
  pattern: |
    retention_in_days = $DAYS
  where:
    - comparison: $DAYS <= 365
>>>,
<<<
  resource "aws_cloudwatch_log_group" "notifier" {
    name = "notifier-${var.cluster_name}"
    `retention_in_days = 30`
  }
>>>
)
}
  __useless($)
\end{frame}

\begin{frame}[fragile]
  \begin{exercisePage}{Flag weak RSA encryption keys}

  Let's do another example. {\color{blue}\href{https://semgrep.dev/playground/r/j2Ug0eE/semgrep.use-of-weak-rsa-key}{Link to the playground}}.
  \solutionLink{https://semgrep.dev/playground/r/9AUd9NK/semgrep.use-of-weak-rsa-key-solution}

  \vspace{\fill}

  We want to ensure that generated RSA keys are of a sufficient strength, in our case
  greater than or equal to 2048. Let's write a Semgrep rule which can find instances
  where RSA keys are too weak.

  \vspace{\fill}

  \begin{codeblock}[language=go]
    // ruleid: use-of-weak-rsa-key-solution
    pvk, err := rsa.GenerateKey(rand.Reader, 1024)
    // ruleid: use-of-weak-rsa-key-solution
    pvk, err := rsa.MultiPrimeKey(rand.Reader, 3, 1024)
  \end{codeblock}

  \vspace{\fill}

  Keywords: \metavariableComparisonLink, \patternAnyLink
  \end{exercisePage}
\end{frame}

\sectionSlide{7}{Advanced: Taint Mode}

\begin{frame}[fragile]
  \frametitle{Pattern Operators}

  Many of the problems in security are not quite so simple as can be phrased in
  a regular search mode rule.

  \vspace{\fill}

  Many problems in security come from the simple idea of sensitive data reaching
  a sensitive area, such as a network call or a file write. This problem is solvable
  via a classic technique in static analysis called \term{dataflow analysis}.

  \vspace{\fill}

  \defBox{}{\term{Taint analysis} is a form of dataflow analysis which tracks whether
  some specified form of data (called the \term{source}) is capable of reaching some
  specified sensitive site (called the \term{sink}). We say that data which comes
  from the source is \term{tainted}.}
\end{frame}

\begin{frame}[fragile]
  \frametitle{Pattern Operators}

  For instance, consider the following code snippet:

  \begin{pythoncodeblock}[language=java]
    public class DataflowExample {
      public static void main(String |fb|args|fb|[]) throws IOException {
        String cmd = args[0];
        Runtime.getRuntime().exec(cmd);
      }
    }
  \end{pythoncodeblock}

  \vspace{\fill}

  In this example, the input \code{args} to the program are a potential source of
  user input, which is sensitive data!

  \vspace{\fill}

  If this user input is allowed to reach the \code{Runtime.exec} method, for instance,
  an attacker could potentially cause dangerous arbitrary code execution.

  \vspace{\fill}

  \begin{itemize}
    \item common sources: URL parameters, cookies, user input
    \item common sinks: SQL queries, shell commands
  \end{itemize}
\end{frame}

\begin{frame}[fragile]
  \frametitle{Pattern Operators}

  \begin{minipage}[t][1.6in][t]{\textwidth}
    Let's trace the flow of taint through this code snippet:

    \begin{pythoncodeblock}[language=java]
      public class DataflowExample {
        public static void main(String |lb|args|lb|[]) throws IOException {
          String cmd = args[0];
          Runtime.getRuntime().exec(cmd);
        }
      }
    \end{pythoncodeblock}
  \end{minipage}
  \begin{minipage}[t][1in][t]{\textwidth}
    \begin{enumerate}
      \item the source, the \code{args} argument to \code{main}
    \end{enumerate}
  \end{minipage}
\end{frame}

\begin{frame}[fragile]
  \frametitle{Pattern Operators}

  \begin{minipage}[t][1.6in][t]{\textwidth}
    Let's trace the flow of taint through this code snippet:

    \begin{pythoncodeblock}[language=java]
      public class DataflowExample {
        public static void main(String |fb|args|fb|[]) throws IOException {
          String cmd = |fb|args|fb|[0];
          Runtime.getRuntime().exec(cmd);
        }
      }
    \end{pythoncodeblock}
  \end{minipage}
  \begin{minipage}[t][1in][t]{\textwidth}
    \begin{enumerate}
      \item the source, the \code{args} argument to \code{main}
      \item the expression \code{args[0]}
    \end{enumerate}
  \end{minipage}
\end{frame}

\begin{frame}[fragile]
  \frametitle{Pattern Operators}

  \begin{minipage}[t][1.6in][t]{\textwidth}
    Let's trace the flow of taint through this code snippet:

    \begin{pythoncodeblock}[language=java]
      public class DataflowExample {
        public static void main(String |fb|args|fb|[]) throws IOException {
          String |fb|cmd|fb| = |fb|args|fb|[0];
          Runtime.getRuntime().exec(cmd);
        }
      }
    \end{pythoncodeblock}
  \end{minipage}
  \begin{minipage}[t][1in][t]{\textwidth}
    \begin{enumerate}
      \item the source, the \code{args} argument to \code{main}
      \item the expression \code{args[0]}
      \item the newly declared variable \code{cmd}
    \end{enumerate}
  \end{minipage}
\end{frame}

\begin{frame}[fragile]
  \frametitle{Pattern Operators}

  \begin{minipage}[t][1.6in][t]{\textwidth}
    Let's trace the flow of taint through this code snippet:

    \begin{pythoncodeblock}[language=java]
      public class DataflowExample {
        public static void main(String |fb|args|fb|[]) throws IOException {
          String |fb|cmd|fb| = |fb|args|fb|[0];
          Runtime.getRuntime().exec(|fb|cmd|fb|);
        }
      }
    \end{pythoncodeblock}
  \end{minipage}
  \begin{minipage}[t][1in][t]{\textwidth}
    \begin{enumerate}
      \item the source, the \code{args} argument to \code{main}
      \item the expression \code{args[0]}
      \item the newly declared variable \code{cmd}
      \item the usage of the tainted variable \code{cmd}
    \end{enumerate}
  \end{minipage}
\end{frame}









% \sectionSlide{8}{Advanced: Miscellaneous}

% \begin{frame}[fragile]
%   \begin{exercisePage}{Enforcing secure defaults}

%   \end{exercisePage}
% \end{frame}

% \begin{frame}[fragile]
%   \begin{exercisePage}{Enforcing secure defaults}
%   \end{exercisePage}
% \end{frame}


\thankyou

\end{document}
