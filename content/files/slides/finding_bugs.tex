% Jacob Neumann

% DOCUMENT CLASS AND PACKAGE USE
\documentclass[aspectratio=169, handout]{beamer}

% Establish the colorlambda boolean, to control whether the lambda is solid color (true), or the same as the picture (false)
\newif\ifcolorlambda
\colorlambdafalse % DEFAULT: false

% Use auxcolor for syntax highlighting
\newif\ifuseaux
\useauxfalse % DEFAULT: false

% Color settings
\useauxtrue

\newcommand{\auxColor}{e0b45a}     % the color of note boxes and stuff
\newcommand{\presentColor}{08A679} % the primary color of the slide borders
\newcommand{\bgColor}{e1f2ea}      % the color of the background of the slide
\newcommand{\darkBg}{8b98ad}
\newcommand{\lambdaColor}{\auxColor}

\colorlambdatrue

\usepackage[pro]{fontawesome5}
\usepackage{comment} % comment blocks
\usepackage{soul} % strikethrough
\usepackage{listings} % code
\usepackage{makecell}
\usepackage{tcolorbox}
\usepackage{amssymb}% http://ctan.org/pkg/amssymb
\usepackage{pifont}% http://ctan.org/pkg/pifont
\usepackage[outline]{contour}
\usepackage{ stmaryrd }

\setbeamertemplate{itemize items}[circle]
% \setbeameroption{show notes on second screen=right}

\usepackage{lectureSlides}
%%%%%%%%%%%%%%%%%%%%%%%%%%%%%%%%%%%%%%%%%| <----- Don't make the title any longer than this
\title{Finding Bugs and Scaling Your Security Program} % TODO
\subtitle{Using Semgrep to take your security game to the next level} % TODO
\date{16 June 2024} % TODO
\author{Brandon Wu} % TODO


\def\checkmark{\tikz\fill[green, scale=0.4](0,.35) -- (.25,0) -- (1,.7) -- (.25,.15) -- cycle;}
\contourlength{.08em}% default is 0.03em
\newcommand{\cmark}{{\color{green!80!black}\ding{51}}}
\newcommand{\xmark}{{\color{red}\ding{55}}}

\newcommand{\exerciseHeader}[2]{
  \begin{minipage}[t][0.4in][t]{\textwidth}
    \Large \term{Exercise:} #1
  \end{minipage}
  \begin{minipage}[t][2in][t]{\textwidth}
    #2
  \end{minipage}
}

\newenvironment{exercisePage}[1]{
  \begin{minipage}[t][0.4in][t]{\textwidth}
    \Large \term{Exercise:} #1
  \end{minipage}
  \begin{minipage}[t][2in][t]{\textwidth}
}{
  \end{minipage}
}

\definecolor{semgrepGreen}{HTML}{2ACFA6}

\graphicspath{ {./img/} }


\newcommand{\dollar}{\mbox{\textdollar}}


    % DONT FORGET TO PUT [fragile] on frames with codeblocks, specs, etc.
        %\begin{frame}[fragile]
        %\begin{codeblock}
        %fun fact 0 = 1
        %  | fact n = n * fact(n-1)
        %\end{codeblock}
        %\end{frame}

    % INCLUDING codefile:
        % 1. In some file under code/NN (where NN is the lecture id num), include:
    %       (* FRAGMENT KK *)
    %           <CONTENT>
    %       (* END KK *)

    %    Remember to not put anything on the same line as the FRAGMENT or END comment, as that won't be included. KK here is some (not-zero-padded) integer. Note that you MUST have fragments 0,1,...,KK-1 defined in this manner in order for fragment KK to be properly extracted.
        %  2. On the slide where you want code fragment K
                % \smlFrag[color]{KK}
        %     where 'color' is some color string (defaults to 'white'. Don't use presentColor.
    %  3. If you want to offset the line numbers (e.g. have them start at line 5 instead of 1), use
                % \smlFragOffset[color]{KK}{5}
\colorlet{bgBlue}{hlBlue!130}
\colorlet{bgOrange}{hlOrange}
\colorlet{fgBlue}{fgCodeBlue}
\colorlet{fgGreen}{fgCodeGreen}
\colorlet{fgRed}{fgCodeRed}
\colorlet{fgOrange}{fgCodeOrange}

\colorlet{codeBackground}{background_color}

\definecolor{bgPurple}{HTML}{dbc7ff}
\definecolor{bgRed}{HTML}{fcaea9}
\definecolor{bgYellow}{HTML}{fffeb5}
\definecolor{bgGreen}{HTML}{b4ffb3}
\definecolor{bgGray}{HTML}{bfbfbf}

\definecolor{fgOrange}{HTML}{fcae4e}
\definecolor{fgYellow}{HTML}{ffbe0a}

\definecolor{unknownPink}{HTML}{e0fff6}

\tikzset{
  every path/.style={line width=0.25mm},
  tok/.style={
    font=\large,
    align=center,
    minimum height=3cm,
  },
  belowtok/.style={
    font=\large,
    align=center,
    minimum height=3cm,
  },
  astNode/.style={
    draw,
    inner sep=2pt,
    line width=0.4mm,
    minimum size=0.6cm,
    font=\large,
    align=center,
  },
  hex/.style={
    astNode,
    signal,
    signal to=east and west,
    signal pointer angle=130,
    fill=bgPurple,
  },
  box/.style={
    astNode,
    rectangle,
    fill=green!20!white,
  },
  bbox/.style={
    astNode,
    rectangle,
    fill=bgBlue,
  },
  circ/.style={
    astNode,
    circle,
    minimum size=0.7cm,
    fill=bgYellow
  },
  bhex/.style={
    astNode,
    signal,
    inner sep=0.5pt,
    signal to=east and west,
    signal pointer angle=130,
    fill=bgBlue,
    minimum width=0.8cm,
  },
  unknown/.style={hex, draw=black, preaction={fill, unknownPink}, pattern=north east lines, pattern color=gray},
  unknown2/.style={unknown, preaction={fill, bgYellow}},
  highlight/.style={draw=fgYellow, line width=0.55mm},
  highlight2/.style={draw=magenta, line width=0.55mm},
  highlight3/.style={draw=blue, line width=0.55mm},
  highlight4/.style={draw=orange, line width=0.55mm},
  highlight5/.style={draw=green!70!white, line width=0.55mm},
  hlbg/.style={fill=bgYellow},
  inline/.style={anchor=base, baseline},
  hledge/.style={thick, red},
  between/.style args={#1 and #2}{
    at = ($(#1)!0.5!(#2)$)
  },
}


\begin{document}

% Make it so ./mkWeb works correctly
\ifweb
    \renewcommand{\pause}{}
\fi

\setbeamertemplate{itemize items}[circle]

% SOLID COLOR TITLE (see SETTINGS.sty)
% {
% \begin{frame}[plain]
%     \colorlambdatrue
%     \titlepage
% \end{frame}
% }

% \begin{frame}[fragile]
%   \frametitle{Lesson Plan}

%   \tableofcontents
% \end{frame}

% \sectionSlide{1}{Introduction}

% TODO: community slack link
% https://semgrep.slack.com/archives/C0770SP7WTZ

\begin{frame}[fragile]
  \frametitle{Staff}

  \begin{center}
    \begin{minipage}{0.65\textwidth}
      My name is Brandon Wu, and I am a program analysis engineer working at {\color{semgrepGreen}\href{https://semgrep.dev/}{Semgrep}}.

      \vspace{10pt}

      I have been working at Semgrep for two years now, and I was educated in computer science
      at Carnegie Mellon University, where I previously lectured on the subject of functional
      programming.

      \vspace{10pt}

      Semgrep is a software security startup and application security platform that
      helps developers find and fix security vulnerabilities in their code, at minimal
      friction to their workflow.
    \end{minipage}
    \hspace{\fill}
    \begin{minipage}{0.30\textwidth}
      \begin{center}
        \includegraphics[scale=0.4]{me_smaller.jpg}

        \vspace{5pt}

        \includegraphics[height=0.35cm]{twitter.png}
        {\color{blue}\href{https://twitter.com/onefiftyman}{@onefiftyman}}

        \vspace{5pt}

        \includegraphics[height=0.35cm]{linkedin.png}
        {\color{blue}\href{https://www.linkedin.com/in/brandon-wu-79935116b/}{LinkedIn}}

        % TODO: punch up with funny link
      \end{center}
    \end{minipage}
  \end{center}
\end{frame}

\begin{frame}[fragile]
  \frametitle{Introductions}

  Some questions to get us started:

  \vspace{5pt}

  \begin{itemize}
    \item Is everyone here a security professional?
    \item Who here has heard of Semgrep before?
    \item Has anyone here written a Semgrep rule before?\footnote{"No" is an acceptable answer.}
    \item Did anyone come to a previous Semgrep training?
  \end{itemize}
\end{frame}

\sectionSlide{2}{Static Analysis Essentials}

\begin{frame}[fragile]
  \frametitle{Endgame}

  We're here because we are interested in \term{software security}.

  \vspace{5pt}

  Broadly, we are interested in categories of undesirable behaviors in programs, and
  in particular, we are interested in minimizing them wherever possible. These may
  take the form of:

  \vspace{5pt}

  \begin{itemize}
    \item \textbf{logic errors} - passing an ill-typed argument to a function
    \item \textbf{code smells} - using a deprecated utility function instead of a more
    up-to-date one
    \item \textbf{security vulnerabilities} - SQL injection, cross-site scripting, broken authentication
  \end{itemize}

  \vspace{5pt}

  All of these are fair game for things that we want to find and fix. Our weapon of
  choice for doing so is \term{static analysis}.
\end{frame}

\begin{frame}[fragile]
  \frametitle{What is static analysis?}

  \defBox{}{\term{Static analysis} is the art of analyzing programs without ever running them.}

  \vspace{5pt}

  This is in contrast to \term{dynamic analysis}, which involves running the program and observing
  its behavior at runtime. While an interesting field in its own right, dynamic analysis
  has its own disadvantages, so it is out of scope for this workshop.

  \vspace{10pt}

  There are many static analysis tools out there which work for different languages, and
  have varying degrees of effectiveness. Semgrep is one such tool.

  \vspace{5pt}

  Other tools may include Checkmarx, SonarQube, Coverity, CodeQL, etc.
\end{frame}

% TODO: more explanation of static analysis?

\begin{frame}[fragile]
  \frametitle{Goals in Static Analysis}

  Static analysis should be:

  \vspace{5pt}

  \begin{enumerate}
      \item \textbf{fast} - Scanning code with a static analysis solution should run
      quickly, so that developers can get feedback and iterate as quickly as possible.
      \item \textbf{fitting} - Static analysis should be customized to \textit{your}
      use case. Teams, codebases, and organizations are \textit{not} one-size-fits-all.
      You should be able to scan your code in a \textbf{way that works for you}.
      \item \textbf{friendly} - Static analysis tools should be simple to use, for
      both developers and security engineers. A tool which is not understood is a tool
      which will be disabled.
  \end{enumerate}

  \vspace{\fill}

  These are all things that Semgrep aims to achieve.
\end{frame}

\sectionSlide{3}{Static Analysis through Semgrep}

\begin{frame}[fragile]
  \frametitle{What is Semgrep?}

  Here's some things about Semgrep that are cool:
  \begin{itemize}
    \item {\color{auxColor}\faBookOpen} \, \textbf{open-source}, with community-contributed language support and rules
    \item {\color{auxColor}\faPen} \, \textbf{customizable} - Semgrep scans for code patterns that you define, and provide
    as an argument to the tool. This means that Semgrep can be customized to your code
    base's idioms, and tuned to your organization's security needs.
    \item {\color{auxColor}\faFont} \, \textbf{multilingual} - Semgrep supports over 25+ languages
    \item {\color{auxColor}\faTimes} \, \textbf{no building required}. Semgrep only needs source code to run, and doesn't
    require that the code it is analyzing can build
    \item {\color{auxColor}\faLightbulb} \, \textbf{No DSL}. Semgrep patterns look like code in the language you're targeting,
    meaning no hours of browsing documentation and API references. All you need to be able
    to do is read and write code.\footnote{This is really a killer, and will form the basis
    of this workshop as we move into Semgrep rule-writing.}
  \end{itemize}
\end{frame}

\begin{frame}[fragile]
  \frametitle{Useful references}

  Here are some useful Semgrep-related references, when it comes to utilizing the platform:

  \vspace{5pt}

  \begin{itemize}
    \item The {\color{blue}\href{https://formulae.brew.sh/formula/semgrep}{Semgrep brew formula}}, where you can
    pull the Semgrep CLI to your computer via \code{brew install semgrep}
    \item The {\color{blue}\href{https://marketplace.visualstudio.com/items?itemName=Semgrep.semgrep}{Semgrep VS Code Extension}},
    which allows developers to scan code directory from their IDE, \textbf{before it is committed}.
    \item The {\color{blue}\href{https://semgrep.dev/r}{Semgrep Rule Registry}}, which contains
    all the pre-written rule packs from Semgrep's security research team, as well as community-contributed
    rules
    \item The {\color{blue}\href{https://semgrep.dev/docs/}Semgrep docs page}, which contains
    information on how to run Semgrep in your CI/CD pipeline, how to write rules, and other
    useful information\footnote{This is mostly here as a reference for \textit{after} the workshop. Leave
    that link blue for now.}
  \end{itemize}
\end{frame}

\begin{frame}[fragile]
  \frametitle{What is Semgrep?}

  \begin{minipage}[t][0.5in][t]{\textwidth}
    \begin{center}
      \begin{tikzpicture}[every node/.style={node distance=2in}]
        \node[box, minimum width=2cm] (A) {code};
        \node[box, right of=A, minimum width=2cm] (B) {AST};
        \node[box, right of=B, minimum width=2cm] (C) {analysis};

        \draw[-stealth] (A) -- (B);
        \draw[-stealth] (B) -- (C);
      \end{tikzpicture}
    \end{center}
  \end{minipage}
  \begin{minipage}[t][1.1in][t]{\textwidth}
    \begin{center}
    \begin{minipage}{0.32\textwidth}
      \begin{center}
        \begin{pythoncodeblock}
          b = 1
          def foo(a):
            return a + b
        \end{pythoncodeblock}
      \end{center}
    \end{minipage}
    \begin{minipage}{0.32\textwidth}
      \begin{center}
      \begin{tikzpicture}
        [level distance=8mm,
        level 1/.style={sibling distance=16mm},
        level 2/.style={sibling distance=8mm},
        level 3/.style={sibling distance=4mm},
        every node/.style={circle, inner sep=4pt, draw=black!80, thick, fill=black!20!white},
        ]
        \node {}
          child{node {}
            child{node {}}
            child{node {}
              child{node{}}
              child[missing]
            }
          }
          child{node {}
            child{node{}
              child[missing]
              child[missing]
            }
            child[missing]
        };
      \end{tikzpicture}
    \end{center}
    \end{minipage}
    \begin{minipage}{0.32\textwidth}
      \begin{center}
        \begin{itemize}
          \item pattern matching
          \item IL translation
          \item taint analysis
          \item constant propagation
        \end{itemize}
      \end{center}
    \end{minipage}
    \end{center}
  \end{minipage}
\end{frame}

\begin{frame}[fragile]
  \frametitle{The Lifecycle of a Program}

  Consider the simple program seen on the previous page. To analyze it, we could
  look at the text directly, or we could notice that the shape of the program forms
  a kind of hierarchical structure.

  \vspace{5pt}

  For instance, we note that the program is made of two statements, which could be
  regarded as the children of a parent node.

  \vspace{8pt}

  \begin{minipage}[t][1in][t]{\textwidth}
    \begin{minipage}{0.3\textwidth}
      \begin{pythoncodeblock}
        `b = 1`
        |lr|def foo(a):|lr|
        |lr|  return a + b|lr|
      \end{pythoncodeblock}
    \end{minipage}
    \begin{minipage}{0.65\textwidth}
      \begin{center}
      \begin{tikzpicture}[level 1/.style={sibling distance=3cm}]
        \node[box, fill=bgGray] {\code{statements}}
          child{node[box, fill=hlYellow] {
            \code{b = 1}
          }}
          child{node [box, align=left, fill=hlLightRed] {
            \code{def foo(a):} \\
            \;\;\code{  return a + b}
          }}
        ;
      \end{tikzpicture}
    \end{center}
    \end{minipage}
  \end{minipage}

  \vspace{8pt}

  \customBox{Important}{\, We say that the AST nodes of the assignment to \code{b} and the definition of
  the function \code{foo} have a \term{range}, which is the contiguous span of
  text in the source file that it covers. These are highlighted in yellow and red.}
\end{frame}


\begin{frame}[fragile]
  \frametitle{The Lifecycle of a Program}

  But, this isn't the farthest granularity we can go. We see that the assignment to
  \code{b = 1}, for instance, is made up of parts also. We could regard the identifier
  \code{b} and the literal \code{1} as children of yet another node, this time for
  an assignment.

  \vspace{5pt}

  Our new tree will look like a more detailed version of the previous one:

  \vspace{8pt}

  \begin{minipage}[t][1.4in][t]{\textwidth}
    \begin{minipage}{0.3\textwidth}
      \begin{pythoncodeblock}
        |g|b|g| = |b|1|b|
        def |o|foo|o|(|lp|a|lp|):
          return |lb|a + b|lb|
      \end{pythoncodeblock}
    \end{minipage}
    \begin{minipage}{0.65\textwidth}
      \begin{center}
      \begin{tikzpicture}[level distance=10mm, level 1/.style={sibling distance=4cm}, level 2/.style={sibling distance=1.5cm}]
        \node[box, fill=bgGray] {\code{statements}}
          child{node[box, fill=bgGray, draw=hlYellow] {\code{assignment}}
            child{node[hex, fill=hlGreen, draw=hlYellow] {
              \code{b}
            }}
            child{node[hex, fill=hlBlue, draw=hlYellow] {
              \code{1}
            }}
          }
          child{node[box, fill=bgGray, draw=hlLightRed] {\code{function}}
            child{node[hex, fill=hlOrange, draw=hlLightRed]{\code{foo}}}
            child{node[hex, fill=hlLightPurple, draw=hlLightRed] {\code{a}}}
            child{node[box, fill=bgGray, xshift=0.5cm, draw=hlLightRed] {\code{return}}
              child{node[hex, fill=hlLightBlue, draw=hlLightRed] {\code{a + b}}}
            }
          }
        ;
        \end{tikzpicture}
      \end{center}
    \end{minipage}
  \end{minipage}

  \vspace{8pt}

  Keep in mind that the old assignment node still exists! It just comprises the entire
  subtree of the assignment, which is now highlighted in yellow. The same is true of the
  function node.
\end{frame}


\begin{frame}[fragile]
  \frametitle{What is Semgrep?}

  % TODO: label nodes
  \vspace{5pt}
  \begin{minipage}[t][1.2in][t]{\textwidth}
    \begin{minipage}{0.3\textwidth}
      \begin{pythoncodeblock}
        `b = 1`
        |lr|def foo(a):|lr|
        |lr|  return a + b|lr|
      \end{pythoncodeblock}
    \end{minipage}
    \begin{minipage}{0.65\textwidth}
      \begin{center}
      \begin{tikzpicture}[level 1/.style={sibling distance=3cm}]
        \node[box, fill=bgGray] {\code{statements}}
          child{node[box, fill=hlYellow] {
            \code{b = 1}
          }}
          child{node [box, align=left, fill=hlLightRed] {
            \code{def foo(a):} \\
            \;\;\code{  return a + b}
          }}
        ;
      \end{tikzpicture}
    \end{center}
    \end{minipage}
  \end{minipage}
  \hrule
  \vspace{15pt}
  \begin{minipage}[t][1.8in][t]{\textwidth}
    \begin{minipage}{0.3\textwidth}
      \begin{pythoncodeblock}
        |g|b|g| = |b|1|b|
        def |o|foo|o|(|lp|a|lp|):
          return |lb|a + b|lb|
      \end{pythoncodeblock}
    \end{minipage}
    \begin{minipage}{0.65\textwidth}
      \begin{center}
      \begin{tikzpicture}[level distance=10mm, level 1/.style={sibling distance=4cm}, level 2/.style={sibling distance=1.5cm}]
        \node[box, fill=bgGray] {\code{statements}}
          child{node[box, fill=bgGray, draw=hlYellow] {\code{assignment}}
            child{node[hex, fill=hlGreen, draw=hlYellow] {
              \code{b}
            }}
            child{node[hex, fill=hlBlue, draw=hlYellow] {
              \code{1}
            }}
          }
          child{node[box, fill=bgGray, draw=hlLightRed] {\code{function}}
            child{node[hex, fill=hlOrange, draw=hlLightRed]{\code{foo}}}
            child{node[hex, fill=hlLightPurple, draw=hlLightRed] {\code{a}}}
            child{node[box, fill=bgGray, xshift=0.5cm, draw=hlLightRed] {\code{return}}
              child{node[hex, fill=hlLightBlue, draw=hlLightRed] {\code{a + b}}}
            }
          }
        ;
        \end{tikzpicture}
      \end{center}
    \end{minipage}
  \end{minipage}
\end{frame}

\begin{frame}[fragile]
  \frametitle{Semgrep Patterns}

  Semgrep is about \textbf{matching nodes} and \textbf{obtaining ranges}.

  \vspace{\fill}

  Suppose we were interested in matching calls to \code{print} in Python. We could
  be concerned with the following program:

  \vspace{5pt}

  \begin{pythoncodeblock}
    # this line will print
    print("Hello, world!")

    isprinted(true)

    print("printing")
  \end{pythoncodeblock}

  \vspace{\fill}

  We could use \code{grep} or \code{ctrl-f} to find these calls, but they won't
  work properly!
\end{frame}


\begin{frame}[fragile]
  \frametitle{Semgrep Patterns}

  By \code{grep}ping for \code{print}, we would get the following:

  \vspace{5pt}

  % FRAGILE
  \begin{codeblock}
    # this line will |lr|print|lr|
    |lr|print|lr|('Hello, world!')

    is|lr|print|lr|ed(true)

    |lr|print|lr|('|lr|print|lr|ing now')
  \end{codeblock}

  \vspace{\fill}

  We were interested in finding \textbf{calls} to \code{print}. We see
  that we have two matches we didn't intend, one in a comment and one in
  a string literal -- two \term{false positives}.

  \vspace{\fill}

  The reason is that \code{grep} is not aware of whether each contiguous \code{print}
  is a call or not. It just operates purely based on characters. Put another way,
  \code{grep} is not \term{semantic}.
\end{frame}

\begin{frame}[fragile]
  \frametitle{Semgrep Patterns}

  Semgrep stands for \term{semantic grep}. It is a code-searching tool that is aware
  of the underlying structure of the code -- that is, the AST.

  \vspace{\fill}

  Consider the following example:

  \vspace{5pt}

__compare(
<<<
  print
>>>,
<<<
    # this line will print
    print("Hello, world!")
    isprinted(true)
    print("printing")
>>>
)
\end{frame}

\begin{frame}[fragile]
  \frametitle{Semgrep Patterns}

  With Semgrep, we will match the two calls to \code{print}, and nothing else.

  \vspace{\fill}

__compare(
<<<
  print
>>>,
<<<
    # this line will print
    `print`("Hello, world!")
    isprinted(true)
    `print`("printing")
>>>
)

  \vspace{\fill}

  Note that it is not simply a matter of ignoring the comments and string literals -- we
  also do not match the identifier \code{isprinted}, because it is not the same as the
  identifier \code{print}.
\end{frame}

\begin{frame}[fragile]
  \frametitle{Semgrep Patterns}

  \begin{center}
    \begin{minipage}{0.23\textwidth}

  __pattern(<<<
    print
  >>>)

    \end{minipage}
    \hspace{10pt}
    \begin{minipage}{0.72\textwidth}

  __target(
  <<<
      # this line will print
      print("Hello, world!")
      isprinted(true)
      print("printing")
  >>>)

    \end{minipage}
  \end{center}

  \vspace{\fill}

  \begin{minipage}{0.3\textwidth}
      \begin{center}
      \begin{tikzpicture}
        \node[hex] {\code{print}}
          child[missing]
          child[missing]
        ;
        \end{tikzpicture}
      \end{center}
  \end{minipage}
  \begin{minipage}{0.65\textwidth}
      \begin{center}
      \begin{tikzpicture}[level distance=10mm, level 1/.style={sibling distance=3.5cm}, level 2/.style={sibling distance=1.5cm}]
        \node[box, fill=bgGray] {\code{statements}}
          child{node[box, fill=bgGray] {\code{call}}
            child{node[hex] {\code{print}}}
            child{node[hex, fill=hlBlue, yshift=-1cm] {\code{"Hello, world!"}}}
          }
          child{node[box, fill=bgGray] {\code{call}}
            child{node[hex, xshift=-0.6cm] {\code{isprinted}}}
            child{node[hex, xshift=0.2cm, fill=hlGreen] {\code{true}}}
          }
          child{node[box, fill=bgGray] {\code{call}}
            child{node[hex] {\code{print}}}
            child{node[hex, fill=hlBlue, yshift=-1cm] {\code{"printing"}}}
          }
        ;
        \end{tikzpicture}
      \end{center}
  \end{minipage}
\end{frame}

\begin{frame}[fragile]
  \frametitle{Semgrep Patterns}

  \begin{center}
    \begin{minipage}{0.23\textwidth}

  __pattern(<<<
    print
  >>>)

    \end{minipage}
    \hspace{10pt}
    \begin{minipage}{0.72\textwidth}

  __target(
  <<<
      # this line will print
      `print`("Hello, world!")
      isprinted(true)
      `print`("printing")
  >>>)

    \end{minipage}
  \end{center}

  \vspace{\fill}

  \begin{minipage}{0.3\textwidth}
      \begin{center}
      \begin{tikzpicture}
        \node[hex, highlight] {\code{print}}
          child[missing]
          child[missing]
        ;
        \end{tikzpicture}
      \end{center}
  \end{minipage}
  \begin{minipage}{0.65\textwidth}
      \begin{center}
      \begin{tikzpicture}[level distance=10mm, level 1/.style={sibling distance=3.5cm}, level 2/.style={sibling distance=1.5cm}]
        \node[box, fill=bgGray] {\code{statements}}
          child{node[box, fill=bgGray] {\code{call}}
            child{node[hex, highlight] {\code{print}}}
            child{node[hex, fill=hlBlue, yshift=-1cm] {\code{"Hello, world!"}}}
          }
          child{node[box, fill=bgGray] {\code{call}}
            child{node[hex, xshift=-0.6cm] {\code{isprinted}}}
            child{node[hex, xshift=0.2cm, fill=hlGreen] {\code{true}}}
          }
          child{node[box, fill=bgGray] {\code{call}}
            child{node[hex, highlight] {\code{print}}}
            child{node[hex, fill=hlBlue, yshift=-1cm] {\code{"printing"}}}
          }
        ;
        \end{tikzpicture}
      \end{center}
  \end{minipage}
\end{frame}

\begin{frame}[fragile]
  \frametitle{Semgrep Patterns}

  Semgrep works by doing three things:
  \begin{enumerate}
    \item parsing the \term{pattern} to an AST
    \item parsing the \term{target} to an AST
    \item \term{matching} both against each other
  \end{enumerate}

  % TODO? add a diagram

  \vspace{\fill}

  Since there are only two nodes on the right-hand side which look like the
  \tikz[inline] \node[hex]{\code{print}}; node, they are the only ones which
  end up being printed.

  \vspace{\fill}

  Note that the \tikz[inline] \node[hex]{\code{isprinted}}; node is \textbf{not}
  matched, because it contains different text!
\end{frame}

\begin{frame}[fragile]
  \exerciseHeader{Try it for yourself!}{
    Try writing the pattern we just saw within the Semgrep playground.

    \vspace{5pt}

    You can access it at {\color{blue}\href{https://semgrep.dev/playground/r/lBU4AY2/semgrep.print-example?editorMode=structure}{this link}}.

    \vspace{\fill}

    \noteBox{}{
    In the Playground page, you will see a couple of tabs, notably ones labeled \code{advanced}
    and \code{taint}. Leave those alone for now.
    }
  }
\end{frame}

\begin{frame}[fragile]
  \frametitle{Test Annotations}

  It's worth expanding on the strange comments that are present in the target source.

  \vspace{\fill}

  \customBox{Feature}{\, In the Semgrep playground, you can add \term{test annotations} to your target,
  which will tell the Playground whether a match is expected or not on the next line:}

  \vspace{\fill}

  \begin{pythoncodeblock}
    # ruleid: rule-id-here
    match_expected()
    # ok: rule-id-here
    match_unexpected()
  \end{pythoncodeblock}

  \vspace{\fill}

  This is useful for verifying that your rule is working as expected. We must
  obtain a match on line 2, and no match on line 4, or else the Playground will throw
  an error and tell you that tests are failing. This is also useful for conveying
  what you intend your rule to do!

\end{frame}

\begin{frame}[fragile]
  \begin{exercisePage}{Enforcing secure defaults}
    Let's try a real Java use case! {\color{blue}\href{https://semgrep.dev/playground/r/6JUvwkA/semgrep.java-secure-cookies}{Link to the playground}}.

    \vspace{5pt}

    {\footnotesize\begin{codeblock}[language=java]
      import javax.servlet.http.Cookie;

      class CookieExample {
          public void SecureCookiesTest(){
              Cookie c = new Cookie("name", "value");
              //ok: java-secure-cookies
              c.setSecure(true);

              Cookie c2 = new Cookie("name", "value");
              // ruleid: java-secure-cookies
              c2.setSecure(false);
          }
      }
    \end{codeblock}
    }
  \end{exercisePage}
\end{frame}

\thankyou

\end{document}
