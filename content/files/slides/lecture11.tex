% Jacob Neumann

% DOCUMENT CLASS AND PACKAGE USE
    \documentclass[aspectratio=169]{beamer}
 
    % Establish the colorlambda boolean, to control whether the lambda is solid color (true), or the same as the picture (false)
    \newif\ifcolorlambda
    \colorlambdafalse % DEFAULT: false
    
    % Use auxcolor for syntax highlighting
    \newif\ifuseaux
    \useauxfalse % DEFAULT: false
   
    % Color settings
    \useauxtrue
    
    \newcommand{\auxColor}{03009A}     % the color of note boxes and stuff
    \newcommand{\presentColor}{0066FF} % the primary color of the slide borders
    \newcommand{\bgColor}{d8eff2}      % the color of the background of the slide
    \newcommand{\darkBg}{8b98ad}
    \newcommand{\lambdaColor}{\auxColor}
  
    \colorlambdatrue

    \usepackage{comment} % comment blocks
    \usepackage{soul} % strikethrough
    \usepackage{listings} % code
    \usepackage{makecell}
    \usepackage{multimedia}

    \newcommand\bh{\tikz[remember picture]
      \node (begin highlight) {};
    }
    \newcommand\eh{\tikz[remember picture]
      \node (end highlight) {};
      \tikz[remember picture, overlay] 
      \draw[yellow,line width=10pt,opacity=0.3] (begin highlight) -- (end
        highlight);
    }

    \setbeamertemplate{itemize items}[circle]
    % \setbeameroption{show notes on second screen=right}

    \usepackage{lectureSlides}
    %%%%%%%%%%%%%%%%%%%%%%%%%%%%%%%%%%%%%%%%%| <----- Don't make the title any longer than this
    \title{Continuation-Passing Style} % TODO
    \subtitle{Making control flow explicit} % TODO
    \date{20 June 2023} % TODO
    \author{Brandon Wu} % TODO

    \graphicspath{ {./img/} }
    % DONT FORGET TO PUT [fragile] on frames with codeblocks, specs, etc.
        %\begin{frame}[fragile]
        %\begin{codeblock}
        %fun fact 0 = 1
        %  | fact n = n * fact(n-1)
        %\end{codeblock}
        %\end{frame}

    % INCLUDING codefile:
        % 1. In some file under code/NN (where NN is the lecture id num), include:
    %       (* FRAGMENT KK *)
    %           <CONTENT>
    %       (* END KK *)
    
    %    Remember to not put anything on the same line as the FRAGMENT or END comment, as that won't be included. KK here is some (not-zero-padded) integer. Note that you MUST have fragments 0,1,...,KK-1 defined in this manner in order for fragment KK to be properly extracted.
        %  2. On the slide where you want code fragment K
                % \smlFrag[color]{KK}
        %     where 'color' is some color string (defaults to 'white'. Don't use presentColor.
    %  3. If you want to offset the line numbers (e.g. have them start at line 5 instead of 1), use
                % \smlFragOffset[color]{KK}{5}

\begin{document}

% Make it so ./mkWeb works correctly
\ifweb
    \renewcommand{\pause}{}
\fi

\setbeamertemplate{itemize items}[circle]

% SOLID COLOR TITLE (see SETTINGS.sty)
{
\begin{frame}[plain]
    \colorlambdatrue
    \titlepage
\end{frame}
}

\begin{frame}[fragile]
  \frametitle{Lesson Plan}

  \tableofcontents
\end{frame}

\begin{frame}[fragile]
  \frametitle{Last time}

  Last lecture, we explored more applications of \term{higher-order functions}.
  In particular, we talked about how we can take advantage of \term{currying}
  to create \term{staged} functions, which can do useful work by shifting
  around computations with respect to when curried arguments are taken in.

  We then looked at more examples of HOFs in action, such as by generalizing
  \code{map} and \code{fold} to tree data structures.

  Finally, we looked at using the \code{|>} operator to sequentialize our
  code for enhanced readability, as well as the option \term{monad}, which
  let us reduce the amount of boilerplate code we needed to write.
\end{frame}

\sectionSlide{1}{Pipelines}

\begin{comment}
\begin{frame}[fragile]
  \frametitle{The Set-Up}

  Suppose that we are interested in reading in a file from the command line,
  which contains a comma-separated list of integers, and then summing all
  of them.

  % this is maybe _too_ interesting
  We might write the following function:
  \begin{codeblock}
    fun getSumOfNumbersInFile filename =
      List.foldr op+ 0 (
        List.map Option.valOf (
          List.map Int.fromString (
            String.tokens (fn c => c = #",") (readFile filename)
          )
        )
      )
  \begin{codeblock}

  This is kind of a mess.
\end{frame}

\begin{frame}[fragile]
  \frametitle{Pipelines}

  We talked about the \code{|>} operator, which lets us sequence operations.
  
  So now, we can write some very straightforward code:

  \begin{codeblock}
    fun getSumOfNumbersInFile filename =
      readFile filename
      |> String.tokens (fn c => c = #",")
      |> List.map Int.fromString  
      |> List.map Option.valOf (* never actually do this *)
      |> List.foldr op+ 0
  \end{codeblock}
\end{frame}

\begin{frame}[fragile]
  \frametitle{Explicitness}

  But what if we wanted to divide by the number of total items? 

  This would be pretty difficult, because we can't refer to the list we're
  summing!

  \begin{codeblock}
    fun getSumOfNumbersInFile filename =
      readFile filename
      |> String.tokens (fn c => c = #",")
      |> List.map Int.fromString  
      |> List.map Option.valOf (* never actually do this *)
      |> List.foldr op+ 0
  \end{codeblock}
\end{frame}
\end{comment}

\begin{frame}[fragile]
  \frametitle{Pipes and Arguments}

  We defined the \code{|>} operator before as
  \begin{codeblock}
    infix |>
    fun x |> f = f x
  \end{codeblock}

  This lets us produce code like
  \begin{codeblock}
    [1, 2, 3] 
    |> map Int.toString
    |> foldr (fn (x, y) => x ^ "," ^ y) ""
  \end{codeblock}

  This works pretty well!
\end{frame}

\begin{frame}[fragile]
  \frametitle{A Less Convenient Example}

  What if we were doing something slightly different? Suppose we now have
  a list of strings, and we want to interpret them as integers and find the
  average.

  \begin{codeblock}
    ["1", "2", "3"] 
    |> map Int.fromString 
    |> map Option.valOf
    |> (fn L => (foldr op+ 0 L) div (List.length L)) 
  \end{codeblock}

  We have to introduce a lambda, because we want to use the value of
  \code{L} twice. We need to know how much the denominator is.
  
  This is kind of gross, and breaks our nice sequencing, though! Because of 
  this \code{div} function, it's not clear what the control flow of our
  function is. Before, our steps were cleanly separated by pipes -- now, 
  we rely on evaluation order. Can we do better?
\end{frame}

\begin{frame}[fragile]
  \frametitle{Pipes and Arguments}

  \begin{codeblock}
    fun mapCool f L k = k (map f L) 
    fun foldrCool f z L k = k (foldr f z L) 
  \end{codeblock}

  \begin{codeblock}
    mapCool Int.fromString ["1", "2", "3"] (fn L1 => 
    mapCool Option.valOf L1                (fn L2 =>
    foldrCool op+ 0 L2                     (fn sum => 
    sum div (List.length L2)               )))
  \end{codeblock}

  Now, we can maintain our nice sequencing, and reading this function
  still looks like a linear list of instructions. We have to name each
  individual step's result, but this is a strength, in this case. 
\end{frame}

\begin{frame}[fragile]
  \frametitle{\code{let} Instead}

  But all of this seems a little complicated. Why don't we just use a \code{let}?

  \begin{codeblock}
    let
      val L1 = map Int.fromString ["1", "2", "3"]
      val L2 = map Option.valOf L1
      val sum = foldr op+ 0 L2
      val res = sum div (List.length L2)
    in
      res
    end
  \end{codeblock}

  It's similarly readable.
\end{frame}

\begin{frame}[fragile]
  \frametitle{\code{let} Instead}

  What if we wanted to do this for a recursive function, though?

  \begin{codeblock}
    fun fact 0 = 1
      | fact n = 
        let 
          val rec_ans = fact (n - 1)
          val res = n * rec_ans
        in
          res
        end
  \end{codeblock}

  Uh oh! We run into a problem that we saw several lectures ago...
\end{frame}

\begin{frame}[fragile]
  \frametitle{A Tail Problem}

  This function is not tail recursive!

  We make the recursive call to \code{fact}, and then multiply it by 
  \code{n}. Actually, this is exactly equivalent to the ordinary \code{fact}
  function we would ordinarily write, just spread out on multiple lines.

  We know the solution, of course. Let's write:
  \begin{codeblock}
    fun tfact 0 acc = acc 
      | tfact n acc = tfact n (acc * n)
  \end{codeblock}

  Is it always straightforward, though?
\end{frame}

\begin{frame}[fragile]
  \frametitle{A Redundant Problem}

  Let's try \code{map}. For this one, we similarly cannot
  naively use a \code{let} for our intermediate computations.
  
  \begin{codeblock}
    fun tmap f [] acc = acc
      | tmap f (x::xs) acc = tmap f xs (f x :: acc) 

    fun map f L = tmap f L []
  \end{codeblock}

  Seems good, right?

  \badBox{}{\, Wrong. This is an incorrect implementation of \code{map}! }
\end{frame}

\begin{frame}[fragile]
  \frametitle{A Redundant Problem}

  We see that \code{tmap Int.toString [1, 2, 3] []} $\hookrightarrow$ 
  \code{["3", "2", "1"]}! It reverses the list!

  What can we do? Well, maybe all is not lost. It happens to be that 
  we know how to implement a tail-recursive version of \code{rev}:

  \begin{codeblock}
    fun tmap_backwards f [] acc = acc
      | tmap_backwards f (x::xs) acc = tmap f xs (f x :: acc) 

    fun trev [] acc = acc
      | trev (x::xs) acc = trev xs (x::acc)

    fun map f L = 
      trev (tmap_backwards f L []) []
  \end{codeblock}

  But this just becomes harder to reason about, and ugly as well. Our goal
  is that every function should admit a tail-recursive version. This one is,
  but can we make this transformation any more natural?
\end{frame}

\begin{frame}[fragile]
  \frametitle{A Tail Problem}

  How did we get here?

  \begin{enumerate}
    \item We wanted to write nicely sequenced, explicit operations that chained together
    \item We wanted to be more explicit about our intermediate computations, because
    we might want to use them in later steps
  \end{enumerate}

  One way to solve is by what we did earlier, having each function in our pipeline 
  take in a lambda which contained the "next step" to be done.

  These criteria are also satisfied with \code{let} expressions, except for the  
  fact that in most cases, simply using a \code{let} ends up being non
  tail recursive!

  \keyBox{}{\, This method of taking in a lambda of the "next step" \textit{always
  produces a tail-recursive function}. We call this \term{continuation-passing style}.}
\end{frame}

\sectionSlide{2}{Continuation-Passing Style}

\begin{frame}[fragile]
  \frametitle{Continuation-Passing Style}

  What is a \term{continuation}?

  \defBox{}{\, A \term{continuation} is a function taken in as an argument,
  which denotes what to do after the current computation.}

  It is named as such because it tells the function taking it in how to
  \textit{continue} once it finishes its computation.

  For instance, the declaration
  \begin{codeblock}
    fun mapCool f L k = k (map f L)
  \end{codeblock}
  has the function \code{k} as a continuation, because it passes its
  return value directly to the continuation function.
\end{frame}

\begin{frame}[fragile]
  \frametitle{Continuation-Passing Style}

  The type of a function taking in a continuation changes in a predictable way.

  For instance, if we want a function of type \code{int -> string -> bool} to
  take in a continuation, then its type would change from
  \begin{codeblock}
    int -> string -> bool
  \end{codeblock}
  to
  \begin{codeblock}
    int -> string -> `(bool -> 'a)` -> `'a`
  \end{codeblock}

  \noteBox{}{\, The return type is polymorphic, because it depends on what the 
  particular continuation that is passed in does!}
\end{frame}

\begin{frame}[fragile]
  \frametitle{A Simple CPS Example}

  For instance, suppose we wanted to write the following function in CPS:
  \begin{codeblock}
    (* add : int -> int -> int *)
    (* REQUIRES: true *)
    (* ENSURES: add x y =~= x + y *)
    fun add x y = x + y
  \end{codeblock}

  The following would suffice:
  \begin{codeblock}
    (* add : int -> int -> (int -> 'a) -> 'a *)
    (* REQUIRES: true *)
    (* ENSURES: addCPS x y k =~= k (x + y) *)
    fun addCPS x y k = k (x + y)
  \end{codeblock}

  Is it always as simple as this, though? Suppose we were trying to
  make \code{fact} in CPS:
\end{frame}

\begin{frame}[fragile]
  \frametitle{Tail Recursion is Transitive}

  \begin{codeblock}
    fun factCPS f L k = k (fact f L)
  \end{codeblock}

  \badBox{}{\, This function is not in CPS.}

  There's something wrong with our implementation. We do indeed call 
  \code{k}, but we make another call to \code{fact}, which we know is 
  not tail recursive!

  \vspace{\fill}

  Our goal here was to transform a non tail-recursive function into
  a tail recursive one using continuations. Just adding continuations
  like this doesn't make it tail recursive, however.

  \vspace{\fill}

  Tail recursion is transitive, in the sense that a tail recursive
  function can only be so if it only calls other tail recursive functions. 
  We need to rewrite \code{fact} entirely!
\end{frame}

\begin{frame}[fragile]
  \frametitle{\code{fact} with \code{let}}

  This is what we had before, when we tried to write sequencing using \code{let}:

  \vspace{\fill}

  \begin{codeblock}
    (* fact : int -> int *) 
    (* REQUIRES: n >= 0 *)
    (* ENSURES: fact n evaluates to n! *)
    fun fact 0 = 1
      | fact n = 
        let 
          val rec_ans = fact (n - 1) 
          val res = n * rec_ans
        in
          res
        end
  \end{codeblock}

  How can we rewrite this in CPS?
\end{frame}

\begin{frame}[fragile]
  \frametitle{CPS for Recursive Functions}

  For recursive functions, the process of converting to CPS is more involved.

  The distinction here has to do with the fact that we have to make sure any
  recursive calls to the function happen last! Otherwise, our function will
  not be tail recursive, and therefore not in CPS.

  The way to think about this is to draw a distinction between
  \textbf{writing down instructions} versus \textbf{remembering instructions}.
  In another sense, the difference is the distinction between \textbf{now}
  and \textbf{later}. 
\end{frame}

\begin{frame}[fragile]
  \frametitle{Now and Later}

  My friend Steve and I are in a band, and every so often we want to play music
  together. I ask Steve to go print out the sheet music for our latest song,
  and he returns to me with the music.

  \vspace{\fill}

  Unfortunately, the stack of papers is all out of order! I sort them so that the
  music is in the right order, and we play.

  \vspace{\fill}

  This happens a couple of times before I realize that the fact that I need to keep
  organizing the music after Steve prints it for me is annoying. I need to drop
  what I'm doing and start sorting it, and the key problem is I need to \textit{remember}
  to sort it! This takes up space in my brain. 

  \vspace{\fill}

  A better way of doing things would be for me to, instead of \textbf{later}
  having to remember to sort the papers, to ask Steve right \textbf{now} to 
  sort the papers for me, and then bring them back to me in the right order. 
\end{frame}

\begin{frame}[fragile]
  \frametitle{Later (Remembering Instructions)}

  \begin{codeblock}
    let 
      val rec_ans = fact (n - 1) 
      val res = n * rec_ans
    in
      res
    end
  \end{codeblock}

  Using \code{let} to make a few bindings illustrates the \textbf{later} mindset,
  which entails \textbf{remembering instructions}.

  \vspace{\fill}

  In this code, I need to first make a call to \code{fact (n - 1)}, and then
  \textit{remember to multiply it afterwards}. This takes up space in the computer,
  because we need to do something after the \code{fact}! This is super not nice.

  \vspace{\fill}

  This means that after the recursive call, we \textbf{later} need to remember to
  do further work. This means we have to \textbf{remember instructions}. 
\end{frame}

\begin{frame}[fragile]
  \frametitle{\code{fact} in CPS}

  But, if we just give the perspective a switch:

  \vspace{\fill}

  \begin{codeblock}
    (* fact : int -> (int -> 'a) -> 'a *) 
    (* REQUIRES: n >= 0 *)
    (* ENSURES: factCPS n k evaluates to k (fact n) *)
    fun factCPS 0 `k` = `k 1`
      | factCPS n `k` = 
        `factCPS (n - 1) (fn rec_ans =>`
          let 
            val rec_ans = `rec_ans`
            val res = n * rec_ans
          in
            `k res`
          end
        `)`
  \end{codeblock}
\end{frame}

\begin{frame}[fragile]
  \frametitle{Now (Writing Instructions)}

  In this example, instead of executing the call to \code{factCPS}
  and then having work to do after, we make a tail call to \code{factCPS}!

  The difference is that we put the work that must be done afterwards 
  into a \term{continuation}, which is to say a lambda expression. This
  means that we don't need to do anything after the recursive call, but
  we \textit{tell the recursive call what it needs to do later}.
\end{frame}

\begin{frame}[fragile]
  \frametitle{Demystifying the Analogy}

  In this analogy, Steve is our recursive call, and the sheet music is the
  value that we want it to return.

  \vspace{\fill}

  The choices are either to be \textbf{direct}, or to use CPS.

  \vspace{\fill}

  In a \term{direct-style} function, we have our recursive call return to us a value,
  which we then need to remember to do something to. We need to \textbf{remember
  instructions} to \textbf{later} execute on the value which it returns to us.
  In the case of \code{fact}, that is to multiply the recursive value by \code{n}.

  \vspace{\fill}

  In a \term{CPS} function, we \textbf{write down instructions} by encoding them into
  a lambda, and then we \textit{give those instructions to the recursive call}.
  This way, we need to do no work on our part -- the recursive call takes care
  of it for us. This means that instead of needing to remember instructions,
  we simply write it down \textbf{now}.
\end{frame}

\begin{frame}[fragile]
  \frametitle{Lambdas as Instructions}

  I've used the analogy of lambda expressions as instructions a few times now.

  The idea is that a lambda expression is a list of steps to be done with a
  currently unknown input. The key observation is that everything in the body
  of a lambda expression \textbf{is not evaluated}. This means that writing
  down something like:

  \begin{codeblock}
    (fn onions => onions |> chop |> grill |> put sandwich) 
  \end{codeblock}

  is equivalent in intention to a list of instructions, which says:
  \begin{itemize}
    \item Take the onions 
    \item Chop the onions 
    \item Grill the onions 
    \item Put the onions on a sandwich 
  \end{itemize}
\end{frame}

\begin{frame}[fragile]
  \frametitle{Finishing the Translation}

  After some extensionally equivalent refactoring, we end up with the 
  CPS translation of \code{fact} as:

  \begin{codeblock}
    fun factCPS 0 k = k 1
      | factCPS n k = 
        factCPS (n - 1) (fn rec_ans => k (n * rec_ans)) 
  \end{codeblock}

  Let's do a trace to see how this function actually works!
\end{frame}

\begin{frame}[fragile]
  \frametitle{The Rules of CPS}

  By following the translation steps above, we can arrive at a CPS function.

  To be more explicit, however, here are the rules defining a functon in CPS:

  \defBox{}{\, We say a function is in \term{continuation-passing} style if it
  fulfills the following criteria:}

  \begin{enumerate}
    \item It takes in and uses continuations
    \item It makes calls to other functions with continuations (including itself)
    as tail calls
    \item It only calls continuations as tail calls 
  \end{enumerate}
\end{frame}

\begin{frame}[fragile]
  \frametitle{A CPS Trace}

  \begin{center}
    \huge \color{blue} \href{https://asciinema.org/a/WxUSfzU1xOdbJkikgyy6wFJ0l}{link to asciinema video}
\end{center}
\end{frame}

\sectionSlide{3}{CPS Translation}

\begin{frame}[fragile]
  \frametitle{CPS Translation}

  Here's the general formula for how we can carry out CPS conversion on a function.

  We are, given a function \code{f : t1 -> t2}, seeking its CPS version, which is
  \code{f_cps : t1 -> (t2 -> 'a) -> 'a} such that \code{f_cps x k} $\eeq$ \code{k (f x)}.

  \begin{enumerate}
    \item For a function with return type \code{t}, add an extra curried argument
    of type \code{t -> 'a}, and then change the return type to \code{'a}
    \item Call the extra continuation argument on every single return value of
    the function.
    \item Suppose there is a recursive call to the function, which is the expression
    \code{e}. Change that to a free variable, let's say \code{rec_ans}. 
    \item Change the body of the recursive case to one which first performs 
    \code{e (fn rec_ans => <body>)}, where the \code{<body>} is just the current 
    body of the function.
  \end{enumerate}
\end{frame}

\begin{frame}[fragile]
  \frametitle{I'm the Map}

  Let's execute these translation steps on \code{map}. First, we start off with
  the vanilla implementation:

  \begin{codeblock}
    fun map f []      = []
      | map f (x::xs) = 
          f x :: map f xs
  \end{codeblock}
\end{frame}

\begin{frame}[fragile]
  \frametitle{I'm the Map: Step 1}

  First, let's add in the continuation \code{k}:

  \begin{codeblock}
    fun map f [] `k`      = []
      | map f (x::xs) `k` = 
          f x :: map f xs
  \end{codeblock}
\end{frame}

\begin{frame}[fragile]
  \frametitle{I'm the Map: Step 2}

  Now, let's call the continuation \code{k} on every expression which
  is returned by the function: 

  \begin{codeblock}
    fun map f [] k      = `k []`
      | map f (x::xs) k = 
          `k (`f x :: map f xs`)`
  \end{codeblock}
\end{frame}

\begin{frame}[fragile]
  \frametitle{I'm the Map: Step 3}

  Next, let's identify the recursive calls to \code{map}. In this case, 
  our recursive call \code{e} is just \code{map f xs}:

  \begin{codeblock}
    fun map f [] k      = k []
      | map f (x::xs) k = 
          k (f x :: `map f xs`)
  \end{codeblock}

  \vspace{\fill}

  Let's assume that we have the answer to the recursive call already, call it
  \code{rec_ans}, and replace \code{e} with it:

  \begin{codeblock}
    fun map f [] k      = k []
      | map f (x::xs) k = 
          k (f x :: `rec_ans`)
  \end{codeblock}
\end{frame}

\begin{frame}[fragile]
  \frametitle{I'm the Map: Step 4}

  Now, we take the recursive case and wrap it in a tail-recursive call to
  \code{e}, the expression we just replaced, except given a continuation binding 
  \code{rec_ans}:

  \begin{codeblock}
    fun map f [] k      = k []
      | map f (x::xs) k = 
          `map f xs (fn rec_ans =>` k (f x :: rec_ans)`)`
  \end{codeblock}
\end{frame}

\begin{frame}[fragile]
  \frametitle{I'm the Map}

  Now we have a complete, CPS version of \code{map}!

  \begin{codeblock}
    fun mapCPS f [] k      = k []
      | mapCPS f (x::xs) k = 
          mapCPS f xs (fn rec_ans => k (f x :: rec_ans))
  \end{codeblock}
\end{frame}

\begin{frame}[fragile]
  \frametitle{The Method to Madness}

  It is important to remember \textit{why} this works in the first place.

  \vspace{\fill}

  All we are doing is making the binding of the recursive call to 
  \code{map} explicit. Instead of leaving it somewhere nested in a big 
  expression, to be used later, we \textit{first} make the recursive call to
  the CPS-ified \code{map}, which is then \textit{given instructions}
  on what to do with the recursive call's value.

  \vspace{\fill}

  The application of \code{k} to each returning expression is necessary to
  complete the "inductive handshake", to fulfill the promise that we 
  pass whatever value we compute to the continuation.
\end{frame}

\begin{comment}
  To demonstrate the difference between CPS and direct-style,
  call for a student volunteer.

  I will ask the student to walk across the room and get a 
  "FUNCTIONS ARE POINTERS" sticker. When they return with
  it, I will rip it up.

  I will then ask them to do it again.

  The key distinction is that, I need to remember to rip
  it up when they come back! This is annoying to me, because
  it takes up space in my brain, and generally inconveniences
  me. It would be better if I just told the student to rip it up
  themselves, and then come back with the pieces.
\end{comment}

\sectionSlide{4}{Control Flow}

\begin{frame}[fragile]
  \frametitle{CPS with Optional Returns}

  We can CPS-ify more interesting examples, that demonstrate the
  ability to express more nuanced control flow using continuations.

  Consider the case of a function which is allowed to fail, i.e.
  returns a type \code{t option} for some type \code{t}, such as
  finding an element that satisfies a predicate in a tree:

  \begin{codeblock}
    fun search p Empty = NONE
      | search p (Node (L, x, R)) =
          if p x then
            SOME x 
          else
            case search p L of
              NONE => search p R
            | SOME res => SOME res
  \end{codeblock}
\end{frame}

\begin{frame}[fragile]
  \frametitle{Continuing from an Option}

  When the recursive call is made to the function, we see that we have to handle
  two cases:
  \begin{codeblock}
    case search p L of
      NONE => search p R
    | SOME res => SOME res
  \end{codeblock}

  In essence, the function needs to "continue" from the recursive call in
  one of two cases: 
  \begin{itemize}
    \item no information, in the \code{NONE} case
    \item a value of type \code{t}, for a \code{t tree}, in the \code{SOME} case
  \end{itemize}

  We can think of this as a \textit{success} case and a \textit{failure} case.
\end{frame}

\begin{frame}[fragile]
  \frametitle{Success and Failure}

  To make our reasoning more explicit, we will separate the logic of these 
  cases into \textit{two} continuations.

  Instead of a single continuation of type \code{t option -> 'a},
  we will instead have a \term{success continuation} of type \code{t -> 'a},
  and a \term{failure continuation} of type \code{unit -> 'a}. 

  Instead of calling our continuation on an optional value to decide which
  case to branch on, we will simply call our failure continuation when we
  would otherwise return \code{NONE}, and call our success continuation
  on the value we would otherwise inject into \code{SOME}.
\end{frame}

\begin{frame}[fragile]
  \frametitle{CPS Translation (Optional Type)}

  Here's the general formula for how we can carry out CPS conversion on a function
  which returns an optional value:

  \begin{enumerate}
    \item For a function with return type \code{t option}, add two extra curried 
    arguments -- one of type \code{t -> 'a} (the \term{success continuation}), 
    and then one of type \code{unit -> 'a} (the \term{failure continuation}). 
    Change the return type to \code{'a}.
    \item For every return of \code{NONE}, call the failure continuation instead.
    For every return of \code{SOME x}, call the success continuation on \code{x}
    instead. 
    \item Suppose there is a \code{case} on a recursive call to the function. 
    Replace the \code{case} with a call to the CPS-ified function itself, but 
    with the success continuation changed to the code in the \code{SOME} case, 
    and the failure continuation changed to the code in the \code{NONE} case. 
  \end{enumerate}
\end{frame}

\begin{frame}[fragile]
  \frametitle{A Star Search: Step 1}

  Let's add in the success and failure continuations, \code{sc} and \code{fc}.
  
  \begin{codeblock}
    fun search p Empty `sc fc` = NONE
      | search p (Node (L, x, R)) `sc fc` =
          if p x then
            SOME x 
          else
            case search p L of
              NONE => search p R
            | SOME res => SOME res
  \end{codeblock}
\end{frame}

\begin{frame}[fragile]
  \frametitle{A Star Search: Step 2}

  Now let's change the implicit returning values to explicit calls to
  \code{sc} and \code{fc}: 
  
  \begin{codeblock}
    fun search p Empty sc fc = `fc ()`
      | search p (Node (L, x, R)) sc fc =
          if p x then
            `sc x` 
          else
            case search p L of
              NONE => search p R
            | SOME res => `sc res`
  \end{codeblock}
\end{frame}

\begin{frame}[fragile]
  \frametitle{A Star Search: Step 3}

  We see one case where we dispatch on the return value of \code{search p L}.
  
  Let's change that to instead be a call to \code{search}, except with the
  success and failure continuations corresponding to the code of each case.

  First, let's identify each branch we are interested in:
  
  \begin{codeblock}
    fun search p Empty sc fc = fc ()
      | search p (Node (L, x, R)) sc fc =
          if p x then
            sc x 
          else
            case search p L of
              &NONE => search p R&
            | `SOME res => sc res`
  \end{codeblock}
\end{frame}

\begin{frame}[fragile]
  \frametitle{A Star Search: Step 3.5}

  Now let's change it to an explicit recursive call:

  \begin{codeblock}
    fun search p Empty sc fc = fc ()
      | search p (Node (L, x, R)) sc fc =
          if p x then
            sc x 
          else
            search p L 
              `(fn res => sc res)`
              &(fn () => search p R sc fc)& 
  \end{codeblock}
\end{frame}

\begin{frame}[fragile]
  \frametitle{A Star Search}

  Now we have a complete CPS version of \code{search}!

  \begin{codeblock}
    fun searchCPS p Empty sc fc = fc ()
      | searchCPS p (Node (L, x, R)) sc fc =
          if p x then
            sc x 
          else
            searchCPS p L 
              (fn res => sc res)
              (fn () => searchCPS p R sc fc)
  \end{codeblock}
\end{frame}

\begin{frame}[fragile]
  \frametitle{Conclusion}

  Continuation-passing style is merely a \textbf{different way of structuring
  computation}.
  
  Instead of following implicit evaluation rules (left to right evaluation,
  outer then inner), we explicitly name and sequence every single computation,
  which makes our control flow more clear.

  The process of CPS conversion that we've seen has been rather mechanical. This
  is intentional, because it is so mechanical that even a compiler can do it!

  Regardless, understanding of CPS is an important skill. Being able to convert functions
  into CPS demonstrates mastery over the distinction between data as 
  ordinary values versus data as function, and CPS-like code shows up in common
  applications, such as in the form of callbacks for web programming.
\end{frame}

\begin{frame}[plain]
	\begin{center} Thank you! \end{center}
\end{frame}


\end{document}

