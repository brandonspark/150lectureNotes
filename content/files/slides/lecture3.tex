% Jacob Neumann

% DOCUMENT CLASS AND PACKAGE USE
    \documentclass[aspectratio=169]{beamer}
 
    % Establish the colorlambda boolean, to control whether the lambda is solid color (true), or the same as the picture (false)
    \newif\ifcolorlambda
    \colorlambdafalse % DEFAULT: false
    
    % Use auxcolor for syntax highlighting
    \newif\ifuseaux
    \useauxfalse % DEFAULT: false
   
    % Color settings
    \useauxtrue
    
    \newcommand{\auxColor}{1A5F7A}     % the color of note boxes and stuff
    \newcommand{\presentColor}{002B5B} % the primary color of the slide borders
    \newcommand{\bgColor}{a6c4e3}      % the color of the background of the slide
    \newcommand{\darkBg}{8b98ad}
    \newcommand{\lambdaColor}{\auxColor}
  
    \colorlambdatrue

    \usepackage{comment} % comment blocks
    \usepackage{soul} % strikethrough
    \usepackage{listings} % code
    \usepackage{makecell}

    \setbeamertemplate{itemize items}[circle]
    % \setbeameroption{show notes on second screen=right}

    \usepackage{lectureSlides}
    %%%%%%%%%%%%%%%%%%%%%%%%%%%%%%%%%%%%%%%%%| <----- Don't make the title any longer than this
    \title{Induction and Recursion} % TODO
    \subtitle{Awesome slides with an awesome subtitle} % TODO
    \date{01 January 2020} % TODO
    \author{Brandon Wu} % TODO

    \graphicspath{ {./img/} }
    % DONT FORGET TO PUT [fragile] on frames with codeblocks, specs, etc.
        %\begin{frame}[fragile]
        %\begin{codeblock}
        %fun fact 0 = 1
        %  | fact n = n * fact(n-1)
        %\end{codeblock}
        %\end{frame}

    % INCLUDING codefile:
        % 1. In some file under code/NN (where NN is the lecture id num), include:
    %       (* FRAGMENT KK *)
    %           <CONTENT>
    %       (* END KK *)
    
    %    Remember to not put anything on the same line as the FRAGMENT or END comment, as that won't be included. KK here is some (not-zero-padded) integer. Note that you MUST have fragments 0,1,...,KK-1 defined in this manner in order for fragment KK to be properly extracted.
        %  2. On the slide where you want code fragment K
                % \smlFrag[color]{KK}
        %     where 'color' is some color string (defaults to 'white'. Don't use presentColor.
    %  3. If you want to offset the line numbers (e.g. have them start at line 5 instead of 1), use
                % \smlFragOffset[color]{KK}{5}

\begin{document}

% Make it so ./mkWeb works correctly
\ifweb
    \renewcommand{\pause}{}
\fi

\setbeamertemplate{itemize items}[circle]

% SOLID COLOR TITLE (see SETTINGS.sty)
{
\begin{frame}[plain]
    \colorlambdatrue
    \titlepage
\end{frame}
}

\begin{frame}[fragile]
  \frametitle{Lesson Plan}

  \tableofcontents
\end{frame}

\begin{frame}[fragile]
  \frametitle{Last time}

  In the last lecture, we learned about some of the more complex types in 
  SML, like tuples and functions. 

  \vspace{\fill}
  
  We learned about how to declare variables and functions. We learned that declaring
  a variable \textbf{binds} it, which is different than assignment. 
  
  \vspace{\fill}

  We also learned how we can mathematically prove things 
  about SML code using a notion of \textbf{extensional equivalence}. 
\end{frame}

\sectionSlide{1}{Induction and Recursion}

\begin{frame}[fragile]
  \frametitle{Recursion}

  In the first lecture, we mentioned how we will be focusing on
  \term{recursion} in this class.

  \vspace{\fill}

  \defBox{\, We call a function \term{recursive} if it calls itself. 
  More generally, something which refers to itself is recursive.}

  \vspace{5pt}

  \exBox{ This sentence is recursive. }

  \vspace{\fill}

  SML doesn't have for loops, so recursion will be our preferred method
  for repeating some computation. We will find that, in conjunction with
  purity and binding, this will greatly help us reason about our code. 
\end{frame}

\begin{frame}[fragile]
  \frametitle{A Recursive Example}

  In the last lecture, we saw this example of a recursive function:

  \vspace{\fill}

  \begin{codeblock}
    (* REQUIRES: n >= 0 *)
    (* ENSURES: fact n reduces to the factorial of n *)
    fun fact (n : int) : int = 
      if n = 0 then 
        1
      else 
        n * fact (n - 1) 
  \end{codeblock}

  \vspace{\fill}

  How do we know that it works, though?
\end{frame}

\begin{frame}[fragile]
  \frametitle{Recursion, Mathematically}

  While recursion is a concept which is usually taught in programming classes, 
  it's just as much a mathematical concept.

  \vspace{\fill}

  The factorial function is usually specified as: 

  \vspace{5pt}

  \begin{equation}
    fact(n) = 
    \left\{
        \begin{array}{lr}
            1, & \text{if } n = 0\\
            n * fact(n - 1), & \text{if } n > 0
        \end{array}
    \right\}
  \end{equation}

  \vspace{5pt}

  which references the factorial function in its own definition. In other words, it's 
  recursive.

  \vspace{\fill}

  We see that the SML code for the factorial function closely resembles the mathematical
  definition. 
\end{frame}

\begin{frame}[fragile]
  \frametitle{The Recursive Formula}

  There is a simple four-step formula to writing any recursive function:

  \vspace{5pt}

  \begin{itemize}
    \item Identify and write the \term{base case},
    \item Identify the recursive case, 
    \item Assume that the function \textbf{already works} on a "smaller" input,
    \item Write the recursive case under that assumption.
  \end{itemize}

  \vspace{\fill}

  \defBox{\, We say that the \term{base case} for a recursive function is the
  branch of the function which does not recurse.}
\end{frame}

\begin{frame}[fragile]
  \frametitle{A Matter of Fact}

  Let's try to write out the factorial function using this formula. 

  \vspace{\fill}

  In this case, we identify that the base case is \code{0}, and that
  the factorial of \code{0} is \code{1}. 

  \vspace{10pt}

  The recursive case must be any other case, given that we \code{REQUIRE}
  a non-negative input. Assuming that \code{fact (n - 1)} is the factorial of
  \code{n - 1} (where \code{n - 1} is a "smaller" input), then it follows that 
  the factorial of \code{n} should just be \code{n * fact(n - 1)}.

  \vspace{\fill}

  \begin{codeblock}
    fun fact (0 : int) : int = 1 
      | fact (n : int) : int = n * fact (n - 1)
  \end{codeblock}
\end{frame}

\begin{frame}[fragile]
  \frametitle{Function Clauses}

  \begin{codeblock}
    fun fact (0 : int) : int = 1 
      | fact (n : int) : int = n * fact (n - 1)
  \end{codeblock}

  \vspace{\fill}

  This syntax we introduced in the last example demonstrates 
  \term{function clauses}, which allow you to specify the
  behavior of the function on different inputs.

  \vspace{\fill}

  When the \code{fact} function receives an input, it tries to 
  "match" it against \code{0}. If it succeeds, then it evaluates to
  \code{1}. Otherwise, it tries the remaining cases.

  \vspace{5pt}

  It's basically just a more powerful \code{if}!
\end{frame}

\begin{frame}[fragile]
  \frametitle{Inductive Proof}

  Recall:

  \vspace{\fill}

  \defBox{\, The principle of mathematical \term{induction} is a proof technique
  for theorems on the natural numbers. In particular, it has the logical form:}

  \vspace{5pt}

  \begin{center}
     $P(0) \land (\forall n. P(n) \implies P(n + 1)) \implies \forall n. P(n)$
  \end{center}

  \vspace{\fill}

  In other words: "if you can prove $P(0)$, and that any step follows from the 
  previous, then the statement is true for all numbers".
\end{frame}

\begin{frame}[fragile]
  \frametitle{The Inductive Formula}

  There is a simple four-step formula to proving any simple inductive theorem: 

  \vspace{5pt}

  \begin{itemize}
    \item Identify and write the \term{base case},
    \item Identify the recursive case, 
    \item Assume that the induction hypothesis holds, 
    \item Prove the recursive case under that assumption. 
  \end{itemize}

  \vspace{\fill}

  \defBox{\, We say that the \term{base case} for an inductive proof is the
  case of the proof which does not require an inductive hypothesis.}
\end{frame}

\begin{frame}[fragile]
  \frametitle{An Example Inductive Proof}
 
  { \small
  Let $S_n$ be the sum of the first $n$ odd natural numbers.

  \vspace{\fill}

  \thmBox{}{\, Prove that $S_n$ is $n^2$}

  \vspace{\fill}

  We proceed by mathematical induction on $n$. 

  \vspace{\fill}
  
  \bcBox{\, $n = 1$} 
  Then, $1 = 1^2$.

  \vspace{\fill}

  \ihBox{}{\, Let $k$ be arbitrary and fixed. Assume that $S_k = k^2$.}

  \isBox{}{\, Then, we want to show that $S_{k + 1} = (k + 1)^2$.}

  \vspace{\fill}

  We know the $k + 1$th odd natural number is $2k + 1$, so: 
  \begin{align*} 
    S_{k + 1} &= S_k + 2k + 1 \\
              &= k^2 + 2k + 1 \\
              &= (k + 1)^2 
  \end{align*}

  Therefore, by the principle of mathematical induction, the theorem holds 
  for all natural numbers $n$.
  }
\end{frame}

\begin{frame}[fragile]
  \frametitle{Seeing Double} 

  We see that \term{recursion} and \term{induction} are really just two sides of the same coin.

  \vspace{\fill}

  Recursion is about using the answers to sub-problems to solve a bigger one.

  \vspace{\fill}

  Induction is about using the inductive assumption to prove the $n + 1$th case.

  \vspace{\fill}

  The key thing to remember is the \term{recursive leap of faith}, which entails
  assuming that the function already works on a smaller input.
\end{frame}

\sectionSlide{2}{Patterns and Constructors}

\begin{frame}[fragile]
  \frametitle{Pattern Matching}

  Function clauses demonstrate a more general concept called 
  \term{pattern matching}.

  \vspace{\fill}

  \defBox{\, A \term{pattern} is a way to describe the \textbf{form} of a value. A
  value can either match to a pattern, or not. }

  \vspace{5pt}

  The goal is that patterns should be used to describe the values that you are interested
  in.

  \vspace{\fill}

  Here are some examples of patterns:
  \begin{itemize}
    \item variables, such as \code{x}, \code{y}, \code{z}. These produce a binding on a successful match.
    \item the \term{wildcard} pattern, \code{_}. This matches \textbf{all} values, and 
    produces no binding.
    \item constants, such as \code{1}, \code{"hi"}, and \code{true}
    \item tuples, where each entry is itself a pattern. For instance, \code{(_, x, 2)} is 
    a valid pattern.
  \end{itemize}
\end{frame}

\begin{frame}[fragile]
  \frametitle{Pattern Matching and Function Clauses}

  We write function clauses in the general form as:
  
  \begin{codeblock}
    fun f (<pat1> : <ty1>) : <ty2> = <expr1> 
      | f (<pat2> : <ty1>) : <ty2> = <expr2>
      ...
  \end{codeblock}

  \vspace{\fill}

  \code{val} bindings also use patterns!

  \begin{codeblock}
    val <pat> : <ty> = <expr>
  \end{codeblock}
\end{frame}

\begin{frame}[fragile]
  \frametitle{Case Expressions}

  An alternative to matching on an input in a function clause is to
  use a \textit{case expression}.

  \begin{codeblock}
    case <expr> of
      <pat1> => <expr1>
    | <pat2> => <expr2>
    ...
    | <patn> => <exprn> 
  \end{codeblock}

  \vspace{\fill}

  \noteBox{\, The first "arm" of a case expression has no bar! }
\end{frame}

\begin{frame}[fragile]
  \frametitle{Typing Case Expressions}

  Case expressions also follow typing rules!

  \vspace{\fill}

  If we have
  \begin{codeblock}
    case <expr> of
      <pat1> => <expr1>
    | <pat2> => <expr2>
    ...
    | <patn> => <exprn> 
  \end{codeblock}
  then we require that, for some \code{t}, \code{expr1 : t}, \code{expr2 : t}, ..., \code{exprn : t}. 

  \vspace{\fill}

  Why? Because otherwise \textit{the type of an expression could change during execution}.
\end{frame}

\begin{frame}[fragile]
  \frametitle{Typing Gone Wrong}

  Consider the following example of ill-typed code:

  \begin{codeblock}
    case x of
      0 => "15" 
    | _ => 150 
  \end{codeblock}

  \vspace{\fill}

  Then, the type of the value the expression would reduce to is unclear! It could
  be either \code{int} or \code{string}.   

  \vspace{\fill}

  As such, this program will be rejected as \textbf{ill-typed}.
\end{frame}

\begin{frame}[fragile]
  \frametitle{Case Expressions}
  So for instance, we could rewrite the \code{fact} function as:

  \begin{codeblock}
    fun fact (n : int) : int = 
      case n of
        0 => 1
      | _ => n * fact (n - 1)
  \end{codeblock}

  \vspace{\fill}

  \noteBox{\, We could have written this with \code{n} instead of the wildcard pattern,
  which would have shadowed the original \code{n} with a new binding with
  exactly the same value.}

  \vspace{\fill}

  It's still an expression, though, so it's totally OK to write something like
  \begin{codeblock}
    (case 2 of
      2 => 3
    | _ => 5) + 1 
  \end{codeblock}
\end{frame}

\begin{frame}[fragile]
  \frametitle{Lists}

  To write more interesting functions, we will also introduce \term{lists}.

  \vspace{\fill}

  \defBox{\, For any type \code{t}, there is a type \code{t list}, which
  describes an ordered collection of 0 or more elements of type \code{t}.}

  \vspace{\fill}

  Here are some examples of lists:
  \begin{itemize}
    \item \code{[1, 2, 3] : int list}
    \item \code{["hi", "there" : string list]}
    \item \code{[] : int list}
    \item \code{[] : bool list}
    \item \code{[[1, 2]] : int list list}
  \end{itemize}

  \vspace{\fill}
  
  What if we want to add to an existing list, though? We can also construct lists out of other lists, using a \term{constructor}!
\end{frame}

\begin{frame}[fragile]
  \frametitle{List Constructors}

  A list is characterized by two \term{constructors}, which are used to construct
  values of some list type. These constructors are 
  \begin{itemize}
    \item \code{[]}\footnotemark, which is the empty list of type \code{t list}
    \item \code{::} (pronounced "cons"), an infix operator such that 
    \code{x :: xs : t list} if \code{x : t} and \code{xs : t list}.
  \end{itemize}

  \vspace{\fill}

  While the bracket notation is simple, it's actually just syntactic sugar! The 
  list \code{[1, 2, 3]} is really just \code{1::2::3::[]}

  \vspace{\fill}

  \noteBox{ The \code{::} operator is \textit{right-associative}, meaning that
  \code{1::2::3::[]} is implicitly understood to be \code{1::(2::(3::[]))}}

  \footnotetext[1]{If you're viewing this presentation online, after the lecture, it's
  important to me that you know this is pronounced "nil". Hard to convey digitally.}
\end{frame}


\begin{frame}[fragile]
  \frametitle{List Constructors}

  Constructors are patterns too! 

  \vspace{\fill}

  This means that we can \code{case} upon a list to figure out what kind of 
  \term{variant} it is. For instance, we can write a function to check whether
  a list is empty as:
  
  \begin{codeblock}
    fun isEmpty ([] : int list) : bool = true
      | isEmpty (x::xs) = false
  \end{codeblock}

  \vspace{\fill}

  We might say that these constructors are all that characterize a list. A list 
  \textit{must be} either empty (\code{[]}) or have a first element, and the rest of 
  the list (\code{x::xs}). Pattern matching just lets us handle either case.
\end{frame}

\begin{frame}[fragile]
  \frametitle{List Constructors}

  Let's apply the recursion formula on a simple length function on lists.

  \vspace{\fill}

  The base case must be the empty list \code{[]}, in which case we just return \code{0}. 

  \vspace{10pt}

  In the recursive case \code{x::xs}, we also assume that \code{length} works on a smaller
  input. 

  \vspace{5pt}
  
  In this case, \code{xs} happens to be exactly a smaller input than the entire 
  list \code{x::xs}, so we assume \code{length xs} works, and we get: 

  \vspace{\fill}

  \begin{codeblock}
    fun length ([] : int list) : int = 0 
      | length (x::xs) = 1 + length xs
  \end{codeblock}

\end{frame}

\sectionSlide{3}{Case Study: The Division Algorithm}

\begin{frame}[fragile]
  \frametitle{A \code{case} Study}

  Let's use our knowledge of patterns, induction, and recursion to implement 
  a function. Specifically, we will be interested in the \code{divmod} function.

  \vspace{\fill}

  \begin{codeblock}
    (* divmod : int * int -> int * int
       REQUIRES: d is not 0 
       ENSURES: divmod (n, d) evaluates to (q, r), where 
       q * d + r = n 
     *)
    fun divmod (n : int, d : int) : int * int = 
      (* to be implemented *)
  \end{codeblock}

  The \code{divmod} function computes the quotient and remainder of a numerator
  and denominator via repeated subtraction.

  \vspace{\fill}

  In math, we want $q, r$ such that $qd + r = n$, for a prospective $\frac{n}{d}$.
\end{frame}

\begin{frame}[fragile]
  \frametitle{Establishing a Base}

  First step in our recursion formula is figuring out the base case. 

  \vspace{\fill}

  But what should that be? We could case on the value of \code{n} or \code{d},
  but \code{d} was already specified to be nonzero, and it probably won't help
  us to know what the divisor specifically is. 

  \vspace{\fill}

  Another way of putting it is: what is the thing that is being inducted on?

  \vspace{\fill}

  In this case, our goal will be to decrease the value of \code{n}, and through 
  recursion obtain our answer, so our induction focus will be \code{n}.
\end{frame}

\begin{frame}[fragile]
  \frametitle{Establishing a Base}

  We specified what a base case is in the context of recursion, earlier, as 
  the \textbf{branch of the function which does not recurse.}

  \vspace{\fill}

  So, put another way, \textbf{when do we not need to make a recursive call}?

  \vspace{5pt}

  This occurs when the answer is obvious. If the numerator is less than the
  denominator, obviously there is no need to decrease it and try again.

  \vspace{\fill}

  The base case is when \code{n < d}.

  \vspace{\fill}
  \begin{codeblock}
    fun divmod (n : int, d : int) : int * int = 
      if n < d then
        (0, d)
      else
        (* to be implemented *)
  \end{codeblock}
\end{frame}

\begin{frame}[fragile]
  \frametitle{The Recursive Call}

  Recall that our "induction hypothesis" will be to assume that the \code{divmod}
  function satisfies its specification on smaller values of \code{n}.

  \vspace{\fill}

  We will, in particular, try to make the recursive call on the smaller numerator 
  of \code{n - d}. But how should we use the results of the recursive call?

  \vspace{\fill}

  We will first explore \code{let} expressions, which allow us to produce bindings
  within an expression.
\end{frame}

\begin{frame}[fragile]
  \frametitle{Let Expressions}

  {\small
  The syntax for binding a variable for use within another expression is as follows:

  \vspace{\fill}

  \begin{codeblock}
    let 
      <declarations>
    in 
      <expr>
    end
  \end{codeblock}

  where the declarations are much the same as the \code{val} and \code{fun} declarations
  that we saw earlier. This means there can be multiple of them!

  \vspace{\fill}

  \noteBox{\, Similarly to what we saw earlier with \code{case} expressions, \code{let}
  expressions are expressions in their own right, so it is valid to write}
  \begin{codeblock}
    (let 
      val x = 2
    in 
      x + x
    end) + 3
  \end{codeblock}
  }
\end{frame}

\begin{frame}[fragile]
  \frametitle{Lexical Scoping}

  SML is a \term{lexically scoped} language, which means that the scope in which
  variable bindings are visible depends on its context in the text of the program.
  
  \vspace{\fill}

  This means, for instance, that anything declared within a \code{let} expression
  is only potentially visible within the expression included with it. This means,
  for instance, that this is invalid code: 

  \begin{codeblock}
    (* INVALID CODE! What is `y`? *)
    val x = 
      (let 
        val y = 3
      in 
        4 
      end) + y 
  \end{codeblock}
\end{frame}

\begin{frame}[fragile]
  \frametitle{The Recursive Call}

  We will finish off our function by assuming that \code{divmod (n - d, d)}
  returns a correct solution. That is, it returns the quotient and remainder 
  of \code{n - d} by \code{d}. 

  \vspace{\fill}

  In which case, the remainder is the same, but \code{d} goes into \code{n}
  once more than it does into \code{n - d}. Thus our finished function is:

  \begin{codeblock}
    fun divmod (n : int, d : int) : int * int = 
      if n < d then
        (0, d)
      else
        let 
          val (q, r) = divmod (n - d, d) 
        in
          (q + 1, r)
        end
  \end{codeblock}
\end{frame}

\begin{frame}[plain]
	\begin{center} Thank you! \end{center}
\end{frame}


\end{document}

