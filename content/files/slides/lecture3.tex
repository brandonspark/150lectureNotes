% Jacob Neumann

% DOCUMENT CLASS AND PACKAGE USE
    \documentclass[aspectratio=169]{beamer}
 
    % Establish the colorlambda boolean, to control whether the lambda is solid color (true), or the same as the picture (false)
    \newif\ifcolorlambda
    \colorlambdafalse % DEFAULT: false
    
    % Use auxcolor for syntax highlighting
    \newif\ifuseaux
    \useauxfalse % DEFAULT: false
   
    % Color settings
    \useauxtrue
    
    \newcommand{\auxColor}{0f8a73}     % the color of note boxes and stuff
    \newcommand{\presentColor}{002B5B} % the primary color of the slide borders
    \newcommand{\bgColor}{a6c4e3}      % the color of the background of the slide
    \newcommand{\darkBg}{8b98ad}
    \newcommand{\lambdaColor}{\auxColor}
  
    \colorlambdatrue

    \usepackage{comment} % comment blocks
    \usepackage{soul} % strikethrough
    \usepackage{listings} % code
    \usepackage{makecell}

    \setbeamertemplate{itemize items}[circle]
    % \setbeameroption{show notes on second screen=right}

    \usepackage{lectureSlides}
    %%%%%%%%%%%%%%%%%%%%%%%%%%%%%%%%%%%%%%%%%| <----- Don't make the title any longer than this
    \title{Induction and Recursion} % TODO
    \subtitle{Iterating to infinity} % TODO
    \date{23 May 2023} % TODO
    \author{Brandon Wu} % TODO

    \graphicspath{ {./img/} }
    % DONT FORGET TO PUT [fragile] on frames with codeblocks, specs, etc.
        %\begin{frame}[fragile]
        %\begin{codeblock}
        %fun fact 0 = 1
        %  | fact n = n * fact(n-1)
        %\end{codeblock}
        %\end{frame}

    % INCLUDING codefile:
        % 1. In some file under code/NN (where NN is the lecture id num), include:
    %       (* FRAGMENT KK *)
    %           <CONTENT>
    %       (* END KK *)
    
    %    Remember to not put anything on the same line as the FRAGMENT or END comment, as that won't be included. KK here is some (not-zero-padded) integer. Note that you MUST have fragments 0,1,...,KK-1 defined in this manner in order for fragment KK to be properly extracted.
        %  2. On the slide where you want code fragment K
                % \smlFrag[color]{KK}
        %     where 'color' is some color string (defaults to 'white'. Don't use presentColor.
    %  3. If you want to offset the line numbers (e.g. have them start at line 5 instead of 1), use
                % \smlFragOffset[color]{KK}{5}

\begin{document}

% Make it so ./mkWeb works correctly
\ifweb
    \renewcommand{\pause}{}
\fi

\setbeamertemplate{itemize items}[circle]

% SOLID COLOR TITLE (see SETTINGS.sty)
{
\begin{frame}[plain]
    \colorlambdatrue
    \titlepage
\end{frame}
}

\begin{frame}[fragile]
  \frametitle{Lesson Plan}

  \tableofcontents
\end{frame}

\begin{frame}[fragile]
  \frametitle{Last time}

  In the last lecture, we learned about some of the more complex types in 
  SML, like tuples and functions. 

  \vspace{\fill}
  
  We learned about how to declare variables and functions. We learned that declaring
  a variable \textbf{binds} it, which is different than assignment. 
  
  \vspace{\fill}

  We also learned how we can mathematically prove things 
  about SML code using a notion of \textbf{extensional equivalence}. 
\end{frame}

\begin{frame}[plain]
  \begin{center}
    \huge \color{presentColor} \intermedium{0 - Equivalence} \\ 
    \vspace{10pt}
    \large \color{presentColor} \intermedium{This material was meant to be covered during the last
    lecture, but was moved here due to time constraints. }
  \end{center}
\end{frame}

\begin{frame}[fragile]
  \frametitle{On the Equivalence of Values}

  Up until this point, we've said that a value is a "final answer". You should be able
  to tell if values are the same just by looking at them, like with \code{2} and \code{2},
  or with \code{(3, 4)} and \code{(3, 4)}.

  \vspace{\fill}

  But what about functions?

  \pause
  \vspace{5pt}

  Are \code{fn (x : int) => x + x} and \code{fn (x : int) => 2 * x} extensionally equivalent?

  \pause
  \vspace{\fill}

  It's now hard to tell, because these lambda expressions are values, but it's not obvious 
  if they are extensionally equivalent or not.
\end{frame}

\begin{frame}[fragile]
  \frametitle{Extensional Equivalence for Functions}

  Functions necessitate their own rule for extensional equivalence.

  \pause
  \vspace{\fill}

  \defBox{\, Two functions \code{f : t1 -> t2} and \code{g : t1 -> t2} are \term{extensionally equivalent}
  if for all values \code{x : t1}, \code{f x} $\eeq$ \code{g x}}.

  \pause
  \vspace{5pt}

  In other words, two functions are equivalent if \textbf{for all inputs, they give equivalent outputs}.

  \pause
  \vspace{\fill}
  
  \noteBox{\, We specified that \code{f x} $\eeq$ \code{g x}, not that \code{f x} and \code{g x}
  reduce to the same value. Why?}
\end{frame}

\begin{frame}[fragile]
  \frametitle{Totality}

  The reason why is simple - some functions do not have defined outputs!

  \pause
  \vspace{\fill}

  We described functions in SML as mathematical functions, which is true, but not exactly in the
  same way as some mathematical functions, such as $+$, or $\sin$.

  \pause
  \vspace{5pt}

  SML functions can be \term{partial}, that is, undefined on some input. There may be \code{f}
  and \code{x} such that there is no \code{v} where \code{f x} $\hookrightarrow$ \code{v}.  

  \pause
  \vspace{\fill}

  \defBox{\, We say that \code{f : t1 -> t2} is \term{total} if for all values \code{v : t1},
  there is a value \code{v' : t2} such that \code{f v} $\hookrightarrow$ \code{v'}.}

  \vspace{5pt}

  For example, the SML functions \code{+}, \code{not}, and \code{^} are all total.
\end{frame}

\begin{frame}[fragile]
  \frametitle{Specification}

  Concepts like purity, extensional equivalence, and totality are all just means to
  an end, which is being able to \textbf{specify} the behavior of code.

  \pause
  \vspace{\fill}

  In particular, for functions, we are interested in writing \textbf{descriptive} code,
  that accurately reflects the function's behavior.
  
  \pause
  \vspace{5pt}

  To that end, it is helpful to write alongside a function the conditions which 
  must be true \textbf{prior} to calling the function, and must be true \textbf{after}
  calling the function.

  \pause
  \vspace{\fill}

  \begin{codeblock}
    (* REQUIRES: x is not 0 *)
    (* ENSURES: divide x is total *)
    fun divide (x : int) : int = 2 div x 
  \end{codeblock}

  \ptmt
\end{frame}

\begin{frame}[fragile]
  \frametitle{The Specification of Specification}

  In this class, our way of writing specifications will follow the 
  five-step methodology.

  \pause
  \vspace{\fill}

  There are five components to this methodology, shockingly:
  \pause
  \begin{enumerate}
    \item the function's type \pause
    \item the \code{REQUIRES} clause (preconditions) \pause
    \item the \code{ENSURES} clause (postconditions) \pause
    \item the function's definition \pause
    \item test cases! \pause
  \end{enumerate}

  \vspace{\fill}

  We will frequently annotate our functions this way this semester.
\end{frame}

\begin{frame}[fragile]
  \frametitle{Specification, pt. 2}

  We use comments in the form of \code{REQUIRES} and \code{ENSURES} to describe 
  what must be true of the inputs the function receives, and what then is guaranteed 
  to hold of the function's behavior. 

  \pause
  \vspace{\fill}

  It is often unrealistic to have a function which has meaning on \textit{every} possible 
  input (like \code{div}, or square root, or logarithm). The \code{REQUIRES} helps to
  describe the range of "relevant inputs", and the \code{ENSURES} helps to describe
  what the function does.

  \pause
  \vspace{\fill}

  These "contracts" pop up in real code all the time:
  \pause
  \begin{itemize}
    \item this function must be called with only safe values, \pause
    \item this library can only be invoked in single-threaded programs, \pause
    \item this API is not guaranteed to work with non-ASCII characters \pause 
  \end{itemize}

  \vspace{\fill}

  It is important that code is documented so users and maintainers know pertinent
  information about it! 
\end{frame}

\begin{frame}[fragile]
  \frametitle{Specification, pt. 3}

  The last piece of the formula is writing tests.\footnotemark

  \pause
  \vspace{\fill}

  In this class, we will use a 150-specific testing framework, where
  we write something like:
  
  \begin{codeblock}
    val () = Test.int ("test1", 2 + 2, 1 + 3)
  \end{codeblock}
  which will raise an exception if the second and third parts of the
  tuple do not evaluate to the same value.

  \vspace{\fill}

  \noteBox{\, I realize writing tests sucks. So does life, sometimes.}

  \footnotetext[1]{There's a lot I could say here about the importance
  of writing tests. Test-driven development is a real thing, and especially
  in imperative languages, it's very important to have a solid backbone
  of tests to make sure you don't accidentally regress behavior when 
  introducing a small change. However, this is a class on functional
  programming, and not software engineering, so I'll decline to comment
  for now.\footnotemark}

  \footnotetext[2]{This is a really long footnote.}
\end{frame}

\begin{frame}[fragile]
  \frametitle{Back to Fact}

  Now that we have all these tools for mathematically analyzing
  code, let's look at a specific example. Let's take the factorial 
  function that we wrote earlier.

  \vspace{\fill}

  \begin{codeblock}
    fun fact (0 : int) : int = 1
      | fact n = n * fact (n - 1)
  \end{codeblock}

  \pause
  \vspace{\fill}

  \customBox{Question}{\, Is \code{fact} total?}
\end{frame}

\begin{frame}[fragile]
  \frametitle{A Specific Specification}

  \textbf{Answer:} It is not.

  \pause
  \vspace{\fill}

  The \code{fact} function loops forever on a negative input.

  \pause
  \vspace{\fill}

  But for our intents and purposes, we don't really
  care about what \code{fact} does on negative inputs, anyways. So 
  let's restrict our domain of interest to strictly non-negative
  numbers.

  \pause
  \vspace{\fill}

  \begin{codeblock}
    (* fact : int -> int *)
    (* REQUIRES: n >= 0 *)
    (* ENSURES: fact n evaluates to the nth factorial *)
    fun fact (0 : int) : int = 1
      | fact n = n * fact (n - 1)
  \end{codeblock}

  \pause
  \customBox{Lesson}{\, Oftentimes, we are interested in only a 
  subset of the domain of a function, and we only get to
  make interesting claims about its behavior when we restrict
  our attention to it.}
\end{frame}

\begin{frame}[fragile]
  \frametitle{Remark: Slide Notation}

  Instead of putting specifications in comments above the code, when
  writing code within these slides, they will often be conveyed via
  specification blocks.

  For instance, for the \code{fact} function, I would instead write:
  \spec
    {fact}
    {int -> int}
    {\code{n >= 0}}
    {\code{fact n} evaluates to $n!$}
\end{frame}

\sectionSlide{1}{Recursion}

\begin{frame}[fragile]
  \frametitle{Recursion}

  In the first lecture, we mentioned how we will be focusing on
  \term{recursion} in this class.

  \pause
  \vspace{\fill}

  \defBox{\, We call a function \term{recursive} if it calls itself. 
  More generally, something which refers to itself is recursive.}

  \pause
  \vspace{5pt}

  \exBox{ This sentence is recursive. }

  \pause
  \vspace{\fill}

  SML doesn't have for loops, so recursion will be our preferred method
  for repeating some computation. We will find that, in conjunction with
  purity and binding, this will greatly help us reason about our code. 
\end{frame}

\begin{frame}[fragile]
  \frametitle{Recursion, Mathematically}

  While recursion is a concept which is usually taught in programming classes, 
  it's just as much a mathematical concept.

  \pause
  \vspace{\fill}

  The factorial function is usually specified as: 

  \vspace{5pt}

  \begin{equation}
    \text{fact}(n) = 
    \left\{
        \begin{array}{lr}
            1, & \text{if } n = 0\\
            n * \text{fact}(n - 1), & \text{if } n > 0
        \end{array}
    \right\}
  \end{equation}

  \vspace{5pt}

  which references the factorial function in its own definition. In other words, it's 
  recursive.

  \pause
  \vspace{\fill}

  We see that the SML code for the factorial function closely resembles the mathematical
  definition. 

  \rprs
\end{frame}

\begin{frame}[fragile]
  \frametitle{The Recursive Formula}

  There is a simple four-step formula to writing any recursive function:

  \pause
  \vspace{5pt}

  \begin{itemize}
    \item Identify and write the \term{base case}, \pause
    \item Identify the recursive case, \pause 
    \item Assume that the function \textbf{already works} on a "smaller" input, \pause
    \item Write the recursive case under that assumption.
  \end{itemize}

  \pause
  \vspace{\fill}

  \defBox{\, We say that the \term{base case} for a recursive function is the
  branch of the function which does not recurse.}
\end{frame}

\begin{frame}[fragile]
  \frametitle{A \code{pow} Function}

  Let's try to write out a different function using this formula. 

  \spec
    {pow}
    {int * int -> int}
    {\code{k >= 0}}
    {\code{pow (n, k)} evaluates to $n^k$}

  \pause
  \vspace{\fill}

  We know that raising any number to the power of $0$ will produce $1$, so 
  our base case will simply be the case where the second value we receive is 
  \code{0}.

  \pause
  \begin{codeblock}
    fun pow (n : int, 0 : int) : int = 1 
      | (* ... *)
  \end{codeblock}

  \pause
  \vspace{\fill}

  Since our base case is casing upon the value of the exponent, this is a good
  indication that the exponent is going to be the value that we recurse on. We
  call this the \term{variable of recurrence}. 
\end{frame}

\begin{frame}[fragile]
  \frametitle{The \code{pow} Function: Recursive Case}

  What should we write for our recursive case? We want to be able to compute
  an arbitrary \code{pow(n, k)}, for non-zero \code{k}.

  \pause
  \vspace{\fill}

  Thinking recursively, we want to be able to assume that our function already
  works on a smaller input. Since our variable of recurrence is \code{k}, then
  we can assume that \code{pow (n, k - 1)} already evaluates to the
  $(k-1)$th power of $n$.
  
  \pause
  \vspace{\fill}

  Given $n^{k - 1}$, we can recover $n^k$ by just multiplying by $n$, so we get:

  \begin{codeblock}
    fun pow (n : int, 0 : int) : int = 1 
      | pow (n, k) = n * pow (n, k - 1) 
  \end{codeblock}

  \pause
  \vspace{\fill}

  This is the magic of recursive thinking!
\end{frame}

\begin{frame}[fragile]
  \frametitle{The Recursive Formula, Revisited}

  There is a simple four-step formula to writing any recursive function:

  \begin{itemize}
    \item Identify and write the \term{base case}, \
    \item Identify the recursive case, 
    \item Assume that the function \textbf{already works} on a "smaller" input,
    \item Write the recursive case under that assumption.
  \end{itemize}

  \vspace{\fill}

  \defBox{\, We say that the \term{base case} for a recursive function is the
  branch of the function which does not recurse.}

  \pause
  \vspace{\fill}

  \noteBox{}{\, But wait, this formula looks similar to something we've already seen...}

  Let's give the perspective a switch...
\end{frame}

\begin{frame}[fragile]
  \frametitle{Recursion, Unmasked}

  \begin{center}
    \includegraphics[scale=0.3]{mask_reveal}
  \end{center}
\end{frame}

\begin{frame}[fragile]
  \frametitle{The Recursive Formula, Unmasked}

  There is a simple four-step formula to proving any simple inductive theorem: 

  \vspace{5pt}

  \begin{itemize}
    \item Identify and write the \term{base case},
    \item Identify the recursive case, 
    \item Assume that the induction hypothesis holds, 
    \item Prove the recursive case under that assumption. 
  \end{itemize}

  \vspace{\fill}

  \defBox{\, We say that the \term{base case} for an inductive proof is the
  case of the proof which does not require an inductive hypothesis.}
  But wait, this formula looks similar to something we've already seen...

  \pause
  \vspace{\fill}

  \noteBox{}{\, Perfect.}
\end{frame}

\sectionSlide{2}{Induction}

\begin{frame}[fragile]
  \frametitle{Inductive Proof}

  Let's take a brief excursion into mathematical induction.

  \pause
  \vspace{\fill}

  \defBox{\, The principle of mathematical \term{induction} (or \term{simple induction}) is a proof technique
  for theorems on the natural numbers. In particular, it has the logical form:}

  \pause
  \vspace{5pt}

  \begin{center}
     $P(0) \land (\forall n. P(n) \implies P(n + 1)) \implies \forall n. P(n)$
  \end{center}

  \pause
  \vspace{\fill}

  In other words: "if you can prove $P(0)$, and that any step follows from the 
  previous, then the statement is true for all numbers".
\end{frame}

\begin{frame}[fragile]
  \frametitle{The Inductive Formula}

  There is a simple four-step formula to proving any simple inductive theorem: 

  \pause
  \vspace{5pt}

  \begin{itemize}
    \item Identify and write the \term{base case}, \pause
    \item Identify the recursive case, \pause
    \item Assume that the induction hypothesis holds, \pause 
    \item Prove the recursive case under that assumption. \pause 
  \end{itemize}

  \pause
  \vspace{\fill}

  \defBox{\, We say that the \term{base case} for an inductive proof is the
  case(s) of the proof which do not require an inductive hypothesis.}
\end{frame}

\begin{frame}[fragile]
  \frametitle{An Example Inductive Proof}
 
  { \small
  Let $S_n$ be the sum of the first $n$ odd natural numbers.

  \pause
  \vspace{\fill}

  \thmBox{}{\, Prove that $S_n$ is $n^2$}

  \pause
  \vspace{\fill}

  We proceed by mathematical induction on $n$. 

  \vspace{\fill}
  
  \bcBox{\, $n = 1$} 
  Then, $1 = 1^2$.

  \vspace{\fill}

  \ihBox{}{\, Let $k$ be arbitrary and fixed. Assume that $S_k = k^2$.}

  \isBox{}{\, Then, we want to show that $S_{k + 1} = (k + 1)^2$.}

  \vspace{\fill}

  We know the $k + 1$th odd natural number is $2k + 1$, so: 
  \begin{align*} 
    S_{k + 1} &= S_k + 2k + 1 \\
              &= k^2 + 2k + 1 \\
              &= (k + 1)^2 
  \end{align*}

  Therefore, by the principle of mathematical induction, the theorem holds 
  for all natural numbers $n$.
  }
\end{frame}

\begin{frame}[fragile]
  \frametitle{Seeing Double} 

  We see that \term{recursion} and \term{induction} are really just two sides of the same coin.

  \pause
  \vspace{\fill}

  Recursion is about \textbf{using the answers to sub-problems to solve a bigger one.}

  \pause
  \vspace{\fill}

  Induction is about \textbf{using the inductive assumption to prove the $n + 1$th case.}

  \pause
  \vspace{\fill}

  The key thing to remember is the \term{recursive leap of faith}, which entails
  assuming that the function already works on a smaller input.

  \ptmt
\end{frame}

\begin{frame}[fragile]
  \frametitle{Writing versus Verifying} 

  We have seen now that induction can be used as a tool which can help us to
  \textit{write} recursive functions.

  \pause
  \vspace{\fill}

  But not all recursive functions are obvious in their logic! We said earlier
  that we are not just interested in writing code, we are interested in writing
  \textit{correct} code.

  \pause
  \vspace{\fill}

  The best way to be certain of a function's correctness is to \textit{prove}
  that it is correct. We will see now that induction helps us with recursion
  in another way, because we can prove that SML functions are correct using
  induction.

\end{frame}

\sectionSlide{3}{Proving Correctness}

\begin{frame}[fragile]
  \frametitle{Recursive Correctness}

  Let's look at the \code{pow} function we just wrote.

  \vspace{\fill}

  \begin{codeblock}
    fun pow (n : int, 0 : int) : int = 1 
      | pow (n, k) = n * pow (n, k - 1) 
  \end{codeblock}

  \pause
  \vspace{\fill}

  Let's try to write an inductive proof of correctness.

  \pause
  \vspace{\fill}

  When we do induction, we usually want a quantity to induct on, however.
  What should we pick for this function?

  \pause
  \vspace{\fill}

  In this case, the \term{variable of recurrence} is usually a prime
  candidate for our induction!
\end{frame}

\begin{frame}[fragile]
  \frametitle{Proving \code{pow} Correct}

  \thmBox{}{\, Prove that \code{pow (n, k)} $\hookrightarrow$ $n^k$, for all 
  \code{k >= 0}}

  \pause
  \vspace{\fill}

  We proceed by mathematical induction on \code{k}. 

  \pause
  \vspace{\fill}
  
  \bcBox{\, \code{k = 0}} 
  \pause
  \begin{align*}
       \code{pow(n, 0)} &\stepsTo \code{1} &\tag{clause 1 of \code{pow}}
  \end{align*}

  \pause
  \vspace{\fill}

  \ihBox{}{\, Assume that \code{pow (n, k)} $\hookrightarrow$ $n^k$, for some \code{k}} 

  \pause

  \isBox{}{\, Then, we want to show that \code{pow (n, k + 1)} $\hookrightarrow$ $n^{k + 1}$}
  \pause
  \begin{align*}
       \code{pow (n, k + 1)} &\stepsTo \code{n * pow (n, k)} &\tag{clause 2 of \code{pow}} \\
       &\stepsTo \code{n * $\,\,\, n^k$} &\tag{induction hypothesis} \\
       &\stepsTo n^{k + 1} &\tag{math}
  \end{align*}
\end{frame}

\begin{frame}[fragile]
  \frametitle{Proving a Function Correct}

  We see that in an inductive proof on a recursive function, the isomorphism 
  between recursion and induction is made even more clear.

  \vspace{\fill}

  \begin{center}
    \begin{tabular}{c || c} 
      \hline & \\[-1.5ex]
      Program & Proof \\ [2pt]
      \hline & \\[-1.5ex]
      \term{Base case} (\code{pow (n, 0)}) & \term{Base case} ($n^0$) \\ [2pt]
      \hline & \\[-1.5ex]
      \term{Recursive call} (\code{pow (n, k)}) & \term{Inductive hypothesis} ($n^k$) \\ [2pt]
      \hline & \\[-1.5ex]
      \term{Variable of recurrence} (\code{k}) & \term{Induction variable} ($k$) \\ [2pt]
      \hline
    \end{tabular}
  \end{center}

  \ptmt
\end{frame}

\quizBreak{FARADAY}

\sectionSlide{4}{More Language Features}

\begin{frame}[fragile]
  \frametitle{Case Expressions}

  An alternative to matching on an input in a function clause is to
  use a \textit{case expression}.

  \pause

  \begin{codeblock}
    case <expr> of
      <pat1> => <expr1>
    | <pat2> => <expr2>
    ...
    | <patn> => <exprn> 
  \end{codeblock}

  \vspace{\fill}

  \noteBox{\, The first "arm" of a case expression has no bar! }

  \pause
  \vspace{\fill}

  \noteBox{\, \code{case} expressions follow similar typing rules as
  function clauses, where each case's expression must share the same type.} 
\end{frame}

\begin{frame}[fragile]
  \frametitle{Case Expressions}
  So for instance, we could rewrite the \code{fact} function as:

  \begin{codeblock}
    fun fact (n : int) : int = 
      case n of
        0 => 1
      | _ => n * fact (n - 1)
  \end{codeblock}

  \pause
  \vspace{\fill}

  \noteBox{\, We could have written this with \code{n} instead of the wildcard pattern,
  which would have shadowed the original \code{n} with a new binding with
  exactly the same value.}

  \pause
  \vspace{\fill}

  It's still an expression, though, so it's totally OK to write something like
  \begin{codeblock}
    (case 2 of
      2 => 3
    | _ => 5) + 1 
  \end{codeblock}
\end{frame}

\begin{frame}[fragile]
  \frametitle{Lists}

  To write more interesting functions, we will also introduce \term{lists}.

  \pause
  \vspace{\fill}

  \defBox{\, For any type \code{t}, there is a type \code{t list}, which
  describes an ordered collection of 0 or more elements of type \code{t}.}

  \pause
  \vspace{\fill}

  Here are some examples of lists:
  \begin{itemize}
    \item \code{[1, 2, 3] : int list}
    \item \code{["hi", "there"] : string list}
    \item \code{[] : int list}
    \item \code{[] : bool list}
    \item \code{[[1, 2]] : int list list}
  \end{itemize}

  \pause
  \vspace{\fill}
  
  What if we want to add to an existing list, though? We can also construct lists out of other lists, using a \term{constructor}!
\end{frame}

\begin{frame}[fragile]
  \frametitle{List Constructors}

  A list is characterized by two \term{constructors}, which are used to construct
  values of some list type. These constructors are 
  \pause
  \begin{itemize}
    \item \code{[]}\footnotemark, which is the empty list of type \code{t list} \pause
    \item \code{::} (pronounced "cons"), an infix operator such that 
    \code{x :: xs : t list} if \code{x : t} and \code{xs : t list}.
  \end{itemize}

  \pause
  \vspace{\fill}

  While the bracket notation is simple, it's actually just syntactic sugar! The 
  list \code{[1, 2, 3]} is really just \code{1::2::3::[]}

  \pause
  \vspace{\fill}

  \noteBox{ The \code{::} operator is \textit{right-associative}, meaning that
  \code{1::2::3::[]} is implicitly understood to be \code{1::(2::(3::[]))}}

  \footnotetext[3]{If you're viewing this presentation online, after the lecture, it's
  important to me that you know this is pronounced "nil". Hard to convey digitally.}

  \tgs
\end{frame}


\begin{frame}[fragile]
  \frametitle{List Constructors}

  Constructors are patterns too! 

  \pause
  \vspace{\fill}

  This means that we can \code{case} upon a list to figure out what kind of 
  \term{variant} it is. For instance, we can write a function to check whether
  a list is empty as:
  
  \pause

  \begin{codeblock}
    fun isEmpty (L : int list) : bool = 
      case L of
        [] => true
      | x::xs => false
  \end{codeblock}

  \pause
  \vspace{\fill}

  We might say that these constructors are all that characterize a list. A list 
  \textit{must be} either empty (\code{[]}) or have a first element, and the rest of 
  the list (\code{x::xs}). Pattern matching just lets us handle either case.
\end{frame}

\begin{frame}[fragile]
  \frametitle{List Constructors}

  Let's apply the recursion formula on a simple length function on lists.

  \pause
  \vspace{\fill}

  The base case must be the empty list \code{[]}, in which case we just return \code{0}. 

  \pause
  \vspace{10pt}

  In the recursive case \code{x::xs}, we also assume that \code{length} works on a smaller
  input. 

  \pause
  \vspace{5pt}
  
  In this case, \code{xs} happens to be exactly a smaller input than the entire 
  list \code{x::xs}, so we assume \code{length xs} works, and we get: 

  \pause
  \vspace{\fill}

  \begin{codeblock}
    fun length ([] : int list) : int = 0 
      | length (x::xs) = 1 + length xs
  \end{codeblock}
\end{frame}

\sectionSlide{5}{Case Study: Fast Exponentiation}

\begin{frame}[fragile]
  \frametitle{Fast Exponentiation}

  Suppose that we are interested in solving a slightly more involved problem.

  \pause
  \vspace{\fill}

  The \code{pow} function we just implemented is fine, but it's not necessarily
  as fast as it could be! From eyeing the function, we can see that it would take
  us $k$ many multiplications to compute the $k$th power of $n$. This is more 
  than it needs to be, because of a nice fact that we can take advantage of:

  \pause
  \vspace{\fill}

  If $k$ is odd, then $$n^k = n * \left (n^{\lfloor k/2 \rfloor} \right )^2$$

  \pause

  If $k$ is even, then $$n^k = \left (n^{k/2} \right )^2$$

  \pause
  \vspace{\fill}

  Can we turn this mathematical definition into a program?
\end{frame}

\begin{frame}[fragile]
  \frametitle{The \code{fast_pow} Function}

  Let's try it out!

  \pause
  \begin{codeblock}
    fun fast_pow (n : int, 0 : int) : int = 1
      | fast_pow (n, k) =
          if k mod 2 = 0 then 
            fast_pow (n, k div 2) * fast_pow (n, k div 2)
          else
            n * fast_pow (n, k - 1)
  \end{codeblock}

  \vspace{\fill}

  How's this look?\footnotemark

  \footnotetext[4]{I actually adjusted the definition slightly from the
  mathematical definition, because it makes the proof easier. It can be
  done either way, though.}
\end{frame}

\begin{frame}[fragile]
  \frametitle{A "Faster" Function}

  There's a problem with this function, however. It's not actually necessarily
  saving us that many multiplications, because we make two recursive calls!

  \vspace{\fill}

  We wanted to save on effort by reusing our answers from the recursive call, 
  but we see here that we are just recomputing each time!

  \pause
  \vspace{\fill}

  \code{fast_pow} is a pure function, meaning that we should be able to do some
  extensionally equivalent refactoring, using the power of referential
  transparency...

  \vspace{\fill}

  To do this, however, we need to be able to bind a variable within an
  expression. How shall we do this?
\end{frame}

\begin{frame}[fragile]
  \frametitle{Let Expressions}

  {\small
  The syntax for binding a variable for use within another expression is as follows:

  \pause
  \vspace{\fill}

  \begin{codeblock}
    let 
      <declarations>
    in 
      <expr>
    end
  \end{codeblock}

  where the declarations are much the same as the \code{val} and \code{fun} declarations
  that we saw earlier. This means there can be multiple of them!

  \pause
  \vspace{\fill}

  \noteBox{\, \code{let} expressions are expressions in their own right, 
  so it is valid to write}
  \begin{codeblock}
    (let 
      val x = 2
    in 
      x + x
    end) + 3
  \end{codeblock}
  }
\end{frame}

\begin{frame}[fragile]
  \frametitle{Lexical Scoping}

  SML is a \term{lexically scoped} language, which means that the scope in which
  variable bindings are visible depends on its context in the text of the program.
  
  \pause
  \vspace{\fill}

  This means, for instance, that anything declared within a \code{let} expression
  is only potentially visible within the expression included with it. This means,
  for instance, that this is invalid code: 

  \begin{codeblock}
    (* INVALID CODE! What is `y`? *)
    val x = 
      (let 
        val y = 3
      in 
        4 
      end) + y 
  \end{codeblock}
\end{frame}

\begin{frame}[fragile]
  \frametitle{Back to \code{fast_pow}}

  So let's modify our original \code{fast_pow}:

  \pause

  { \small
  \begin{codeblock}
    fun fast_pow (n : int, 0 : int) : int = 1
      | fast_pow (n, k) =
          if k mod 2 = 0 then 
            let
              val half_ans = fast_pow (n, k div 2)
            in
              half_ans * half_ans 
            end
          else
            n * fast_pow (n, k - 1) 
  \end{codeblock}
  }

  Now, we only ever have at maximum one recursive call to \code{fast_pow}!

  \pause
  \vspace{\fill}

  How do we know that behavior is preserved, however? We want to be able to
  prove this.
\end{frame}

\begin{frame}[fragile]
  \frametitle{Inducting on a Strange Call}

  However, we see that the recursive call is to \code{fast_pow (n, k div 2)}.

  \pause
  \vspace{\fill}
  
  This is not as straightforward as what we usually saw, with the recursive
  call being on one less than the current value.

  \pause
  \vspace{\fill}

  If the recursive call corresponds to our inductive hypothesis, then will
  a change to our recursive call change what our inductive hypothesis 
  should be?

  \pause
  \vspace{\fill}

  \noteBox{}{\, Yes.}

  \pause
  \vspace{\fill}

  Since we need to know something about $k/2$ rather than $k - 1$, we will
  proceed by \term{strong induction} instead of \term{simple induction}.
\end{frame}

\begin{frame}[fragile]
  \frametitle{Strong Induction}

  \defBox{}{\, \term{Strong induction} is a variant of inductive proof where
  instead of assuming the inductive hypothesis for just $n - 1$, the hypothesis
  is assumed for all numbers between $0$ and $n$.}

  \pause
  \vspace{\fill}

  You can think of it as, if we have gone to all the trouble of proving our
  theorem incrementally by adding larger and larger numbers to our collection,
  we should still be able to make use of all of the intermediary theorems
  we proved (i.e. $P(0), P(1), ..., P(n)$), not just the last one.

  \vspace{\fill}

  Now, let's do the proof.
\end{frame}

\begin{frame}[fragile]
  \frametitle{\code{fast_pow} Is \code{pow}}

  \thmBox{}{\, Prove that \code{fast_pow} $\eeq$ \code{pow}}

  \pause
  \vspace{\fill}

  We proceed by strong induction on $k$. 

  \pause
  \vspace{\fill}
  
  \bcBox{\, $k = 0$} 
  \pause
  \begin{align*}
       \code{fast_pow (n, 0)} &\eeq \code{1} &\tag{clause 1 of \code{fast_pow}} \\
       &\eeq \code{pow (n, 0)} &\tag{clause 1 of \code{pow}}
  \end{align*}

  \pause
  \vspace{\fill}

  \ihBox{}{\, Assume \code{fast_pow (n, k)} $\eeq$ \code{pow (n, i)}, for all \code{0 <= i < k},
  for some \code{k}} 

  \pause
  \vspace{\fill}

  \isBox{}{\, Then, we want to show that \code{fast_pow (n, k)} $\eeq$ \code{pow (n, k)}}
  But what do we do in this case?
\end{frame}

\begin{frame}[fragile]
  \frametitle{Proving through Cases}

  Here, we have a dilemma, because we have two cases, the case where \code{k} is even,
  and the case where \code{k} is odd.

  \pause
  \vspace{\fill}
  
  The way we deal with this in a proof is that we must prove that the theorem holds
  \textit{no matter which case we are in}.

  \pause
  \vspace{\fill}

  So we must prove the theorem for both cases. To do this, we will make use of a
  \term{lemma}:

  \vspace{\fill}

  \lemmaBox{1}{\, When \code{k mod 2 = 0}, then} 
  $$\code{pow (n, k div 2) * pow (n, k div 2)} \eeq \code{pow (n, k)}$$
\end{frame}

\begin{frame}[fragile]
  \frametitle{\code{fast_pow} Is \code{pow}: Inductive Step}
  
  \isBox{}{\, Then, we want to show that \code{fast_pow (n, k)} $\eeq$ \code{pow (n, k)}}

  \pause
  \vspace{\fill}

  Case 1: \code{k mod 2 = 0}
  \begin{align*}
       \code{fast_pow (n, k)} &\eeq \code{half_ans * half_ans} &\tag{clause 2 of \code{fast_pow}} \\
       &\eeq \code{fast_pow (n, k div 2) * fast_pow (n, k div 2)} &\tag{def of \code{half_ans}} \\
       &\eeq \code{pow (n, k div 2) * pow (n, k div 2)} &\tag{induction hypothesis} \\
       &\eeq \code{pow (n, k)} &\tag{lemma 1, case assumption} \\
  \end{align*}
\end{frame}

\begin{frame}[fragile]
  \frametitle{\code{fast_pow} Is \code{pow}: Inductive Step}

  Case 2: \code{k mod 2 <> 0}
  \begin{align*}
    \code{fast_pow (n, k)} &\eeq \code{n * fast_pow (n, k - 1)} &\tag{clause 2 of \code{fast_pow}} \\
    &\eeq \code{n * pow (n, k - 1)} &\tag{induction hypothesis} \\
    &\eeq \code{pow (n, k)} &\tag{clause 2 of \code{pow}} \\
  \end{align*}

  \pause

  Now, we have finished proving our theorem!
\end{frame}

\begin{frame}[fragile]
  \frametitle{Programs and Proofs}

  \begin{center}
    \begin{tabular}{c || c} 
      \hline & \\[-1.5ex]
      Program & Proof \\ [2pt]
      \hline & \\[-1.5ex]
      \term{Base case} (\code{pow (n, 0)}) & \term{Base case} ($n^0$) \\ [2pt]
      \hline & \\[-1.5ex]
      \term{Recursive call} (\code{pow (n, k)}) & \term{Inductive hypothesis} ($n^k$) \\ [2pt]
      \hline & \\[-1.5ex]
      \term{Variable of recurrence} (\code{k}) & \term{Induction variable} ($k$) \\ [2pt]
      \hline & \\[-1.5ex]
      \term{Simple recursive call} (\code{k - 1}) & \term{Simple induction} \\ [2pt]
      \hline & \\[-1.5ex]
      \term{Complex recursive call} (\code{k div 2}) & \term{Strong induction} \\ [2pt]
      \hline & \\[-1.5ex]
      \term{Branching behavior} (\code{if}) & \term{Proof casing} \\ [2pt]
      \hline
    \end{tabular}
  \end{center}

  \ptmt
\end{frame}
 
\begin{frame}[plain]
	\begin{center} Thank you! \end{center}

	\begin{center} 
    Post-lecture survey: \\
    \vspace{5pt}
    \includegraphics[scale=0.035]{qr_may23}
  \end{center}
\end{frame}

\end{document}

