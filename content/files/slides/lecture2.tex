% Jacob Neumann

% DOCUMENT CLASS AND PACKAGE USE
    \documentclass[aspectratio=169]{beamer}
 
    % Establish the colorlambda boolean, to control whether the lambda is solid color (true), or the same as the picture (false)
    \newif\ifcolorlambda
    \colorlambdafalse % DEFAULT: false
    
    % Use auxcolor for syntax highlighting
    \newif\ifuseaux
    \useauxfalse % DEFAULT: false
   
    % Color settings
    \useauxtrue

    \newcommand{\auxColor}{655DBB}     % the color of note boxes and stuff
    \newcommand{\presentColor}{3E54AC} % the primary color of the slide borders
    \newcommand{\bgColor}{fafafa}      % the color of the background of the slide
    \newcommand{\darkBg}{8b98ad}
    \newcommand{\lambdaColor}{\auxColor}
  
    \colorlambdatrue

    \usepackage{comment} % comment blocks
    \usepackage{soul} % strikethrough
    \usepackage{listings} % code
    \usepackage{makecell}

    \setbeamertemplate{itemize items}[circle]
    % \setbeameroption{show notes on second screen=right}

    \usepackage{lectureSlides}
    %%%%%%%%%%%%%%%%%%%%%%%%%%%%%%%%%%%%%%%%%| <----- Don't make the title any longer than this
    \title{Equivalence, Functions, and Scope} % TODO
    \subtitle{Awesome slides with an awesome subtitle} % TODO
    \date{01 January 2020} % TODO
    \author{Brandon Wu} % TODO

    \graphicspath{ {./img/} }
    % DONT FORGET TO PUT [fragile] on frames with codeblocks, specs, etc.
        %\begin{frame}[fragile]
        %\begin{codeblock}
        %fun fact 0 = 1
        %  | fact n = n * fact(n-1)
        %\end{codeblock}
        %\end{frame}

    % INCLUDING codefile:
        % 1. In some file under code/NN (where NN is the lecture id num), include:
    %       (* FRAGMENT KK *)
    %           <CONTENT>
    %       (* END KK *)
    
    %    Remember to not put anything on the same line as the FRAGMENT or END comment, as that won't be included. KK here is some (not-zero-padded) integer. Note that you MUST have fragments 0,1,...,KK-1 defined in this manner in order for fragment KK to be properly extracted.
        %  2. On the slide where you want code fragment K
                % \smlFrag[color]{KK}
        %     where 'color' is some color string (defaults to 'white'. Don't use presentColor.
    %  3. If you want to offset the line numbers (e.g. have them start at line 5 instead of 1), use
                % \smlFragOffset[color]{KK}{5}

\begin{document}

% Make it so ./mkWeb works correctly
\ifweb
    \renewcommand{\pause}{}
\fi

\setbeamertemplate{itemize items}[circle]

% SOLID COLOR TITLE (see SETTINGS.sty)
{
\begin{frame}[plain]
    \colorlambdatrue
    \titlepage
\end{frame}
}

\begin{frame}[fragile]
  \frametitle{Lesson Plan}

  \tableofcontents
\end{frame}

\begin{frame}[fragile]
  \frametitle{Last time}

  In the last lecture, we learned about \textbf{expressions}, \textbf{values}, and \textbf{types}.

  \vspace{5pt}

  We learned that \textbf{only well-typed expressions are evaluated}, and that expressions can 
  exhibit one of three behaviors:

  \vspace{\fill}

  \begin{itemize}
    \item Evaluate to a value
    \item Raise an exception
    \item Loop forever
  \end{itemize}
\end{frame}

\sectionSlide{1}{More Types}

\begin{frame}[fragile]
  \frametitle{More on types}

  There are a few basic types to know in SML.

  \vspace{10pt}

  The \term{base types} comprise of a few simple primitives:

  \vspace{5pt}

  \begin{itemize}
    \item \code{int} (\code{1}, \code{150}, \code{412})
    \item \code{real} (\code{1.0}, \code{1.50})
    \item \code{char} (\code{#"a"}, \code{#"1"})
    \item \code{bool} (\code{true}, \code{false})
    \item \code{string} (\code{"functions"}, \code{"are"}, \code{"values"})
  \end{itemize}

  \vspace{\fill}

  Types get a lot more interesting, though!
\end{frame}

\begin{frame}[fragile]
  \frametitle{Tuples}

  SML has \term{tuples}, which are just collections of values of 
  other types.

  \vspace{\fill}

  So for instance, valid tuples include:

  \vspace{5pt}

  \begin{itemize} 
    \item \code{(1, 2) : int * int}
    \item \code{(1, "hi") : int * string}
    \item \code{("a", true, 1.0) : string * bool * real}
  \end{itemize}

  \vspace{\fill}

  \defBox{ We call a type like \code{int * int} or \code{string * bool * real} a \term{tuple type},
  or \term{product type}.
  }

  \vspace{\fill}

  \noteBox{ Tuples evaluate from \textbf{left to right}. So \code{(1 + 1, 2 + 2)} $\stepsTo$ \code{(2, 2 + 2)} $\stepsTo$ \code{(2, 4)}}.
\end{frame}

\begin{frame}[fragile]
  \frametitle{More on Tuples}

  \defBox{\, The typing rule for tuples is that if \code{e1 : t1}, \code{e2 : t2}, ..., \code{en : tn}, then \code{(e1, e2, ..., en) : t1 * t2 * ... * tn}}

  \vspace{\fill}

  \noteBox{\, Parentheses matter! The type \code{int * string * bool} is \textbf{different} than \code{int * (string * bool)}}

  \vspace{5pt}

  The first type is a tuple of three things, but the second is a tuple of two things, where the second part is itself a tuple.

\end{frame}

\sectionSlide{2}{Functions}

\begin{frame}[fragile]
  \frametitle{Functions}

  Similarly to how we can use \code{*} to make tuple types out of other types, we can use \code{->} to make 
  function types out of other types.

  \vspace{\fill}

  \defBox{\, We call a type like \code{int -> int} a \term{function type}, which takes in a value of
  type \code{int} and evaluates to an expression of type \code{int}}.

  \vspace{\fill}

  For instance, we have \code{not : bool -> bool}, such that \code{not true} $\stepsTo$ \code{false}.

  \vspace{\fill}

  \noteBox{\, In SML, parentheses are not needed to separate a function from its argument!}

  \vspace{5pt}

  Writing \code{e1 e2} means the \term{application} of the function \code{e1} to its argument \code{e2}.

  \vspace{\fill}

  \defBox{\, The typing rule for function application is \code{e1 e2 : t2} if \code{e1 : t1 -> t2} and \code{e2 : t1}.}
\end{frame}

\begin{frame}[fragile]
  \frametitle{Lambda Expressions}

  Similarly to how we can create expressions with tuple type, we can create expressions
  with function type.

  \vspace{\fill}

  We write \code{fn (x : int) => x + 1} for the function which takes in an int, and adds one to it. 
  We call this a \term{lambda expression}.

  \vspace{5pt}

  \noteBox{\, Lambda expressions are expressions! They do not declare a function that can be referenced,
  they are \textbf{anonymous}, and do not have names}.

  \vspace{\fill}

  So \code{(fn (x : int) => x + 1) 2} $\stepsTo$ \code{3}.

  \vspace{\fill}

  Lambda expressions are themselves values, meaning that they do not reduce to anything else.
\end{frame}

\begin{frame}[fragile]
  \frametitle{Function Evaluation}

  \noteBox{\, An expression \code{e1 e2} \textbf{first evaluates} \code{e1}, and then \code{e2}}.

  \vspace{5pt}

  SML is an \term{eagerly evaluated} language, meaning that arguments to functions are 
  \textbf{always} reduced to values, before the function can be called.

  \vspace{\fill}

  \begin{align*}
    & \code{(fn (x : int * int) => 150) (1 + 1, 3 * 4)} \\ 
    & \stepsTo \code{(fn (x : int * int) => 150) (2, 3 * 4)} \\
    & \stepsTo \code{(fn (x : int * int) => 150) (2, 12)} \\
    & \stepsTo \code{150} \\ 
  \end{align*}
\end{frame}

\begin{frame}[fragile]
  \frametitle{Function Declarations}

  What if we wanted to declare a function, rather than just use a lambda expression?

  \vspace{\fill}

  We can write something like
  \begin{codeblock} 
    fun double (x : int) : int = x + x
  \end{codeblock}

  to declare a function \code{double : int -> int} which doubles its input.

  \vspace{\fill}

  Then, we could write an expression like \code{double 2} $\stepsTo$ \code{4}.
\end{frame}

\sectionSlide{3}{Binding and Scope}

\begin{frame}[fragile]
  \frametitle{Variable Declarations}

  So far we've talked a lot about expressions, values, and types, but not about one of the
  most important things in a programming language - variables!

  \vspace{\fill}

  The syntax: 

  \begin{codeblock}
    val x : int = 2
  \end{codeblock}

  \textbf{binds} the value \code{2} to the variable \code{x}. 

  \vspace{\fill}

  \defBox{\,\term{Binding} is the act of producing a new association of a value to
  a variable name.}

  \vspace{\fill}

  \noteBox{ \textbf{Binding is not assignment}. }
\end{frame}

\begin{frame}[fragile]
  \frametitle{Binding vs. Assignment}

  The easy way to see how binding differs from assignment is to consider 
  the following code:

  \vspace{\fill}

  \begin{codeblock}
    val x : int = 2 
    fun foo (y : int) : int = x + y
    val x : int = 4 
  \end{codeblock}

  \vspace{\fill}

  After this code, what is the value of \code{foo 1}?
  
  \vspace{\fill}

  The imperative answer is \code{5}. 

  \vspace{\fill}

  \textbf{The SML answer is \code{3}}.
\end{frame}

\begin{frame}[fragile]
  \frametitle{Binding vs. Assignment}

  \begin{codeblock}
    val x : int = 2 
    fun foo (y : int) : int = x + y
    val x : int = 4 
  \end{codeblock}

  \vspace{\fill}

  In an imperative language, you \textbf{change the world} by \textbf{reassigning} the 
  value of the variable \code{x}.

  \vspace{5pt}

  This changes the value referenced by \code{fun foo (y : int) = x + y}.

  \vspace{\fill}

  In a functional language, you bind a \textbf{new, unrelated} variable called \code{x},
  whose value is \code{4}, but is \textbf{not} the same as the one referenced in \code{foo}.

  \vspace{\fill}

\end{frame}

\begin{frame}[fragile]
  \frametitle{A Binding Analogy}

  Suppose you are named Brandon, and you have a 9-5 job. 

  \vspace{\fill}

  Your manager walks in and says "Your performance
  has been suffering lately", and walks out. 

  \vspace{\fill}

  Then, another engineer named Brandon walks in and sits down.

  \vspace{\fill}

  \textbf{You are still in trouble.}

  \vspace{\fill}

  The point: Just because something (or someone) named the same walked in, doesn't
  change who your manager was talking about!
\end{frame}


\begin{frame}[fragile]
  \frametitle{Environments}

  \defBox{\, The \term{environment} at a particular point in a program is the 
  collection of all currently active bindings.}

  \vspace{\fill}

  \code{val} bindings and \code{fun} declarations introduce \textbf{new bindings} into 
  the environment, and displace old ones. This is called \term{shadowing}.
  
  \vspace{5pt}

  When a binding is shadowed, we say the old binding leaves \term{scope}.

  \vspace{\fill}

  We use the mathematical notation $[\code{5}/\code{x}, \code{true}/\code{y}]$, for instance, for the environment
  where \code{5} is bound to \code{x} and \code{true} is bound to \code{y}.
\end{frame}

\begin{frame}[fragile]
  \frametitle{A Trace of Environment}

  Let's look at an example of a trace with environments attached. 

  \vspace{10pt}

  Suppose our program is:

  \begin{codeblock}
    val x : int = 1
    val y : int = 1 + x
    val x : int = 3
    val z : int = x + y
  \end{codeblock}

  \vspace{\fill}

  \begin{align*}
    & \code{val x : int = 1} &\implies &&[\code{1}/\code{x}] \\ 
    & \code{val y : int = 1 + x} &\implies &&[\code{1}/\code{x}, \code{2}/\code{y}] \\ 
    & \code{val x : int = 3} &\implies &&[\code{3}/\code{x}, \code{2}/\code{y}] \\ 
    & \code{val z : int = x + y} &\implies &&[\code{3}/\code{x}, \code{2}/\code{y}, \code{5}/\code{z}] \\ 
  \end{align*}
\end{frame}


\begin{frame}[fragile]
  \frametitle{Functions and Environments}

  It might seem like this is no different than assignment.

  \vspace{5pt}

  Environments and bindings are shown as different when functions get involved.

  \vspace{\fill}

  \defBox{\, A function binding is composed of two things, a lambda expression and the 
  environment \textbf{at the time of binding}. This is known as a \term{closure}.}

  \vspace{5pt}

  This means that a function always only knows about what was in the environment when
  it was first bound. It doesn't see any bindings that happen later.

  \vspace{\fill}

  \noteBox{\,Functions are elephants. (they remember everything)
  }
\end{frame}


\begin{frame}[fragile]
  \frametitle{Functions and Environments}

  \begin{codeblock}
    val x : int = 2 
    fun foo (y : int) : int = x + y
    val x : int = 4 
  \end{codeblock}

  \vspace{\fill}

  After the first binding of \code{x}, we have the environment $[\code{2}/\code{x}]$.

  \vspace{\fill}

  Then, we could represent the closure as:

  \vspace{5pt}

  \makebox[\textwidth][c]{
    \begin{tikzpicture}
      \node[rectangle,draw, minimum width=2.5in, fill=auxColor] (r) at (0,0) {[\code{2}/\code{x}]};
      \node[rectangle,draw, minimum width=2.5in, fill=cyan] (r) at (0,0.75) {\code{fn (y : int) => x + y}};
    \end{tikzpicture}
  }

  \vspace{5pt}

  which is bound to the identifier \code{foo}. Future bindings will not change this.
\end{frame}

\lectureBreak

\sectionSlide{4}{Equivalence}

\begin{frame}[fragile]
  \frametitle{What's the point?}

  We have now described the difference between variable binding in SML and 
  assignment in other languages.

  \vspace{5pt}

  A question remains - \textbf{what's the point?}

  \vspace{\fill}

  In the first lecture, we described some of the ideals of a good programming language,
  and one of them was \textbf{modularity}. We should be able to change a part of a 
  program without affecting another. 

  \vspace{5pt}
  
  Remember that functions in SML are like mathematical functions, they always 
  give the same outputs for the same inputs (purity).

  \vspace{\fill}

  Variable binding allows a stronger property, which is that \textbf{no binding after
  a function declaration can change that function's behavior}.

  \vspace{5pt}

  In other words, \textbf{binding preserves function} \term{equivalence}.
\end{frame}

\begin{frame}[fragile]
  \frametitle{Extensional Equivalence}

  Functional programming lends itself to reasoning mathematically about code.

  \vspace{5pt}

  To supplement that, we will have a notion of when code is equivalent.

  \vspace{\fill}

  \defBox{\,Two expressions \textbf{of the same type} are said to be \term{extensionally equivalent} if they:}
  \begin{itemize}
    \item evaluate to the same value,
    \item both loop forever,
    \item or both raise the same kind of exception
  \end{itemize}

  \vspace{5pt}

  We write \code{e1} $\eeq$ \code{e2} when \code{e1} and \code{e2} are extensionally equivalent.

  \vspace{\fill}

  So for instance, \code{2 + 2} $\eeq$ \code{4} $\eeq$ \code{1 + 3}. 
\end{frame}

\begin{frame}[fragile]
  \frametitle{Referential Transparency}

  It turns out that this idea gives us some very nice properties.

  \vspace{\fill}

  \defBox{\, \term{Referential Transparency} - If \code{e1} $\eeq$ \code{e2}, then 
  replacing \code{e1} with \code{e2} anywhere in a program will produce an extensionally 
  equivalent program.
  }
  \vspace{\fill}

  So for instance, you could replace \code{2 + 2} with \code{1 + 3} anywhere, and always
  be assured that program behavior \textbf{will not change}.
  
  \vspace{5pt}

  In other words, you can swap "equals for equals".

  \vspace{\fill}

  \noteBox{\, This is the foundation of \textbf{equational reasoning}, which lets us 
  reason about code like we would reason about math.}
\end{frame}

\begin{frame}[fragile]
  \frametitle{On the Equivalence of Values}

  Up until this point, we've said that a value is a "final answer". You should be able
  to tell if values are the same just by looking at them, like with \code{2} and \code{2},
  or with \code{(3, 4)} and \code{(3, 4)}.

  \vspace{\fill}

  But what about functions?

  \vspace{5pt}

  Are \code{fn (x : int) => x + x} and \code{fn (x : int) => 2 * x} extensionally equivalent?

  \vspace{\fill}

  It's now hard to tell, because these lambda expressions are values, but it's not obvious 
  if they are extensionally equivalent or not.
\end{frame}

\begin{frame}[fragile]
  \frametitle{Extensional Equivalence for Functions}

  Functions necessitate their own rule for extensional equivalence.

  \vspace{\fill}

  \defBox{\, Two functions \code{f : t1 -> t2} and \code{g : t1 -> t2} are \term{extensionally equivalent}
  if for all values \code{x : t1}, \code{f x} $\eeq$ \code{g x}}.

  \vspace{5pt}

  In other words, two functions are equivalent if \textbf{for all inputs, they give equivalent outputs}.

  \vspace{\fill}
  
  \noteBox{\, Note that we specified that \code{f x} $\eeq$ \code{g x}, not that \code{f x} and \code{g x}
  reduce to the same value. Why?}
\end{frame}

\begin{frame}[fragile]
  \frametitle{Totality}

  The reason why is simple - some functions do not have defined outputs!

  \vspace{\fill}

  We described functions in SML as mathematical functions, which is true, but not exactly in the
  same way as some mathematical functions, such as $+$, or $sin$.

  \vspace{5pt}

  \textbf{SML functions can be partial}, that is, undefined on some input. There may be \code{f}
  and \code{x} such that there is no \code{v} where \code{f x} $\hookrightarrow$ \code{v}.  

  \vspace{\fill}

  \defBox{\, We say that \code{f : t1 -> t2} is \term{total} if for all values \code{v : t1},
  there is a value \code{v' : t2} such that \code{f v} $\hookrightarrow$ \code{v'}.}

  \vspace{5pt}

  For example, the SML functions \code{+}, \code{not}, and \code{^} are all total.
\end{frame}

\begin{frame}[fragile]
  \frametitle{Specification}

  Concepts like purity, extensional equivalence, and totality are all just means to
  an end, which is being able to \textbf{specify} the behavior of code.

  \vspace{\fill}

  In particular, for functions, we are interested in writing \textbf{descriptive} code,
  that accurately reflects the function's behavior.
  
  \vspace{5pt}

  To that end, it is helpful to write alongside a function the conditions which 
  must be true \textbf{prior} to calling the function, and must be true \textbf{after}
  calling the function.

  \vspace{\fill}

  \begin{codeblock}
    (* REQUIRES: x is not 0 *)
    (* ENSURES: divide x is total *)
    fun divide (x : int) : int = 2 div x 
  \end{codeblock}
\end{frame}

\begin{frame}[fragile]
  \frametitle{Specification, pt. 2}

  We use comments in the form of \code{REQUIRES} and \code{ENSURES} to describe 
  what must be true of the inputs the function receives, and what then is guaranteed 
  to hold of the function's behavior. 

  \vspace{\fill}

  It is often unrealistic to have a function which has meaning on \textit{every} possible 
  input (like \code{div}, or square root, or logarithm). The \code{REQUIRES} helps to
  describe the range of "relevant inputs", and the \code{ENSURES} helps to describe
  what the function does.

  \vspace{\fill}

  These "contracts" pop up in real code all the time:
  \begin{itemize}
    \item this function must be called with only safe values,
    \item this library can only be invoked in single-threaded programs,  
    \item this API is not guaranteed to work with non-ASCII characters 
  \end{itemize}

  \vspace{\fill}

  It is important that code is documented so users and maintainers know pertinent
  information about it! 
\end{frame}

\begin{frame}[fragile]
  \frametitle{A Specific Specification}

  For example, take the factorial function.

  \vspace{\fill}

  \begin{codeblock}
    (* REQUIRES: n >= 0 *)
    (* ENSURES: fact n reduces to the factorial of n *)
    fun fact (n : int) : int = 
      if n = 0 then 
        1
      else 
        n * fact (n - 1) 
  \end{codeblock}
\end{frame}

\begin{frame}[plain]
	\begin{center} Thank you! \end{center}
\end{frame}


\end{document}

