% Jacob Neumann

% DOCUMENT CLASS AND PACKAGE USE
    \documentclass[aspectratio=169]{beamer}
 
    % Establish the colorlambda boolean, to control whether the lambda is solid color (true), or the same as the picture (false)
    \newif\ifcolorlambda
    \colorlambdafalse % DEFAULT: false
    
    % Use auxcolor for syntax highlighting
    \newif\ifuseaux
    \useauxfalse % DEFAULT: false
   
    % Color settings
    \useauxtrue

    \newcommand{\auxColor}{009999}     % the color of note boxes and stuff
    \newcommand{\presentColor}{50009F} % the primary color of the slide borders
    \newcommand{\bgColor}{d3cade}      % the color of the background of the slide
    \newcommand{\darkBg}{d4cedb}
    \newcommand{\lambdaColor}{\auxColor}
  
    \colorlambdatrue

    \usepackage{comment} % comment blocks
    \usepackage{soul} % strikethrough
    \usepackage{listings} % code
    \usepackage{makecell}

    \setbeamertemplate{itemize items}[circle]
    % \setbeameroption{show notes on second screen=right}

    \usepackage{lectureSlides}
    %%%%%%%%%%%%%%%%%%%%%%%%%%%%%%%%%%%%%%%%%| <----- Don't make the title any longer than this
    \title{Equivalence, Binding, and Scope} % TODO
    \subtitle{A mathematical perspective on programs} % TODO
    \date{18 May 2023} % TODO
    \author{Brandon Wu} % TODO

    \graphicspath{ {./img/} }
    % DONT FORGET TO PUT [fragile] on frames with codeblocks, specs, etc.
        %\begin{frame}[fragile]
        %\begin{codeblock}
        %fun fact 0 = 1
        %  | fact n = n * fact(n-1)
        %\end{codeblock}
        %\end{frame}

    % INCLUDING codefile:
        % 1. In some file under code/NN (where NN is the lecture id num), include:
    %       (* FRAGMENT KK *)
    %           <CONTENT>
    %       (* END KK *)
    
    %    Remember to not put anything on the same line as the FRAGMENT or END comment, as that won't be included. KK here is some (not-zero-padded) integer. Note that you MUST have fragments 0,1,...,KK-1 defined in this manner in order for fragment KK to be properly extracted.
        %  2. On the slide where you want code fragment K
                % \smlFrag[color]{KK}
        %     where 'color' is some color string (defaults to 'white'. Don't use presentColor.
    %  3. If you want to offset the line numbers (e.g. have them start at line 5 instead of 1), use
                % \smlFragOffset[color]{KK}{5}

\begin{document}

% Make it so ./mkWeb works correctly
\ifweb
    \renewcommand{\pause}{}
\fi

\setbeamertemplate{itemize items}[circle]

% SOLID COLOR TITLE (see SETTINGS.sty)
{
\begin{frame}[plain]
    \colorlambdatrue
    \titlepage
\end{frame}
}

\begin{frame}[fragile]
  \frametitle{Lesson Plan}

  \tableofcontents
\end{frame}

\begin{frame}[fragile]
  \frametitle{Last time}

  In the last lecture, we learned about \term{expressions}, \term{values}, and \term{types}.

  \pause
  \vspace{5pt}

  We learned that \textbf{only well-typed expressions are evaluated}, and that expressions can 
  exhibit one of three behaviors:

  \pause
  \vspace{\fill}

  \begin{itemize}
    \item Evaluate to a value \pause
    \item Raise an exception \pause
    \item Loop forever
  \end{itemize}

  \pause
  \vspace{\fill}

  We also saw some examples of \term{typing rules} which SML uses to determine whether
  an expression is well-typed or not.
\end{frame}

\sectionSlide{1}{More Types}

\begin{frame}[fragile]
  \frametitle{More on types}

  There are a few basic types to know in SML.
  
  \pause
  \vspace{10pt}

  The \term{base types} comprise of a few simple primitives:

  \pause
  \vspace{5pt}

  \begin{itemize}
    \item \code{int} - \code{1}, \code{150}, \code{412}
    \item \code{real} - \code{1.0}, \code{1.50}
    \item \code{char} - \code{#"a"}, \code{#"1"}
    \item \code{bool} - \code{true}, \code{false}
    \item \code{string} - \code{"functions"}, \code{"are"}, \code{"values"}
  \end{itemize}

  \vspace{\fill}

  Types get a lot more interesting, though!
\end{frame}

\begin{frame}[fragile]
  \frametitle{Tuples}

  \tgs

  SML has \term{tuples}, which are just collections of values of 
  other types.

  \pause
  \vspace{\fill}

  So for instance, valid tuples include:

  \vspace{5pt}

  \begin{itemize} 
    \item \code{(1, 2) : int * int}
    \item \code{(1, "hi") : int * string}
    \item \code{("a", true, 1.0) : string * bool * real}
  \end{itemize}

  \pause
  \vspace{\fill}

  \defBox{ We call a type like \code{int * int} or \code{string * bool * real} a \term{tuple type},
  or \term{product type}.
  }

  \pause
  \vspace{\fill}

  \noteBox{ Tuples evaluate from \textbf{left to right}. So \code{(1 + 1, 2 + 2)} $\stepsTo$ \code{(2, 2 + 2)} $\stepsTo$ \code{(2, 4)}}.
\end{frame}

\begin{frame}[fragile]
  \frametitle{More on Tuples}

  \defBox{\, The typing rule for tuples is that if \code{e1 : t1}, \code{e2 : t2}, ..., \code{en : tn}, then \code{(e1, e2, ..., en) : t1 * t2 * ... * tn}}

  \pause
  \vspace{\fill}

  \noteBox{\, Parentheses matter! The type \code{int * string * bool} is \textbf{different} than \code{int * (string * bool)}}

  \pause
  \vspace{\fill}

  So for instance, \code{(1, 2, 3) : int * int * int} is a valid tuple, but so is 
  \code{(1, (2, 3)) : int * (int * int)}. These are different values!

  \vspace{5pt}

  The first value is a tuple of three things, but the second is a tuple of two things, where the second part is itself a tuple.

\end{frame}

\sectionSlide{2}{Functions}

\begin{frame}[fragile]
  \frametitle{Functions}

  Similarly to how we can use \code{*} to make tuple types out of other types, we can use \code{->} to make 
  function types out of other types.

  \vspace{5pt}

  \customBox{Remark}{\, This is because \code{*} and \code{->} are known
  as \textit{type constructors}.}

  \pause
  \vspace{\fill}

  \defBox{\, We call a type like \code{int -> int} a \term{function type}, which takes in a value of
  type \code{int} and evaluates to an expression of type \code{int}}.

  \pause
  \vspace{\fill}

  For instance, we have \code{not : bool -> bool}, such that \code{not true} $\stepsTo$ \code{false}.
\end{frame}

\begin{frame}[fragile]
  \frametitle{Lambda Expressions}

  Similarly to how we can create expressions with tuple type, we can create expressions
  with function type.

  \pause
  \vspace{\fill}

  We write \code{fn (x : int) => x + 1} for the function which takes in an int, and adds one to it. 
  We call this a \term{lambda expression}.

  \pause
  \vspace{5pt}

  \noteBox{\, Lambda expressions are expressions! They do not declare a function that can be referenced,
  they are \textbf{anonymous}, and do not have names}.

  \pause
  \vspace{\fill}

  So \code{(fn (x : int) => x + 1) 2} $\stepsTo$ \code{3}.

  \pause
  \vspace{\fill}

  Lambda expressions are themselves values, meaning that they do not reduce to anything else.

  \pause
  \vspace{\fill}

  \noteBox{\, Functions are values.}
\end{frame}

\begin{frame}[fragile]
  \frametitle{Functions are Values}

  As we will say many times over the course of the semester, \textbf{functions are values}.

  \pause
  \vspace{\fill}

  This means that they can be bound to variables like any other value! So while we earlier
  wrote 
  \begin{codeblock}
    fun double (n : int) : int = n + n
  \end{codeblock}

  \pause
  \vspace{\fill}

  we could equivalently have written
  \begin{codeblock}
    val double = fn (n : int) => n + n
  \end{codeblock}

  \vspace{\fill}

  We will have more to say on this idea later.
\end{frame}



\begin{frame}[fragile]
  \frametitle{Lambdas Aren't Recursive}

  Suppose I wanted to declare the \code{fact} function using a lambda expression. 

  \pause
  \vspace{\fill}

  I might write something like:
  \begin{codeblock}
    fn (n : int) => 
      if n = 0 then 1
      else (* ??? *)
  \end{codeblock}

  \vspace{\fill}

  But what do I call in the recursive case? How do I call a lambda recursively?

  \pause
  \vspace{\fill}

  \noteBox{\, You can't. Lambdas are non recursive. They don't
  have names, so they can't reference themselves.}
\end{frame}

\begin{frame}[fragile]
  \frametitle{Function Evaluation}

  \noteBox{\, An expression \code{e1 e2} \textbf{first evaluates} \code{e1}, and then \code{e2}}.

  \pause
  \vspace{5pt}

  SML is an \term{eagerly evaluated} language, meaning that arguments to functions are 
  \textbf{always} reduced to values, before the function can be called.

  \pause
  \vspace{\fill}

  Consider the expression \code{(fn (x : int * int) => 150) (1 + 1, 3 * 4)}
  \pause
  \begin{align*}
    & \code{(fn (x : int * int) => 150) (1 + 1, 3 * 4)} \\ 
    & \stepsTo \code{(fn (x : int * int) => 150) (2, 3 * 4)} & \tag{def of \code{+}}\\
    & \stepsTo \code{(fn (x : int * int) => 150) (2, 12)} & \tag{def of \code{*}}\\
    & \stepsTo \code{150} & \tag{function application}
  \end{align*}
  \begin{align*}
    \code{(fn (x : int * int) => 150) (1 + 1, 3 * 4)} \stepsTo 1 + fact \\ 

  \end{align*}
  \pause
  Note that, when stepping expressions like this, usually we will want 
  to cite a \textbf{justification} for each step to the right of the 
  newly-obtained expression. You should do this on your homework too.
\end{frame}

\begin{frame}[fragile]
  \frametitle{Stepping into a Function}

  We glossed over this in the previous lecture, but when we 
  apply a function, it's with the goal of stepping into the
  \term{function body}.

  \pause
  \vspace{\fill}

  \defBox{\, The \term{body} of a function is the expression
  that the function should evaluate to, given arguments.}

  \vspace{\fill}
  
  So for instance, in \code{fn (n : int) => n + n}, the 
  body is the expression \code{n + n}.

  \pause
  \vspace{\fill}

  When the arguments to a function are values, we can then 
  \term{substitute} those values for the function's arguments,
  within its body. This produces \term{bindings}.
\end{frame}

\sectionSlide{3}{Binding and Scope}

\begin{frame}[fragile]
  \frametitle{Variable Declarations}

  What is a variable declaration really doing?

  \pause
  \vspace{\fill}

  The syntax: 

  \begin{codeblock}
    val x : int = 2
  \end{codeblock}

  \textbf{binds} the value \code{2} to the variable \code{x}. 

  \pause
  \vspace{\fill}

  \defBox{\, \term{Binding} is the act of producing a new association of a value to
  a variable name.}

  \pause
  \vspace{\fill}

  \noteBox{\, \textbf{Binding is not assignment}. }
\end{frame}

\begin{frame}[fragile]
  \frametitle{Binding vs. Assignment}

  The easy way to see how binding differs from assignment is to consider 
  the following code:

  \pause
  \vspace{\fill}

  \begin{codeblock}
    val x : int = 2 
    fun foo (y : int) : int = x + y
    val x : int = 4 
  \end{codeblock}

  \pause
  \vspace{\fill}

  After this code, what is the value of \code{foo 1}?
  
  \pause
  \vspace{\fill}

  The imperative answer is \code{5}. 

  \pause
  \vspace{\fill}

  \textbf{The SML answer is \code{3}}.
\end{frame}

\begin{frame}[fragile]
  \frametitle{Binding vs. Assignment}

  \begin{codeblock}
    val x : int = 2 
    fun foo (y : int) : int = x + y
    val x : int = 4 
  \end{codeblock}

  \pause
  \vspace{\fill}

  In an imperative language, you \textbf{change the world} by \textbf{reassigning} the 
  value of the variable \code{x}.

  \vspace{5pt}

  This changes the value referenced by \code{fun foo (y : int) = x + y}.

  \pause
  \vspace{\fill}

  In a functional language, you bind a \textbf{new, unrelated} variable called \code{x},
  whose value is \code{4}, but is \textbf{not} the same as the one referenced in \code{foo}.

  \vspace{\fill}

\end{frame}

\begin{frame}[fragile]
  \frametitle{A Binding Analogy}

  Suppose you are named Brandon, and you have a 9-5 job. 

  \pause
  \vspace{\fill}

  Your manager walks in and says "Brandon, your performance
  has been suffering lately", and walks out. 

  \pause
  \vspace{\fill}

  Then, another engineer named Brandon walks in and sits down.

  \pause
  \vspace{\fill}

  \textbf{You are still in trouble.}

  \pause
  \vspace{\fill}

  \textbf{The point:} Just because something (or someone) named the same walked in, doesn't
  change who your manager was talking about!
\end{frame}


\begin{frame}[fragile]
  \frametitle{Environments}

  \defBox{\, The \term{environment} at a particular point in a program is the 
  collection of all currently active bindings.}

  \pause
  \vspace{\fill}

  \code{val} bindings and \code{fun} declarations introduce \textbf{new bindings} into 
  the environment, and displace old ones. This is called \term{shadowing}.
  
  \pause
  \vspace{5pt}

  When a binding is shadowed, we are no longer in its scope. 

  \pause
  \vspace{\fill}

  We use the mathematical notation $[\code{5}/\code{x}, \code{true}/\code{y}]$, for instance, for the environment
  where \code{5} is bound to \code{x} and \code{true} is bound to \code{y}.
\end{frame}

\begin{frame}[fragile]
  \frametitle{A Trace of Environment}

  Let's look at an example of a trace with environments attached. 

  \pause
  \vspace{10pt}

  Suppose our program is:

  \begin{codeblock}
    val x : int = 1
    val y : int = 1 + x
    val x : int = 3
    val z : int = x + y
  \end{codeblock}

  \pause
  \vspace{\fill}

  \begin{align*}
    & \code{val x : int = 1} &\text{(results in)} &&[\code{1}/\code{x}] \\ 
    & \code{val y : int = 1 + x} &\text{(results in)} &&[\code{1}/\code{x}, \code{2}/\code{y}] \\ 
    & \code{val x : int = 3} &\text{(results in)} &&[\code{3}/\code{x}, \code{2}/\code{y}] \\ 
    & \code{val z : int = x + y} &\text{(results in)} &&[\code{3}/\code{x}, \code{2}/\code{y}, \code{5}/\code{z}] \\ 
  \end{align*}
\end{frame}


\begin{frame}[fragile]
  \frametitle{Functions and Environments}

  It might seem like this is no different than assignment.

  \pause
  \vspace{5pt}

  \keyBox{}{\, Environments and bindings are shown as different when functions get involved.}

  \pause
  \vspace{\fill}

  \defBox{\, A function binding is composed of two things, a lambda expression and the 
  environment \textbf{at the time of binding}. This is known as a \term{closure}.}

  \vspace{5pt}

  This means that a function always only knows about what was in the environment when
  it was first bound. It doesn't see any bindings that happen later.

  \pause
  \vspace{\fill}

  \noteBox{\, Functions are elephants. (they remember everything)
  }
\end{frame}


\begin{frame}[fragile]
  \frametitle{Functions and Environments}

  \ptmt

  \begin{codeblock}
    val x : int = 2 
    fun foo (y : int) : int = x + y
    val x : int = 4 
  \end{codeblock}

  \pause
  \vspace{\fill}

  After the first binding of \code{x}, we have the environment $[\code{2}/\code{x}]$.

  \pause
  \vspace{\fill}

  Then, we could represent the closure as:

  \vspace{5pt}

  \makebox[\textwidth][c]{
    \begin{tikzpicture}
      \node[rectangle,draw, minimum width=2.5in, fill=auxColor] (r) at (0,0) {[\code{2}/\code{x}]};
      \node[rectangle,draw, minimum width=2.5in, fill=lightgray] (r) at (0,0.75) {\code{fn (y : int) => x + y}};
    \end{tikzpicture}
  }

  \vspace{5pt}

  which is bound to the identifier \code{foo}. \textbf{Future bindings will not change this.}
\end{frame}

\quizBreak{PACHYDERM}

\sectionSlide{4}{Pattern Matching}

\begin{frame}[fragile]
  \frametitle{Unpacking Tuples}

  We can write declarations that bind tuples to variables, for instance:
  \begin{codeblock}
    val x : int * int = (1, 2)
  \end{codeblock}

  \pause
  \vspace{\fill}

  But now, how do we access the different parts of the tuple? What if we want to get
  \code{1} and \code{2} back out of the tuple?

  \pause
  \vspace{\fill}

  We can unpack it using a \term{pattern}.

  \pause
  \begin{codeblock}
    val (x, y) : int * int = (1, 2)
  \end{codeblock}

  \pause
  \vspace{\fill}

  This produces the environment $[\code{1}/\code{x}, \code{2}/\code{y}]$
\end{frame}

\begin{frame}[fragile]
  \frametitle{Pattern Matching}

  \defBox{\, A \term{pattern} is a way to describe the \textbf{form} of a value. A
  value can either match to a pattern, or not. }

  \pause
  \vspace{5pt}

  The goal is that patterns should be used to \textit{describe} the values that you are interested
  in.

  \pause
  \vspace{\fill}

  For the example we gave above, we used the pattern \code{(x, y)} to \term{deconstruct}
  the value \code{(1, 2)}, because we knew that the tuple \code{(1, 2)} has two entries. 

  \pause
  \vspace{\fill}

  The right-hand side of the expression doesn't even need to be a value! It turns out,
  all you need to know to figure out the right pattern to use is the expression's
  \textbf{type}.
\end{frame}

\begin{frame}[fragile]
  \frametitle{Types and Patterns}

  \tgs

  Types and patterns have a nice relationship, where they both correspond to each other.

  \pause
  \vspace{\fill}

  For instance, if we wanted to have a declaration like:
  \begin{codeblock}
    val (x, y) = (* something goes here *)
  \end{codeblock}

  then it would be ill-typed to put something like \code{1}. Why?

  \pause
  \vspace{\fill}

  \textbf{Answer:} \code{1} is not a tuple! It doesn't have components to unpack.

  \pause
  \vspace{\fill}

  The fact that we have the pattern \code{(x, y)} \textit{implies} that the right-hand side
  should be of type \code{t1 * t2}, for some types \code{t1}, \code{t2}. In other words,
  it needs to be a tuple of two components.
\end{frame}

\begin{frame}[fragile]
  \frametitle{More Patterns}

  Patterns can be used for a lot more than tuples, however. Here's some other examples of patterns:

  \pause
  \vspace{\fill}

  \begin{itemize}
    \item variables, such as \code{x}, \code{y}, \code{z}. These produce a binding on a successful match.
    \item the \term{wildcard} pattern, \code{_}. This matches \textbf{all} values, and 
    produces no binding.
    \item constants, such as \code{1}, \code{"hi"}, and \code{true}
    \item tuples, where each entry is itself a pattern. For instance, \code{(_, x, 2)} is 
    a valid pattern.
  \end{itemize}
\end{frame}

\begin{frame}[fragile]
  \frametitle{Val Bindings and Patterns}

  Now, we are ready to talk about the form of a \code{val} declaration. A \code{val} declaration looks like:

  \vspace{\fill}

  \begin{codeblock}
    val <pattern> : <type> = <expr>
  \end{codeblock}

  \pause
  \vspace{\fill}

  You can put any pattern on the left-hand side of a binding! Some patterns might not type-check,
  however.
  \begin{codeblock}
    val 2 = (1, 2) (* doesn't typecheck! never runs *)
  \end{codeblock}

  \pause
  \vspace{\fill}

  If a value matches to a pattern, then it will produce some number of bindings, and then proceed
  with the program. If the value doesn't match, then an exception will be raised, for instance
  on the binding
  \begin{codeblock}
    val 2 = 1 (* doesn't match! exception raised *) 
  \end{codeblock}
\end{frame}

\begin{frame}[fragile]
  \frametitle{More Control Flow}

  We have seen some simple examples of
  functions, such as
  \begin{codeblock}
    fun double (n : int) : int = n + n
  \end{codeblock}

  \pause
  \vspace{\fill}

  But what if we want functions with more complicated control flow? We might introduce 
  \term{conditionals}.

  \begin{codeblock}
    fun isEven (n : int) : bool =
      if n mod 2 = 0 then true 
      else false 
  \end{codeblock}

  \pause
  \vspace{\fill}

  \badBox{\, Do not do this.}
\end{frame}

\begin{frame}[fragile]
  \frametitle{Conditionals}

  Hold for a brief interjection.

  \pause
  \vspace{\fill}

  In SML, we have the \term{if expression}. Note that it is an \textit{expression}. It looks like:
  \begin{codeblock}
    if <expr1> then <expr2> else <expr3>
  \end{codeblock}

  \pause
  \vspace{\fill}

  \defBox{\, The typing rule for if expressions is that for \code{if e1 then e2 else e3} : t 
  iff \code{e1 : bool}, and \code{e2 : t} and \code{e3 : t}}.

  \pause
  \vspace{\fill}

  Why? Because otherwise we could get type unsafety! Consider the following conditional
  expression:
  \begin{codeblock}
    if sasquatchIsReal then 2 else "foo"
  \end{codeblock}

  \pause
  \vspace{\fill}

  Depending on if \code{sasquatchIsReal}, we might return an \code{int}, or we might return
  a \code{string}! This is really bad, because this breaks our whole future-sight of 
  type-checking.
\end{frame}

\begin{frame}[fragile]
  \frametitle{What They Don't Want To Tell You About If Expressions, But I Will}

  Back to the warning.

  \pause
  \vspace{\fill}

  The way that \code{if e1 then e2 else e3} works is that if \code{e1} evaluates to
  \code{true}, then the whole expression evaluates to \code{e2}. Otherwise, if 
  \code{e1} evaluates to \code{false}, then the whole expression evaluates to \code{e3}.

  \pause
  \vspace{\fill}

  So let's write the computation trace of \code{isEven 2}.
  \pause
  \begin{align*}
    \code{isEven 2} &\stepsTo \code{if 2 mod 2 = 0 then true else false} & \tag{def of \code{isEven}} \\
                    &\stepsTo \code{if 0 = 0 then true else false} & \tag{def of \code{mod}} \\
                    &\stepsTo \code{if true then true else false} & \tag{def of \code{=}} \\ 
                    &\stepsTo \code{true} & \tag{if expression} 
  \end{align*}

  \pause
  \vspace{\fill}

  \textbf{This is really redundant.}

  \pause
  \vspace{\fill}

  \customBox{Lesson}{\, Never write \code{if e then true else false}!}
\end{frame}

\begin{frame}[fragile]
  \frametitle{More Complex Functions}

  \rprs

  Now that we have if expressions, we can write more complicated functions.

  \pause
  \vspace{\fill}

  \begin{codeblock}
    fun fact (n : int) : int = 
      if n = 0 then 1 
      else n * fact (n - 1)
  \end{codeblock}

  \pause
  \vspace{\fill}

  Function syntax in SML actually offers a better way of writing certain things. For instance,
  we could write:
  \begin{codeblock}
    fun fact (0 : int) : int = 1 
      | fact (n : int) : int = n * fact (n - 1)
  \end{codeblock}

  \pause
  \vspace{\fill}

  We call these \term{function clauses}.
\end{frame}

\begin{frame}[fragile]
  \frametitle{Function Clauses}

  We write function clauses in the general form as:

  \pause
  \vspace{\fill}
  
  \begin{codeblock}
    fun f (<pat1> : <ty1>) : <ty2> = <expr1> 
      | f (<pat2> : <ty1>) : <ty2> = <expr2>
      ...
  \end{codeblock}

  \pause
  \vspace{\fill}

  When a function written in this way takes in an argument, that
  argument has already been evaluated to a value.

  That value is then matched against each pattern linearly, until
  it finds a pattern that matches.

  \pause
  \vspace{\fill}

  So \code{fact 1} matches against the pattern \code{0}, fails, and
  then matches against \code{n}. This succeeds, so then we enter the case
  of \code{n * fact (n - 1)}, with the environment $[\code{1}/\code{n}]$.
\end{frame}

\begin{frame}[fragile]
  \frametitle{Function Clauses}

  Function clauses behave similarly to if expressions, in that
  they must always return the same type. Otherwise, we might get 
  type unsafety.

  \pause
  \vspace{\fill}

  Writing functions via clauses is generally more powerful than when
  using if expressions. Pattern matching is a more 
  fundamental notion than conditionals, and we will see in the next
  lecture what else we can use it for. 
\end{frame}


\sectionSlide{5}{Equivalence}

\begin{frame}[fragile]
  \frametitle{What's the point?}

  \ptmt

  We have now described the difference between variable binding in SML and 
  assignment in other languages.

  \pause
  \vspace{5pt}

  A question remains - \textbf{what's the point?}

  \pause
  \vspace{\fill}

  In the first lecture, we described some of the ideals of a good programming language,
  and one of them was \textbf{modularity}. We should be able to change a part of a 
  program without affecting another. 

  \pause
  \vspace{5pt}
  
  Remember that functions in SML are like mathematical functions, they always 
  give the same outputs for the same inputs (purity).

  \pause
  \vspace{\fill}

  Variable binding allows a stronger property, which is that \textbf{no binding after
  a function declaration can change that function's behavior}.

  \pause
  \vspace{5pt}

  In other words, \textbf{binding preserves function} \term{equivalence}.
\end{frame}

\begin{frame}[fragile]
  \frametitle{Extensional Equivalence}

  Functional programming lends itself to reasoning mathematically about code.

  \pause
  \vspace{5pt}

  To supplement that, we will have a notion of when code is equivalent.

  \pause
  \vspace{\fill}

  \defBox{\, Two expressions \textbf{of the same type} are said to be \term{extensionally equivalent} if they:}
  \begin{itemize}
    \item evaluate to the same value,
    \item both loop forever,
    \item or both raise the same kind of exception
  \end{itemize}

  \pause
  \vspace{5pt}

  We write \code{e1} $\eeq$ \code{e2} when \code{e1} and \code{e2} are extensionally equivalent.

  \vspace{\fill}

  So for instance, \code{2 + 2} $\eeq$ \code{4} $\eeq$ \code{1 + 3}. 
\end{frame}

\begin{frame}[fragile]
  \frametitle{Equivalence vs. Reduction}

  You might think that we already saw something like this.

  \pause
  \vspace{\fill}

  In the last lecture, we explored the idea of \term{reduction}, which
  is when an expression is simplified to another. For instance, we learned
  that \code{1 + 1} $\stepsTo$ \code{2}.

  \pause
  \vspace{\fill}

  Based on the definition given before, we also know that \code{1 + 1} $\eeq$ 
  \code{2}. What gives? What's the difference?

  \pause
  \vspace{\fill}

  The reason is that \textbf{reduction is stronger than equivalence}.
\end{frame}

\begin{frame}[fragile]
  \frametitle{Reduction is Stronger than Equivalence}

  To say that a condition, or conclusion, is \textit{stronger} than
  another, is to say that it implies the thing that it is stronger than.

  \pause
  \vspace{\fill}

  For instance, if some fact $A$ being true implies that $B$ must be true,
  then $A$ is a stronger condition, because knowing $A$ also gives you
  information about $B$. But, knowing $B$ doesn't necessarily tell you
  anything about $A$.

  \pause
  \vspace{\fill}

  So for instance, "I am a human" would be a stronger condition than
  "I breathe air", because if I know I am a human, then I also already
  know I breathe air.
\end{frame}

\begin{frame}[fragile]
  \frametitle{Back to Equivalence}

  What I'm getting at here is that \textit{reduction implies equivalence}.

  \pause
  \vspace{\fill}

  Any time that you know that \code{e1} $\stepsTo$ \code{e2}, you also
  know that \code{e1} $\eeq$ \code{e2}. So reduction is a stronger
  condition.

  \pause
  \vspace{\fill}

  What this means is that, whenever you are deciding which to use, know
  that you can't freely go from one to another! If you know equivalence,
  you don't necessarily know reduction. You need to take care to make sure
  you use the right notation\footnotemark.

  \footnotetext[1]{There is at least one homework problem in this class
  where the distinction is important.}
\end{frame}

\begin{frame}[fragile]
  \frametitle{A Picture of Evaluation}
  
  \makebox[\textwidth][c]{
    \begin{tikzpicture}
      \node[] (E1) at (-4.5,5) {\textellipsis};
      \node[shape=rectangle,draw=black] (T1) at (-4.5,4) {\small \code{1+(1+2)}};

      \node[] (E2) at (-2,5) {\textellipsis};
      \node[shape=rectangle,draw=black] (T2) at (-2,4) {\small \code{1+(2+1)}};

      \node[] (E3) at (0.5,5) {\textellipsis};
      \node[shape=rectangle,draw=black] (T3) at (0.5,4) {\small \code{(1+1)+2}};

      \node[] (E4) at (3.5,5) {\textellipsis};
      \node[shape=rectangle,draw=black] (T4) at (3.5,4) {\small \code{2+(1+1)}};

      \node[] (E5) at (6,5) {\textellipsis};
      \node[shape=rectangle,draw=black] (T5) at (6,4) {\small \code{(1+2)+1}};

      \node[] (E6) at (8.5,5) {\textellipsis};
      \node[shape=rectangle,draw=black] (T6) at (8.5,4) {\small \code{(2+1)+1}};

      \node[shape=rectangle,draw=black] (M1) at (-1,2) {\small \code{1 + 3}};
      \node[shape=rectangle,draw=black] (M2) at (2,2) {\small \code{2 + 2}};
      \node[shape=rectangle,draw=black] (M3) at (5,2) {\small \code{3 + 1}};
      \node[shape=rectangle,draw=black] (L) at (2,0) {\small \code{4}} ;

      \draw[->] (E1) -- (T1);
      \draw[->] (E2) -- (T2);
      \draw[->] (E3) -- (T3);
      \draw[->] (E4) -- (T4);
      \draw[->] (E5) -- (T5);
      \draw[->] (E6) -- (T6);

      \draw[->] (T1) -- (M1);
      \draw[->] (T2) -- (M1);
      \draw[->] (T3) -- (M2);
      \draw[->] (T4) -- (M2);
      \draw[->] (T5) -- (M3);
      \draw[->] (T6) -- (M3);

      \draw[->] (M1) -- (L);
      \draw[->] (M2) -- (L);
      \draw[->] (M3) -- (L);
  \end{tikzpicture}
  }

  \vspace{\fill}

  Every node\footnotemark in this graph is extensionally equivalent, but reduction 
  only flows one way!

  \footnotetext[2]{If you want to sound really fancy, say "lattice".}
\end{frame}



\begin{frame}[fragile]
  \frametitle{Referential Transparency}

  \ptmt

  It turns out that this idea gives us some very nice properties.

  \pause
  \vspace{\fill}

  \defBox{\, \term{Referential Transparency} - If \code{e1} $\eeq$ \code{e2}, then 
  replacing \code{e1} with \code{e2} anywhere in a program will produce an extensionally 
  equivalent program.
  }

  \pause
  \vspace{\fill}

  So for instance, you could replace \code{2 + 2} with \code{1 + 3} anywhere, and always
  be assured that program behavior \textbf{will not change}.
  
  \pause
  \vspace{5pt}

  In other words, you can swap "equals for equals".

  \pause
  \vspace{\fill}

  \noteBox{\, This is the foundation of \textbf{equational reasoning}, which lets us 
  reason about code like we would reason about math.}
\end{frame}

\begin{frame}[fragile]
  \frametitle{The Refactoring Lemma}

  Another way to think of referential transparency is as the
  \textbf{refactoring lemma}. Refactoring is easy, safe, and 
  mathematically guaranteed. 

  \pause
  \vspace{\fill}

  For instance, suppose that you are the engineering manager at a large
  tech firm named after a red fruit that grows on apple trees. You
  hire an intern for the summer, and you review their first PR, and you
  see: 

  \pause
  \begin{codeblock}
    if flagIsSet then (
      flagisSet andalso ( 
        if permissionsGranted then
          true
        else
          permissionsGranted 
      )
    ) else false
  \end{codeblock}

  \pause
  \vspace{\fill}

  What the hell is this.
\end{frame}

\begin{frame}[fragile]
  \frametitle{Equivalence Saves Lives}

  The intern didn't remember their equivalences!

  \pause
  \vspace{\fill}

  For instance, if we are within the \code{then} branch of
  an if expression, we know that the expression we cased on 
  must be equivalent to \code{true}. The same is true for
  the \code{else} branch and \code{false}.
   
  \vspace{\fill}

  Let's do some refactoring based on equivalences:
  \pause
  \begin{codeblock}
    if flagIsSet then (
      true andalso ( 
        if permissionsGranted then
          true
        else
          false 
      )
    ) else false
  \end{codeblock}
\end{frame}

\begin{frame}[fragile]
  \frametitle{Referential Transparency Saves Lives}

  Well, now we see \code{true andalso e}, which we know should
  always evaluate to \code{e}. So let's simplify again:

  \pause
  \vspace{\fill}

  \begin{codeblock}
    if flagIsSet then (
      if permissionsGranted then
        true
      else
        false 
    ) else false
  \end{codeblock}
\end{frame}

\begin{frame}[fragile]
  \frametitle{Referential Transparency Saves Lives}

  We also see that we have \code{if e then true else false},
  which we learned a few slides ago is a cardinal sin. So let's
  get rid of that:

  \pause
  \vspace{\fill}

  \begin{codeblock}
    if flagIsSet then permissionsGranted
    else false
  \end{codeblock}
\end{frame}

\begin{frame}[fragile]
  \frametitle{Referential Transparency Saves Lives}

  We can make a final observation that this condition is only
  \code{true} when both \code{flagIsSet} and \code{permissionsGranted}
  are true. Therefore, we can simply write it as:

  \pause
  \vspace{\fill}

  \begin{codeblock}
    flagIsSet andalso permissionsGranted
  \end{codeblock}

  \pause
  \vspace{\fill}

  \noteBox{\, Referential transparency saves lives.}

\end{frame}

\begin{frame}[fragile]
  \frametitle{On the Equivalence of Values}

  Up until this point, we've said that a value is a "final answer". You should be able
  to tell if values are the same just by looking at them, like with \code{2} and \code{2},
  or with \code{(3, 4)} and \code{(3, 4)}.

  \vspace{\fill}

  But what about functions?

  \pause
  \vspace{5pt}

  Are \code{fn (x : int) => x + x} and \code{fn (x : int) => 2 * x} extensionally equivalent?

  \pause
  \vspace{\fill}

  It's now hard to tell, because these lambda expressions are values, but it's not obvious 
  if they are extensionally equivalent or not.
\end{frame}

\begin{frame}[fragile]
  \frametitle{Extensional Equivalence for Functions}

  Functions necessitate their own rule for extensional equivalence.

  \pause
  \vspace{\fill}

  \defBox{\, Two functions \code{f : t1 -> t2} and \code{g : t1 -> t2} are \term{extensionally equivalent}
  if for all values \code{x : t1}, \code{f x} $\eeq$ \code{g x}}.

  \pause
  \vspace{5pt}

  In other words, two functions are equivalent if \textbf{for all inputs, they give equivalent outputs}.

  \pause
  \vspace{\fill}
  
  \noteBox{\, We specified that \code{f x} $\eeq$ \code{g x}, not that \code{f x} and \code{g x}
  reduce to the same value. Why?}
\end{frame}

\begin{frame}[fragile]
  \frametitle{Totality}

  The reason why is simple - some functions do not have defined outputs!

  \pause
  \vspace{\fill}

  We described functions in SML as mathematical functions, which is true, but not exactly in the
  same way as some mathematical functions, such as $+$, or $sin$.

  \pause
  \vspace{5pt}

  \textbf{SML functions can be partial}, that is, undefined on some input. There may be \code{f}
  and \code{x} such that there is no \code{v} where \code{f x} $\hookrightarrow$ \code{v}.  

  \pause
  \vspace{\fill}

  \defBox{\, We say that \code{f : t1 -> t2} is \term{total} if for all values \code{v : t1},
  there is a value \code{v' : t2} such that \code{f v} $\hookrightarrow$ \code{v'}.}

  \vspace{5pt}

  For example, the SML functions \code{+}, \code{not}, and \code{^} are all total.
\end{frame}

\begin{frame}[fragile]
  \frametitle{Specification}

  Concepts like purity, extensional equivalence, and totality are all just means to
  an end, which is being able to \textbf{specify} the behavior of code.

  \pause
  \vspace{\fill}

  In particular, for functions, we are interested in writing \textbf{descriptive} code,
  that accurately reflects the function's behavior.
  
  \pause
  \vspace{5pt}

  To that end, it is helpful to write alongside a function the conditions which 
  must be true \textbf{prior} to calling the function, and must be true \textbf{after}
  calling the function.

  \pause
  \vspace{\fill}

  \begin{codeblock}
    (* REQUIRES: x is not 0 *)
    (* ENSURES: divide x is total *)
    fun divide (x : int) : int = 2 div x 
  \end{codeblock}
\end{frame}

\begin{frame}[fragile]
  \frametitle{The Specification of Specification}

  In this class, our way of writing specifications will follow the 
  five-step methodology.

  \pause
  \vspace{\fill}

  There are five components to this methodology, shockingly:
  \pause
  \begin{enumerate}
    \item the function's type \pause
    \item the \code{REQUIRES} clause (preconditions) \pause
    \item the \code{ENSURES} clause (postconditions) \pause
    \item the function's definition \pause
    \item test cases! \pause
  \end{enumerate}

  \vspace{\fill}

  We will frequently annotate our functions this way this semester.
\end{frame}

\begin{frame}[fragile]
  \frametitle{Specification, pt. 2}

  We use comments in the form of \code{REQUIRES} and \code{ENSURES} to describe 
  what must be true of the inputs the function receives, and what then is guaranteed 
  to hold of the function's behavior. 

  \pause
  \vspace{\fill}

  It is often unrealistic to have a function which has meaning on \textit{every} possible 
  input (like \code{div}, or square root, or logarithm). The \code{REQUIRES} helps to
  describe the range of "relevant inputs", and the \code{ENSURES} helps to describe
  what the function does.

  \pause
  \vspace{\fill}

  These "contracts" pop up in real code all the time:
  \pause
  \begin{itemize}
    \item this function must be called with only safe values, \pause
    \item this library can only be invoked in single-threaded programs, \pause
    \item this API is not guaranteed to work with non-ASCII characters \pause 
  \end{itemize}

  \vspace{\fill}

  It is important that code is documented so users and maintainers know pertinent
  information about it! 
\end{frame}

\begin{frame}[fragile]
  \frametitle{Specification, pt. 3}

  The last piece of the formula is writing tests.\footnotemark

  \pause
  \vspace{\fill}

  In this class, we will use a 150-specific testing framework, where
  we write something like:
  
  \begin{codeblock}
    val () = Test.int ("test1", 2 + 2, 1 + 3)
  \end{codeblock}
  which will raise an exception if the second and third parts of the
  tuple do not evaluate to the same value.

  \vspace{\fill}

  \noteBox{\, I realize writing tests sucks. So does life, sometimes.}

  \footnotetext[3]{There's a lot I could say here about the importance
  of writing tests. Test-driven development is a real thing, and especially
  in imperative languages, it's very important to have a solid backbone
  of tests to make sure you don't accidentally regress behavior when 
  introducing a small change. However, this is a class on functional
  programming, and not software engineering, so I'll decline to comment
  for now.\footnotemark}

  \footnotetext[4]{This is a really long footnote.}
\end{frame}

\begin{frame}[fragile]
  \frametitle{Back to Fact}

  Now that we have all these tools for mathematically analyzing
  code, let's look at a specific example. Let's take the factorial 
  function that we wrote earlier.

  \vspace{\fill}

  \begin{codeblock}
    fun fact (0 : int) : int = 1
      | fact n = n * fact (n - 1)
  \end{codeblock}

  \pause
  \vspace{\fill}

  \customBox{Question}{\, Is \code{fact} total?}
\end{frame}

\begin{frame}[fragile]
  \frametitle{A Specific Specification}

  \textbf{Answer:} It is not.

  \pause
  \vspace{\fill}

  The \code{fact} function loops forever on a negative input.

  \pause
  \vspace{\fill}

  But for our intents and purposes, we don't really
  care about what \code{fact} does on negative inputs, anyways. So 
  let's restrict our domain of interest to strictly non-negative
  numbers.

  \pause
  \vspace{\fill}

  \begin{codeblock}
    (* fact : int -> int *)
    (* REQUIRES: n >= 0 *)
    (* ENSURES: fact n evaluates to the nth factorial *)
    fun fact (0 : int) : int = 1
      | fact n = n * fact (n - 1)
  \end{codeblock}

  \pause
  \customBox{Lesson}{\, Oftentimes, we are interested in only a 
  subset of the domain of a function, and we only get to
  make interesting claims about its behavior when we restrict
  our attention to it.}
\end{frame}

\begin{frame}[fragile]
  \frametitle{Conclusion}

  In this lecture, we learned a lot about some of the core
  tools that we have at our disposal when writing SML programs!

  \pause
  \vspace{\fill}

  We also learned about some of the more mathematical underpinnings
  of the language, which comes up in \term{binding and scope},
  as well as our ability to reason about code via using concepts
  like \term{referential transparency}, \term{totality}, and
  \term{specifications}.

  \pause
  \vspace{\fill}

  We also saw a little bit of the interplay between \term{patterns}
  and \term{types}. We will develop this more in the future.
\end{frame}

\begin{frame}[plain]
	\begin{center} Thank you! \end{center}

	\begin{center} 
    Post-lecture survey: \\
    \vspace{5pt}
    \includegraphics[scale=0.035]{qr_may18}
  \end{center}
\end{frame}


\end{document}

