% Jacob Neumann

% DOCUMENT CLASS AND PACKAGE USE
    \documentclass[aspectratio=169]{beamer}

    % Establish the colorlambda boolean, to control whether the lambda is solid color (true), or the same as the picture (false)
    \newif\ifcolorlambda
    \colorlambdafalse % DEFAULT: false

    % Use auxcolor for syntax highlighting
    \newif\ifuseaux
    \useauxfalse % DEFAULT: false

    % Color settings
    \useauxtrue

    \newcommand{\auxColor}{375EE9}     % the color of note boxes and stuff
    \newcommand{\presentColor}{dbb21f} % the primary color of the slide borders
    \newcommand{\bgColor}{fffbd9}      % the color of the background of the slide
    \newcommand{\darkBg}{8b98ad}
    \newcommand{\lambdaColor}{\auxColor}

    \colorlambdatrue

    \usepackage{comment} % comment blocks
    \usepackage{soul} % strikethrough
    \usepackage{listings} % code
    \usepackage{makecell}

    \setbeamertemplate{itemize items}[circle]
    % \setbeameroption{show notes on second screen=right}

    \usepackage{lectureSlides}
    %%%%%%%%%%%%%%%%%%%%%%%%%%%%%%%%%%%%%%%%%| <----- Don't make the title any longer than this
    \title{Sequences} % TODO
    \subtitle{Making our data structures parallel} % TODO
    \date{20 July 2023} % TODO
    \author{Brandon Wu} % TODO

    \graphicspath{ {./img/} }
    % DONT FORGET TO PUT [fragile] on frames with codeblocks, specs, etc.
        %\begin{frame}[fragile]
        %\begin{codeblock}
        %fun fact 0 = 1
        %  | fact n = n * fact(n-1)
        %\end{codeblock}
        %\end{frame}

    % INCLUDING codefile:
        % 1. In some file under code/NN (where NN is the lecture id num), include:
    %       (* FRAGMENT KK *)
    %           <CONTENT>
    %       (* END KK *)

    %    Remember to not put anything on the same line as the FRAGMENT or END comment, as that won't be included. KK here is some (not-zero-padded) integer. Note that you MUST have fragments 0,1,...,KK-1 defined in this manner in order for fragment KK to be properly extracted.
        %  2. On the slide where you want code fragment K
                % \smlFrag[color]{KK}
        %     where 'color' is some color string (defaults to 'white'. Don't use presentColor.
    %  3. If you want to offset the line numbers (e.g. have them start at line 5 instead of 1), use
                % \smlFragOffset[color]{KK}{5}

    \expandafter\definecolor\expandafter{myblue}{HTML}{93c8ed}
    \expandafter\definecolor\expandafter{hexcolor}{HTML}{d8c0fc}
    \usetikzlibrary {shapes.symbols}
    \usetikzlibrary {calc}
    \tikzset{
      every path/.style={line width=0.25mm},
      hex/.style={
        draw,
        inner sep=0.5pt,
        line width=0.4mm,
        fill=green!20!white,
        minimum size=0.6cm,
        align=center,
        font=\small,
        signal,
        signal to=east and west,
        signal pointer angle= 130
      },
      between/.style args={#1 and #2}{
        at = ($(#1)!0.5!(#2)$)
      },
      dot/.style={
        fill=myblue, draw=black, circle, inner sep=2pt,
      }
    }

    \newcommand{\hex}[4][]{\node[hex, #1, minimum size=1cm] (#2) {$W$: #3 \\ $S$: #4}}
    \newcommand{\fhex}[3][]{\node[hex, #1, fill=hexcolor, minimum width=0.4in] (#2) {#3}}

\begin{document}

% Make it so ./mkWeb works correctly
\ifweb
    \renewcommand{\pause}{}
\fi

\setbeamertemplate{itemize items}[circle]

% SOLID COLOR TITLE (see SETTINGS.sty)
{
\begin{frame}[plain]
    \colorlambdatrue
    \titlepage
\end{frame}
}

\menti{5146 4874}

\begin{frame}[fragile]
  \frametitle{Lesson Plan}

  \tableofcontents
\end{frame}

\sectionSlide{1}{Fundamental Data Structures}

\begin{frame}[fragile]
  \frametitle{Fundamental Data Structures}

  We previously claimed that, in computer science, there are only a few
  fundamental data structures, from which everything else pretty much
  is derived from.

  \pause
  \vspace{\fill}

  One of these items is the \code{list}. The other is the \code{tree}. Many
  things, such as priority queues, tries, queues, and sets can be implemented
  with just these two ideas.

  \pause
  \vspace{\fill}

  The third is the \term{array}, which we have given substantially less
  treatment of, so far in this course. That changes today.
\end{frame}

\begin{frame}[fragile]
  \frametitle{Arrays}

  \defBox{}{\term{Arrays} are a kind of mutable data structure that have a
  fixed number of elements, all of the same type. These elements support
  \textit{indexing}, which allows access of any given element by its position in
  the array, in constant time.}

  \pause
  \vspace{\fill}

  Arrays are usually associated with lower-level programming languages, meaning
  that they are mutable by default -- indices of the array can be rewritten and
  assigned, which will change future accesses.

  \pause
  \vspace{\fill}

  Constant time access is cool. We can skip the mutability.
\end{frame}

\begin{frame}[fragile]
  \frametitle{From Arrays to Sequences}

  \tgs

  Instead of arrays, we will discuss a similar type of data structures called
  \term{sequences}.

  \pause
  \vspace{\fill}

  \defBox{}{\term{Sequences}\footnote{Worth noting that as far as I can tell, "sequences" to refer
  to essentially immutable arrays is a CMU thing. Other people don't use this
  terminology.} are a kind of \textbf{immutable} data structure that have a
  fixed size of elements, all of the same type. These elements support
  \textit{indexing}, which allows access of any given element by its position in
  the array, in constant time.}

  \pause
  \vspace{\fill}

  The key is that sequences are not mutable, however! Changing an given element of
  a sequence entails producing a new sequence entirely. This is how we will be able
  to take advantage of the various benefits of arrays, without falling for the
  trap of mutability.

  \pause
  \vspace{\fill}

  \noteBox{}{Other than the fact that sequences are immutable, you can think of
  them as being implemented as arrays.}
\end{frame}

\begin{frame}[fragile]
  \frametitle{The Part Where I Tell You About The Inner Implementation of Sequences}

  But how are sequences implemented?

  \pause
  \vspace{\fill}

  The answer: \textbf{It's a secret}.

  \pause
  \vspace{\fill}

  No, actually. For our purposes, we are taking advantage of the module system,
  and using a library which implements sequences as an abstract type, which
  prevents us from knowing anything about the way that it is implemented under
  the hood. This means we cannot pattern-match, nor can we interact with
  sequences in any way which is not prescribed to us by the sequences interface.

  \vspace{\fill}

  The signature of the \code{Seq} module is available on the next few slides (and
  {\color{blue}\href{http://www.cs.cmu.edu/~15150/resources/libraries/sequence.pdf}{online, at this link}}):
\end{frame}

\begin{frame}[fragile]
  \frametitle{The \code{SEQUENCE} Signature}

  \small
  \begin{codeblock}
    signature SEQUENCE =
      sig
        type 'a t
        type 'a seq = 'a t (* abstract *)

        exception Range of string


        (* Constructing a Sequence *)

        val empty : unit -> 'a seq
        val singleton : 'a -> 'a seq
        val tabulate : (int -> 'a ) -> int -> 'a seq
        val fromList : 'a list -> 'a seq

        (* ... *)
  \end{codeblock}
\end{frame}

\begin{frame}[fragile]
  \frametitle{The \code{SEQUENCE} Signature}

  \small
  \begin{codeblock}
    (* ... *)
        (* Deconstructing a Sequence *)

        val nth : 'a seq -> int -> 'a
        val null : 'a seq -> bool
        val length : 'a seq -> int
        val toList : 'a seq -> 'a list
        val toString : ('a -> string) -> 'a seq -> string
        val equal : ('a * 'a -> bool) -> 'a seq * 'a seq -> bool

        (* Simple Transformations *)

        val rev : 'a seq -> 'a seq
        val append : 'a seq * 'a seq -> 'a seq
        val flatten : 'a seq seq -> 'a seq
        val cons : 'a -> 'a seq -> 'a seq
    (* ... *)
  \end{codeblock}
\end{frame}

\begin{frame}[fragile]
  \frametitle{The \code{SEQUENCE} Signature}

  \tiny
  \begin{codeblock}
    (* ... *)
      (* Combinators and Higher - Order Functions *)

      val filter : ('a -> bool) -> 'a seq -> 'a seq
      val map : ('a -> 'b) -> 'a seq -> 'b seq
      val reduce : ('a * 'a -> 'a) -> 'a -> 'a seq -> 'a
      val reduce1 : ('a * 'a -> 'a) -> 'a seq -> 'a
      val mapreduce : ('a -> 'b) -> 'b -> ('b * 'b -> 'b) -> 'a seq -> 'b
      val zip : ('a seq * 'b seq) -> ('a * 'b) seq
      val zipWith : ('a * 'b -> 'c) -> 'a seq * 'b seq -> 'c seq

      (* Indexing-Related Functions *)
      val enum : 'a seq -> (int * 'a) seq
      val mapIdx : (int * 'a -> 'b) -> 'a seq -> 'b seq
      val update : ('a seq * (int * 'a)) -> 'a seq
      val inject : 'a seq * (int * 'a) seq -> 'a seq

      val subseq : 'a seq -> int * int -> 'a seq
      val take : 'a seq -> int -> 'a seq
      val drop : 'a seq -> int -> 'a seq
      val split : 'a seq -> int -> 'a seq * 'a seq

      (* Sorting and Searching *)

      val sort : ('a * 'a -> order) -> 'a seq -> 'a seq
      val merge : ('a * 'a -> order) -> 'a seq * 'a seq -> 'a seq
      val search : ('a * 'a -> order) -> 'a -> 'a seq -> int option
      (* ... *)
    end
  \end{codeblock}
\end{frame}

\begin{frame}[fragile]
  \frametitle{Taking It In}

  OK, that's a lot. Moreover, there isn't any description on what each
  function does!\footnote{There is at the online reference, though.}

  \vspace{\fill}

  If you glance at the names of each function, however, it doesn't look
  terribly different than the list library (though a little better equipped).
  What gives?

  \pause
  \vspace{\fill}

  Sequences don't offer us anything that can't be done with lists, in terms
  of their structure. We could very well be using lists instead. The difference
  will be in each function's \textbf{cost}, as sequences admit a different
  cost model.

  \pause
  \vspace{\fill}

  \keyBox{}{In particular, \textbf{sequences are very parallelizable}.}

  \pause
  \vspace{\fill}

  For instance, we can execute \code{Seq.map f} within constant time, for a
  constant function \code{f}.
\end{frame}

\begin{frame}[fragile]
  \frametitle{Advantages of Sequences}

  We mentioned previously that sequences will admit constant time access to
  each element, whereas in a list you must perform $O(i)$ work to access the
  $i$th element.

  \pause
  \vspace{\fill}

  We can implement the \code{nth} function for lists as such:

  \begin{codeblock}
    fun nth ([], 0)   = raise Subscript
      | nth (x::_, 0) = x
      | nth (x::xs, n) = nth (xs, n - 1)
  \end{codeblock}

  \pause
  \vspace{\fill}

  Other advantages of sequences are that length can be computed in constant time,
  and that they are parallel-friendly, meaning that bulk operations (like
  \code{map}, \code{fold}, and \code{filter}) can be done
  without having to always pay a linear cost up front.
\end{frame}

\begin{frame}[fragile]
  \frametitle{Disadvantages of Sequences}

  For all those advantages, however, we have some disadvantages that come as a
  consequence. The most prominent one, from a programming standpoint, is that
  you cannot pattern match upon a sequence. This means that something as simple
  as the following, with lists:

  \begin{codeblock}
    case L of
      []    => (* 1 *)
    | x::xs => (* 2 *)
  \end{codeblock}
\end{frame}

\begin{frame}[fragile]
  \frametitle{Disadvantages of Sequences}

  ...requires the following, for a sequence \code{S}:
  \begin{codeblock}
    case Seq.length S of
      0 => (* 1 *)
    | _ =>
      let
        val (x, xs) = (Seq.nth S 0, Seq.drop S 1)
      in
        (* 2 *)
      end
  \end{codeblock}

  \pause
  \vspace{\fill}

  Tradition dictates that I mention: \textit{this is disgusting}.\footnote{Time
  permitting, at the end of this lecture we can discuss a nicer way of doing
  this. The point stands.}
\end{frame}

\begin{frame}[fragile]
  \frametitle{Disadvantages of Sequences}

  The other disadvantage of sequences is the same as its primary benefit:
  \textit{sequences are designed with
  parallelism in mind}. This means that bulk operations, as in, operations
  which require dealing with a fixed number of elements all at once, are
  very easy, but sequential operations are slow.

  \pause
  \vspace{\fill}

  This is most salient through the fact that \textbf{cons is expensive}.
  For sequences, consing an element onto the sequence takes linear time in
  the length of the sequence, as opposed to constant time for lists.

  \pause
  \vspace{\fill}

  \customBox{Key Fact}{\, The root cause for this is that when you cons an element onto a sequence,
  you must create a brand new sequence, meaning you must copy over everything
  that was in the old sequence! This is an easy linear cost.\footnote{Usually, talk of things like "copying" and "memory" are
  beneath us. But this is a case where it actually matters, because it shows
  up in our cost bound.}}
\end{frame}

\begin{frame}[fragile]
  \frametitle{Sequence Basics}

  When discussing sequences, we will usually textually represent them using
  the mathematical notation $\langle x_1, x_2, ..., x_n \rangle$, for the
  sequence with elements $x_1, ..., x_n$ at each index.

  \pause
  \vspace{\fill}

  For instance, we might write that \code{Seq.map f }$\,\, \langle x_1, x_2, ... x_n \rangle$
  has the cost of $\max_{x_i \in S}(W($\code{f xi}$))$, meaning the
  maximum cost over any given application of \code{f} to an element of
  the sequence, and evaluates to the
  sequence $\langle \code{f x1}, \code{f x2}, ..., \code{f xn} \rangle$.

  \pause
  \vspace{\fill}

  Before we can do a deeper dive into the cost of sequences, we need to come
  up with a conceptual model of how to think about the cost of sequence
  functions.

  \pause
  \vspace{\fill}

  To that end, we need to discuss \term{cost graphs}.
\end{frame}

\sectionSlide{2}{Cost Graphs}

\begin{frame}[fragile]
  \frametitle{The Cost of Sequences}

  We use \term{cost graphs} to visually reason about the cost of a given
  operation of a sequence. Although we don't know specifically the underlying
  implementation of sequences, cost graphs still give us a way of reasoning
  about their cost.

  \pause
  \vspace{\fill}

  \defBox{}{A \term{cost graph} is a graph which indicates the cost of
  performing a certain function, but with visual indications for when
  parallelism can be achieved.}

  \pause
  \vspace{\fill}

  This is exactly the same as the idea of task dependency graphs, from
  the previous lecture on asymptotic analysis! We will boil in a few more
  sequence-specific things, however.
\end{frame}

\begin{frame}[fragile]
  \frametitle{Fundamentals of Cost Graphs}

  \ptmt

  \begin{center}
  \begin{minipage}[t]{0.32\textwidth}
    \centering
    \begin{tikzpicture}
      \node (G1) at (0,0) {[graph 1]};
      \node (G2) at (0,2) {[graph 2]};
      \draw (G1) -- (G2);
    \end{tikzpicture}

    \vspace{10pt}

    \term{sequential composition} of cost graphs
  \end{minipage}
  \hfill
  \begin{minipage}[t]{0.32\textwidth}
    \centering
    \begin{tikzpicture}
      \node (G1) {\small[graph 1]};
      \node (G2) [right of = G1, node distance=1in] {\small[graph 2]};
      \node[dot, between=G1 and G2, yshift=1cm] (J1) {};
      \node[dot, between=G1 and G2, yshift=-1cm] (J2) {};
      \draw[shorten >= 5pt] (G1) -- (J1);
      \draw[shorten >= 5pt] (G2) -- (J1);
      \draw[shorten >= 5pt] (G1) -- (J2);
      \draw[shorten >= 5pt] (G2) -- (J2);
    \end{tikzpicture}

    \vspace{10pt}

    \term{parallel composition} of cost graphs
  \end{minipage}
  \hfill
  \begin{minipage}[t]{0.32\textwidth}
    \centering
    \begin{tikzpicture}
      \fhex[yshift=1cm]{UA}{\code{f}};
      % \hex[yshift=1.15cm] {A}{work}{span};
      \node at (0, 0) {};
    \end{tikzpicture}

    \vspace{10pt}

    node denoting computation of \code{f}
  \end{minipage}
  \end{center}

  \vspace{\fill}

  Cost graphs are \term{inductively defined} by these three constructs.

  \pause
  \vspace{\fill}

  Cost graphs always run from top to bottom, so while we can view them as directed,
  we will omit the arrows for brevity. Every graph also has a \term{source} and
  \term{sink}, which denote the starting and ending positions of executing that
  cost graph.
\end{frame}

\begin{frame}[fragile]
  \frametitle{Sources and Sinks}

  For the sequential case, the source and sink are just the source and sink of
  graphs 1 and 2, respectively.

  \pause
  \vspace{\fill}

  For the parallel case, the source is the originating dot, which then
  \term{forks} to perform graph 1 and graph 2 in parallel, before \term{joining}
  back together.

  \pause
  \vspace{\fill}

  The single node is itself both source and sink.
\end{frame}

\begin{frame}[fragile]
  \frametitle{More on Cost Graphs}

  \begin{center}
  \begin{minipage}{0.5\textwidth}
    \raggedright

  Cost graphs can be mixed and matched and put together! This means that
  although forking has a source and sink which are just before
  and after the two graphs in between, it can be sequentially merged with
  other graphs.

  \vspace{10pt}

    So for instance, we might say that the cost graph of the computation
    \code{(1 + 2) * (3 + 4)} has cost graph:
  \end{minipage}
  \begin{minipage}{0.49\textwidth}
    \centering
    \begin{tikzpicture}[scale=0.6]
      \fhex {G1} {\code{+}};
      \fhex[right of = G1, node distance=1.5in] {G2} {\code{+}};
      \node[dot, between=G1 and G2, yshift=1cm] (J1) {};
      \fhex[between=G1 and G2, yshift=-1cm] {G3} {\code{*}};
      \draw[shorten >= 5pt] (G1.north) -- (J1);
      \draw[shorten >= 5pt] (G2.north) -- (J1);
      \draw (G1.south) -- (G3);
      \draw (G2.south) -- (G3);
    \end{tikzpicture}
  \end{minipage}
  \end{center}

  \pause
  \vspace{10pt}

  This comes from \term{forking} to perform each addition in parallel
  (in constant work and span), before \term{joining} to then perform
  the multiplication, also in constant work and span. Note that each
  purple node here has a constant cost.
\end{frame}

\begin{frame}[fragile]
  \frametitle{Computation Nodes and Cost Nodes}

  \begin{center}
  \begin{minipage}{0.5\textwidth}
    \raggedright

    Once we know the cost of a particular computation node, we might also
    replace it with a work/span node (or \textbf{cost node}), like in the resulting graph:

    \vspace{10pt}

    Now, we can easily see that the work of this graph is just a constant $O(1)$,
    and the span is also.
  \end{minipage}
  \begin{minipage}{0.49\textwidth}
    \centering
    \begin{tikzpicture}[scale=0.6]
      \hex {G1} {$O(1)$} {$O(1)$};
      \hex[right of = G1, node distance=1.5in] {G2} {$O(1)$}{$O(1)$};
      \node[dot, between=G1 and G2, yshift=1cm] (J1) {};
      \hex[between=G1 and G2, yshift=-1.5cm] {G3} {$O(1)$}{$O(1)$};
      \draw[shorten >= 5pt] (G1.north) -- (J1);
      \draw[shorten >= 5pt] (G2.north) -- (J1);
      \draw (G1.south) -- (G3);
      \draw (G2.south) -- (G3);
    \end{tikzpicture}
  \end{minipage}
  \end{center}
\end{frame}

\begin{frame}[fragile]
  \frametitle{More on Cost Graphs}

  \begin{center}
  \begin{minipage}{0.65\textwidth}
    \raggedright
    As before, the work of the entire cost graph is simply the work of each
    node, summed up. The span of a cost graph is the greatest cost among
    paths from the source to the sink.

    \vspace{10pt}

    Actually, we don't necessarily need labeled cost nodes, as in the last
    slide, as we can rederive it from just the parallel and sequential
    composition cases. For instance, we could rederive this node:

    \vspace{5pt}

    \begin{center}
      \begin{tikzpicture}
        \hex{}{$O(n)$}{$O(n)$};
      \end{tikzpicture}
    \end{center}

   \vspace{5pt}

   as the cost graph on the right:
  \end{minipage}
  \hfill
  \begin{minipage}{0.34\textwidth}
    \centering
    \begin{tikzpicture}
      \node[dot] (N1) {};
      \node[dot, below of=N1] (N2) {};
      \node[below of=N2, node distance=0.5in] (N3) {[$O(n)$ many more nodes]};
      \node[dot, below of=N3, node distance=0.5in] (N4) {};
      \node[dot, below of=N4] (N5) {};

      \draw[shorten >= 5pt, shorten <= 5pt] (N1) -- (N2);
      \draw[shorten <= 5pt] (N2) -- (N3);
      \draw[shorten >= 5pt] (N3) -- (N4);
      \draw[shorten >= 5pt, shorten <= 5pt] (N4) -- (N5);
    \end{tikzpicture}
  \end{minipage}
  \end{center}
\end{frame}

\begin{frame}[fragile]
  \frametitle{More on Cost Graphs}

  We will also use the following shorthand for many simultaneously
  forking paths:

  \begin{center}
    \begin{tikzpicture}
      \node (N1) {(graph 1)};
      \node[right of=N1, node distance=1in] (N2) {[graph 2]};
      \node[right of=N2, node distance=2in] (N3) {[graph $n - 1$]};
      \node[between=N2 and N3] (C) {$\cdots$};
      \node[right of=N3, node distance=1in] (N4) {[graph $n$]};
      \node[dot, between=N1 and N4, yshift=0.35in] (J1) {};
      \node[dot, between=N1 and N4, yshift=-0.35in] (J2) {};

      \draw[shorten >= 5pt] (N1.north) -- (J1);
      \draw[shorten >= 5pt] (N2.north) -- (J1);
      \draw[shorten >= 5pt] (N3.north) -- (J1);
      \draw[shorten >= 5pt] (N4.north) -- (J1);
      \draw[shorten >= 5pt] (N1.south) -- (J2);
      \draw[shorten >= 5pt] (N2.south) -- (J2);
      \draw[shorten >= 5pt] (N3.south) -- (J2);
      \draw[shorten >= 5pt] (N4.south) -- (J2);
    \end{tikzpicture}
  \end{center}

  \pause
  \vspace{\fill}

  This could be desugared as just many parallel composed graphs on top of each
  other, but nobody has time for that.

  \pause
  \vspace{\fill}

  When all of the subgraphs are themselves constant cost nodes, this ends up being
  the same as the cost node:
  \begin{center}
    \begin{tikzpicture}
      \hex{}{$O(n)$}{$O(1)$};
    \end{tikzpicture}
  \end{center}
\end{frame}

\begin{frame}[fragile]
  \frametitle{Cost Graphs, Concluded}

  That's pretty much all you need to know about cost graphs.

  \pause
  \vspace{\fill}

  It's worth noting that all of the ideas present in this section
  are strictly conceptually interesting, and the actual encoding doesn't
  matter. You will not be tested on your ability to faithfully render a
  cost graph exactly, so long as you are approximately correct. Don't
  worry too much about the smaller details.

  \pause
  \vspace{\fill}

  Now, we can move on to discussion of sequence functions in general.
\end{frame}

\quizBreak{COPLANAR}

\sectionSlide{3}{Sequence Functions}

\begin{frame}[fragile]
  \frametitle{Sequence Functions}

  We don't have nearly enough time to go through every sequence function,
  so we won't. Fortunately, many functions within the sequence library
  are themselves derivable in terms of others, meaning that we only need
  to discuss a few fundamental functions to be able to understand the
  whole library.

  \pause
  \vspace{\fill}

  For our purposes, the interesting functions to note will be
  \code{Seq.tabulate}, \code{Seq.map}, \code{Seq.filter}, and \code{Seq.reduce}.
  We will also have some brief notes on some other sequence functions, and their
  cost.

  \pause
  \vspace{\fill}

  \noteBox{}{We will occasionally make reference to values as though we are
  inside of the \code{Seq} structure. That means we would ordinarily write
  \code{Seq.map} or \code{'a Seq.seq}, but for brevity I will write \code{map}
  and \code{'a seq}.}
\end{frame}

\begin{frame}[fragile]
  \frametitle{Sequence \code{tabulate}}

  \spec
    {tabulate}
    {(int -> 'a) -> int -> 'a seq}
    {For all $0 \leq \code{i} < \code{n}$, \code{f i} is valuable.}
    {\code{tabulate f n} evaluates to $\langle \code{f 0}, \, \code{f 1}, \,..., \,\code{f (n - 1)}\rangle$}

  \vspace{\fill}

  This function actually exists for lists too, but it's rarely used, and not very
  parallelizable.

  \pause
  \vspace{\fill}

  This specification means that, for instance, \code{Seq.tabulate (fn x => x) n}
  is equivalent to the sequence $\langle \code{0}, \code{1}, ..., \code{n - 1} \rangle$.
\end{frame}

\begin{frame}[fragile]
  \frametitle{Sequence \code{tabulate}: Cost}

  Cost graph:

  \begin{center}
    \begin{tikzpicture}
      \fhex{N1}{\code{f 0}};
      \fhex[right of=N1, node distance=1in] {N2} {\code{f 1}};
      \fhex[right of=N2, node distance=2in] {N3} {\code{f (n-2)}};
      \node[between=N2 and N3] (C) {$\cdots$};
      \fhex[right of=N3, node distance=1in] {N4} {\code{f (n-1)}};
      \node[dot, between=N1 and N4, yshift=0.4in] (J1) {};
      \node[dot, between=N1 and N4, yshift=-0.4in] (J2) {};

      \draw[shorten >= 5pt] (N1.north) -- (J1);
      \draw[shorten >= 5pt] (N2.north) -- (J1);
      \draw[shorten >= 5pt] (N3.north) -- (J1);
      \draw[shorten >= 5pt] (N4.north) -- (J1);
      \draw[shorten >= 5pt] (N2.south) -- (J2);
      \draw[shorten >= 5pt] (N1.south) -- (J2);
      \draw[shorten >= 5pt] (N3.south) -- (J2);
      \draw[shorten >= 5pt] (N4.south) -- (J2);
    \end{tikzpicture}
  \end{center}

  \pause
  \vspace{\fill}

  Recall that this cost graph indicates that each of the calls to \code{f}
  can be parallelized! This means that, for instance, given a constant time
  function \code{f}, \code{Seq.tabulate f n} is $O(n)$ work and $O(1)$ span.

  \pause
  \vspace{\fill}

  This is much improved over the $O(n)$ span for the list equivalent. The
  main difference is that for lists, the list must be made
  by consing on elements repeatedly. For sequences, the entire sequence can be
  created at once, with each part independent, without any sequential operations
  at all.
\end{frame}

\begin{frame}[fragile]
  \frametitle{Sequence \code{nth}}

  As promised before, one of the most important functions for sequences
  will be \code{nth}, which allows constant time access to any given element
  of the sequence.

  \pause
  \vspace{\fill}

    \spec
      {nth}
      {'a seq -> int -> 'a}
      {\code{true}}
      {\code{nth S i} evaluates to the \code{i}th element of \code{S}.
      If \code{i} is negative or greater than or equal to \code{length S},
      then \code{Range} is raised.}

  \pause
  \vspace{\fill}

  \begin{center}
  \begin{minipage}{0.7\textwidth}
    The cost graph just looks like:

    \vspace{20pt}

    It has exactly one edge, denoting that there is only one operation, and
    it runs in $O(1)$ work and span.
  \end{minipage}
  \hfill
  \begin{minipage}{0.25\textwidth}
    \centering
    \begin{tikzpicture}
      \node[dot] (N1) {};
      \node[dot, below of=N1, node distance=0.8in] (N2) {};
      \draw[shorten >= 5pt, shorten <= 5pt] (N1) -- (N2);
    \end{tikzpicture}
  \end{minipage}
  \end{center}
\end{frame}

\begin{frame}[fragile]
  \frametitle{Sequence \code{length}}

  Similarly, sequences promise a length function in $O(1)$ work and span as
  well. This is achieved by just storing the length of the sequence within
  the sequence itself.

  \pause
  \vspace{\fill}

  \spec
    {length}
    {'a seq -> int}
    {\code{true}}
    {\code{length} $\langle \code{x}_0,\, \code{x}_1,\, ...,\, \code{x}_{n - 1} \rangle$
    $\eeq$ \code{n}}

  \pause
  \vspace{\fill}

  \begin{center}
  \begin{minipage}{0.7\textwidth}
    The cost graph looks the same as \code{nth}:
  \end{minipage}
  \hfill
  \begin{minipage}{0.25\textwidth}
    \centering
    \begin{tikzpicture}
      \node[dot] (N1) {};
      \node[dot, below of=N1, node distance=0.8in] (N2) {};
      \draw[shorten >= 5pt, shorten <= 5pt] (N1) -- (N2);
    \end{tikzpicture}
  \end{minipage}
  \end{center}
\end{frame}

\begin{frame}[fragile]
  \frametitle{Sequence \code{map}}

  \spec
    {map}
    {('a -> 'b) -> 'a seq -> 'b seq}
    {$\code{f x}_i$ is valuable for all elements $\code{x}_i$ in the input sequence}
    {\code{map f} $\langle \code{x}_0,\, \code{x}_1,\, ...,\, \code{x}_{n - 1}\rangle$ $\eeq$
    $\langle \code{f x}_0,\, \code{f x}_1,\, ...,\, \code{f x}_{n - 1}\rangle$}

  \pause
  \vspace{\fill}

  \code{map} will end up having an identical cost profile to \code{tabulate}, but
  this ends up being because we don't need special case it in terms of our
  definitions, because we can derive it in terms of the functions we've already seen!

  \pause
  \vspace{\fill}

  \begin{codeblock}
    fun map f S = `tabulate` (fn i => f (`nth` S i)) S
  \end{codeblock}

  \pause
  \vspace{\fill}

  Namely, \code{tabulate} and \code{nth}.
\end{frame}

\begin{frame}[fragile]
  \frametitle{Sequence \code{map}: Cost}

  How do we analyze the cost of \code{map}? Well, intuitively it should be
  the same, but what do we say about the cost graph?

  \pause
  \vspace{\fill}

  The \code{f} function in the expression
  \code{tabulate (fn i => f (nth S i)) S} is just the lambda \code{fn i => f (nth S i)},
  which is the sequential combination of the call to \code{nth}, and the call to
  \code{f}. This means that our cost graph actually looks like:\footnote{Actually, I have simplified notation a bit. There should
  technically be another edge above each purple node, but I have just contracted
  a constant number of edges into one.}

  \pause
  \vspace{\fill}

  \begin{center}
    \begin{tikzpicture}
      \fhex{N1}{\code{f 0}};
      \fhex[right of=N1, node distance=1in] {N2} {\code{f 1}};
      \node[between=N2 and N3] (C) {$\cdots$};
      \fhex[right of=N2, node distance=2in] {N3} {\code{f (n-2)}};
      \fhex[right of=N3, node distance=1in] {N4} {\code{f (n-1)}};

      \node[dot, above of=N1, node distance=0.5in] (U1) {};
      \node[dot, above of=N2, node distance=0.5in] (U2) {};
      \node[dot, above of=N3, node distance=0.5in] (U3) {};
      \node[dot, above of=N4, node distance=0.5in] (U4) {};

      \node[dot, between=N1 and N4, yshift=0.9in] (J1) {};
      \node[dot, between=N1 and N4, yshift=-0.4in] (J2) {};

      \draw[shorten >= 5pt] (N1.south) -- (J2);
      \draw[shorten >= 5pt] (N2.south) -- (J2);
      \draw[shorten >= 5pt] (N3.south) -- (J2);
      \draw[shorten >= 5pt] (N4.south) -- (J2);

      \draw[shorten <= 5pt] (U1) -- (N1);
      \draw[shorten <= 5pt] (U2) -- (N2);
      \draw[shorten <= 5pt] (U3) -- (N3);
      \draw[shorten <= 5pt] (U4) -- (N4);

      \draw[shorten >= 5pt, shorten <= 5pt] (U1) -- (J1);
      \draw[shorten >= 5pt, shorten <= 5pt] (U2) -- (J1);
      \draw[shorten >= 5pt, shorten <= 5pt] (U3) -- (J1);
      \draw[shorten >= 5pt, shorten <= 5pt] (U4) -- (J1);
    \end{tikzpicture}
  \end{center}
\end{frame}

\begin{frame}[fragile]
  \frametitle{Sequence \code{map}: Cost}

  This produces the same cost, because a constant addition to each path does
  not alter the asymptotic complexity of the function.

  \pause
  \vspace{\fill}

  The key conceptual understanding here is that the way to derive this cost
  graph, is simply to compose the cost graphs of \code{nth} and the original
  \code{f}, and then substitute it into the cost graph of \code{tabulate}!

  \pause
  \vspace{\fill}

  These kinds of nested computations are common with sequences, and it's
  important to be able to accurately derive their cost.
\end{frame}

\begin{frame}[fragile]
  \frametitle{A Fold for Sequences}

  The most fundamental operator we would like to analyze is that of
  folding on sequences.

  \pause
  \vspace{\fill}

  We are familiar with left folds and right folds for lists, and for
  sequences it turns out we are going to have a natural equivalent. There
  will be some differences, however, to take advantage of the parallel
  nature of sequences.

  \pause
  \vspace{\fill}

  The standard specification of \code{foldl} is that \code{foldl f acc [x1, ..., xn]}
  is equivalent to \code{f (xn, ..., f (x2, f (x1, acc)) ... )}. \code{foldl}
  achieves this by literally computing that expression in sequence\footnote{It's funny how "in sequence" ends up being the worst case
  for sequences. It's even funnier how a data structure specifically built for
  parallelism ended up named after the word "sequential".},
  but this is a naturally $O(n)$ work and span operation! We can't possibly
  do any better than sequentially march along, because there's a giant data
  dependency, as each computation depends on its inner consistuents.
\end{frame}

\begin{frame}[fragile]
  \frametitle{A Fold for Sequences}

  But how will we go about producing this fold for sequences, in a way that
  takes advantage of parallelism? We can't avoid this data dependency.

  \pause
  \vspace{\fill}

  Indeed, we aren't going to be able to avoid the fact that each outer
  \code{f} depends on the inner ones, but we can change the problem statement
  slightly.

  \pause
  \vspace{\fill}

  By analogy, let's consider an operation like addition. If we were summing
  all of the elements of a list, must this take $O(n)$ span?

  \pause
  \vspace{\fill}

  The answer: \textbf{No, it does not}!
\end{frame}

\begin{frame}[fragile]
  \frametitle{An Additive Example}

  Suppose we were summing a list of elements, and that list is
  \code{[1, 2, 3, 4, 5, 6, 7, 8]}. There's a closed form, but pretend
  we weren't actually privy to the contents of the list.

  \pause
  \vspace{\fill}

  The naive way to do this is to march from left to right and compute the sum:

  \pause
  \vspace{-15pt}

  \begin{align*}
    & 1 + 2 + 3 + 4 + 5 + 6 + 7 + 8 \\
    &= 3 + 3 + 4 + 5 + 6 + 7 + 8 \\
    &= 6 + 4 + 5 + 6 + 7 + 8 \\
    &= 10 + 5 + 6 + 7 + 8 \\
    &= 15 + 6 + 7 + 8 \\
    &= 21 + 7 + 8 \\
    &= 28 + 8 \\
    &= 36
  \end{align*}

  \pause
  \vspace{\fill}

  How exhausting.
\end{frame}

\begin{frame}[fragile]
  \frametitle{An Additive Example}

  \ptmt

  But, let's give the perspective a switch. What if, instead, we computed the
  sum of the following:
  \pause

  $$((1 + 2) + (3 + 4)) + ((5 + 6) + (7 + 8))$$

  \pause
  \vspace{\fill}

  It's the same, but now \textit{everything} is different. The different
  components of the addition can be done in parallel.

  \begin{align*}
    & ((1 + 2) + (3 + 4)) + ((5 + 6) + (7 + 8)) \\
    &= (3 + 7) + (11 + 15) \\
    &= 10 + 26 \\
    &= 36
  \end{align*}
\end{frame}

\begin{frame}[fragile]
  \frametitle{An Additive Example}

  Now three steps, when previously it took seven. All because of a simple
  reparenthesization.

  \pause
  \vspace{\fill}

  What gives? The key distinction is that, for lists, the trace on the previous
  slide is \textit{still} the best you can do, in parallel, because lists are
  sequential. With sequences, we are going to exploit its parallel properties to
  have the freedom to "reparenthesize" in this manner.
\end{frame}

\begin{frame}[fragile]
  \frametitle{A Subtractive Example}

  Is this always possible, though? Consider if we were simply folding from
  left to right with the subtraction operator, on a smaller list of \code{[1, 2, 3, 4]}.

  \pause
  \vspace{\fill}

  Then, \code{foldl} should produce:
  \pause
  \begin{align*}
    & \code{foldl (op-) 0 [1, 2, 3, 4]} \\
    &\eeq\code{foldl (op-) (0 - 1) [2, 3, 4]} \\
    &\eeq\code{foldl (op-) (\~1) [2, 3, 4]} \\
    &\eeq\code{foldl (op-) (\~1 - 2) [3, 4]} \\
    &\eeq\code{foldl (op-) (\~3) [3, 4]} \\
    &\eeq\code{foldl (op-) (\~3 - 3) [4]} \\
    &\eeq\code{foldl (op-) (\~6) [4]} \\
    &\eeq\code{foldl (op-) (\~6 - 4) []} \\
    &\eeq\code{foldl (op-) (\~10) []} \\
    &\eeq\code{\~10}
  \end{align*}
\end{frame}

\begin{frame}[fragile]
  \frametitle{A Subtractive Example}

  Let's try the same trick:
  \pause
  \begin{align*}
    & (1 - 2) - (3 - 4) \\
    &= (-1) - (-1) \\
    &= 0
  \end{align*}
  \pause
  ...What just happened?

  \pause
  \vspace{\fill}

  \keyBox{}{Addition is not the same as subtraction.}

  \pause
  \vspace{\fill}

  More particularly, addition is different than subtraction via a mathematical
  property that allows it to employ this "reparenthesization" and still be
  equivalent. This property is \term{associativity}.
\end{frame}

\begin{frame}[fragile]
  \frametitle{Associativity}

  \defBox{}{We say a binary operation $\odot$ of type \code{t * t -> t} is
  \term{associative} if, for all \code{x, y, z} of type \code{t}, we have
  that \code{x} $\odot$ (\code{y} $\odot$ \code{z}) is equivalent to
  (\code{x} $\odot$ \code{y}) $\odot$ \code{z}.
  }

  \pause
  \vspace{\fill}

  You can also think of it as, in a long chain of binary operations, if it
  is associative, then you can place parentheses wherever you want, and
  \textit{reassociate} the operations, and still get the same thing out.
  They must remain in the same order, however.

  \pause
  \vspace{\fill}

  Addition is thus associative, since it doesn't matter which side of the
  addition you do first. For subtraction, it does, so we can't use this
  tactic.

  \pause
  \vspace{\fill}

  We will thus restrict our attention solely to those operations which
  are associative, for the purpose of our folding function. Since it's
  not really a "fold" in the same sense as we saw for lists, we will call
  it another name: \term{reduce}.
\end{frame}

\begin{frame}[fragile]
  \frametitle{Sequence \code{reduce}}

  Here's our spec for \code{reduce}:

  \pause
  \spec
    {reduce}
    {('a * 'a -> 'a) -> 'a -> 'a seq -> 'a}
    {\code{g} is both total and associative}
    {\code{reduce g z S} is equivalent to \code{foldr g z L}, where \code{L}
    is the corresponding list to the sequence \code{S}}

  \pause
  It's implemented spiritually in the same way as the algoritm we just described,
  for fast computation of the sum of a list. We pair up the elements if we can,
  and apply the binary operation to them in parallel. The base case of \code{z}
  is also placed at the very end, at the right of the sequence.

  \pause
  \vspace{\fill}

  We then repeat this process until we get a final value.
\end{frame}

\begin{frame}[fragile]
  \frametitle{Sequence \code{reduce}: Cost}

  \rprs

  What's our cost graph look like? It actually looks similar to the computation
  trace:
  \pause

  \begin{center}
    \begin{tikzpicture}
      \node[dot] (N1) {};
      \node[dot, right of=N1, node distance=0.7in] (N2) {};
      \node[dot, right of=N2, node distance=0.7in] (N3) {};
      \node[dot, right of=N3, node distance=0.7in] (N4) {};

      \node[dot, right of=N4, node distance=1in] (N5) {};
      \node[dot, right of=N5, node distance=0.7in] (N6) {};
      \node[dot, right of=N6, node distance=0.7in] (N7) {};
      \node[dot, right of=N7, node distance=0.7in] (N8) {};

      \fhex[between=N1 and N2, yshift=-0.4in] {M1} {\code{g}};
      \fhex[between=N3 and N4, yshift=-0.4in] {M2} {\code{g}};
      \fhex[between=N5 and N6, yshift=-0.4in] {M3} {\code{g}};
      \fhex[between=N7 and N8, yshift=-0.4in] {M4} {\code{g}};

      \fhex[between=M1 and M2, yshift=-0.2in] {L1} {\code{g}};
      \fhex[between=M3 and M4, yshift=-0.2in] {L2} {\code{g}};
      \node[below of=L1, xshift=0.4in] (A1) {$\cdots$};
      \node[below of=L2, xshift=-0.4in] (A2) {$\cdots$};

      \node[dot, between=N4 and N5, yshift=0.4in] (J1) {};
      \node[between=N4 and N5] (J2) {$\cdots$};
      \node[between=N4 and N5, yshift=-0.5in] (J3) {$\cdots$};

      \fhex[below of=M2, yshift=-0.8in, xshift=0.4in] {LL1} {\code{g}};
      \fhex[below of=M3, yshift=-0.8in, xshift=-0.4in] {LL2} {\code{g}};
      \fhex[between=LL1 and LL2, yshift=-0.4in] {B} {\code{g}};

      \node[between=A1 and A2] (C) {(lots of stuff here)};

      \draw[shorten >= 5pt, shorten <= 5pt] (N1.north) -- (J1);
      \draw[shorten >= 5pt, shorten <= 5pt] (N2.north) -- (J1);
      \draw[shorten >= 5pt, shorten <= 5pt] (N3.north) -- (J1);
      \draw[shorten >= 5pt, shorten <= 5pt] (N4.north) -- (J1);
      \draw[shorten >= 5pt, shorten <= 5pt] (N5.north) -- (J1);
      \draw[shorten >= 5pt, shorten <= 5pt] (N6.north) -- (J1);
      \draw[shorten >= 5pt, shorten <= 5pt] (N7.north) -- (J1);
      \draw[shorten >= 5pt, shorten <= 5pt] (N8.north) -- (J1);

      \draw[shorten <= 5pt] (N1) -- (M1);
      \draw[shorten <= 5pt] (N2) -- (M1);
      \draw[shorten <= 5pt] (N3) -- (M2);
      \draw[shorten <= 5pt] (N4) -- (M2);
      \draw[shorten <= 5pt] (N5) -- (M3);
      \draw[shorten <= 5pt] (N6) -- (M3);
      \draw[shorten <= 5pt] (N7) -- (M4);
      \draw[shorten <= 5pt] (N8) -- (M4);

      \draw (M1) -- (L1);
      \draw (M2) -- (L1);
      \draw (M3) -- (L2);
      \draw (M4) -- (L2);

      \draw (LL1) -- (B);
      \draw (LL2) -- (B);

      \draw (L1) -- (A1);
      \draw (L2) -- (A2);
      \draw (LL1) -- (A1);
      \draw (LL2) -- (A2);
    \end{tikzpicture}
  \end{center}
\end{frame}

\begin{frame}[fragile]
  \frametitle{Sequence \code{reduce}: Cost}

  The cost graph might look scary, but it's really just the same idea as the
  divide-and-conquer algorithm we saw for addition! All we do is divide into
  our pairs, and then recursively continue doing that until completion.

  \pause
  \vspace{\fill}

  This means that the height of the cost graph is $O(\log n)$, in the length
  of the sequence, and the number of nodes is linear. Beware that the actual
  work and span of the \code{reduce} call might be worse, though, depending
  on if \code{g} is constant time or not.

  \pause
  \vspace{\fill}

  For a constant function \code{g} (like addition), however, we have that
  \code{reduce g z S} is $O(n)$ work and $O(\log n)$ span. Not bad!
\end{frame}

\begin{frame}[fragile]
  \frametitle{Sequence \code{filter}}

  We love filtering, so we wanted one for sequences, too:
  \pause
  \spec
    {filter}
    {('a -> bool) -> 'a seq -> 'a seq}
    {\code{p} is total}
    {\code{filter p S} is equivalent to the same sequence as \code{S}, but
    with all the elements that do not satisfy \code{p} removed}

  \pause
  How is it implemented, though? Naively, we might think that we could just
  apply the function \code{p} to each element, and then fold over the sequence
  to collect the results.

  \pause
  \vspace{\fill}

  Folding is a sequential concept, though! This would incur an $O(n)$ span
  bound, because we couldn't do it in parallel. We can do better than that.
\end{frame}

\begin{frame}[fragile]
  \frametitle{Sequence \code{filter}: Cost}

  We actually can't draw a cost graph here, because the implementation of
  \code{filter} is kind of subtle. The cost of filter is, given a constant-time
  predicate \code{p}, $O(n)$ work and $O(\log n)$ span.

  \pause
  \vspace{\fill}

  You can think of it as being something like the following pseudocode:

  \pause
  \begin{itemize}
    \item map each element to \code{NONE} or \code{SOME} depending on if it
    satisfies \code{p}
    \item \code{reduce} the sequence by combining the remaining elements into
    sub-sequences, joining them with \code{Seq.append}
  \end{itemize}

  \pause
  \vspace{\fill}

  This won't have the right work bound, but it's the right intuition for
  why the span is logarithmic. Further treatment is reserved for 15-210.

  % The cost graph for \code{filter} will end up looking like this:

  % \begin{center}
  %   \begin{tikzpicture}
  %     \fhex{N1}{\code{f 0}};
  %     \fhex[right of=N1, node distance=1in] {N2} {\code{f 1}};
  %     \node[between=N2 and N3] (C) {$\cdots$};
  %     \fhex[right of=N2, node distance=2in] {N3} {\code{f (n-2)}};
  %     \fhex[right of=N3, node distance=1in] {N4} {\code{f (n-1)}};

  %     \node[between=N1 and N4, yshift=0.4in] (J1) {$\circ$};
  %     \node[between=N1 and N4, yshift=-0.4in] (J2) {$\circ$};

  %     \draw (N1) -- (J1);
  %     \draw (N2) -- (J1);
  %     \draw (N3) -- (J1);
  %     \draw (N4) -- (J1);
  %   \end{tikzpicture}
  % \end{center}
\end{frame}

\begin{frame}[fragile]
  \frametitle{More Sequence Functions}

  There are many more sequence functions, but we will not give a rigorous
  treatment of them here. Their descriptions and cost bounds can be found
  at the
  {\color{blue}\href{http://www.cs.cmu.edu/~15150/resources/libraries/sequence.pdf}{150 sequences reference}}.

  \pause
  \vspace{\fill}

  These are the fundamental ones that will get us going with programming with
  sequences. We will find that many problems having to do with dealing with
  bulk data become much more performant due to our use of sequences.
\end{frame}

\sectionSlide{4}{And Then Sum}

\begin{frame}[fragile]
  \frametitle{Sum Kind of Problem}

  For a simple example on sequences, see the problem of summing all
  the entries in a 2D matrix, modelled by two sequences.

  \pause
  \vspace{\fill}

  \spec
    {sumMatrix}
    {int seq seq -> int}
    {\code{true}}
    {\code{sumMatrix S} evaluates to the sum of all the elements in
    the 2D matrix \code{S}}

  \pause
  \vspace{\fill}

  We can write some terse code as follows:

  \begin{codeblock}
    fun sum S = Seq.reduce (op+) 0 S

    fun sumMatrix S =
      Seq.map sum S
      |> sum
  \end{codeblock}
\end{frame}

\begin{frame}[fragile]
  \frametitle{Sum Kind of Problem}

  This comprises of the logic to take a 2D matrix and sum across its rows using
  the faster \code{reduce}:

  \begin{center}
  \pause
  \begin{minipage}{0.2\textwidth}
    \centering
    $\begin{bmatrix}
      0 & 1 & 2 & 3\\
      4 & 5 & 6 & 7 \\
      8 & 9 & 10 & 11
    \end{bmatrix}$
  \end{minipage}
  \pause
  \begin{minipage}{0.12\textwidth}
    \centering
    $\longmapsto^*$
  \end{minipage}
  \begin{minipage}{0.3\textwidth}
    \centering
    $\begin{bmatrix}
      (0 + 1) + (2 + 3) \\
      (4 + 5) + (6 + 7) \\
      (8 + 9) + (10 + 11)
    \end{bmatrix}$
  \end{minipage}
  \pause
  \begin{minipage}{0.12\textwidth}
    \centering
    $\longmapsto^*$
  \end{minipage}
  \begin{minipage}{0.12\textwidth}
    \centering
    $\begin{bmatrix}
      6 \\
      12 \\
      28
    \end{bmatrix}$
  \end{minipage}
  \end{center}

  \pause
  \vspace{\fill}

  ...and then to sum across each row's sum, using the same method:

  \vspace{\fill}

  $$(6 + 12) + 28$$
\end{frame}

\begin{frame}[fragile]
  \frametitle{Sum Kind of Cost Graph}

  The cost graph looks like so:

  \begin{center}
    \begin{tikzpicture}
      \fhex {N1} {$\code{sum R}_1$};
      \fhex[right of=N1, node distance=1in] {N2} {$\code{sum R}_2$};
      \fhex[right of=N2, node distance=1.5in] {N3} {$\code{sum R}_{m - 2}$};
      \fhex[right of=N3, node distance=1in] {N4} {$\code{sum R}_{m - 1}$};

      \node[dot, between=N1 and N4, yshift=0.5in] (J1) {};
      \fhex[between=N1 and N4, yshift=-0.5in] {J2} {\code{sum}};
      \node[dot, between=N1 and N4, yshift=-1in] (J3) {};
      \node[between=N2 and N3] (M) {\textellipsis};

      \draw[shorten >= 5pt] (N1.north) -- (J1);
      \draw[shorten >= 5pt] (N2.north) -- (J1);
      \draw[shorten >= 5pt] (N3.north) -- (J1);
      \draw[shorten >= 5pt] (N4.north) -- (J1);

      \draw (N1.south) -- (J2);
      \draw (N2.south) -- (J2);
      \draw (N3.south) -- (J2);
      \draw (N4.south) -- (J2);

      \draw[shorten <= 5pt] (J3) -- (J2);
      \draw[shorten <= 5pt] (J3) -- (J2);
      \draw[shorten <= 5pt] (J3) -- (J2);
      \draw[shorten <= 5pt] (J3) -- (J2);
    \end{tikzpicture}
  \end{center}

  \pause
  \vspace{\fill}

  Here, we have $m$ calls in the middle, for an $m \times n$ matrix.

  \vspace{\fill}

  What's the cost?
\end{frame}

\begin{frame}[fragile]
  \frametitle{Sum Kind of Cost Analysis}

  Well, not all of the \code{sum}s in this graph are made equal. In particular,
  each of the inner calls to $\code{sum R}_i$ are on the $i$th row, which is
  a sequence $n$ elements long, so that is a cost of $O(n)$ work and
  $O(\log n)$ span.

  \pause
  \vspace{\fill}

  Then, we know that the sequence passed to the final \code{sum} must be of
  the same length, which is the number of rows $m$. So that is a cost of $O(m)$ work
  and $O(\log m)$ span.

\end{frame}

\begin{frame}[fragile]
  \frametitle{Sum Kind of Cost Graph, Again}

  So now we can get our cost graph, but with cost nodes instead of
  computation nodes:

  \pause
  \begin{center}
    \begin{tikzpicture}
      \hex {N1} {$O(n)$} {$O(\log n)$};
      \hex[right of=N1, node distance=1.25in] {N2} {$O(n)$} {$O(\log n)$};
      \hex[right of=N2, node distance=1.75in] {N3} {$O(n)$} {$O(\log n)$};
      \hex[right of=N3, node distance=1.25in] {N4} {$O(n)$} {$O(\log n)$};

      \node[dot, between=N1 and N4, yshift=0.5in] (J1) {};
      \hex[between=N1 and N4, yshift=-0.75in] {J2} {$O(m)$} {$O(\log m)$};
      \node[dot, between=N1 and N4, yshift=-1.25in] (J3) {};
      \node[between=N2 and N3] (M) {\textellipsis};

      \draw[shorten >= 5pt] (N1.north) -- (J1);
      \draw[shorten >= 5pt] (N2.north) -- (J1);
      \draw[shorten >= 5pt] (N3.north) -- (J1);
      \draw[shorten >= 5pt] (N4.north) -- (J1);

      \draw (N1.south) -- (J2);
      \draw (N2.south) -- (J2);
      \draw (N3.south) -- (J2);
      \draw (N4.south) -- (J2);

      \draw[shorten <= 5pt] (J3) -- (J2);
      \draw[shorten <= 5pt] (J3) -- (J2);
      \draw[shorten <= 5pt] (J3) -- (J2);
      \draw[shorten <= 5pt] (J3) -- (J2);
    \end{tikzpicture}
  \end{center}

  \pause
  \vspace{\fill}

  From this, it should be clear that the work (area) of the graph is
  $m * O(n) + O(m)$, or $O(nm)$, and the span (longest path) is
  $O(\log n + \log m)$.
\end{frame}

\begin{frame}[fragile]
  \frametitle{Sum Kind of Cost Graph, Again}

  So we finally simplify, and obtain the cost node which is simply our
  final result.

  \pause
  \vspace{\fill}

  \begin{center}
    \begin{tikzpicture}
      \hex {N1} {$O(nm)$} {$O(\log n + \log m)$};
    \end{tikzpicture}
  \end{center}

  \pause
  \vspace{\fill}

  That's all there is to it! We could then iterate this process, and substitute
  this cost node somewhere else as well, if this \code{sumMatrix} function were
  to be involved in another computation somewhere.
\end{frame}

\begin{frame}[fragile]
  \frametitle{Conclusions}

  Sequences are useful for a couple of reasons. They:
  \begin{itemize}
    \item are an excellent use of abstraction to not think of lower-level
    details of a given implementation
    \item allow us to solve problems that would normally be solved by arrays,
    without needing to compromise our immutability
    \item allow us to think about span in a more nuanced way, with parallel-friendly
    operations
  \end{itemize}

  \vspace{\fill}

  They aren't as nice to work with as lists, which are the bread and butter of
  a functional programmer's toolkit, but they are still quite usable and quite
  convenient.
\end{frame}









% \begin{frame}[fragile]
%   \frametitle{More Sequence Functions}

%   \begin{minipage}{0.45\textwidth}
%   \begin{tikzpicture}
%     \fhex {N1} {\code{+}};
%     \fhex[right of=N1, node distance=1in] {N2} {\code{+}};
%     \fhex[right of=N2, node distance=1in] {N3} {\code{+}};
%     \node[dot, between=N1 and N3, yshift=0.6in] (J1) {};
%     \node[dot, between=N2 and N3, yshift=0.4in] (J2) {};
%     \fhex[between=N2 and N3, yshift=-0.4in] {J3} {\code{*}};
%     \fhex[between=N1 and N3, yshift=-0.8in] {J4} {\code{*}};

%     \draw[shorten >= 5pt] (N1.north) -- (J1);
%     \draw[shorten >= 5pt] (N2.north) -- (J2);
%     \draw[shorten >= 5pt] (N3.north) -- (J2);
%     \draw (N1.south) -- (J4);
%     \draw (N2.south) -- (J3);
%     \draw (N3.south) -- (J3);

%     \draw[shorten >= 5pt, shorten <= 5pt] (J1) -- (J2);
%     \draw (J3.south) -- (J4);
%   \end{tikzpicture}
%   \end{minipage}
%   \begin{minipage}{0.45\textwidth}
%   \begin{tikzpicture}
%     \fhex {N1} {\code{+}};
%     \fhex[right of=N1, node distance=1in] {N2} {\code{+}};
%     \fhex[right of=N2, node distance=1in] {N3} {\code{+}};
%     \node[dot, between=N1 and N3, yshift=0.6in] (J1) {};
%     \fhex[between=N2 and N3, yshift=-0.4in] {J3} {\code{*}};
%     \fhex[between=N1 and N3, yshift=-0.8in] {J4} {\code{*}};

%     \draw[shorten >= 5pt] (N1.north) -- (J1);
%     \draw[shorten >= 5pt] (N2.north) -- (J1);
%     \draw[shorten >= 5pt] (N3.north) -- (J1);
%     \draw (N1.south) -- (J4);
%     \draw (N2.south) -- (J3);
%     \draw (N3.south) -- (J3);

%     \draw (J3.south) -- (J4);
%   \end{tikzpicture}
%   \end{minipage}

% \end{frame}


% \begin{frame}[fragile]
%   \frametitle{More Sequence Functions}

%   \begin{minipage}{0.45\textwidth}
%   \begin{tikzpicture}
%     \fhex {N1} {\code{+}};
%     \fhex[right of=N1, node distance=1in, yshift=0.4in] {N2} {\code{+}};
%     \fhex[right of=N1, node distance=1in] {N3} {\code{+}};
%     \fhex[right of=N1, node distance=1in, yshift=-0.4in] {N4} {\code{*}};
%     \node[dot, between=N1 and N3, yshift=0.8in] (J1) {};
%     \fhex[between=N1 and N3, yshift=-0.8in] {J2} {\code{*}};

%     \draw[shorten >= 5pt] (N1.north) -- (J1);
%     \draw[shorten >= 5pt] (N2.north) -- (J1);
%     \draw (N3.north) -- (N2.south);
%     \draw (N4.north) -- (N3.south);
%     \draw (N1.south) -- (J2);
%     \draw (N4.south) -- (J2);
%   \end{tikzpicture}
%   \end{minipage}
%   \begin{minipage}{0.45\textwidth}
%       \begin{tikzpicture}
%         \fhex {N1} {\code{+}};
%         \fhex[right of=N1, node distance=1in] {N2} {\code{+}};
%         \fhex[right of=N2, node distance=1in] {N3} {\code{+}};
%         \node[dot, between=N1 and N3, yshift=0.5in] (J1) {};
%         \fhex[between=N1 and N3, yshift=-0.5in] {J4} {\code{*}};

%         \draw[shorten >= 5pt] (N1.north) -- (J1);
%         \draw[shorten >= 5pt] (N2.north) -- (J1);
%         \draw[shorten >= 5pt] (N3.north) -- (J1);
%         \draw (N1.south) -- (J4);
%         \draw (N2.south) -- (J4);
%         \draw (N3.south) -- (J4);
%       \end{tikzpicture}
%   \end{minipage}
% \end{frame}


\begin{frame}[plain]
	\begin{center} Thank you! \end{center}

	\begin{center}
    {\color{blue} \href{https://docs.google.com/forms/d/e/1FAIpQLSd-hUtmsPshw7PxEBf4mZFiNHEromtgmIa85bsHu232yEpdXA/viewform?usp=sf_link}{Post-lecture survey:}} \\
    \vspace{5pt}
    \includegraphics[scale=0.035]{qr_july20} \\
    \vspace{5pt}
    And the House Quiz winner is...
  \end{center}
\end{frame}

\end{document}
