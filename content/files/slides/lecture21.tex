% Jacob Neumann

% DOCUMENT CLASS AND PACKAGE USE
    \documentclass[aspectratio=169]{beamer}

    % Establish the colorlambda boolean, to control whether the lambda is solid color (true), or the same as the picture (false)
    \newif\ifcolorlambda
    \colorlambdafalse % DEFAULT: false

    % Use auxcolor for syntax highlighting
    \newif\ifuseaux
    \useauxfalse % DEFAULT: false

    % Color settings
    \useauxtrue

    \newcommand{\auxColor}{38D1C8}     % the color of note boxes and stuff
    \newcommand{\presentColor}{511CE8} % the primary color of the slide borders
    \newcommand{\bgColor}{ebe3ff}      % the color of the background of the slide
    \newcommand{\darkBg}{8b98ad}
    \newcommand{\lambdaColor}{\auxColor}

    \colorlambdatrue

    \usepackage{comment} % comment blocks
    \usepackage{soul} % strikethrough
    \usepackage{listings} % code
    \usepackage{makecell}
    \usepackage{tcolorbox}
    \usepackage{amssymb}% http://ctan.org/pkg/amssymb
    \usepackage{pifont}% http://ctan.org/pkg/pifont
    \usepackage[outline]{contour}
    \usepackage{ stmaryrd }

    \setbeamertemplate{itemize items}[circle]
    % \setbeameroption{show notes on second screen=right}

    \usepackage{lectureSlides}
    %%%%%%%%%%%%%%%%%%%%%%%%%%%%%%%%%%%%%%%%%| <----- Don't make the title any longer than this
    \title{Program Analysis} % TODO
    \subtitle{Writing programs to analyze programs} % TODO
    \date{03 August 2023} % TODO
    \author{Brandon Wu} % TODO


    \def\checkmark{\tikz\fill[green, scale=0.4](0,.35) -- (.25,0) -- (1,.7) -- (.25,.15) -- cycle;}
    \contourlength{.08em}% default is 0.03em
    \newcommand{\cmark}{{\color{green!80!black}\ding{51}}}
    \newcommand{\xmark}{{\color{red}\ding{55}}}

    \graphicspath{ {./img/} }
    % DONT FORGET TO PUT [fragile] on frames with codeblocks, specs, etc.
        %\begin{frame}[fragile]
        %\begin{codeblock}
        %fun fact 0 = 1
        %  | fact n = n * fact(n-1)
        %\end{codeblock}
        %\end{frame}

    % INCLUDING codefile:
        % 1. In some file under code/NN (where NN is the lecture id num), include:
    %       (* FRAGMENT KK *)
    %           <CONTENT>
    %       (* END KK *)

    %    Remember to not put anything on the same line as the FRAGMENT or END comment, as that won't be included. KK here is some (not-zero-padded) integer. Note that you MUST have fragments 0,1,...,KK-1 defined in this manner in order for fragment KK to be properly extracted.
        %  2. On the slide where you want code fragment K
                % \smlFrag[color]{KK}
        %     where 'color' is some color string (defaults to 'white'. Don't use presentColor.
    %  3. If you want to offset the line numbers (e.g. have them start at line 5 instead of 1), use
                % \smlFragOffset[color]{KK}{5}

\begin{document}

% Make it so ./mkWeb works correctly
\ifweb
    \renewcommand{\pause}{}
\fi

\setbeamertemplate{itemize items}[circle]

% SOLID COLOR TITLE (see SETTINGS.sty)
{
\begin{frame}[plain]
    \colorlambdatrue
    \titlepage
\end{frame}
}

\begin{frame}[fragile]
  \frametitle{Lesson Plan}

  \tableofcontents
\end{frame}

\begin{frame}[fragile]
  \frametitle{Last time}
\end{frame}

\sectionSlide{1}{The State of Software}

\begin{frame}[fragile]
  \frametitle{Software and the World}

  On August 20th, 2011, Silicon Valley venture capitalist and and
  entrepreneur Marc Andreessen\footnote{Currently a board director for Meta Platforms.} published an essay entitled
  "Software is eating the world".

  This essay included a lot of business-oriented reasons for why software
  was immensely disrupting each individual economic sector, for reasons of
  ease of use, speed of execution, and reach of influence, among others.

  Now, more than a decade after this article, it's an incredibly obvious fact that
  software already has eaten the world. You cannot get away from it -- it is
  everywhere, and it is everything.
\end{frame}

\begin{frame}[fragile]
  \frametitle{Software and the World}

  Part of what makes software engineering a lucrative profession is that there
  is not, and will never be, a shortage for software engineers.

  Regardless of if a company is a recruiting company, a think tank, a massage parlor,
  or a pet food retailer, everyone needs software developers. Every business needs
  a website, every business deals with data, and every business needs a way to
  keep up with every other business, which is doing exactly the same.

  Unfortunately, not all of them are educated at Carnegie Mellon, and
  have taken 15-150, so not all of them are very well-informed on the
  importance of writing safe code.
\end{frame}

\begin{frame}[fragile]
  \frametitle{Software and the World}

  One theme that has cropped up throughout this course is to try
  to produce as little code as possible, because any human writing any
  amount of code has some probability of producing a bug.

  The less code we write, the less possibility of writing a bug.

  So what can we say about the immense volume of code produced by
  the tens of millions of software developers around the world?

  \textbf{Answer:} It is utterly, immensely buggy, and full of mistakes.
\end{frame}

\begin{frame}[fragile]
  \frametitle{Software and the World}

  When you write a mistake in your code, what does it often look like?

  Maybe you made a typo:

  \begin{center}
    \begin{minipage}[t][0.7in][t]{\textwidth}
      \begin{minipage}{0.2\textwidth}
        \centering
        \xmark
      \end{minipage}
      \begin{minipage}{0.75\textwidth}
        {\small
          \begin{codeblock}[rulecolor=\color{red}, framerule=0.3mm]
            fun fact 0 = 1
              | fact n = n * fac (n - 1)
          \end{codeblock}
          }
        \end{minipage}
    \end{minipage}
    \begin{minipage}[t][0.7in][t]{\textwidth}
      \begin{minipage}{0.2\textwidth}
        \centering
        \cmark
      \end{minipage}
      \begin{minipage}{0.75\textwidth}
        {\small
        \begin{codeblock}[rulecolor=\color{green!80!black}, framerule=0.3mm]
          fun fact 0 = 1
            | fact n = n * `fact` (n - 1)
        \end{codeblock}
        }
      \end{minipage}
    \end{minipage}
  \end{center}
\end{frame}

\begin{frame}[fragile]
  \frametitle{Software and the World}

  Or maybe you declared a variable, and then forgot to use it:

  \begin{center}
    \begin{minipage}[t][1.4in][t]{\textwidth}
      \begin{minipage}{0.2\textwidth}
        \centering
        \xmark
      \end{minipage}
      \begin{minipage}{0.75\textwidth}
        {\small
          \begin{codeblock}[rulecolor=\color{red}, framerule=0.3mm]
            fun treefoldl f acc Empty = acc
              | treefoldl f acc (Node (L, x, R)) =
                  let
                    val left_folded = treefoldl f acc L
                  in
                    treefoldl f (f (x, acc)) R
                  end
          \end{codeblock}
          }
        \end{minipage}
    \end{minipage}
    \begin{minipage}[t][1.4in][t]{\textwidth}
      \begin{minipage}{0.2\textwidth}
        \centering
        \cmark
      \end{minipage}
      \begin{minipage}{0.75\textwidth}
        {\small
        \begin{codeblock}[rulecolor=\color{green!80!black}, framerule=0.3mm]
          fun treefoldl f acc Empty = acc
            | treefoldl f acc (Node (L, x, R)) =
                let
                  val left_folded = treefoldl f acc L
                in
                  treefoldl f (f (x, `left_folded`)) R
                end
        \end{codeblock}
        }
      \end{minipage}
    \end{minipage}
  \end{center}
\end{frame}

\begin{frame}[fragile]
  \frametitle{Software and the World}

  Or maybe you just have a simple type error:

  \begin{center}
    \begin{minipage}[t][0.8in][t]{\textwidth}
      \begin{minipage}{0.2\textwidth}
        \centering
        \xmark
      \end{minipage}
      \begin{minipage}{0.75\textwidth}
        {\small
          \begin{codeblock}[rulecolor=\color{red}, framerule=0.3mm]
            fun foldr f acc [] = acc
              | foldr f acc (x::xs) =
                  f (x, foldr acc xs)
          \end{codeblock}
          }
        \end{minipage}
    \end{minipage}
    \begin{minipage}[t][0.8in][t]{\textwidth}
      \begin{minipage}{0.2\textwidth}
        \centering
        \cmark
      \end{minipage}
      \begin{minipage}{0.75\textwidth}
        {\small
        \begin{codeblock}[rulecolor=\color{green!80!black}, framerule=0.3mm]
          fun foldr f acc [] = acc
            | foldr f acc (x::xs) =
                f (x, foldr `f` acc xs)
        \end{codeblock}
        }
      \end{minipage}
    \end{minipage}
  \end{center}
\end{frame}

\begin{frame}[fragile]
  \frametitle{Software and the World}

  These kinds of simple mistakes crop up all of the time!

  Thankfully, we are working in a language which is disciplined enough
  to warn you about most of these things, albeit not all.

  What about all the software being produced elsewhere, though? In some languages,
  such as Python, \textbf{none of these errors} are able to be caught, until they
  happen at runtime!

  We say that making the programmer aware of these errors at \term{compile time},
  before the program runs, is a \term{static} warning, versus a \term{dynamic}
  warning, which would only occur once the program runs into it while executing.
\end{frame}

\begin{frame}[fragile]
  \frametitle{Software and the World}

  I love telling this story whenever anyone asks about why it is important
  to catch errors statically.

  Imagine that you are a machine learning engineer\footnote{The horror}.

  You have spent the past six months working on a state of the art large
  language model, and finally you are ready to put it to the test. You just
  make a few adjustments (mostly adding comments and clarifying names),
  before you run the model and then decide to go on a vacation to France for
  two weeks.

  When you return from your vacation, you discover that your model failed
  with:
  {\small
  \begin{lstlisting}[style=15150code, numbers=none]
    NameError: name 'modle' is not defined. Did you mean: 'model'?
  \end{lstlisting}
  }
\end{frame}




\begin{frame}[plain]
	\begin{center} Thank you! \end{center}
\end{frame}


\end{document}
