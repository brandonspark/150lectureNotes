% Jacob Neumann

% DOCUMENT CLASS AND PACKAGE USE
    \documentclass[aspectratio=169]{beamer}
 
    % Establish the colorlambda boolean, to control whether the lambda is solid color (true), or the same as the picture (false)
    \newif\ifcolorlambda
    \colorlambdafalse % DEFAULT: false
    
    % Use auxcolor for syntax highlighting
    \newif\ifuseaux
    \useauxfalse % DEFAULT: false
   
    % Color settings
    \useauxtrue
    
    \newcommand{\auxColor}{5725E4}     % the color of note boxes and stuff
    \newcommand{\presentColor}{FABC60} % the primary color of the slide borders
    \newcommand{\bgColor}{fff8ed}      % the color of the background of the slide
    \newcommand{\darkBg}{8b98ad}
    \newcommand{\lambdaColor}{\auxColor}
  
    \colorlambdatrue

    \usepackage{comment} % comment blocks
    \usepackage{soul} % strikethrough
    \usepackage{listings} % code
    \usepackage{makecell}

    \setbeamertemplate{itemize items}[circle]
    % \setbeameroption{show notes on second screen=right}

    \usepackage{lectureSlides}
    %%%%%%%%%%%%%%%%%%%%%%%%%%%%%%%%%%%%%%%%%| <----- Don't make the title any longer than this
    \title{Sorting and Parallelism} % TODO
    \subtitle{Awesome slides with an awesome subtitle} % TODO
    \date{01 January 2020} % TODO
    \author{Brandon Wu} % TODO

    \graphicspath{ {./img/} }
    % DONT FORGET TO PUT [fragile] on frames with codeblocks, specs, etc.
        %\begin{frame}[fragile]
        %\begin{codeblock}
        %fun fact 0 = 1
        %  | fact n = n * fact(n-1)
        %\end{codeblock}
        %\end{frame}

    % INCLUDING codefile:
        % 1. In some file under code/NN (where NN is the lecture id num), include:
    %       (* FRAGMENT KK *)
    %           <CONTENT>
    %       (* END KK *)
    
    %    Remember to not put anything on the same line as the FRAGMENT or END comment, as that won't be included. KK here is some (not-zero-padded) integer. Note that you MUST have fragments 0,1,...,KK-1 defined in this manner in order for fragment KK to be properly extracted.
        %  2. On the slide where you want code fragment K
                % \smlFrag[color]{KK}
        %     where 'color' is some color string (defaults to 'white'. Don't use presentColor.
    %  3. If you want to offset the line numbers (e.g. have them start at line 5 instead of 1), use
                % \smlFragOffset[color]{KK}{5}

\begin{document}

% Make it so ./mkWeb works correctly
\ifweb
    \renewcommand{\pause}{}
\fi

\setbeamertemplate{itemize items}[circle]

% SOLID COLOR TITLE (see SETTINGS.sty)
{
\begin{frame}[plain]
    \colorlambdatrue
    \titlepage
\end{frame}
}

\begin{frame}[fragile]
  \frametitle{Lesson Plan}

  \tableofcontents
\end{frame}

\begin{frame}[fragile]
  \frametitle{Last time}

  Last time, we reviewed the idea of \term{asymptotic analysis}, which is the analysis
  of the performance of programs, as the input size grows.

  We learned that for recursive, functional programs, we could write mathematical
  \term{recurrences} that described the \code{work} of the code, and that could be solved
  via the \term{unrolling} method to obtain a closed form, and a bound.

  We also learned that, by assuming we had infinitely many processors, we could
  obtain recurrences that measured the \code{span} of the code, or the amount of time
  using parallelism. We saw that this gave performance benefits for \code{treesum} on 
  balanced trees. 
\end{frame}

\sectionSlide{1}{Analyzing a Tree via Depth}

\begin{frame}[fragile]
  \frametitle{Measuring a Tree (again)} 

  Before, we discussed how we could use the number of nodes in a tree as the input
  size, and obtain two recurrences for \code{treesum} -- $O(n)$ in the imbalanced
  case, and $O(\log n)$ in the balanced case.

  That's not the only way to measure a tree, though. The other way is that we could
  use the \term{depth} of the tree, which is the longest path through the tree to
  get to the bottom.

  Let's try it!
\end{frame}

\begin{frame}[fragile]
  \frametitle{Tree Recurrence: Depth} 

  Where $d$ is the depth of the tree \code{T}, in the expression \code{treesum T}:
  $$S_{\code{treesum}}(0) = c_0$$
  $$S_{\code{treesum}}(d) = ???$$

  Again, we don't know how deep the tree is in the left and right subtrees. We find 
  ourselves in the same situation as before, and we'll solve it the same way, by
  assuming the worst case. In the worst case, the tree is just a line, so the depth
  of the left is $d - 1$, and the depth of the right is $0$.
\end{frame}

\begin{frame}[fragile]
  \frametitle{Tree Recurrence: Depth (Unbalanced)} 

  So:
  $$S_{\code{treesum}}(d) = \max(S_{\code{treesum}}(d - 1), S_{\code{treesum}}(0)) + c_1$$
  $$S_{\code{treesum}}(d) = S_{\code{treesum}}(d - 1) + c_1$$

  If we exchange $d$ for $n$, we've seen this recurrence before. This solves to $O(d)$.

  So the complexity is linear in the case of an unbalanced tree. 
\end{frame}

\begin{frame}[fragile]
  \frametitle{Tree Recurrence: Depth (Balanced)} 

  What if it's balanced? Then, the left subtree still has depth $d - 1$, as does the
  right subtree.

  So:
  $$S_{\code{treesum}}(d) = \max(S_{\code{treesum}}(d - 1), S_{\code{treesum}}(d - 1)) + c_1$$
  $$S_{\code{treesum}}(d) = S_{\code{treesum}}(d - 1) + c_1$$

  What gives?? This is the same recurrence! Span is $O(d)$ in both the unbalanced and
  balanced cases.
\end{frame}

\begin{frame}[fragile]
  \frametitle{On Depth and Balance}

  This might be counterintuitive, because we expect a better bound, but it makes sense if
  you remember the task dependency graphs we discussed earlier.

  In a task dependency graph, the cost of executing some amount of tasks in parallel is
  just the \textit{longest path through the graph}, or tree. The length of the longest
  path through a tree is just $d$, the depth of the tree!

  We can relate it to our previous bounds, $O(n)$ and $O(\log n)$, for unbalanced and
  balanced trees, in the number of nodes $n$, to see they are the same.
\end{frame}

\begin{frame}[fragile]
  \frametitle{On Depth and Balance}

  In an unbalanced tree, the depth of the tree $d$ is just the number of nodes $n$! So 
  $d = n$, so $O(d)$ is the same as $O(n)$, which is the same bound we received earlier.

  What is the number of nodes in a balanced tree of depth $d$ though? Well, each level has
  double the nodes of the previous, so it's equal to
  $$1 + 2 + 4 + ... + 2^d$$

  There's a lovely geometric proof that shows that this is in $O(2^d)$.

  So in a balanced tree, $n = 2^d$, so our previous bound is $O(\log n) = O(\log (2^d)) = O(d)$.
  We get the same thing either way, so this is perfectly consistent!
\end{frame}

\begin{frame}[fragile]
  \frametitle{Nodes or Depth}

  Ultimately, if you do the math and reason it out, you find that getting bounds in terms of
  depth and nodes looks different, but ultimately say the same thing.

  Whichever is "easier" is up to your discretion. Both are valid ways of solving a 
  recurrence\footnotemark.

  \footnotetext[1]{We will usually specify whenever we have a particular way we want to 
  see you solve it, which is often.}
\end{frame}

\sectionSlide{2}{The Tree Method}

% TODO: move this to lecture6, move parallelism stuff here?
% actually, inclined to keep it this way, i think...
\begin{frame}[fragile]
  \frametitle{Ordering a Tree}

  Recall our notion of an \textit{ordering} on a tree, which produces
  a list from a tree by traversing the tree in some prescribed order.

  We are interested in \textit{inorder} traversal, which traverses a
  tree the same way that someone would traverse it by reading from
  left-to-right. 

  \begin{codeblock}
    fun inord (Empty : tree) : int list = []
      | inord (Node (L, x, R)) = inord L @ [x] @ inord R 
  \end{codeblock}
\end{frame}

\begin{frame}[fragile]
  \frametitle{\code{inord}, Naively}

  When you see recursive calls being given as arguments to append, you
  should double-check, because something fishy is probably going on. 

  But, better than thinking about it, we can mathematically solve for the
  performance! Let's assume a balanced tree, and solve for the work of this
  function, in terms of the nodes of the tree.

  Where $n$ 
  $$W_{\code{inord}}(0) = c_0$$
  $$W_{\code{inord}}(n) = W_{\code{@}}(\frac{n}{2}) + 2 \cdot W_{\code{inord}}(\frac{n}{2})$$

  (because we append a list of half the size, and compute \code{inord} recursively twice)
\end{frame}

\begin{frame}[fragile]
  \frametitle{Solving \code{inord}}

  $$W_{\code{inord}}(n) = W_{\code{@}}(\frac{n}{2}) + 2 \cdot W_{\code{inord}}(\frac{n}{2}) + c_1$$

  So we have:
  \begin{align*} 
    &= W_{\code{inord}}(n) 
    &= W_{\code{@}}(\frac{n}{2}) + 2 \cdot W_{\code{inord}}(\frac{n}{2}) + c_1
    &= O(n) + c_1 + 2 \cdot W_{\code{inord}}(\frac{n}{2})
    &= O(n) + c_1 + 2 \cdot (O(\frac{n}{2}) + W_{\code{inord}}(\frac{n}{4}) + c_1)
    &= ... 
    &= ??? 
  \end{align*}

  This is... messy.
\end{frame}

\begin{frame}[fragile]
  \frametitle{Unrolling in the Deep}

  Sometimes, unrolling is messy. Sometimes, like in \code{length}, we only get
  one extra term per "unrolling", and so it's not hard to solve by just 
  finding the pattern.
  
  In the case of functions on trees, this usually isn't the case! We actually
  get \textit{two} terms per unrolling, which quickly becomes four by the
  next unrolling, and so on.

  We will employ a new technique of solving for such recurrences, using the
  \term{tree method}, instead of the unrolling method.
\end{frame}

\begin{frame}[fragile]
  \frametitle{The Tree Method}

  The tree method gets its name, from noticing that the amount of recursive
  calls done by a function like \code{inord} induces a tree structure.

  We see that calling \code{inord} on a tree with $n$ nodes causes two
  calls, to \code{inord} with $\frac{n}{2}$ nodes (in the balanced case).

  So we get a tree which looks like:

  image
\end{frame}

\begin{frame}[fragile]
  \frametitle{Tree Call Structure}

  But those two calls to \code{inord} have their own recursive calls,
  which have size $\frac{n}{4}$.

  So our "call tree" expands to:

  image
\end{frame}

\begin{frame}[fragile]
  \frametitle{Tree Call Structure}

  We can think of the following recurrence as two parts:
  $$W_{\code{@}}(\frac{n}{2}) + 2 \cdot W_{\code{inord}}(\frac{n}{2}) + c_1$$

  \begin{itemize}
    \item the \textit{nonrecursive work}, $W_{\code{@}}(\frac{n}{2}) + c_1$
    \item the \textit{recursive work}, $W_{\code{inord}}(\frac{n}{2})$ 
  \end{itemize}

  With respect to the call tree, the nonrecursive work is present at each
  node, but the recursive work is taken care of by all its children.

  So, if we sum all the nonrecursive work in each node, we'll get the work
  done by the entire function.
\end{frame}

\begin{frame}[fragile]
  \frametitle{The Work of a Node}

  To do this, we'll need to somehow figure out the nonrecursive work
  done by each node. This is $O(n) + c_1$ for a node with size $n$, 
  except each node has a different size!  

  In addition, there's a differing number of nodes of each size,
  since there's 2 of size $\frac{n}{2}$, and 4 of size $\frac{n}{4}$,
  and so on.
  
  This isn't easier at all!
\end{frame}

\begin{frame}[fragile]
  \frametitle{Get On My Level}

  The innovation comes from noticing that the nonrecursive work
  at each \textit{level} of the tree might come out to the same thing.

  If the work at each level was the same, then we could just multiply
  that quantity by $\log n$, the number of times we can recursively
  call \code{inord} by halving the input. 

  Let's try it out.
\end{frame}

\begin{frame}[fragile]
  \frametitle{The Tree Method, Concluded}

  Let the \textit{level} of a tree denote how far we are from the root.
  So the root is at level $0$, and there are two nodes at level $1$, and 
  so on.

  Then, level $i$ has $2^i$ nodes. In addition, each node has size
  $\frac{n}{2^i}$, and thus work $O(\frac{n}{2^i}) + c_1$.

  So this is 
  \begin{align*} 
    &= 2^i \cdot (O(\frac{n}{2^i}) + c_1)
    &= O(n) + 2^i \cdot c_1
  \end{align*}

  So the work at level $i$, in total, is still in $O(n)$! There's
  $\log n$ levels, so we ultimately come out with a bound of 
  $O(n \log n)$.
\end{frame}

\begin{frame}[fragile]
  \frametitle{Tail \code{inord}}

  It feels like we should be able to do better than that. Let's do 
  \code{inord} again, but this time with an accumulator argument.

  \begin{codeblock}
    fun tinord (Empty : tree, acc : int list) = acc
      | tinord (Node (L, x, R), acc) = 
          tinord (L, x :: tinord (R, acc))
  \end{codeblock}

  Theoretically, the complexity should be better. Let's figure it out!
\end{frame}

\begin{frame}[fragile]
  \frametitle{\code{tinord}: Work (Balanced)}

  Let's deal with the balanced case first.

  Where $n$ is the number of nodes in \code{T} in the expression \code{tinord (T, L)}:
  $$W_{\code{tinord}}(0) = c_0$$
  $$W_{\code{tinord}}(n) = 2 \cdot W_{\code{tinord}}(\frac{n}{2}) + c_1$$

  We get two calls to $W_{\code{tinord}}(\frac{n}{2})$, because we first compute
  \code{tinord(R, acc)}, and then pass that in as \code{acc'} to \code{tinord (L, x :: acc')}.

  In either case, the size of the tree is roughly half.
\end{frame}

\begin{frame}[fragile]
  \frametitle{\code{tinord}: Work}

  So we solve to:

  \begin{align*}
    &= W_{\code{tinord}}(n) 
    &= 2 \cdot W_{\code{tinord}}(\frac{n}{2}) + c_1
    &= 4 \cdot (W_{\code{tinord}}(\frac{n}{4}) + c_1) + c_1 
    &= ... 
  \end{align*}

  Same issue as before, now we have two recursive calls at each unrolling. Better to solve this
  with the tree method!   
\end{frame}

\begin{frame}[fragile]
  \frametitle{\code{tinord}: Call Structure}

  $$W_{\code{tinord}}(n) = 2 \cdot W_{\code{tinord}}(\frac{n}{2}) + c_1$$

  As before, we have a structure where we have one node with size $n$ at the top, which calls
  itself twice with size $\frac{n}{2}$, so we get the same call tree as before.

  But what is the nonrecursive work done at each node? This time, there's no append, so the
  work is just $c_1$, the constant cost of consing, and various other things.

  So we have that each level has $2^i$ nodes, and does $c_1$ work, so our summation looks
  like
  $$\sum_{i = 0}^{\log n} 2^i c_1 = c_1 \sum_{i = 0}^{\log n} 2^i$$
\end{frame}

\begin{frame}[fragile]
  \frametitle{\code{tinord}: Call Structure}

  $$\sum_{i = 0}^{\log n} 2^i c_1 = c_1 \sum_{i = 0}^{\log n} 2^i$$

  This expands to a term like:
  $$c_1(1 + 2 + 4 + ... + n)$$ 

  where we know the inner term to be in $O(n)$. So ultimately, our
  bound is $O(n)$. That's a logarithmic improvement over $\code{inord}$!
\end{frame}

\begin{frame}[fragile]
  \frametitle{\code{tinord}: Work (Unbalanced)}

  What if our tree were to be unbalanced?

  \begin{codeblock}
    fun tinord (Empty : tree, acc : int list) = acc
      | tinord (Node (L, x, R), acc) = 
          tinord (L, x :: tinord (R, acc))
  \end{codeblock}

  Then we would get:
  $$W_{\code{tinord}}(0) = c_0$$
  $$W_{\code{tinord}}(n) = W_{\code{tinord}}(n - 1) + W_{\code{tinord}}(0) + c_1$$

  By analogy, we've seen this recurrence before. This solves to 
  $$W_{\code{tinord}}(n) = W_{\code{tinord}}(n - 1) + c_0 + c_1$$
  which is in $O(n)$. So we do the same amount of work.
\end{frame}

\begin{frame}[fragile]
  \frametitle{\code{tinord}: Span}

  \begin{codeblock}
    fun tinord (Empty : tree, acc : int list) = acc
      | tinord (Node (L, x, R), acc) = 
          tinord (L, x :: tinord (R, acc))
  \end{codeblock}

  Now finally, let's do the span analysis. Let's assume the best case, which is a balanced tree.

  $$S_{\code{tinord}}(0) = c_0$$
  $$S_{\code{tinord}}(n) = 2 \cdot S_{\code{tinord}}(\frac{n}{2}) + c_1$$

  What gives? We still have two calls to $S_{\code{tinord}}(\frac{n}{2})$, even though we usually
  get to take the max of them.

  The reason is because there is a \term{data dependency} between the two calls to \code{tinord}.
  The call to \code{tinord(R, acc)} is being given as an argument to the other, meaning that
  the second call cannot be executed until the first finishes!

  So our span bound ends up still being $O(n)$. This holds in the unbalanced case too.
\end{frame}

\begin{frame}[fragile]
  \frametitle{\code{tinord}: A Retrospective}

  \code{tinord} is an excellent example of how better parallel or sequential performance is not
  necessarily correlated!

  Certain functions can admit a fast implementation that is not parallelizable, and certain
  problems are more parallelizable-friendly. For more on this, the class 15-210 will go into
  more detail.
\end{frame}



\sectionSlide{3}{Sorting}

\begin{frame}[fragile]
  \frametitle{Sorting}
  We've now discussed trees and lists in detail. We've seen how we can analyze the performance
  of functions on these data structures, which cover a wide variety of classic computer science
  problems.

  We will now turn to one of the most classic problems of all in computer science: sorting a
  list of integers.
\end{frame}

\begin{frame}[fragile]
  \frametitle{Sorting}

  There are a variety of sorting algorithms that have been invented. We're
  going to try our hand at implementing a classic one -- insertion sort.

  First, let's prepare our comparison function. Recall our \code{order} type from before:

  \begin{codeblock}
    datatype order = LESS | EQUAL | GREATER

    fun compare (x : int, y : int) =
      if x < y then
        LESS
      else if x = y then
        EQUAL
      else
        GREATER
  \end{codeblock}
\end{frame}

\begin{frame}[fragile]
  \frametitle{Insertion}

  Insertion sort works via repeatedly inserting an element into an already-sorted
  list. By doing this for every element in the list, we will eventually sort
  the entire list.

  First, we need an insertion function:
  \begin{codeblock}
    (* insert : int * int list -> int list *)
    (* REQUIRES: L is sorted *) 
    (* ENSURES: insert (x, L) is a sorted permutation of x::L *)
    fun insert (x : int, [] : int list) : int list = [x]
      | insert (x, y::ys) = 
          case compare (x, y) of 
            LESS => x::y::ys 
          | _ => x :: insert (x, ys)
  \end{codeblock}
\end{frame}

\begin{frame}[fragile]
  \frametitle{Insertion Sort}

  Now we can proceed to defining our sorting function!

  \begin{codeblock}
    (* insert : int list -> int list *)
    (* REQUIRES: true *) 
    (* ENSURES: insort L is a sorted permutation of L *)
    fun insort ([] : int list) : int list = []
      | insort (x::xs) = insert (x, insort xs)
  \end{codeblock}

  How simple!
\end{frame}

\begin{frame}[fragile]
  \frametitle{Insertion Sort: Work Analysis}
  \begin{codeblock} 
    fun insert (x : int, [] : int list) : int list = [x]
      | insert (x, y::ys) = 
          case compare (x, y) of 
            LESS => x::y::ys 
          | _ => x :: insert (x, ys)
  \end{codeblock}

  We see that insertion sort admits a very simple implementation in SML.

  Now, let's analyze it!

  Where $n$ is the length of the list \code{L} in the expression \code{insert (x, L)}:
  $$W_{\code{insert}}(0) = c_0$$
  $$W_{\code{insert}}(n) = W_{\code{insert}}(n - 1) + c_1 = O(n)$$

  (we saw that the \code{compare} function from earlier is $O(1)$)
\end{frame}


\begin{frame}[fragile]
  \frametitle{Insertion Sort: Work Analysis}
  \begin{codeblock}
    fun insort ([] : int list) : int list = []
      | insort (x::xs) = insert (x, insort xs)
  \end{codeblock}

  Now, if we analyze \code{insort}, we get:

  Where $n$ is the length of the list \code{L} in the expression \code{insort L}:
  $$W_{\code{insort}}(0) = c_0$$
  $$W_{\code{insort}}(n) = W_{\code{insert}}(n) + W_{\code{insort}}(n - 1) + c_1$$

  where the second equation is because the length of \code{insort xs} is $n - 1$,
  since it's the same length as \code{xs}.
\end{frame}

\begin{frame}[fragile]
  \frametitle{Insertion Sort: Work Analysis}

  Now we solve:
  \begin{align*}
    &= W_{\code{insort}}(n) \\ 
    &= W_{\code{insert}}(n) + W_{\code{insort}}(n - 1) + c_1 \\
    &= W_{\code{insort}}(n - 1) + c_2 \cdot n + c_1 \\
    &= W_{\code{insort}}(n - 2) + c_2 \cdot (n - 1) + c_1 + c_2 \cdot n + c_1 \\
    &= ... \\ 
    &= c_2 \cdot (1 + 2 + ... + n) + c_1 \cdot n \\
    &= O(n^2) 
  \end{align*}

  So insertion sort is quadratic time, which is expected.
\end{frame}

\begin{frame}[fragile]
  \frametitle{Insertion Sort: Span Analysis?}

  Unfortunately, there is no real span analysis to be had here. On a list,
  the amount of opportunities for parallelism is low.

  This might mean our dreams of analyzing the span of a sorting algorithm
  are dead!
\end{frame}

\begin{frame}[fragile]
  \frametitle{A Parallel Sort}

  Fortunately, someone else invented merge sort\footnotemark.

  \defBox{}{\, \term{Merge sort} is a sorting algorithm involving dividing
  the list to be sorted in half, and recursively sorting each half.}

  Not only does merge sort achieve a better sequential complexity, but we will
  see how its span bound improves as well.
  
  \footnotetext[1]{Well, quick sort too. But we will discuss merge sort for today.}
\end{frame}

\begin{frame}[fragile]
  \frametitle{The Mergesort Algorithm}

  Our algorithm will be as follows:

  \begin{itemize}
    \item Split the list into two halves. It doesn't really matter how.
    \item Recursively sort either half.
    \item Merge the two sorted halves to make a sorted list. 
  \end{itemize}

  The main important thing here is that it is pretty easy to split a list in 
  half, as well as put two sorted lists together into another sorted list.

  We will implement this, and call those functions \code{split} and \code{merge}.
\end{frame}

\begin{frame}[fragile]
  \frametitle{Mergesort Helpers}

  \begin{codeblock}
    (* split : int list -> int list * int list *)
    (* REQUIRES: true *)
    (* ENSURES: split L => (A, B) such that L is a permutation
     * of A @ B, and A and B are roughly the same length *) 
    fun split ([] : int list) : int list * int list = []
      | split [x] = [x]
      | split (x::y::xs) = 
          let
            val (A, B) = split xs
          in
            (x::A, y::B)
          end
  \end{codeblock}

  Where $n$ is the length of \code{L} in the expression \code{split L}:
  $$W_{\code{split}}(0) = c_0$$
  $$W_{\code{split}}(1) = c_1$$
  $$W_{\code{split}}(n) = W_{\code{split}}(n - 1) + c_1 = O(n)$$
\end{frame}

\begin{frame}[fragile]
  \frametitle{Mergesort Helpers}

  \begin{codeblock}
    (* merge : int list * int list -> int list *)
    (* REQUIRES: L and R are sorted *)
    (* ENSURES: merge (L, R) is a sorted permutation of L @ R *)
    fun merge ([] : int list, R : int list) : int list = R
      | merge (L, []) = L
      | merge (x::xs, y::ys) =
          case compare (x, y) of
            LESS => x :: merge (xs, y::ys)
          | _ => y :: merge (x::xs, ys)
  \end{codeblock}

  Where $n$ is the sum of the lengths of \code{L} and \code{R} in \code{merge (L, R)}:
  $$W_{\code{merge}}(0) = c_0$$
  $$W_{\code{merge}}(n) = W_{\code{merge}}(n - 1) + c_1 = O(n)$$
\end{frame}

\begin{frame}[fragile]
  \frametitle{Implementing Mergesort}

  Now that we've defined \code{split} and \code{merge}, we're ready to write \code{msort}.

  \begin{codeblock}
    (* msort : int list -> int list *)
    (* REQUIRES: true *) 
    (* ENSURES: msort L is a sorted permutation of L *) 
    fun msort ([] : int list) : int list = []
      | msort [x] = [x]
      | msort L = 
          let
            val (A, B) = split L 
          in
            merge (msort A, msort B) 
          end
  \end{codeblock}
\end{frame}


\begin{frame}[fragile]
  \frametitle{The Final Product}
  {\tiny
  \begin{codeblock}
    fun split ([] : int list) : int list * int list = []
      | split [x] = [x]
      | split (x::y::xs) = 
          let
            val (A, B) = split xs
          in
            (x::A, y::B)
          end

    fun merge ([] : int list, R : int list) : int list = R
      | merge (L, []) = L
      | merge (x::xs, y::ys) =
          case compare (x, y) of
            LESS => x :: merge (xs, y::ys)
          | _ => y :: merge (x::xs, ys)

    fun msort ([] : int list) : int list = []
      | msort [x] = [x]
      | msort L = 
          let
            val (A, B) = split L 
          in
            merge (msort A, msort B) 
          end
  \end{codeblock}
  }

  That's all!
\end{frame}

\begin{frame}[fragile]
  \frametitle{\code{msort}: Work Recurrence}

  \begin{codeblock}
    fun msort ([] : int list) : int list = []
      | msort [x] = [x]
      | msort L = 
          let
            val (A, B) = split L 
          in
            merge (msort A, msort B) 
          end
  \end{codeblock}

  What's the complexity of \code{msort}?

  Where $n$ is the length of the list \code{L} in the expression \code{msort L}:
  $$W_{\code{msort}}(0) = c_0$$
  $$W_{\code{msort}}(1) = c_1$$
  $$W_{\code{msort}}(n) = W_{\code{split}}(n) + 2 \cdot W_{\code{msort}}(n) + W_{\code{merge}}(n) + c_2$$
\end{frame}

\begin{frame}[fragile]
  \frametitle{\code{msort}: Work Recurrence}

  $$W_{\code{msort}}(n) = W_{\code{split}}(n) + 2 \cdot W_{\code{msort}}(n) + W_{\code{merge}}(n) + c_2$$

  Now we can solve to:

  \begin{align*} 
    &= W_{\code{msort}}(n) \\
    &= W_{\code{split}}(n) + 2 \cdot W_{\code{msort}}(n) + W_{\code{merge}}(n) + c_2\\
    &= 2 \cdot W_{\code{msort}}(n) + c_3 \cdot n + c_4 \cdot n + c_2 \\
  \end{align*}

  Let's use the tree method to solve this recurrence!
\end{frame}

\begin{frame}[fragile]
  \frametitle{\code{msort}: Work Recurrence}

  There are two calls to \code{msort} at each level, so there are $2^i$ nodes at level $i$.

  At each node, we do $O(n)$ work. At level $i$, this is $\frac{n}{2^i}$ work.

  So we are looking at $\sum_{i = 0}^{\log n} 2^i \frac{n}{2^i} = \sum_{i = 0}^{\log n} n = n \log n$

  So this is $O(n \log n)$. 
\end{frame}

\begin{frame}[fragile]
  \frametitle{\code{msort}: Span Recurrence}

  What about span? \code{msort} makes two calls to itself in parallel, so there is an opportunity for
  a speedup.

  \begin{codeblock}
    fun msort ([] : int list) : int list = []
      | msort [x] = [x]
      | msort L = 
          let
            val (A, B) = split L 
          in
            merge (msort A, msort B) 
          end
  \end{codeblock}

  Where $n$ is the length of the list \code{L} in the expression \code{msort L}:
  $$S_{\code{msort}}(0) = c_0$$
  $$S_{\code{msort}}(1) = c_1$$
  $$S_{\code{msort}}(n) = S_{\code{split}}(n) + \max(S_{\code{msort}}(n), S_{\code{msort}}(n)) + S_{\code{merge}}(n) + c_2$$
  $$S_{\code{msort}}(n) = S_{\code{split}}(n) + S_{\code{msort}}(n) + S_{\code{merge}}(n) + c_2$$
\end{frame}

\begin{frame}[fragile]
  \frametitle{\code{msort}: Span Recurrence}

  $$S_{\code{msort}}(n) = S_{\code{split}}(n) + S_{\code{msort}}(n) + S_{\code{merge}}(n) + c_2$$

  Now we solve:
  \begin{align*}
    &= S_{\code{msort}}(n) \\ 
    &= S_{\code{split}}(n) + S_{\code{msort}}(n) + S_{\code{merge}}(n) + c_2 \\
    &= S_{\code{msort}}(n) + c_3 \cdot n + c_4 \cdot n + c_2 \\
    &= S_{\code{msort}}(n) + c_3 \cdot n + c_4 \cdot n + c_2 \\
    &= c_3 \cdot \frac{n}{2} + c_4 \cdot \frac{n}{2} + c_2 + c_3 \cdot n + c_4 \cdot n + c_2 \\
    &= ... \\
    &= (c_3 + c_4) \cdot (1 + 2 + 4 + ... + n) + c_2 \\ 
    &= (c_3 + c_4) \cdot O(n) + c_2 \\ 
  \end{align*}

  So we get that, in parallel, merge sort is in $O(n)$.
\end{frame}

\begin{frame}[fragile]
  \frametitle{A Linear Time Sort}

  That's pretty huge! The power of parallelism offers us to not just get a 
  speedup when doing computations, but mathematically prove that we achieve
  a better asymptotic bound. That's pretty cool.
\end{frame}

\begin{frame}[fragile]
	\begin{center} Thank you! \end{center}
\end{frame}


\end{document}

