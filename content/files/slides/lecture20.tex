% Jacob Neumann

% DOCUMENT CLASS AND PACKAGE USE
    \documentclass[aspectratio=169]{beamer}

    % Establish the colorlambda boolean, to control whether the lambda is solid color (true), or the same as the picture (false)
    \newif\ifcolorlambda
    \colorlambdafalse % DEFAULT: false

    % Use auxcolor for syntax highlighting
    \newif\ifuseaux
    \useauxfalse % DEFAULT: false

    % Color settings
    \useauxtrue

    \newcommand{\auxColor}{0AA0F4}     % the color of note boxes and stuff
    \newcommand{\presentColor}{2DC5A9} % the primary color of the slide borders
    \newcommand{\bgColor}{e6fffa}      % the color of the background of the slide
    \newcommand{\darkBg}{8b98ad}
    \newcommand{\lambdaColor}{\auxColor}

    \colorlambdatrue

    \usepackage{comment} % comment blocks
    \usepackage{soul} % strikethrough
    \usepackage{listings} % code
    \usepackage{makecell}


    \setbeamertemplate{itemize items}[circle]
    % \setbeameroption{show notes on second screen=right}

    \usepackage{lectureSlides}
    %%%%%%%%%%%%%%%%%%%%%%%%%%%%%%%%%%%%%%%%%| <----- Don't make the title any longer than this
    \title{Compilers} % TODO
    \subtitle{Making programming languages real} % TODO
    \date{01 August 2023} % TODO
    \author{Brandon Wu} % TODO

    \graphicspath{ {./img/} }

    \colorlet{bgBlue}{hlBlue!130}
    \colorlet{bgOrange}{hlOrange}
    \colorlet{fgBlue}{fgCodeBlue}
    \colorlet{fgGreen}{fgCodeGreen}
    \colorlet{fgRed}{fgCodeRed}
    \colorlet{fgOrange}{fgCodeOrange}

    \definecolor{bgPurple}{HTML}{dbc7ff}
    \definecolor{bgRed}{HTML}{fcaea9}
    \definecolor{bgYellow}{HTML}{fffeb5}
    \definecolor{bgGreen}{HTML}{b4ffb3}

    \definecolor{fgOrange}{HTML}{fcae4e}

    \usetikzlibrary{positioning}
    \usetikzlibrary{decorations.pathreplacing,calligraphy}

    \tikzset{
      every path/.style={line width=0.25mm},
      tok/.style={
        font=\large,
        align=center,
        minimum height=3cm,
      },
      belowtok/.style={
        font=\large,
        align=center,
        minimum height=3cm,
      },
      astNode/.style={
        draw,
        inner sep=0.5pt,
        line width=0.4mm,
        minimum size=0.6cm,
        font=\large,
        align=center,
      },
      hex/.style={
        astNode,
        signal,
        signal to=east and west,
        signal pointer angle=130,
        fill=bgPurple,
      },
      box/.style={
        astNode,
        rectangle,
        fill=green!20!white,
      },
      bbox/.style={
        astNode,
        rectangle,
        fill=bgBlue,
      },
      circ/.style={
        astNode,
        circle,
        minimum size=0.7cm,
        fill=bgYellow
      },
    }

    \newcommand{\kwdToken}[4][]{
      \node[tok, text centered, #1] (#2) {{\color{fgBlue}\code{#3}}};
      \node[belowtok, text centered, below=-2.4cm of #2] {{\color{fgBlue}\code{#4}}};
    }
    \newcommand{\idToken}[3][]{
      \node[tok, text centered, #1] (#2) {{\color{fgGreen}\code{#3}}};
      \node[belowtok, text centered, below=-2.4cm of #2] {{\color{fgGreen}\code{ID}}};
    }
    \newcommand{\numToken}[3][]{
      \node[tok, text centered, #1] (#2) {{\color{fgRed}\code{#3}}};
      \node[belowtok, text centered, below=-2.4cm of #2] {{\color{fgRed}\code{NUM}}};
    }
    \newcommand{\opToken}[4][]{
      \node[tok, text centered, #1] (#2) {{\color{fgOrange}\code{#3}}};
      \node[belowtok, text centered, below=-2.4cm of #2] {{\color{fgOrange}\code{#4}}};
    }
    \newcommand{\eqToken}[3][]{
      \node[tok, text centered, #1] (#2) {{\color{fgOrange}\code{#3}}};
      \node[belowtok, text centered, below=-2.38cm of #2] {{\color{fgOrange}\code{EQ}}};
    }

    % DONT FORGET TO PUT [fragile] on frames with codeblocks, specs, etc.
        %\begin{frame}[fragile]
        %\begin{codeblock}
        %fun fact 0 = 1
        %  | fact n = n * fact(n-1)
        %\end{codeblock}
        %\end{frame}

    % INCLUDING codefile:
        % 1. In some file under code/NN (where NN is the lecture id num), include:
    %       (* FRAGMENT KK *)
    %           <CONTENT>
    %       (* END KK *)

    %    Remember to not put anything on the same line as the FRAGMENT or END comment, as that won't be included. KK here is some (not-zero-padded) integer. Note that you MUST have fragments 0,1,...,KK-1 defined in this manner in order for fragment KK to be properly extracted.
        %  2. On the slide where you want code fragment K
                % \smlFrag[color]{KK}
        %     where 'color' is some color string (defaults to 'white'. Don't use presentColor.
    %  3. If you want to offset the line numbers (e.g. have them start at line 5 instead of 1), use
                % \smlFragOffset[color]{KK}{5}

\begin{document}

% Make it so ./mkWeb works correctly
\ifweb
    \renewcommand{\pause}{}
\fi

\setbeamertemplate{itemize items}[circle]

% SOLID COLOR TITLE (see SETTINGS.sty)
{
\begin{frame}[plain]
    \colorlambdatrue
    \titlepage
\end{frame}
}

\begin{frame}[fragile]
  \frametitle{Lesson Plan}

  \tableofcontents
\end{frame}

\sectionSlide{1}{The History of Programming Languages}

\begin{frame}[fragile]
  \frametitle{Prologue}

  Let me tell you a story.

  \vspace{\fill}

  Once upon a time, programming languages weren't real.

  \vspace{\fill}

  They were an idea in someone's head, of something that might be, but
  which had never been done yet.

  \pause
  \vspace{\fill}

  There weren't IDEs, there weren't VS Code extensions, there weren't
  syntax highlighters, and there was no Standard ML.\footnote<2->{A true horror story.}
\end{frame}

\begin{frame}[fragile]
  \frametitle{Early Computer Hardware}

  Actually, once upon a time, computers weren't real either.\footnote<2->{I'm less bothered about that one.}

  \vspace{\fill}

  People did everything by hand or by simple machine. There were no calculators,
  and there was no Twitter\footnote<3->{Soon, there might not be again.}.

  \vspace{\fill}

  An early invention in the 1800s led to a loom which could weave a certain pattern
  based upon usage of punched cards, literally cards with holes in them. This was
  one of the first examples of what could be considered programming, albeit
  strictly for loom-weaving.

  \vspace{\fill}

  A little later on in the 1800s, Ada Lovelace and Charles Babbage worked on a
  theoretical proposal of a machine called the \term{Analytical Engine}, that
  could execute arbitrary computations. This was the first advent of the general
  purpose computer.
\end{frame}

\begin{frame}[fragile]
  \frametitle{Early Computer Hardware}

  By the beginning of the 1900s, Herman Hollerith had the idea of being able to
  feed data into a machine via use of punch card, for the purposes of data
  processing, record-keeping, and basic arithmetic operations.

  \vspace{\fill}

  He would found the Tabulating Machine Company on this premise, primarily
  using punched cards for the purpose of data storage. He then would realize
  this was a stupid name, and rename it to International Business Machines
  Corporation, now known as IBM.

  \vspace{\fill}

  Fast forward a couple of decades, and Konrad Zuse comes out with the Z3, a
  working, programmable electrical computer. Punched cards were used both to
  submit program code to the machine, as well as to store the data that it
  kept and output.
\end{frame}

\begin{frame}[fragile]
  \frametitle{Punch Card Computing}

  Some people have expressed that programming in SML/NJ can be an unpleasant
  experience, due to a lack of ecosystem and documentation, and rather
  interesting error messages.

  \vspace{\fill}

  I want to put this scenario into your head.

  \vspace{\fill}

  You are an engineer in the mid 1900s, and you are one of the first programmers
  in the world. You painstakingly punch characters, hole by hole, into punch cards,
  and then gather your cards that comprise your program and then walk across the
  hall to the other room, where you can wait in line to submit a request to
  run your program.

  \vspace{\fill}

  If you make a mistake while punching a card, you have to start over. If you
  drop your deck of cards, the program is gone.

  \vspace{\fill}

  And still, programming languages are not real.
\end{frame}

\begin{frame}[fragile]
  \frametitle{Assembly Programming}

  In the 1940s, programs were now being written in assembly language, the lowest
  form of language understandable by computers. These are essentially just blocks
  of bits that have some particular meaning to the computer, bijected with
  English words so that we can remember what they do.

  \vspace{\fill}

  For instance, the following might be an assembly-like syntax:
  \begin{codeblock}
    ADD r1 r2
    CMP r1 r2
    GOTO 0x00067AB3EF
  \end{codeblock}

  \vspace{\fill}

  I quote Wikipedia here on the topic, as follows:
  \begin{quote}
    It was eventually realized that programming in assembly language required a
    great deal of intellectual effort. [{\color{blue} \textit{citation needed}}]
  \end{quote}

  \vspace{\fill}

  Wikipedia claims that a citation is needed. I think that citation is only needed
  for people who have never had to write assembly language before.
\end{frame}

\begin{frame}[fragile]
  \frametitle{Assembly Programming}

  So programmers spend hours, days, weeks poring over these very basic symbols,
  having to reconstruct every single operation that the computer is doing in their
  heads, and needing to remember the exact state of the computer at each step,
  at the risk of making an incorrect assumption and writing a bug.

  \vspace{\fill}

  There is no recovery from a typo. There is no such thing as a type error.

  \vspace{\fill}

  And still, programming languages are not real.
\end{frame}

\begin{frame}[fragile]
  \frametitle{A Language for Computing}

  Then, one day, someone has an idea. Maybe multiple people have an idea.

  \vspace{\fill}

  One such person was John W. Backus, in the 1950s, whose idea was that
  maybe we don't need to write assembly by hand. Maybe we can write instructions
  that could then be converted into the assembly itself.

  \vspace{\fill}

  John Backus called the idea in his head FORTRAN, and by the end of the decade,
  a program was implemented to do exactly that.

  \vspace{\fill}

  This was the first FORTRAN compiler,

  \vspace{\fill}

  And then, programming languages were real.
\end{frame}

\sectionSlide{2}{Compiler Theory}

\begin{frame}[fragile]
  \frametitle{What is a Compiler?}

  What on earth really is a compiler?

  \vspace{\fill}

  \defBox{}{A \term{compiler} can be put in very broad terms as a \textbf{program
  that translates data from one form to another}. Usually, we refer to a compiler
  for programming languages, which turns text in some language into some other form.}

  \vspace{\fill}

  For instance, SML/NJ is a compiler which takes in SML programs, and turns it into
  native machine code for your computer, for a variety of computer architectures.

  \vspace{\fill}

  A sister idea is also that of:
  \defBox{}{An \term{interpreter} is a program which reads in a program, and executes
  it on the computer directly, without necessarily explicitly translating it to a
  different form.}

  \vspace{\fill}

  The SML/NJ REPL is an interpreter.
\end{frame}

\begin{frame}[fragile]
  \frametitle{Compilers and Interpreters}

  Put in SML terms, we can come up with a type signature for compilers and
  interpreters:
  \begin{codeblock}
    (* SML text -> assembly language text *)
    val compile : string -> string
    (* a function which executes some assembly code directly.
     * all computers come with this prepackaged. *)
    val run : string -> unit

    (* execute SML text directly *)
    val interpret : string -> unit
  \end{codeblock}

  \vspace{\fill}

  In a broad sense, we should have that \code{interpret} $\eeq$ \code{run o compile}.
  Put simply, interpreting a program is the same as compiling it first, and then
  running it.
\end{frame}

\begin{frame}[fragile]
  \frametitle{Compilers and Interpreters}

  A way to understand interpreters and compilers comes from natural language. For
  instance, I am both a compiler and an interpreter.

  \vspace{\fill}

  I am a compiler, because I have a rudimentary understanding of Mandarin
  Chinese. This means that when I hear Mandarin, I can translate it into English
  first, and then understand the resulting English.

  \vspace{\fill}

  I am also an interpreter, because I natively speak English. When I hear English,
  I immediately understand its meaning, without needing to mentally translate it.
\end{frame}

\begin{frame}[fragile]
  \frametitle{Bootstrapping Compilers}

  One idea that always is hard to accept is the fact that a programming language
  can be implemented in itself.

  \vspace{\fill}

  For instance, \code{pypy} is a Python implementation written in Python. Similarly,
  the SML/NJ compiler is written in SML, and the Rust compiler is written in Rust.

  \vspace{\fill}

  Why can this happen?

  \vspace{\fill}

  Let's use the shorthand of $\code{compile}_{\text{\textlangle lang\textrangle}}$
  to denote a function of type \code{string -> string}, which takes in code written
  in the language \textlangle lang\textrangle, and outputs assembly instructions.
\end{frame}

\begin{frame}[fragile]
  \frametitle{Programming Languages as Ideas}

  Understand that a programming language originates as just an \textit{idea}.

  \vspace{\fill}

  Before John Backus actually implemented FORTRAN, he had the idea in his head that
  there should be some high-level language that he could program in. He possibly
  had the syntax in his head, and called it FORTRAN, but a program written in
  FORTRAN couldn't do anything, because no computer could understand FORTRAN at
  the time.

  \vspace{\fill}

  But, luckily enough, we had assembly and punch cards. This meant that it was
  already possible to compute -- in particular, it was possible to tell a
  computer how to do \textit{anything}, how to compute any computable function.

  \vspace{\fill}

  One such function of interest is the $\code{compile}_{\text{FORTRAN}}$ function,
  or in other words, a FORTRAN compiler. This happens to be a concrete instance of
  a function which is computable, so it is \textit{possible} to use assembly and
  punch cards to write it.
\end{frame}

\begin{frame}[fragile]
  \frametitle{Bootstrapping Compilers}

  But, once we now have $\code{compile}_{\text{FORTRAN}}$, it is possible to write
  FORTRAN code and actually turn it into machine instructions, and run it.

  \vspace{\fill}

  Therefore, it is possible to write FORTRAN code which computes anything.

  \vspace{\fill}

  In particular, it is still possible to write a function $\code{compile}_{\text{FORTRAN}}$,
  except now instead of being written in assembly language, it is written in FORTRAN
  itself, and given meaning by the existing FORTRAN compiler.

  \vspace{\fill}

  This is an example of what is called \term{bootstrapping} in compiler theory. We
  can implement a language in itself, but only so long as there already exists a
  compiler which can understand that language.
\end{frame}

\begin{frame}[fragile]
  \frametitle{Bootstrapping Compilers}

  So, pypy and SML/NJ and the Rust compiler all work the same way.

  \vspace{\fill}

  Guido van Rossum started with an idea in his head of a language called Python,
  with a particular syntax and a particular kind of evaluation model. It didn't
  exist yet, but it was an idea in his head.

  \vspace{\fill}

  Then, he took his language of choice, and then implemented a function
  $\code{compile}_{\text{Python}}$ in it. This was the first Python compiler, and
  made it possible to write and run Python code.

  \vspace{\fill}

  Once that was done, now we could write any function, including
  \textit{rewriting} $\code{compile}_{\text{Python}}$, but this time in Python.
  This was then called \code{pypy}.
\end{frame}

\begin{frame}[fragile]
  \frametitle{Compilers from 5000 Feet}

  Now that we've set the stage for what compilers are, and for what they do,
  we can talk about the actual implementation of a compiler.

  \vspace{\fill}

  Recall that a compiler, such as $\code{compile}_{\text{SML}}$, is written
  in some language, and has type \code{string -> string}, where it takes in
  the text of an SML program and returns an executable.

  \vspace{\fill}

  A compiler achieves this goal by translating the input program through a
  series of intermediate forms, simplifying and optimizing along the way,
  until finally producing an assembly program.
\end{frame}

\begin{frame}[fragile]
  \frametitle{Compilers and Functional Programming}

  A meta question comes to mind, which is -- why are we studying compilers in
  a course on functional programming?

  \vspace{\fill}

  \textbf{Even the most dogged, determined hater of functional programming cannot deny
  that functional programming is incredibly suited towards writing compilers.
  }

  \vspace{\fill}

  The killer application for functional programming is in writing compilers, because
  compilers are just transformations on data. Functional programming mediates that
  relationship by enforcing typed guarantees on that data, as well as offering
  fundamental constructs (pattern matching, datatype declarations) that make
  the entire process an absolute joy.

  \vspace{\fill}

  For the rest of this presentation, we will assume we are trying to implement
  a toy SML compiler in SML.
\end{frame}

\sectionSlide{3}{Implementing Compilers}

\begin{frame}[fragile]
  \frametitle{Steps of a Compiler}

  Most all compilers follow the same structure:
  \begin{itemize}
    \item \term{lexing}, which takes in a \code{string} program, and outputs a
    \code{token list}, which simply groups together fundamental units of the program
    \item \term{parsing}, which takes in a \code{token list}, and outputs an
    \term{abstract syntax tree} of type \code{ast}, which is a tree representing
    the program's structure
    \item \term{intermediate representation}, which turns the abstract
    syntax tree into \term{abstract assembly}, that breaks apart the high-level
    constructs into assembly-like primitives
    \item \term{optimization}, which tries to fine-tune the abstract assembly to
    be as performant as possible
    \item \term{code generation}, which involves turning the abstract assembly
    into real assembly
  \end{itemize}
\end{frame}

\begin{frame}[fragile]
  \frametitle{Steps of a Compiler}

    \hspace{0.8cm}
    \begin{tikzpicture}
      [primary/.style={rectangle, draw, node distance=0.6in}]
      % \node[text width=0.6in] (A) {lexing};
      % \node[right of=A, text width=0.6in, node distance=0.8in] (B) {parsing};
      % \node[right of=B] (C) {\small intermediate \\ representation};
      % \node[right of=C] (D) {optimization};
      % \node[right of=D] (E) {code \\ generation};
      \node[primary, fill=bgBlue] (A) {program text};
      \node[primary, below of=A, fill=bgGreen] (B) {token list};
      \node[primary, below of=B, fill=bgPurple] (C) {abstract syntax tree};
      \node[primary, below of=C, fill=bgRed] (D) {abstract assembly / control-flow graph};
      \node[primary, below of=D, fill=bgYellow] (E) {real assembly};
      \node[left of=A, node distance=1.5in] (S) {start};
      \node[right of=E, node distance=1.5in] (P) {profit!};
      \draw[-stealth] (A) -- node [right] {\small lexing} (B);
      \draw[-stealth] (B) -- node [right] {\small parsing} (C);
      \draw[-stealth] (C) -- node [right] {\small IR generation} (D);
      \draw[-stealth] (D) -- node [right] {\small codegen} (E);

      \draw[-stealth] (D.south west) to [out=225, in=135, looseness=4] node [left] {optimization} (D.north west);
      \draw[-stealth] (S) -- (A);
      \draw[-stealth] (E) -- (P);
    \end{tikzpicture}
\end{frame}

\begin{frame}[fragile]
  \frametitle{Steps of a Compiler}

    \hspace{0.8cm}
    \begin{tikzpicture}
      [primary/.style={rectangle, draw, node distance=0.6in}]
      % \node[text width=0.6in] (A) {lexing};
      % \node[right of=A, text width=0.6in, node distance=0.8in] (B) {parsing};
      % \node[right of=B] (C) {\small intermediate \\ representation};
      % \node[right of=C] (D) {optimization};
      % \node[right of=D] (E) {code \\ generation};
      \node[primary, fill=bgBlue!130] (A) {program text};
      \node[primary, below of=A, fill=bgGreen!130] (B) {token list};
      \node[primary, below of=B, fill=bgPurple!50, text=black!45] (C) {abstract syntax tree};
      \node[primary, below of=C, fill=bgRed!50, text=black!45] (D) {abstract assembly / control-flow graph};
      \node[primary, below of=D, fill=bgYellow!50, text=black!45] (E) {real assembly};
      \node[left of=A, node distance=1.5in] (S) {start};
      \node[right of=E, node distance=1.5in] (P) {profit!};
      \draw[-stealth] (A) -- node [right] {\small \textbf{lexing}} (B);
      \draw[-stealth] (B) -- node [right, text=black!45] {\small parsing} (C);
      \draw[-stealth] (C) -- node [right, text=black!45] {\small IR generation} (D);
      \draw[-stealth] (D) -- node [right, text=black!45] {\small codegen} (E);

      \draw[-stealth] (D.south west) to [out=225, in=135, looseness=4] node [left, text=black!45] {optimization} (D.north west);
      \draw[-stealth] (S) -- (A);
      \draw[-stealth] (E) -- (P);
    \end{tikzpicture}
\end{frame}

\begin{frame}[fragile]
  \frametitle{Steps of a Compiler}

  Let's visually see how we can think about a compiler. We'll start with an
  example SML program, of type \code{string}.

  \vspace{\fill}

  \begin{codeblock}
    val x = 2 - 1

    fun foo (y : int) = 5 + x
  \end{codeblock}
\end{frame}

\begin{frame}[fragile]
  \frametitle{Steps of a Compiler}

  We then \term{tokenize} the input program, so that instead of thinking of it
  as a list of characters, we group together all characters that are part of
  the same semantic unit.

  \vspace{\fill}

  Linguistically, this is similar to reading sentences as words, instead of as
  a list of letters. Let's highlight all the "words" of this program.

  \vspace{\fill}

  % need to fix coloring here...
  \begin{codeblock}
    <bvalb> <gxg> <p=p> <r2r> <p-p> <r1r>

    <bfunb> <gfoog> <p(p><gyg> <p:p> <gintg><p)p> = <r5r> <p+p> <gxg>
  \end{codeblock}

  \vspace{\fill}

  SML is a whitespace-agnostic language, so we don't actually care about
  the whitespace here. It just serves to separate distinct tokens.

  \vspace{\fill}

  Now, let's turn it into a list of tokens, annotated with each token's meaning.
\end{frame}

\begin{frame}[fragile]
  \frametitle{Steps of a Compiler}

  In SML, we could define a type \code{token} which simply denotes each
  "word" of a program. It might look something like this:

  \begin{codeblock}
    datatype token =
      (* data *)
        ID of string | NUM of int

      (* keywords *)
      | VAL | FUN | TYPE (* ... *)

      (* syntax *)
      | LPAREN | RPAREN | COLON | EQ | PLUS | MINUS
      (* ... *)
  \end{codeblock}
\end{frame}

\begin{frame}[fragile]
  \frametitle{Steps of a Compiler}

  So after tokenization into the \code{token} type we just defined, we might
  get something like this:

  \begin{center}
    \vspace{-1.5cm}
    \begin{tikzpicture}
      [ node distance=1.35cm,
        ]
      \kwdToken{N1}{val}{VAL};
      \idToken[right of=N1]{N2}{x};
      \eqToken[right of=N2]{N3}{=};
      \numToken[right of=N3]{N4}{2};
      \opToken[right of=N4]{N5}{-}{MINUS};
      \numToken[right of=N5]{N6}{1};
    \end{tikzpicture}

    \vspace{-2cm}
    \begin{tikzpicture}
      [ node distance = 1.4cm,
        ]
      \kwdToken{N1}{fun}{FUN};
      \idToken[right of=N1]{N2}{foo};
      \opToken[right of=N2]{N3}{(}{LPAREN};
      \idToken[right of=N3]{N4}{y};
      \opToken[right of=N4]{N5}{:}{COLON};
      \idToken[right of=N5]{N6}{int};
      \opToken[right of=N6]{N7}{)}{RPAREN};
      \eqToken[right of=N7]{N8}{=};
      \numToken[right of=N8]{N9}{5};
      \opToken[right of=N9]{N10}{+}{PLUS};
      \idToken[right of=N10]{N11}{x};
    \end{tikzpicture}
  \end{center}

  \vspace{-1.3cm}
  Or, written as actual SML code:

  \begin{codeblock}
    [ VAL, ID "x", EQ, NUM 2, MINUS, NUM 1]
    @ [ FUN, ID "foo", LPAREN, ID "y", COLON, ID "int", RPAREN ]
    @ [ EQ, NUM 5, PLUS, ID "x"]
  \end{codeblock}
\end{frame}

\begin{frame}[fragile]
  \frametitle{Steps of a Compiler}

    \hspace{0.8cm}
    \begin{tikzpicture}
      [primary/.style={rectangle, draw, node distance=0.6in}]
      % \node[text width=0.6in] (A) {lexing};
      % \node[right of=A, text width=0.6in, node distance=0.8in] (B) {parsing};
      % \node[right of=B] (C) {\small intermediate \\ representation};
      % \node[right of=C] (D) {optimization};
      % \node[right of=D] (E) {code \\ generation};
      \node[primary, fill=bgBlue!50, text=black!45] (A) {program text};
      \node[primary, below of=A, fill=bgGreen!130] (B) {token list};
      \node[primary, below of=B, fill=bgPurple!130] (C) {abstract syntax tree};
      \node[primary, below of=C, fill=bgRed!50, text=black!45] (D) {abstract assembly / control-flow graph};
      \node[primary, below of=D, fill=bgYellow!50, text=black!45] (E) {real assembly};
      \node[left of=A, node distance=1.5in] (S) {start};
      \node[right of=E, node distance=1.5in] (P) {profit!};
      \draw[-stealth] (A) -- node [right, text=black!45] {\small lexing} (B);
      \draw[-stealth] (B) -- node [right] {\small \textbf{parsing}} (C);
      \draw[-stealth] (C) -- node [right, text=black!45] {\small IR generation} (D);
      \draw[-stealth] (D) -- node [right, text=black!45] {\small codegen} (E);

      \draw[-stealth] (D.south west) to [out=225, in=135, looseness=4] node [left, text=black!45] {optimization} (D.north west);
      \draw[-stealth] (S) -- (A);
      \draw[-stealth] (E) -- (P);
    \end{tikzpicture}
\end{frame}



\begin{frame}[fragile]
  \frametitle{Recursive Problems, Recursive Programs}

  One thing we note before proceeding is the fact that programs are naturally
  recursively defined. We can witness this by the fact that all the following
  are programs:
  \begin{codeblock}
    if true then big else ()
  \end{codeblock}

  \begin{codeblock}
    if true then if true then big else () else ()
  \end{codeblock}

  \begin{codeblock}
    if true then if true then if true then big else () else () else ()
  \end{codeblock}
\end{frame}

\begin{frame}[fragile]
  \frametitle{Towards Abstract Syntax}

  It turns out this makes programs a prime candidate for a recursive \code{datatype}
  declaration!

  \vspace{\fill}

  By analogy, if you are familiar with the idea of \term{op trees}, recall that we can
  have a tree corresponding to some arithmetic expression:

  \begin{center}
    \begin{tikzpicture}
      \node[circ] {\code{+}}
        child{node[circ] {\code{-}}
          child{node[hex] {\code{1}}}
          child{node[hex] {\code{2}}}
        }
        child{node[hex] {\code{3}}}
      ;

    \end{tikzpicture}
  \end{center}

  \vspace{\fill}

  This tree happens to denote the expression \code{(1 - 2) + 3}.
\end{frame}

\begin{frame}[fragile]
  \frametitle{Abstract Syntax is Abstract}

  This is just \term{abstract syntax}, though, since it elides some of the
  specific syntactic details, like the fact that there is a left and right paren
  around the subtraction.

  \vspace{\fill}

  In the end, this doesn't matter, because the tree structure serves as a
  proxy for what the parentheses were trying to tell us. We thus can get away
  from the precise coding details, while preserving the meaning, by using an
  \term{abstract syntax tree}, or AST for short.
\end{frame}

\begin{frame}[fragile]
  \frametitle{Programs are Trees}

  Well, we can do something very similar to op trees with programs. We will
  instead have an \term{abstract syntax tree} which denotes the structure
  of the program.

  \vspace{\fill}

  \begin{center}
    \begin{minipage}{0.6\textwidth}
      \raggedright
      This tree denotes the program
      \begin{codeblock}
        val x = (1 - 2) + 3
      \end{codeblock}

      Note how it has no mention of parens or the \code{=} sign, because they don't
      actually matter in terms of what the program \textit{means}!

      Generally, we can get rid of things like colons, equals signs, keywords,
      and parentheses in abstract syntax. These syntactic details only existed
      to let us know what the actual underlying tree looked like.
    \end{minipage}
    \begin{minipage}{0.38\textwidth}
      \centering
      \begin{tikzpicture}[level distance=12mm]
        \node[box] {\code{ValDec}}
          child{node[box, fill=bgBlue] {\code{x}}}
          child{node[circ] {\code{+}}
            child{node[circ] {\code{-}}
              child{node[hex] {\code{1}}}
              child{node[hex] {\code{2}}}
            }
            child{node[hex] {\code{3}}}
          }
          ;
      \end{tikzpicture}
    \end{minipage}
  \end{center}
\end{frame}

\begin{frame}[fragile]
  \frametitle{Steps of a Compiler}

  So, for our running example program, we could obtain
  the following abstract syntax tree:
  \begin{center}
    \begin{tikzpicture}
      [
      ]
      \node[box] (T) {\code{TopDecs}\strut}
        child{node[box, xshift=-1in] (M1) {\code{ValDec}}
          child{node[bbox] (L1) {\code{"x"}}}
          child{node[circ, minimum size=0.8cm] (L2) {\code{-}}
            child{node[hex, xshift=-0.3cm] (L3) {\code{Int 2}}}
            child{node[hex, xshift=0.3cm] (L4) {\code{Int 1}}}
          }
        }
        child{node[box, xshift=1in] (M2) {\code{FunDec}}
          child{node[bbox, xshift=-1cm] (R1) {\code{"foo"}}}
          child{node[box, fill=bgRed] (R2) {\code{TypedPat}\strut}
            child{node[bbox] (RU1) {\code{"x"}}}
            child{node[box, fill=bgOrange] (RU2) {\code{int}}}
          }
          child{node[circ, xshift=2cm, minimum size=0.8cm] (R3) {\code{+}}
            child{node[hex, xshift=-0.3cm] (RU3) {\code{Int 5}}}
            child{node[hex, xshift=0.3cm] (RU3) {\code{Id "x"}}}
          }
        };
      % \node[box, below of=L2, node distance=1.5cm] (LU2) {\code{Int 2}};
    \end{tikzpicture}
  \end{center}
\end{frame}

\begin{frame}[fragile]
  \frametitle{Steps of a Compiler}

  The SML code for how we might represent an abstract syntax tree looks
  similarly to the $C_{not}$ problem from your homework. We could write:

  \small
  \begin{codeblock}
    datatype exp =
        Int of int
      | Id of string
      | Plus of exp * exp               (* e1 + e2 *)
    and declaration =
        ValDec of pat * exp             (* val <pat> = <exp> *)
      | FunDec of string * pat list * exp
          (* fun <id> <p1> ... <pn> = <e> *)
      | (* ... *)
    and pattern =
        IdPat of string
      | TuplePat of pattern list        (* (p1, p2, ..., pn) *)
      | (* ... *)
  \end{codeblock}
\end{frame}

\begin{frame}[fragile]
  \frametitle{Steps of a Compiler}

  Usually, an abstract syntax tree can be obtained from a list of tokens via a
  straightforward \term{recursive-descent parser}.

  \vspace{\fill}

  \defBox{}{A \term{recursive-descent parser} is one comprised of many mutually
  recursive functions, each of which simply is responsible for parsing a single
  construct (such as expressions, patterns, declarations) from a list of tokens.}

  \vspace{\fill}

  So for instance, we might have a bunch of functions, such as:
  \begin{codeblock}
    val parsePat : token list -> pat * token list
    val parseExp : token list -> exp * token list
    val parseDec : token list -> dec * token list
  \end{codeblock}

  which parse a tree corresponding to a pattern, expression, or declaration from
  the front of a \code{token list}, and then return the rest of the \code{token list}.
  This should feel like the regex matcher!
\end{frame}

\begin{frame}[fragile]
  \frametitle{Steps of a Compiler}

  The SML code for that might look like:

  \small
  \begin{codeblock}
    fun parseDec (ts : token list) : dec * token list =
      case ts of
        (* val <pat> = <exp> *)
        VAL::ts2 =>
          let
            val (pat, ts2) = parsePat ts
            val ts3 = expect EQ ts2
            val (exp, ts4) = parseExp ts3
          in
            (ValDec (pat, exp), ts4)
          end
      | FUN::ts2 => (* ... *)
      | TYPE::ts2 => (* ... *)
      | (* ... *)
  \end{codeblock}
\end{frame}

\begin{frame}[fragile]
  \frametitle{Steps of a Compiler}

  We can see how it works by going back to our \code{token list} from earlier.

  \vspace{\fill}

  Using the \code{parseDec} function, we essentially split the token list into
  regions based on whether they correspond to patterns or expressions or
  related:

  \begin{center}
    \begin{tikzpicture}
      [node distance=1.5cm]
      \kwdToken{N1}{val}{VAL};
      \idToken[right of=N1]{N2}{x};
      \eqToken[right of=N2]{N3}{=};
      \numToken[right of=N3]{N4}{2};
      \eqToken[right of=N4]{N5}{-};
      \numToken[right of=N5]{N6}{1};
      \kwdToken[right of=N6]{N7}{fun};
      \idToken[right of=N7]{N8}{foo};

      \node[above of=N1, xshift=-0.6cm, yshift=-1cm] (D1) {};
      \node[above of=N6, xshift=0.6cm, yshift=-1cm] (D2) {};
      \draw [decorate,
             decoration={calligraphic brace, raise=-5pt, amplitude=5pt},
             line width=0.5mm] (D1) -- (D2)
        node[pos=0.5,above=2pt,black]{\code{parseDec}}
      ;


      \node[above of=N7, xshift=-0.6cm, yshift=-1cm] (R1) {};
      \node[above of=N8, xshift=0.6cm, yshift=-1cm] (R2) {};
      \draw [decorate,
             decoration={calligraphic brace, raise=-5pt, amplitude=5pt},
             line width=0.5mm] (R1) -- (R2)
        node[pos=0.5,above=2pt,black]{(rest of token list)}
      ;

      \node[below of=N2, xshift=-0.6cm, yshift=0.1cm] (P1) {};
      \node[below of=N2, xshift=0.6cm, yshift=0.1cm] (P2) {};
      \draw [decorate, decoration={calligraphic brace, mirror, raise=-10pt, amplitude=5pt}, line width=0.5mm, yshift=-0.5cm] (P1) -- (P2)
        node[pos=0.5,below=-2pt,black]{\code{parsePat}}
      ;

      \node[below of=N4, xshift=-0.6cm, yshift=0.1cm] (E1) {};
      \node[below of=N6, xshift=0.6cm, yshift=0.1cm] (E2) {};
      \draw [decorate, decoration={calligraphic brace, mirror, raise=-10pt, amplitude=5pt}, line width=0.5mm, yshift=-0.5cm] (E1) -- (E2)
        node[pos=0.5,below=-2pt,black]{\code{parseExp}}
      ;
    \end{tikzpicture}
  \end{center}

\end{frame}

\begin{frame}[fragile]
  \frametitle{On Implementing Parsing and Lexing}

    The highlighted regions then return to us subtrees of certain types,
    as the return values of \code{parsePat} and \code{parseExp}.

    \vspace{\fill}

    Then, \code{parseDec} just needs to string them together, and place
    them under the constructor \code{ValDec}, corresponding to a node of the
    syntax tree representing a \code{val} declaration.

    \vspace{\fill}

    \begin{minipage}{0.45\textwidth}
      \code{parsePat} $\longmapsto$ \tikz[baseline] \node[anchor=base, bbox]{\code{"x"}}; \\

      \vspace{15pt}

      \code{parseExp} $\longmapsto$ \tikz[baseline]
        \node[circ, minimum size=0.8cm] (L2) {\code{-}}
            child{node[hex, xshift=-0.3cm] (L3) {\code{Int 2}}}
            child{node[hex, xshift=0.3cm] (L4) {\code{Int 1}}}
        ;
    \end{minipage}
    \begin{minipage}{0.45\textwidth}
      \code{parseDec} $\longmapsto$ \tikz[baseline]
      \node[box, xshift=-1in] (M1) {\code{ValDec}}
          child{node[bbox] (L1) {\code{"x"}}}
          child{node[circ, minimum size=0.8cm] (L2) {\code{-}}
            child{node[hex, xshift=-0.3cm] (L3) {\code{Int 2}}}
            child{node[hex, xshift=0.3cm] (L4) {\code{Int 1}}}
          }
      ;
    \end{minipage}
\end{frame}

\begin{frame}[fragile]
  \frametitle{On Implementing Parsing and Lexing}

  I have so far been brief on how you implement lexers and parsers.

  \vspace{\fill}

  If you look at the code above for the \code{parseDec} function, it is
  reasonably boring-looking. The code mostly follows the structure of
  the program.

  \vspace{\fill}

  Lexing and parsing are generally considered to be "solved" problems, in the
  field of computer science, due to decades of advances in lexing and parsing
  theory. It turns out that lexers and parsers are so straightforward that most
  lexers and parser nowadays are no longer written by humans, but actually
  autogenerated from a specification.

  \vspace{\fill}

  So they are generally not super important. I enjoy writing parsers, though.
  \footnote{Such generators are called \term{lexer and parser generators}.
  You could also call them \term{lexer compilers} and \term{parser compilers},
  confusingly.}
\end{frame}

\begin{frame}[fragile]
  \frametitle{Playing with Program Trees}

  Once you have ASTs, everything interesting can happen.

  \vspace{\fill}

  ASTs essentially encode all the essential syntactic structure of a program,
  while getting rid of all of the fat that is not strictly necessary to understand
  what a program \textit{means}.

  \vspace{\fill}

  It is at this point we can write transformations and analyses on those ASTs,
  to imbue the tree with actual meaning.

  \vspace{\fill}

  \keyBox{}{Seen in this way, compilation is nothing more than transformation
  and operations on trees.}
\end{frame}

\begin{frame}[fragile]
  \frametitle{Playing with Program Trees}

  For instance, something very pivotal that occurs soon after you obtain
  a syntax tree is that you can type-check a program. The code for that
  might look like:

  {\small
  \begin{codeblock}
    datatype ty = IntTy | StringTy | RealTy | (* ... *)

    fun typecheck (e : exp) : ty =
      case e of
        Int _ => IntTy
      | Plus (e1, e2) =>
          (case typecheck (e1, e2) of
            (IntTy, IntTy) => IntTy
          | (RealTy, RealTy) => RealTy
          | _ => raise Fail "expected pair of ints or reals"
          )
      | (* ... *)
  \end{codeblock}
  }

  \vspace{\fill}

  We see it's just a straightforward recursive function on a tree.
\end{frame}

\begin{frame}[fragile]
  \frametitle{Compiler Optimizations}

  Type-checking is just one operation you can perform on an AST.

  \vspace{\fill}

  Another critical part of compilers is \term{compiler optimizations},
  which is the problem of turning an AST into a \textit{semantically equivalent}
  AST that is simpler in some form.

  \vspace{\fill}

  \begin{center}
  \begin{minipage}[t]{0.45\textwidth}
    \raggedright

    For instance, consider the sub-tree of our running program, corresponding
    to the declaration of \code{x}:
    \begin{center}
    \begin{tikzpicture}[level distance=12mm]
      \node[box, xshift=-1in] (M1) {\code{ValDec}}
        child{node[bbox] (L1) {\code{"x"}}}
        child{node[circ, minimum size=0.8cm] (L2) {\code{-}}
          child{node[hex, xshift=-0.3cm] (L3) {\code{Int 2}}}
          child{node[hex, xshift=0.3cm] (L4) {\code{Int 1}}}
        };
    \end{tikzpicture}
    \end{center}
  \end{minipage}
  \hspace{10pt}
  \begin{minipage}[t]{0.45\textwidth}
    \raggedright

    However, on inspection, it might as well be:

    \begin{center}
    \begin{tikzpicture}[level distance=12mm]
      \node[box, xshift=-1in] (M1) {\code{ValDec}}
        child{node[bbox] (L1) {\code{"x"}}}
        child{node[hex] (L2) {\code{Int 1}}
        };
    \end{tikzpicture}
    \end{center}
  \end{minipage}
  \end{center}
\end{frame}

\begin{frame}[fragile]
  \frametitle{Constant Folding}

  This is a classic compiler optimization called \term{constant folding}.

  \vspace{\fill}

  \defBox{}{\term{Constant folding} is a compiler optimization where the AST
  is simplified to evaluate any constant arithmetic expressions at compile time,
  simplifying the AST.}

  \vspace{\fill}

  It may seem fairly trivial to simplify expressions like \code{2 - 1}. After
  all, this takes a fairly miniscule amount of time at runtime! Just remember:
  \begin{itemize}
    \item \textbf{Optimizations add up}. Being able to simplify an expression
    might lead to further simplifications, especially nearer to
    assembly, when pretty much everything just becomes arithmetic operations.
    \item \textbf{Computers repeat themselves}. Computers often run the same routine many, many times, so small
    things like less addition operations can lead to a huge difference, when
    it's happening billions of times.
    \item \textbf{Less operations, less bloat}. Less operations lead to smaller file
    sizes. If I compile the expression \code{2 - 1}, I need to issue multiple
    instructions, versus if I just have \code{1}.
  \end{itemize}
\end{frame}

\begin{frame}[fragile]
  \frametitle{Constant Folding}

  It can be implemented simply via the following recursive function:

  {\small
  \begin{codeblock}
    fun cfold (e : exp) : exp =
      case e of
        Plus  (Int i1, Int i2) => Int (i1 + i2)
      | Minus (Int i1, Int i2) => Int (i1 - i2)
      | Div   (Int i1, Int i2) => Int (i1 div i2)
      | _ => e
  \end{codeblock}
  }

  \vspace{\fill}

  At least, it would be, if this wasn't incredibly wrong.
\end{frame}

\begin{frame}[fragile]
  \frametitle{Constant Folding}

  This code seems reasonable. If we are trying to fold the constant integer
  or an identifier, we can't do anything, so we return just that expression.

  \vspace{\fill}

  Otherwise, if we are folding an arithmetic operation of some literal integer
  arguments, we simplify the tree to that operation's result. However:

  \vspace{\fill}

  Error 1: \textit{This program can crash}.

  \vspace{\fill}

  {\small
  \begin{codeblock}
    fun cfold (e : exp) : exp =
      case e of
        Plus  (Int i1, Int i2) => Int (i1 + i2)
      | Minus (Int i1, Int i2) => Int (i1 - i2)
      `| Div   (Int i1, Int i2) => Int (i1 div i2)`
      | _ => e
  \end{codeblock}
  }
\end{frame}

\begin{frame}[fragile]
  \frametitle{Constant Folding}

  Our program will crash upon trying to compile \code{val x = 1 div 0}!

  \vspace{\fill}

  Compiling is just a staged version of interpreting code -- it is meant to
  produce something which can \textit{then} be run, but a compiler definitely
  cannot crash or loop depending on its input program's behavior. So this is
  a big no-no in compilers.

  \vspace{\fill}

  We need to fix it like so:

  {\small
  \begin{codeblock}
    fun cfold (e : exp) : exp =
      case e of
        Plus  (Int i1, Int i2) => Int (i1 + i2)
      | Minus (Int i1, Int i2) => Int (i1 - i2)
      `| Div   (Int i1, Int 0) => e`
      | Div   (Int i1, Int i2) => Int (i1 div i2)
      | _ => e
  \end{codeblock}
  }
\end{frame}

\begin{frame}[fragile]
  \frametitle{Constant Folding}

  A further problem: our constant folding function doesn't recurse!

  \vspace{\fill}

  This means if the expression to be folded is not literally at the top
  of the expression, it won't happen, for instance for the following tree:

  \begin{center}
  \begin{tikzpicture}[level distance=12mm]
    \centering
    \node[circ, minimum size=0.8cm] {\code{-}}
      child{node[circ, minimum size=0.8cm, xshift=-0.2cm] {\code{+}}
        child{node[hex, xshift=-0.3cm] (L3) {\code{Int 1}}}
        child{node[hex, xshift=0.3cm] (L4) {\code{Int 2}}}
      }
      child{node[hex, xshift=0.2cm] (L3) {\code{Int 3}}}
    ;
  \end{tikzpicture}
  \end{center}

  \vspace{\fill}

  So, let's fix our code again:
\end{frame}

\begin{frame}[fragile]
  \frametitle{Constant Folding}

  {\small
  \begin{codeblock}
    fun cfold (e : exp) : exp =
      case e of
        Plus  (Int i1, Int i2) => Int (i1 + i2)
      | Minus (Int i1, Int i2) => Int (i1 - i2)
      | Div   (Int i1, Int 0)  => e
      | Div   (Int i1, Int i2) => Int (i1 div i2)
      `| Plus  (e1, e2) => Plus  (cfold e1, cfold e2)`
      `| Minus (e1, e2) => Minus (cfold e1, cfold e2)`
      `| Div   (e1, e2) => Div   (cfold e1, cfold e2)`
      `(* ... many more cases ... *)`
  \end{codeblock}
  }

  \vspace{\fill}

  Unfortunately, this means we will need a case for every single
  constructor in the \code{exp} type.\footnote{There are techniques in
  functional programming that mean this can actually be written much more
  tersely than this, without all the boilerplate.}
\end{frame}

\begin{frame}[fragile]
  \frametitle{Constant Folding}

  Constant folding is nice, but in practice usually there's not a whole lot of
  compile-time evaluatable expressions in the actual source of the code.

  \vspace{\fill}

  This would require someone to write something like
  \begin{codeblock}
    val x = 1 + 2
  \end{codeblock}
  in their source code.\footnote{Which happens, and sometimes by some of you,
  but not often.}

  \vspace{\fill}

  For things like pattern matching, function calls, loops, and complex
  data structures, it's difficult to optimize, because these structures are
  too high-level. For most optimizations, we need to descend closer to assembly
  language.
\end{frame}

\begin{frame}[fragile]
  \frametitle{Steps of a Compiler}

    \hspace{0.8cm}
    \begin{tikzpicture}
      [primary/.style={rectangle, draw, node distance=0.6in}]
      % \node[text width=0.6in] (A) {lexing};
      % \node[right of=A, text width=0.6in, node distance=0.8in] (B) {parsing};
      % \node[right of=B] (C) {\small intermediate \\ representation};
      % \node[right of=C] (D) {optimization};
      % \node[right of=D] (E) {code \\ generation};
      \node[primary, fill=bgBlue!50, text=black!45] (A) {program text};
      \node[primary, below of=A, fill=bgGreen!50, text=black!45] (B) {token list};
      \node[primary, below of=B, fill=bgPurple!130] (C) {abstract syntax tree};
      \node[primary, below of=C, fill=bgRed!130] (D) {abstract assembly / control-flow graph};
      \node[primary, below of=D, fill=bgYellow!50, text=black!45] (E) {real assembly};
      \node[left of=A, node distance=1.5in] (S) {start};
      \node[right of=E, node distance=1.5in] (P) {profit!};
      \draw[-stealth] (A) -- node [right, text=black!45] {\small lexing} (B);
      \draw[-stealth] (B) -- node [right, text=black!45] {\small parsing} (C);
      \draw[-stealth] (C) -- node [right] {\small \textbf{IR generation}} (D);
      \draw[-stealth] (D) -- node [right, text=black!45] {\small codegen} (E);

      \draw[-stealth] (D.south west) to [out=225, in=135, looseness=4] node [left] {\textbf{optimization}} (D.north west);
      \draw[-stealth] (S) -- (A);
      \draw[-stealth] (E) -- (P);
    \end{tikzpicture}
\end{frame}


\begin{frame}[fragile]
  \frametitle{Constant Folding}

  For this purpose, we have \term{abstract assembly}, which is a representation
  of primitive code that looks like assembly language, but doesn't need to
  touch things like registers.

  \vspace{\fill}

  Such instructions break apart the nested structure of programs to achieve
  a very simple layout. For instance, we might translate the body of this function as:
  \begin{codeblock}
    fun f x y z =
      x + (y + z)
  \end{codeblock}

  \vspace{\fill}

  \begin{codeblock}
    t1 <- y + z   (* a temp variable *)
    t2 <- x + t1  (* another temp *)
    ret t2        (* return the result *)
  \end{codeblock}

  \vspace{\fill}

  Here, binary operations only take in variables or temps, so we can't
  nest expressions!
\end{frame}

\begin{frame}[fragile]
  \frametitle{Constant Folding}

  This is fine for some examples, such
  This is fine for what is known as \term{straight-line code}, which is code
  that doesn't perform any branches of control flow, or jumps in program logic.

  \vspace{\fill}

  Unfortunately, both of these things are quite common in programming languages.
  To that end, we translate abstract syntax into a \term{control-flow graph},
  or CFGs, which links blocks of straight-line code by pointers which denote
  which blocks can reach each other.

  \vspace{\fill}

  So for instance, in the following code:
  \begin{codeblock}
    fun f x =
      if x then f x
      else f (not x)

    val y = f 0
  \end{codeblock}
\end{frame}

\begin{frame}[fragile]
  \frametitle{Constant Folding}

  \begin{center}
    \begin{minipage}{0.35\textwidth}
      \begin{codeblock}
        fun f x =
          ?if x? then &f x&
          else `f (not x)`

        /val y = f 0/
      \end{codeblock}
    \end{minipage}
    \begin{minipage}{0.6\textwidth}
      \begin{center}
      \begin{tikzpicture}
          [every node/.style={rectangle, draw}]
        \node[fill=bgBlue] (A) {
          \code{y <- call f true}\strut
        };
        \node[below of=A, node distance=1.5in, xshift=0.2in, fill=bgOrange] (B) {
          \code{cmp arg1}\strut
        };
        \node[right of=B, node distance=1in, yshift=0.5in, fill=red!45] (C) {
          \code{ret (call f arg1)}\strut
        };
        \node[right of=B, node distance=1in, yshift=-0.5in, text width=1.5in, fill=yellow!45] (D) {
          \code{t1 <- not arg1} \\
          \code{ret call f t1}
        };
        \draw[-stealth] (A) to [out=270, in=180] (B);
        \draw[-stealth] (B) to [out=0, in=270] (C);
        \draw[-stealth] (B) to [out=270, in=180] (D);
        \draw[-stealth] (C.west) to [out=160, in=90] (B);
      \end{tikzpicture}
      \end{center}
    \end{minipage}
  \end{center}
\end{frame}

\begin{frame}[fragile]
  \frametitle{Constant Folding}

  But, what's something that we notice about this CFG?

  \vspace{\fill}

  \textbf{No matter which way we go, we always end up at a call to \code{f}!}

  \vspace{\fill}

  \begin{center}
  \begin{tikzpicture}
      [every node/.style={rectangle, draw}]
    \node[] (A) {
      \code{y <- call f true}\strut
    };
    \node[below of=A, node distance=1in, xshift=0.2in] (B) {
      \code{cmp arg1}\strut
    };
    \node[right of=B, node distance=1in, yshift=0.5in] (C) {
      {\color{red}\code{ret (call f arg1)}\strut}
    };
    \node[right of=B, node distance=1in, yshift=-0.5in, text width=1.5in] (D) {
      \code{t1 <- not arg1} \\
      {\color{red}\code{ret call f t1}}
    };
    \draw[-stealth] (A) to [out=270, in=180] (B);
    \draw[-stealth] (B) to [out=0, in=270] (C);
    \draw[-stealth] (B) to [out=270, in=180] (D);
    \draw[-stealth] (C.west) to [out=160, in=90] (B);
  \end{tikzpicture}
  \end{center}

  \vspace{\fill}

  In other words, a surefire infinite loop.
\end{frame}

\begin{frame}[fragile]
  \frametitle{Constant Folding}

  With CFGs, you can do easy optimizations, because the flow of the program
  is very apparent. This leads us to a whole host of possible optimizations,
  the list of which is far too comprehensive to go over here.

  \vspace{\fill}

  The basic idea is that these are two main kinds of optimizations:

  \vspace{\fill}

  \defBox{}{\term{Local optimizations} are optimizations which run
  purely within a single basic block of straight-line code.}

  Some examples of these include constant folding and copy propagation.

  \vspace{\fill}

  \defBox{}{\term{Global optimizations} are optimizations which run interprocedurally,
  that is, by considering the flow of data through multiple basic blocks or functions.}

  Some examples of these include dead code elimination, unused variables,
  and function inlining.
\end{frame}

\begin{frame}[fragile]
  \frametitle{Constant Folding}

  Consider the following analogy.

  \vspace{\fill}

  I am really popular. I have lots of friends.

  \vspace{\fill}

  Unfortunately most of them hate each other and I only have 8 seats at
  my birthday party, which is happening at Chuck E. Cheese's.\footnote{Where a kid can be a kid.}

  \vspace{\fill}

  I want as many of them to come, but they are \code{all} only \code{available}
  at certain times, and they'll be cranky if they can't be there for the full time
  they're available.

  \vspace{\fill}

  How can I plan a precise schedule with the time that each friend is allowed to
  come, to maximize their (and my) happiness?
\end{frame}

\begin{frame}[fragile]
  \frametitle{Steps of a Compiler}

    \hspace{0.8cm}
    \begin{tikzpicture}
      [primary/.style={rectangle, draw, node distance=0.6in}]
      % \node[text width=0.6in] (A) {lexing};
      % \node[right of=A, text width=0.6in, node distance=0.8in] (B) {parsing};
      % \node[right of=B] (C) {\small intermediate \\ representation};
      % \node[right of=C] (D) {optimization};
      % \node[right of=D] (E) {code \\ generation};
      \node[primary, fill=bgBlue!50, text=black!45] (A) {program text};
      \node[primary, below of=A, fill=bgGreen!50, text=black!45] (B) {token list};
      \node[primary, below of=B, fill=bgPurple!50, text=black!45] (C) {abstract syntax tree};
      \node[primary, below of=C, fill=bgRed!130] (D) {abstract assembly / control-flow graph};
      \node[primary, below of=D, fill=bgYellow!130] (E) {real assembly};
      \node[left of=A, node distance=1.5in] (S) {start};
      \node[right of=E, node distance=1.5in] (P) {profit!};
      \draw[-stealth] (A) -- node [right, text=black!45] {\small lexing} (B);
      \draw[-stealth] (B) -- node [right, text=black!45] {\small parsing} (C);
      \draw[-stealth] (C) -- node [right, text=black!45] {\small IR generation} (D);
      \draw[-stealth] (D) -- node [right] {\small \textbf{codegen}} (E);

      \draw[-stealth] (D.south west) to [out=225, in=135, looseness=4] node [left, text=black!45] {optimization} (D.north west);
      \draw[-stealth] (S) -- (A);
      \draw[-stealth] (E) -- (P);
    \end{tikzpicture}
\end{frame}

\begin{frame}[fragile]
  \frametitle{Constant Folding}

  Once the control flow graph has been simplified to satisfaction, we can
  then move on to the step of generating actual assembly instructions from
  the abstract assembly.

  \vspace{\fill}

  The primary difference is that the abstract assembly assumes that you can
  move values to and from an infinite array of temporary variables.

  \vspace{\fill}

  Unfortunately, actual computers exist in the real world, and do not have
  an infinite amount of places you can put data. This isn't a big deal always,
  though, because we don't need to keep all data around forever.

  \vspace{\fill}

  This means that essentially, a compiler must have future sight, and perfectly
  plan out everywhere that it puts its data during the program's run. This
  problem is called \term{register allocation}, and is far beyond the
  scope of this course, and my ability to explain right now.
\end{frame}

\begin{frame}[fragile]
  \frametitle{Constant Folding}

  Then, finally, real assembly is generated from the abstract assembly, and
  the executable is fit to run. The story is over.

  \vspace{\fill}

  This is the bird's eye view of a compiler. There are many facets to each
  phase that were not elaborated upon, but in terms of the intuition behind
  what a compiler is and does, this should suffice.

  \vspace{\fill}

  The key thing to realize here is that at the end of the day, compilers are
  not magic!
\end{frame}

\begin{frame}[fragile]
  \frametitle{Functional Programming and Compilers}

  Functional programming ultimately fits compilers well for a few reasons:
  \begin{itemize}
    \item \textbf{Compilers are fancy tree transformations.} Functional languages
    are very good at dealing with recursively defined structures which look like
    trees.
    \item \textbf{Compilers cannot be wrong}. If you compile your code, and the
    resulting code does not do what it should, this is the doomsday scenario. If
    we cannot trust the compilers we use, then there is no hope for programming.
    Functional programming is safer in general, leading to less silly mistakes and
    unsafe behavior.
    \item \textbf{Compilers are pure}. When translating a compiler from one
    stage to another, we generally expect deterministic results. Thus, no
    mutability needed.
  \end{itemize}
\end{frame}

% \begin{frame}[fragile]
%   \frametitle{Constant Folding}

%   So, what happens if we run constant folding on the tree we got earlier?
%   \begin{center}
%     \begin{minipage}{0.4\textwidth}
%     \centering
%     \begin{tikzpicture}[level distance=12mm]
%       \node[circ, fill=bgPurple, minimum size=0.8cm] {\code{-}}
%         child{node[circ, fill=bgPurple, minimum size=0.8cm, xshift=-0.2cm] {\code{+}}
%           child{node[hex, xshift=-0.3cm] (L3) {\code{Int 1}}}
%           child{node[hex, xshift=0.3cm] (L4) {\code{Int 2}}}
%         }
%         child{node[hex, xshift=0.2cm] (L3) {\code{Int 3}}}
%       ;
%     \end{tikzpicture}
%     \end{minipage}
%     \begin{minipage}{0.1\textwidth}
%       \centering
%       $\longmapsto$
%     \end{minipage}
%     \begin{minipage}{0.4\textwidth}
%     \centering
%     \begin{tikzpicture}[level distance=12mm]
%       \node[circ, fill=bgPurple, minimum size=0.8cm] {\code{-}}
%         child{node[hex, xshift=-0.2cm] {\code{Int 3}}
%         }
%         child{node[hex, xshift=0.2cm] (L3) {\code{Int 3}}}
%       ;
%     \end{tikzpicture}
%     \end{minipage}
%   \end{center}

%   We then enter this case:
%   \begin{codeblock}
%     | Plus  (e1, e2) => Plus  (cfold e1, cfold e2)
%   \end{codeblock}

%   But, this only simplifies the arguments to the \code{Plus}, and just then
%   returns the \code{Plus} of those things, naively.

%   This means we don't simplify, considering our new arguments!
% \end{frame}

% \begin{frame}[fragile]
%   \frametitle{Constant Folding}

%   So, we need one final optimization:

%   {\small
%   \begin{codeblock}
%     fun cfold (e : exp) : exp =
%       case e of
%         Plus  (Int i1, Int i2) => Int (i1 + i2)
%       | Minus (Int i1, Int i2) => Int (i1 - i2)
%       | Div   (Int i1, Int 0) => e
%       | Div   (Int i1, Int i2) => Int (i1 div i2)
%       | Plus  (e1, e2) => Plus  (cfold e1, cfold e2)
%       | Minus (e1, e2) => Minus (cfold e1, cfold e2)
%       | Div   (e1, e2) => Div   (cfold e1, cfold e2)
%       (* ... many more cases ... *)
%     and cfold' (e : exp) : exp =
%       cfold e
%   \end{codeblock}
%   }
% \end{frame}


\subsection{foo}

\begin{frame}[fragile]
  \frametitle{Constant Folding}

\end{frame}


\begin{frame}[plain]
	\begin{center} Thank you! \end{center}
\end{frame}


\end{document}
