% Jacob Neumann

% DOCUMENT CLASS AND PACKAGE USE
    \documentclass[aspectratio=169, handout]{beamer}

    % Establish the colorlambda boolean, to control whether the lambda is solid color (true), or the same as the picture (false)
    \newif\ifcolorlambda
    \colorlambdafalse % DEFAULT: false

    % Use auxcolor for syntax highlighting
    \newif\ifuseaux
    \useauxfalse % DEFAULT: false

    % Color settings
    \useauxtrue

    \newcommand{\auxColor}{2791e3}
    \newcommand{\presentColor}{2e2e2e} % the primary color of the slide borders
    \newcommand{\bgColor}{d6d6d6}      % the color of the background of the slide
    \newcommand{\darkBg}{8b98ad}
    \newcommand{\lambdaColor}{\auxColor}

    \colorlambdatrue

    \usepackage{comment} % comment blocks
    \usepackage{soul} % strikethrough
    \usepackage{listings} % code
    \usepackage{makecell}

    \setbeamertemplate{itemize items}[circle]
    % \setbeameroption{show notes on second screen=right}

    \usepackage{lectureSlides}
    %%%%%%%%%%%%%%%%%%%%%%%%%%%%%%%%%%%%%%%%%| <----- Don't make the title any longer than this
    \title{Prologue} % TODO
    \subtitle{A humble beginning} % TODO
    \date{16 May 2023} % TODO
    \author{Brandon Wu} % TODO

    \graphicspath{ {./img/} }
    % DONT FORGET TO PUT [fragile] on frames with codeblocks, specs, etc.
        %\begin{frame}[fragile]
        %\begin{codeblock}
        %fun fact 0 = 1
        %  | fact n = n * fact(n-1)
        %\end{codeblock}
        %\end{frame}

    % INCLUDING codefile:
        % 1. In some file under code/NN (where NN is the lecture id num), include:
    %       (* FRAGMENT KK *)
    %           <CONTENT>
    %       (* END KK *)

    %    Remember to not put anything on the same line as the FRAGMENT or END comment, as that won't be included. KK here is some (not-zero-padded) integer. Note that you MUST have fragments 0,1,...,KK-1 defined in this manner in order for fragment KK to be properly extracted.
        %  2. On the slide where you want code fragment K
                % \smlFrag[color]{KK}
        %     where 'color' is some color string (defaults to 'white'. Don't use presentColor.
    %  3. If you want to offset the line numbers (e.g. have them start at line 5 instead of 1), use
                % \smlFragOffset[color]{KK}{5}

\begin{document}

% Make it so ./mkWeb works correctly
\ifweb
    \renewcommand{\pause}{}
\fi

\setbeamertemplate{itemize items}[circle]

% SOLID COLOR TITLE (see SETTINGS.sty)
\begin{frame}[plain]
    \colorlambdatrue
    \titlepage
\end{frame}

\begin{comment}
  Future Retro from ANIMUSIC plays as students walk in
  When the music ends, BRANDON bursts in through the door

  HELLO EVERYONE! Welcome to 15-150, Principles of Functional Programming.

  I'm super excited to have all of you here. My name is Brandon Wu, and I'll be your
  primary instructor for this summer. Please call me Brandon, though confusingly
  there is also a TA named Brandon, so I guess we'll have to just figure that one out.

  To properly introduce myself, I'm actually a recent graduate of CMU, I graduated
  with my bachelor's degree in computer science last spring, and when I was in undergrad,
  I TA'd this class five times, and I was Head TA for two semesters. While I am lucky
  enough to teach this course to you this summer, I have a full-time job working
  for a company called Semgrep in the Bay, where I actually do functional programming
  every single day.

  With intros over, let's hop to it. Today we're here to learn functional programming.
\end{comment}

\begin{frame}[fragile]
  \frametitle{Lesson Plan}

  \tableofcontents
\end{frame}

\sectionSlide{1}{Administrivia}

\begin{frame}[fragile]
  \frametitle{Myself}

  I'm an alum who graduated last year with a degree in computer science from CMU.

  \vspace{\fill}

  I TA'd 150 for three years while I was at CMU, and I currently work a
  full-time job as a software engineer at a security company called
  {\color{blue}\href{https://semgrep.dev/}{Semgrep}}, doing functional
  programming\footnote{Using a language very similar to the one you will learn in this class.} for program analysis.

  \vspace{\fill}

  This means I am uniquely equipped as someone who works in the industry, to say that
  \textbf{functional programming is really useful}.

  \vspace{\fill}

  \noteBox{\, \textbf{I am not a professor}.}
\end{frame}

\begin{frame}[fragile]
    \frametitle{The Course Staff}

    \begin{center}
      \includegraphics[width=0.12\columnwidth]{nancy} \qquad \qquad \qquad
      \includegraphics[scale=0.3]{me_smaller.jpg} \qquad \qquad \qquad
      \includegraphics[width=0.12\columnwidth]{sonya}
    \end{center}
    \begin{center}
      \large Nancy \qquad \qquad \quad
      \Large Brandon Wu \qquad \qquad \quad
      \large Sonya
    \end{center}

    \vspace{\fill}

    \begin{center}\begin{tabular}{c c c c c c c c }
      Kaz & Deya & Caroline & Stephen & Brandon & Michael \\
      \includegraphics[width=0.1\columnwidth]{kaz} &
      \includegraphics[width=0.1\columnwidth]{deya} &
      \includegraphics[width=0.1\columnwidth]{caroline} &
      \includegraphics[width=0.1\columnwidth]{stephen} &
      \includegraphics[width=0.1\columnwidth]{brandon_little} &
      \includegraphics[width=0.1\columnwidth]{michael} \\
    \end{tabular}\end{center}
\end{frame}

\begin{frame}[fragile]
  \frametitle{Course Logistics}

  \begin{itemize}
    \item Homework every week -- due on Tuesdays or Thursdays (see the lecture schedule!)
    \item Turn in assignments via Gradescope
    \item Receive assignments via Canvas
    \item Piazza for any questions
    \item If you are not in all of \{ Gradescope, Piazza, Canvas \} please let me know
    \item Slides, course policy, and other information at the {\color{blue}\href{http://www.cs.cmu.edu/~15150/}{course website here}}
  \end{itemize}
\end{frame}

\begin{frame}[fragile]
  \frametitle{The Schedule}
  \begin{center}
    \large
    \color{blue}
    \href{https://docs.google.com/spreadsheets/d/1Q6utpsc1vm2WbW2aDyR4mGiFoDJSiBsVSiQmkBarqT0/edit?usp=sharing}{The full summer schedule can be found here (click me!)}
  \end{center}
\end{frame}

\begin{frame}[fragile]
  \frametitle{Grading}

  The grading policy for this semester will follow a \textit{choose-your-own-adventure} format.

  \pause
  \vspace{\fill}

  There are two tracks of grades that we will support, the "Lecture Track" and the "Homework Track".

  \pause
  \vspace{\fill}

  The Lecture Track is for students who would like to earn points via lecture attendance. Homework
  will count for 42\% of the final grade, and lecture attendance will be 3\%.

  \vspace{5pt}

  The Homework Track is for students who would rather not be graded for lecture attendance. Homework
  will instead count for 45\%, and there will be no grading on lecture attendance.

  \vspace{\fill}

  Grading scheme selection will occur via Piazza after the first week.
\end{frame}

\begin{frame}[fragile]
  \frametitle{Grading: Lecture Track}

  \begin{minipage}{0.59\textwidth}
    (visualized via \% of Farnam) \\

    \vspace{\fill}

    \begin{itemize}
      \item Homework: 42\%
      \item Lecture Attendance: 3\%
      \item Lab Attendance: 10\%
      \item Midterm 1: 10\%
      \item Midterm 2: 15\%
      \item Final: 20\%
    \end{itemize}
  \end{minipage}
  \begin{minipage}{0.4\textwidth}
    \includegraphics[scale=0.45]{farnam_lecture}
  \end{minipage}
\end{frame}

\begin{frame}[fragile]
  \frametitle{Grading: Homework Track}

  \begin{minipage}{0.59\textwidth}

    (visualized via \% of Farnam) \\

    \vspace{\fill}

    \begin{itemize}
      \item Homework: 45\%
      \item Lab Attendance: 10\%
      \item Midterm 1: 10\%
      \item Midterm 2: 15\%
      \item Final: 20\%
    \end{itemize}
  \end{minipage}
  \begin{minipage}{0.4\textwidth}
    \includegraphics[scale=0.45]{farnam_homework}
  \end{minipage}
\end{frame}



\begin{frame}[fragile]
  \frametitle{House System}

  This semester, we are implementing a house system.\footnote{Yes, like Harry Potter.}

  \pause
  \vspace{\fill}

  The class has been divided into three labs of roughly 20 students, which each forms a
  "house". Each house will be eligible to earn points on the merits of:
  \pause
  \begin{itemize}
    \item Asking good questions on Piazza \pause
    \item Giving good answers on Piazza \pause
    \item Answering questions during lecture
  \end{itemize}

  \pause
  \vspace{\fill}

  \textbf{Twice a semester, the house with the most points will earn boba for their entire lab}.
\end{frame}

\begin{frame}[fragile]
  \frametitle{Active Learning}

  It is better to learn by exercising your mind actively, rather than passively absorbing
  information in a lecture setting.

  \pause
  \vspace{\fill}

  To facilitate this, we will have active learning exercises during lecture. There will be
  a short (roughly 5 minute) quiz in the middle of every single lecture, that \textbf{will not
  count towards your grade}. Instead, it will simply be used by TAs to judge where your
  understanding may be lacking.

  \pause
  \vspace{\fill}

  In particular, these quizzes are a whole-house effort! Every house will take the quiz
  together, and whichever house scores the highest cumulatively earn points for their house,
  and be crowned that lecture's victor, along with earning other prizes.

  \pause
  \vspace{\fill}

  This means that winning the house competition is a whole-house effort! It's in your interest
  to discuss with others during these quizzes, and fill in gaps in yours and others'
  understanding.
\end{frame}

\begin{frame}[fragile]
  \frametitle{Wellness}

  \textbf{Your health is important.}

  \pause
  \vspace{\fill}

  The things you will learn in this class are important, but remember that
  \textbf{it's just a class}. If you're sleeping 4 hours a night, skipping meals,
  or otherwise compromising your health for the sake of this class, \textbf{something
  is wrong}.

  \pause
  \vspace{\fill}

  Things that are required:
  \begin{itemize}
    \item Sleeping \pause
    \item Eating \pause
    \item Socializing (for most people) \pause
    \item Paying taxes
  \end{itemize}

  \pause
  \vspace{\fill}

  Things that are not required:
  \begin{itemize}
    \item Finishing 150 homework at the expense of these four things\footnote{Especially the last one.}
  \end{itemize}
\end{frame}

\begin{frame}[fragile]
  \frametitle{Wellness}
  Take care of yourself, and if you're experiencing difficulties, don't hesitate to
  reach out to myself or a TA you trust. For more severe cases, please go to
  {\color{blue}\url{http://www.cmu.edu/counseling/}}

  \pause
  \vspace{\fill}

  My job is to teach you, not to ruin your life. If you're struggling, \textbf{talk to me}.
  Let's make a plan to help you succeed.
\end{frame}




\sectionSlide{2}{The Philosophy of Functional Programming}

\begin{frame}[fragile]
  \frametitle{What is Functional Programming?}

  \pause

  \begin{center}
    What is functional programming?
  \end{center}

  \pause

  \vspace{\fill}

  \begin{center}
    Well, hold up. What is programming?
  \end{center}

  \note{
    To many of you, this is likely a brand new topic, and that's perfectly OK. How do I
    like to describe it? Functional programming is a kind of paradigm of programming,
    like object-oriented or imperative programming. It's a kind of style, but deeper than
    the sense of MLA or Chicago style. It's not just how you format the text, but it's a
    philosophically different style of programming.

    Before I can go into the philosophy of functional programming though, I'd like to
    detour a bit and talk about programming.
  }
\end{frame}


\begin{frame}[fragile]
  \frametitle{What is Programming?}

  \note{
    So, some of you might have an answer to this question. What is programming? \\
  }

  \pause
  \fontfamily{phv}\selectfont

  \begin{center}
    Programming is the act of \textbf{instructing a computer on how to achieve some kind of operation.}
  \end{center}

  \note{
    Programming is the act of instructing a computer on how to achieve some kind of operation. \\

    This might be something like computing the square root of a number, or searching a paragraph
    of text for a particular keyword, or ordering a pizza online. All of you have programmed before,
    so you're familiar with the concept, I'm sure. \\
  }

  \pause

  \vspace{5pt}

  \begin{center}
    Programming is \textbf{inherently a communicative act.}
  \end{center}

  \pause
  \vspace{\fill}

  \textbf{Instructing} is the key word. Good communication exists, so what is good programming?

  \pause
  \vspace{\fill}

  \begin{center}
    Functional programming is \textbf{an improvement on our ability to communicate as programmers}.
  \end{center}

  \note{
    The key part of that description, however, is \textit{instruction}. You can instruct someone in lots
    of different ways. It all boils down to someone's particular communication style. In
    that way, programming is inherently a communicative act. It's not just math or computer science, it's
    \textit{linguistic}. I'm sure many of you have read bad code before. \\

    So, we know that some things make good communication, but what things make good programming?
  }

  \vspace{\fill}
\end{frame}

\begin{frame}[fragile]
  \frametitle{What should programming be?}

  \begin{center}
    Good programming should be \textbf{descriptive}. \\

    \vspace{\fill}
  \end{center}

  \centerline{
    \begin{minipage}{0.4\textwidth}
      \pause
      \includegraphics[scale=0.35]{goto.jpeg}
    \end{minipage}
    \begin{minipage}{0.4\textwidth}
      \includegraphics[scale=0.5]{spaghetti.png}
    \end{minipage}
  }

  \note{
    So how can we characterize good programming? \\

    Well, for starters, programming should be descriptive. Good code should be self-documenting,
    in the sense that you can read it easily and figure out what's going on. This is something that is
    more subtle than just the names of your variables or the layout of your code, however. \\

    When I think about descriptive code, I'm always reminded of this old paper of Edgar Dijkstra,
    one of the pioneers of the field of computer science. You might know about it. It was called
    "Go To Statement Considered Harmful", and it was basically a treatise against something called
    a "go to statement". Before the days of loops and conditionals, there was the goto, which was
    basically an unrestricted teleportation cheat code. It said, go here and start doing this
    instead. It was the wild west of programming, where programs could sometimes just go and
    start doing something completely different, and it made programs extremely hard to read. \\

    So when I say programs should be descriptive, I mean in a more \textit{fundamental} way that is
    internal to what the program \textit{actually means}. And this can vary, from language to language.
    In some languages, it's quite easy to express something than in another, but it goes deeper than
    just syntax. \\
  }
\end{frame}


\begin{frame}[fragile]
  \frametitle{What should programming be?}
  \begin{center}
    Good programming should be \textbf{modular}.

    \pause
    \vspace{\fill}

    Logically distinct parts should be separated, for separate maintenance and reuse.

    \pause
    \vspace{\fill}

    You should be able to think about a single area of a codebase without needing to
    concern yourself with unrelated logic.


  \end{center}

  \note{
    Next, good programming should be modular. You've probably heard this before, but modularity
    in code basically just means that things should be broken up into logically distinct parts.
    If I go and order a pizza online, there's probably a step where I need to figure out what
    size pizza I want, how many pizzas to order. There's a step to pick the toppings I want on
    my pizza, a step to enter my information, like where to deliver it to, and a step to enter my
    payment credentials so I can order it officially. \\

    All of these steps are logically distinct, when I explain them to you with words. Programming
    should be the same way. I should be able to write a program which carries out all those steps,
    without them stepping on each others toes and getting in each other's way. So, we should write
    modular code, which is ideally reusable and easy to understand. \\
  }
\end{frame}

\begin{frame}[fragile]
  \frametitle{What should programming be?}

  \begin{center}
    Good programming should be \textbf{maintainable}.
  \end{center}

  \pause
  \vspace{\fill}

  \begin{center}
    Programs should be written with future maintainability in mind.
  \end{center}

  \pause
  \vspace{\fill}

  \begin{center}
    This goes hand-in-hand with descriptivity and modularity, but code which is written
    expressively, and has future expansion in mind, will be easiest to maintain.
  \end{center}

  \note {
    And finally, good programming should be maintainable. It should be very easy to go and make
    changes to code, when a problem statement is revised slightly, and it should be very easy to
    change one part of a codebase without affecting all the rest of it. This goes hand-in-hand
    with modularity and descriptivity -- I can't go and modify code I don't understand.
  }
\end{frame}

\begin{frame}[fragile]
  \frametitle{What is Not Functional Programming?}

  \note {
    So, now that we've identified some things that programming should ideally be, we can return
    to this discussion of functional programming. What is functional programming? \\

    Functional programming is a conglomeration of different ideas, so there's not a one-size
    fits-all description for it, but I think the primary differences are along a few axes.
    So, I'll push back on that again, and answer the slightly easier question of what functional
    programming is not. Let's start with an example of code written in Python. \\
  }

  Let's look at an example of non-functional code in Python.

  \pause

  \begin{lstlisting}[language=Python]
    count = 0

    def increment():
      global count
      count += 1

      return count
  \end{lstlisting}

  \pause

  What does \code{increment()} return?

  \pause

  \vspace{\fill}

  \note {
    The answer: It depends! It's not clear what this function might return, because it
    depends on when you call it. This is the notion of "state" in a program. \\
  }

  The answer: \textbf{it depends}. On the first call, it returns \code{1}, and on the
  second call, \code{2}, and so on.

  \pause

  \vspace{\fill}

  This demonstrates what we call \term{state}.

\end{frame}

\begin{frame}[fragile]
  \frametitle{Programming, By Analogy}

  Suppose you are a master chef at a 5-star restaurant.

  \pause
  \vspace{\fill}

  An imperative program is like a fully crowded kitchen with no rules.
  \begin{itemize}
    \item Everyone uses the same ingredients and the same cookware. \\

    \item \textbf{Each cook is an individual actor that can mess with the others}, if
    care is not taken. The health of the kitchen depends on each individual chef.
  \end{itemize}

  \pause
  \vspace{\fill}

  A functional program is like a kitchen where each cook has their own working space.

  \begin{itemize}
    \item Everyone has their own pots, pans, and ingredients, and they only share things when
    they finish producing their individual parts.

    \item \textbf{Each cook only interacts when sharing finished results}. This means
    it is impossible for one cook's actions to mess up another's cooking.
  \end{itemize}

  \note {
    The imperative kitchen is faster, at the risk of salmonella.
  }
\end{frame}

\begin{frame}[fragile]
  \frametitle{Programming, Two Ways}

  \begin{center}
  \begin{tabular}{p{0.34\linewidth} @{\hspace{0.75in}} p{0.35\linewidth}}
    Programming, Imperatively & Programming, Functionally \\ \\
    Computation by \textbf{modifying the computer's state} &
    Computation by \textbf{reduction of expressions to values} \\ \pause \\
    \makecell{\Large \code{x := 2;} \\ \\ \vspace{25pt} \Large \code{x + x}} &
    \makecell{\Large \code{2 + 2}}
  \end{tabular}
  \end{center}
\end{frame}

\begin{frame}[fragile]
  \frametitle{Programming, Two Ways}

  \note {
    There are a few ways that we can contrast imperative and functional programming,
    but this notion of "state" is going to be a big one. \\

    In an imperative program, you compute by changing the state of the computer,
    until eventually you can produce the answer you need. This means that any given
    computation depends on what has happened before, and any step might not make sense if
    you reorder them in a certain way. \\
  }

  In stateful programs, we use commands like \code{x := 2} to \textit{change the world},
  to be one where \code{x + x} is 4.

  \vspace{4pt}

  This doesn't stop another part of the program from changing that later!

  \pause
  \vspace{\fill}

  In functional programs, we apply simplifying rules to expressions like \code{2 + 2},
  to obtain the value of \code{4}.

  \vspace{4pt}

  These expressions are \textbf{disjoint}, in that evaluation of one expression
  is unrelated to the evaluation of another.


  \note {
    In a functional program, you compute by taking expressions, and reducing them to
    values. You can then combine the results of these expressions in different ways to
    compute a final answer. The difference is that each individual expression, each step,
    is independent of the other. This means that they can be done in parallel, they can be
    done in an arbitrary order, and they can be analyzed and determined independently
    of each other.
  }

  \pause
  \vspace{\fill}

  \note {
    In an imperative program, you need to not only understand what code does, but you
    might need to know exactly when the code has been run. This makes it really hard
    to understand programs, but functional programming has a big win here.
  }

  In stateful programs, understanding a program entails not only understanding what the code does,
  but knowing the entire history of the program up until that point.

\end{frame}

\begin{frame}[fragile]
  \frametitle{What is functional programming?}

  Functional programming \textbf{avoids modification of state}.

  \pause
  \vspace{\fill}

  \defBox{\, A function is \textit{pure} if it does not have any observable side effects, and always
  returns the same outputs, given the same inputs.}

  \pause
  \vspace{\fill}

  A large amount of problems in computer science are of a pure nature. This means that they
  give the same outputs for the same inputs.
  (For example, finding the shortest path through a graph, computing the \textit{n}th prime
  number, or compressing a file)

  \pause
  \vspace{5pt}

  An important motivation behind functional programming will end up being that we should
  prefer to solve pure problems with pure components. In other words, don't introduce state
  when it's not necessary!
\end{frame}

\begin{frame}[fragile]
  \frametitle{Three Theses}
  \vspace{\fill}

  Functional programming can be characterized by three theses, which will be a recurring
  theme throughout the semester.

  \pause

  \vspace{\fill}

  { \Large
  \begin{enumerate}
    \item Recursive Problems, Recursive Solutions
    \pause
    \vspace{\fill}
    \item Programmatic Thinking is Mathematical Thinking
    \pause
    \vspace{\fill}
    \item Types Guide Structure
  \end{enumerate}
  }

\end{frame}

\begin{frame}[fragile]
  \frametitle{Recursive Problems, Recursive Solutions}

  In this class, \textbf{almost every single function you write will be recursive}.

  \pause
  \vspace{\fill}

  Many problems in computer science lend themselves to a recursive formulation. These
  are naturally solved by recursive solutions.

  \pause
  \vspace{\fill}

  We will see that lists, trees, and other important structures employ a simple
  recursive description.

\end{frame}

\begin{frame}[fragile]
  \frametitle{Programmatic Thinking is Mathematical Thinking}

  Programs are simpler and more understandable when viewed in a mathematical lens.

  \pause
  \vspace{\fill}

  Functional programming allows reasoning about programs and their subcomponents
  in the same way that you would reason about a mathematical expression.

  \pause
  \vspace{\fill}

  Furthermore, mathematical analysis of code grants techniques to reason about
  things like time complexity, parallel time complexity, and program correctness.
  Functional programs are very amenable to proofs of correctness!

  \pause
  \vspace{\fill}

  We're not just in the business of writing code, but \textit{correct} code!

\end{frame}

\begin{frame}[fragile]
  \frametitle{Types Guide Structure}

  Before, we described programming as an explanatory, communicative process.

  \pause
  \vspace{\fill}

  We also stated how \textit{descriptivity} and \textit{maintainability} are
  key goals for good programming.

  \pause
  \vspace{\fill}

  Functional programming places a great emphasis on \textit{types}, which serve
  the purpose of documenting the purpose of code, and restricting the range of
  behaviors that a program is allowed to exhibit.

  \pause
  \vspace{\fill}

  In this way, types guide the structure of a program, by providing clean interfaces
  for how different parts should interact, and what it should be allowed to do.

\end{frame}




\begin{frame}[plain]

  \begin{center}
    \Large 5-minute break!
  \end{center}

\end{frame}

\sectionSlide{3}{Types, Expressions, Values}

\begin{frame}[fragile]
  \frametitle{The Standard ML Language}

  In this class, we will be using a functional programming language called Standard ML (SML).

  \pause
  \vspace{10pt}

  \customBox{Mantra}{\, In Standard ML, \textbf{computation is evaluation}.}

  \pause
  \vspace{\fill}

  Evaluation of what, though? The most fundamental unit of an SML program is called an \term{expression}.

  \pause
  \vspace{\fill}

  \defBox{\, An \term{expression} is the building block of an SML program. These
  may or may not evaluate to a value.}

  \pause
  \vspace{\fill}

  \defBox{\, A \term{value} is a \textbf{final answer}, that cannot be simplified further.}

  \pause
  \vspace{\fill}

  Examples of values include \code{2}, \code{\"hi\"}, and \code{true}.

  \vspace{5pt}

  Examples of expressions include \code{2}, \code{2 + 2}, and \code{4 * 5}
\end{frame}

\begin{frame}[fragile]
  \frametitle{Evaluation}

  To evaluate an expression like \code{(2 + 3) * 4}, we apply simplifying rules. So we get:

  \pause

  { \Large
  \begin{align*}
    \code{(2 + 3) * 4} &\stepsTo \code{5 * 4} \\
    &\stepsTo \code{20}
  \end{align*}
  }

  \pause
  \vspace{\fill}

  \defBox{\, We use the $\stepsTo$ symbol to denote \term{stepping} (or \term{reducing}) of expressions,
  which means to simplify an expression by one step.}

  \pause
  \vspace{\fill}

  \defBox{\, We use the $\stepsTo^*$ symbol to denote the application of the $\stepsTo$ relation
  an arbitrary number of times, usually until completion.}

  \pause
  \vspace{5pt}

  So the expression \code{5 * 4} \textit{steps to} the expression \code{20}, and the expression
  \code{(2 + 3) * 4} $\stepsTo^*$ \code{20}.
\end{frame}

\begin{frame}[fragile]
  \frametitle{Computation as Evaluation}

  We call the previous slide the \term{computation trace} of the expression \code{(2 + 3) * 4}. \\

  \vspace{5pt}

  The goal of a computation trace is to produce a value.

  \pause
  \vspace{\fill}

  If we know that the expression \code{e} eventually reduces down to value \code{v}, we might
  say that \code{e} reduces to \code{v}, or write \code{e} $\hookrightarrow$ \code{v}. We then say
  that \code{e} is \term{valuable}.

  \vspace{5pt}

  So \code{(2 + 3) * 4} $\hookrightarrow$ \code{20}.

\end{frame}

\begin{frame}[fragile]
  \frametitle{Computation without Valuation}

  We said before that the goal of a computation trace is to produce a value, but not
  all expressions do!

  \pause
  \vspace{\fill}

  What value does the expression \code{1 div 0} reduce to?

  \pause
  \vspace{\fill}

  The answer: \textbf{there is no such value!} Division by zero is undefined, and in
  Standard ML, raises an exception.

  \pause
  \vspace{\fill}

  Evaluating an expression has three possible behaviors:
  \begin{itemize}
    \item Reducing to a value
    \item Raising an exception
    \item Looping forever
  \end{itemize}
\end{frame}

\begin{frame}[fragile]
  \frametitle{Computation without Meaning}

  In Standard ML, the string concatenation operator is \code{^}.

  \vspace{5pt}

  So we would say that \code{"hi" ^ "there"} $\stepsTo$ \code{"hithere"}.

  \pause
  \vspace{5pt}

  But what does \code{"1" ^ 50} step to?

  \pause
  \vspace{\fill}

  Some programming languages might try to make sense of this expression.

  \vspace{5pt}

  \textbf{Standard ML will not.}
\end{frame}

\begin{frame}[fragile]
  \frametitle{Types}

  \defBox{\, A \term{type} is a specification of the behavior of a piece of code. It
  \textbf{predicts} what a program is allowed to do.}

  \pause
  \vspace{5pt}

  We write \code{e : t} to say that the expression \code{e} has type \code{t}, so
  we could write \code{ 1 + (2 * 3) : int }.

  \pause
  \vspace{\fill}

  For instance, something with type \code{int} must produce a number, if it reduces to a value.

  \vspace{5pt}

  Similarly with something of type \code{string}.

  \pause
  \vspace{\fill}

  What can we say about the runtime behavior of \code{"1" ^ 50}? It's not clear, so
  the expression \code{"1" ^ 50} \textbf{does not have a type}.
\end{frame}

\begin{frame}[fragile]
  \frametitle{Typing Trace}

  How does Standard ML know the type of an expression?

  \vspace{5pt}

  It follows \term{typing rules} to determine this. For instance:

  \pause
  \vspace{\fill}

  \defBox{\, The typing rule for \code{+} is: \code{e1 + e2 : int} if \code{e1 : int}
  and \code{e2 : int}
  }

  \pause
  \vspace{\fill}

  Take the expression \code{1 + (2 + 3)}.

  \pause
  \vspace{\fill}

  Then, we know \code{1 + (2 + 3) : int} if \code{1 : int} and \code{2 + 3 : int}.
  \pause
  \vspace{3pt}

  We know \code{1 : int}.
  \pause
  \vspace{3pt}

  Then, we know \code{2 + 3 : int} if \code{2 : int} and \code{3 : int}.
  \pause
  \vspace{3pt}

  We know \code{2 : int}.
  \pause
  \vspace{3pt}

  We know \code{3 : int}.
  \pause
  \vspace{3pt}

  So \code{1 + (2 + 3) : int}.
\end{frame}

\begin{frame}[fragile]
  \frametitle{Ill-Typing Trace}

  What about for \code{"1" ^ 50}?

  \pause
  \vspace{\fill}

  \defBox{\, The typing rule for \code{^} is: \code{e1 ^ e2 : string} if \code{e1 : string}
  and \code{e2 : string}
  }

  \pause
  \vspace{\fill}

  Take the expression \code{"1" ^ 50}.

  \pause
  \vspace{\fill}

  Then, we know \code{"1" ^ 50 : string} if \code{"1" : string} and \code{50 : string}.

  \pause
  \vspace{3pt}

  We know \code{"1" : string}.

  \pause
  \vspace{3pt}

  However, it is not true that \code{50 : string}, because \code{50 : int}.

  \pause
  \vspace{\fill}

  So \code{"1" ^ 50} does not have a type, and we say it is an \term{ill-typed expression}.
\end{frame}


\begin{frame}[fragile]
  \frametitle{Static Typing}

  SML is a \term{statically typed} language, meaning that all typing rules are applied
  \textbf{before the program is ever run}.

  \pause
  \vspace{\fill}

  \defBox{ We say that a piece of code or a program which obeys all the typing rules
  \term{type-checks}, or is \term{well-typed}.
  }

  \pause
  \vspace{\fill}

  \keyBox{\, \textbf{Ill-typed programs are not evaluated.} }
\end{frame}

\sectionSlide{4}{Declarations}

\begin{frame}[fragile]
  \frametitle{Function Declarations}

  We've so far discussed types and values, but we haven't introduced the machinery
  we need to actually work in a programming language! We need some way to declare
  variables and functions.

  \pause
  \vspace{\fill}

  In SML, we can declare functions using the \code{fun} syntax:

  \vspace{5pt}

  \begin{codeblock}
    fun double (n : int) : int = n + n
  \end{codeblock}

  \pause
  \vspace{\fill}

  This comprises of a few parts:
  \begin{itemize}
    \item the \code{fun} keyword that signals the start of the declaration
    \item the name of the function (\code{double})
    \item the arguments to the function, annotated with type (\code{n} and \code{int})
    \item the return type of the function (\code{int})
    \item the body of the function \code{n + n}
  \end{itemize}
\end{frame}

\begin{frame}[fragile]
  \frametitle{Function Application}

  To use the function we just defined, we have to \term{apply} it. This is done via
  placing the function expression directly adjacent to the argument that it is meant
  to take in.

  \pause
  \vspace{\fill}

  So for instance, instead of writing \code{double(2)} like we would in some programming
  languages, we would write \code{double 2}. This is known as \term{function application}.

  \pause
  \vspace{\fill}

  So we would have that \code{double 2} $\stepsTo$ \code{4}.
\end{frame}

\begin{frame}[fragile]
  \frametitle{Typing for Functions}

  Since \code{double} is a function which takes in an \code{int} and returns an \code{int},
  we would refer to it as having the \term{function type} \code{int -> int}.

  \pause
  \vspace{\fill}

  How do we know that \code{double : int -> int}, however? Is it just because we annotated
  it as taking in \code{int} and returning \code{int}?

  \pause
  \vspace{\fill}

  That isn't true, however, because the following declaration fails to type-check, and thus
  will not be executed by Standard ML:

  \vspace{5pt}

  \begin{codeblock}
    fun double (n : int) : string = n + n
  \end{codeblock}

  \pause
  \vspace{\fill}

  In reality, SML is checking the type of \code{double} to make sure that it is consistent
  with the annotations that we state.
\end{frame}

\begin{frame}[fragile]
  \frametitle{Typing for Functions}

  \begin{codeblock}
    fun double (n : int) : int = n + n
  \end{codeblock}

  \pause

  To check the type of \code{double}, SML will take our word for what the input type
  of \code{n} is, here.

  \pause
  \vspace{\fill}

  It will then try to produce a type for the body of the function, \textit{given that}
  \code{n : int}. If that matches up with the return type as stated, then the function is
  well-typed.

  \pause
  \vspace{\fill}

  In this case, we see that if \code{n : int}, then by our previous logic \code{n + n : int},
  which matches the return type. When we change it to \code{string} however, that becomes
  different, so the declaration is rejected as ill-typed.
\end{frame}

\begin{frame}[fragile]
  \frametitle{Typing for Function Application}

  How do we know that \code{double 2} is well-typed?

  \pause
  \vspace{\fill}

  \defBox{\, The typing rule for function application is that \code{e1 e2 : t2} iff
   \code{e1 : t1 -> t2} and \code{e1 : t1}, for some types \code{t1}, \code{t2}.}

   \pause
  \vspace{\fill}

  In essence, this is saying applying a function only returns a type \code{t2} if
  the function has type \code{t1 -> t2}, and it is given an input of type \code{t1}.

  \pause
  \vspace{\fill}

  In this case we know \code{double 2 : int} if \code{double : int -> int} and \code{2 : int}.

  \pause
  \vspace{3pt}

  We know \code{double : int -> int}, because of our previous reasoning.

  \pause
  \vspace{3pt}

  We also know that \code{2 : int}.

  \pause
  \vspace{\fill}

  So \code{double 2 : int}, the return type of \code{double}.
\end{frame}

\begin{frame}[fragile]
  \frametitle{Variable Declarations}

  Now we need to cover how to declare variables. The key word of interest here is
  \code{val}.

  \vspace{\fill}

  \begin{codeblock}
    val favoriteCourseNumber : int = 150
  \end{codeblock}

  \pause
  \vspace{\fill}

  It functions similarly to a function declaration, but there are now no arguments
  to type-annotate.

  \pause
  \vspace{\fill}

  We will have more to say about variable declarations in the next lecture.
\end{frame}

\thankyou

\end{document}
