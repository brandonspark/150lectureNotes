% Jacob Neumann

% DOCUMENT CLASS AND PACKAGE USE
    \documentclass[aspectratio=169, handout]{beamer}

    % Establish the colorlambda boolean, to control whether the lambda is solid color (true), or the same as the picture (false)
    \newif\ifcolorlambda
    \colorlambdafalse % DEFAULT: false

    % Use auxcolor for syntax highlighting
    \newif\ifuseaux
    \useauxfalse % DEFAULT: false

    % Color settings
    \useauxtrue

    \newcommand{\auxColor}{2eab9e}     % the color of note boxes and stuff
    \newcommand{\presentColor}{511CE8} % the primary color of the slide borders
    \newcommand{\bgColor}{ebe3ff}      % the color of the background of the slide
    \newcommand{\darkBg}{8b98ad}
    \newcommand{\lambdaColor}{\auxColor}

    \colorlambdatrue

    \usepackage{comment} % comment blocks
    \usepackage{soul} % strikethrough
    \usepackage{listings} % code
    \usepackage{makecell}
    \usepackage{tcolorbox}
    \usepackage{amssymb}% http://ctan.org/pkg/amssymb
    \usepackage{pifont}% http://ctan.org/pkg/pifont
    \usepackage[outline]{contour}
    \usepackage{ stmaryrd }

    \setbeamertemplate{itemize items}[circle]
    % \setbeameroption{show notes on second screen=right}

    \usepackage{lectureSlides}
    %%%%%%%%%%%%%%%%%%%%%%%%%%%%%%%%%%%%%%%%%| <----- Don't make the title any longer than this
    \title{Program Analysis} % TODO
    \subtitle{Writing programs to analyze programs} % TODO
    \date{03 August 2023} % TODO
    \author{Brandon Wu} % TODO


    \def\checkmark{\tikz\fill[green, scale=0.4](0,.35) -- (.25,0) -- (1,.7) -- (.25,.15) -- cycle;}
    \contourlength{.08em}% default is 0.03em
    \newcommand{\cmark}{{\color{green!80!black}\ding{51}}}
    \newcommand{\xmark}{{\color{red}\ding{55}}}

    \graphicspath{ {./img/} }

    \colorlet{bgBlue}{hlBlue!130}
    \colorlet{bgOrange}{hlOrange}
    \colorlet{fgBlue}{fgCodeBlue}
    \colorlet{fgGreen}{fgCodeGreen}
    \colorlet{fgRed}{fgCodeRed}
    \colorlet{fgOrange}{fgCodeOrange}

    \colorlet{hlset}{fgBlue}

    \definecolor{bgPurple}{HTML}{dbc7ff}
    \definecolor{bgRed}{HTML}{fcaea9}
    \definecolor{bgYellow}{HTML}{fffeb5}
    \definecolor{bgGreen}{HTML}{b4ffb3}

    \definecolor{fgOrange}{HTML}{fcae4e}
    \definecolor{fgYellow}{HTML}{ffbe0a}

    \definecolor{unknownPink}{HTML}{e0fff6}

    \usetikzlibrary{positioning}

    \usetikzlibrary{positioning}
    \usetikzlibrary{decorations.pathreplacing,calligraphy}

    \usetikzlibrary{patterns}
    \usetikzlibrary {calc}
    \tikzset{
      every path/.style={line width=0.25mm},
      tok/.style={
        font=\large,
        align=center,
        minimum height=3cm,
      },
      belowtok/.style={
        font=\large,
        align=center,
        minimum height=3cm,
      },
      astNode/.style={
        draw,
        inner sep=2pt,
        line width=0.4mm,
        minimum size=0.6cm,
        font=\large,
        align=center,
      },
      hex/.style={
        astNode,
        signal,
        signal to=east and west,
        signal pointer angle=130,
        fill=bgPurple,
      },
      box/.style={
        astNode,
        rectangle,
        fill=green!20!white,
      },
      bbox/.style={
        astNode,
        rectangle,
        fill=bgBlue,
      },
      circ/.style={
        astNode,
        circle,
        minimum size=0.7cm,
        fill=bgYellow
      },
      bhex/.style={
        astNode,
        signal,
        inner sep=0.5pt,
        signal to=east and west,
        signal pointer angle=130,
        fill=bgBlue,
        minimum width=0.8cm,
      },
      unknown/.style={hex, draw=black, preaction={fill, unknownPink}, pattern=north east lines},
      highlight/.style={draw=fgYellow, line width=0.55mm},
      highlight2/.style={draw=magenta, line width=0.55mm},
      highlight3/.style={draw=blue, line width=0.55mm},
      hlbg/.style={fill=bgYellow},
      hledge/.style={thick, red},
      between/.style args={#1 and #2}{
        at = ($(#1)!0.5!(#2)$)
      },
    }
    \newcommand{\dollar}{\mbox{\textdollar}}


    % DONT FORGET TO PUT [fragile] on frames with codeblocks, specs, etc.
        %\begin{frame}[fragile]
        %\begin{codeblock}
        %fun fact 0 = 1
        %  | fact n = n * fact(n-1)
        %\end{codeblock}
        %\end{frame}

    % INCLUDING codefile:
        % 1. In some file under code/NN (where NN is the lecture id num), include:
    %       (* FRAGMENT KK *)
    %           <CONTENT>
    %       (* END KK *)

    %    Remember to not put anything on the same line as the FRAGMENT or END comment, as that won't be included. KK here is some (not-zero-padded) integer. Note that you MUST have fragments 0,1,...,KK-1 defined in this manner in order for fragment KK to be properly extracted.
        %  2. On the slide where you want code fragment K
                % \smlFrag[color]{KK}
        %     where 'color' is some color string (defaults to 'white'. Don't use presentColor.
    %  3. If you want to offset the line numbers (e.g. have them start at line 5 instead of 1), use
                % \smlFragOffset[color]{KK}{5}

\begin{document}

% Make it so ./mkWeb works correctly
\ifweb
    \renewcommand{\pause}{}
\fi

\setbeamertemplate{itemize items}[circle]

% SOLID COLOR TITLE (see SETTINGS.sty)
{
\begin{frame}[plain]
    \colorlambdatrue
    \titlepage
\end{frame}
}

\begin{frame}[fragile]
  \frametitle{Lesson Plan}

  \tableofcontents
\end{frame}

\begin{frame}[fragile]
  \frametitle{Last time}
\end{frame}

\sectionSlide{1}{The State of Software}

\begin{frame}[fragile]
  \frametitle{Software and the World}

  On August 20th, 2011, Silicon Valley venture capitalist and and
  entrepreneur Marc Andreessen\footnote{Currently a board director for Meta Platforms.} published an essay entitled
  "Software is eating the world".

  \pause
  \vspace{\fill}

  This essay included a lot of business-oriented reasons for why software
  was immensely disrupting each individual economic sector, for reasons of
  ease of use, speed of execution, and reach of influence, among others.

  \pause
  \vspace{\fill}

  Now, more than a decade after this article, it's an incredibly obvious fact that
  software already has eaten the world. You cannot get away from it -- it is
  everywhere, and it is everything.
\end{frame}

\begin{frame}[fragile]
  \frametitle{The State of Software Engineering}

  Part of what makes software engineering a lucrative profession is that there
  is not, and will never be, a shortage for software engineers.

  \pause
  \vspace{\fill}

  Regardless of if a company is a recruiting company, a think tank, a massage parlor,
  or a pet food retailer, everyone needs software developers. Every business needs
  a website, every business deals with data, and every business needs a way to
  keep up with every other business, which is doing exactly the same.

  \pause
  \vspace{\fill}

  Unfortunately, not all of them are educated at Carnegie Mellon, and
  have taken 15-150, so not all of them are very well-informed on the
  importance of writing safe code.
\end{frame}

\begin{frame}[fragile]
  \frametitle{The State of Software Engineering}

  One theme that has cropped up throughout this course is to try
  to produce as little code as possible, because any human writing any
  amount of code has some probability of producing a bug.

  \pause
  \vspace{\fill}

  The less code we write, the less possibility of writing a bug.

  \pause
  \vspace{\fill}

  So what can we say about the immense volume of code produced by
  the tens of millions of software developers around the world?

  \pause
  \vspace{\fill}

  \textbf{Answer:} It is horribly, immensely buggy, and full of mistakes.
\end{frame}

\begin{frame}[fragile]
  \frametitle{What's in an Error?}

  When you write a mistake in your code, what does it often look like?

  \pause
  \vspace{\fill}

  Maybe you made a typo:

  \begin{center}
    \begin{minipage}[t][0.7in][t]{\textwidth}
      \begin{minipage}{0.2\textwidth}
        \centering
        \vspace{\fill}
        {\huge\xmark}
        \vspace{\fill}
      \end{minipage}
      \begin{minipage}{0.75\textwidth}
        {\small
        \begin{codeblock}[rulecolor=\color{red}, framerule=0.3mm]
          fun fact 0 = 1
            | fact n = n * fac (n - 1)
        \end{codeblock}
        }
      \end{minipage}
    \end{minipage}
    \pause
    \begin{minipage}[t][0.7in][t]{\textwidth}
      \begin{minipage}{0.2\textwidth}
        \centering
        \vspace{\fill}
        {\huge\cmark}
        \vspace{\fill}
      \end{minipage}
      \begin{minipage}{0.75\textwidth}
        {\small
        \begin{codeblock}[rulecolor=\color{green!80!black}, framerule=0.3mm]
          fun fact 0 = 1
            | fact n = n * `fact` (n - 1)
        \end{codeblock}
        }
      \end{minipage}
    \end{minipage}
  \end{center}
\end{frame}

\begin{frame}[fragile]
  \frametitle{What's in an Error?}

  Or maybe you declared a variable, and then forgot to use it:

  \pause
  \begin{center}
    \begin{minipage}[t][1.4in][t]{\textwidth}
      \begin{minipage}{0.2\textwidth}
        \centering
        \vspace{\fill}
        {\huge\xmark}
        \vspace{\fill}
      \end{minipage}
      \begin{minipage}{0.75\textwidth}
        {\small
          \begin{codeblock}[rulecolor=\color{red}, framerule=0.3mm]
            fun treefoldl f acc Empty = acc
              | treefoldl f acc (Node (L, x, R)) =
                  let
                    val left_folded = treefoldl f acc L
                  in
                    treefoldl f (f (x, acc)) R
                  end
          \end{codeblock}
          }
        \end{minipage}
    \end{minipage}
    \pause
    \begin{minipage}[t][1.4in][t]{\textwidth}
      \begin{minipage}{0.2\textwidth}
        \centering
        \vspace{\fill}
        {\huge\cmark}
        \vspace{\fill}
      \end{minipage}
      \begin{minipage}{0.75\textwidth}
        {\small
        \begin{codeblock}[rulecolor=\color{green!80!black}, framerule=0.3mm]
          fun treefoldl f acc Empty = acc
            | treefoldl f acc (Node (L, x, R)) =
                let
                  val left_folded = treefoldl f acc L
                in
                  treefoldl f (f (x, `left_folded`)) R
                end
        \end{codeblock}
        }
      \end{minipage}
    \end{minipage}
  \end{center}
\end{frame}

\begin{frame}[fragile]
  \frametitle{What's in an Error?}

  Or maybe you just have a simple type error:

  \pause
  \begin{center}
    \begin{minipage}[t][0.8in][t]{\textwidth}
      \begin{minipage}{0.2\textwidth}
        \centering
        \vspace{\fill}
        {\huge\xmark}
        \vspace{\fill}
      \end{minipage}
      \begin{minipage}{0.75\textwidth}
        {\small
          \begin{codeblock}[rulecolor=\color{red}, framerule=0.3mm]
            fun foldr f acc [] = acc
              | foldr f acc (x::xs) =
                  f (x, foldr acc xs)
          \end{codeblock}
          }
        \end{minipage}
    \end{minipage}
    \pause
    \begin{minipage}[t][0.8in][t]{\textwidth}
      \begin{minipage}{0.2\textwidth}
        \centering
        \vspace{\fill}
        {\huge\cmark}
        \vspace{\fill}
      \end{minipage}
      \begin{minipage}{0.75\textwidth}
        {\small
        \begin{codeblock}[rulecolor=\color{green!80!black}, framerule=0.3mm]
          fun foldr f acc [] = acc
            | foldr f acc (x::xs) =
                f (x, foldr `f` acc xs)
        \end{codeblock}
        }
      \end{minipage}
    \end{minipage}
  \end{center}
\end{frame}

\begin{frame}[fragile]
  \frametitle{Static and Dynamic Warnings}

  These kinds of simple mistakes crop up all the time!

  \pause
  \vspace{\fill}

  Thankfully, we are working in a language which is disciplined enough
  to warn you about most of these things, albeit not all.\footnote{Though it could.
  SML/NJ in particular just doesn't.}

  \pause
  \vspace{\fill}

  What about all the software being produced elsewhere, though? In some languages,
  such as Python, \textbf{none of these errors} are able to be caught, until they
  happen at runtime!

  \pause
  \vspace{\fill}

  We say that making the programmer aware of these errors at \term{compile time},
  before the program runs, is a \term{static} warning, versus a \term{dynamic}
  warning, which would only occur once the program runs into it while executing.
\end{frame}

\begin{frame}[fragile]
  \frametitle{A Parable of Machine Learning}

  I love telling this story whenever anyone asks about why it is important
  to catch errors statically.

  \pause
  \vspace{\fill}

  Imagine that you are a machine learning engineer.\footnote{The horror.}

  \pause
  \vspace{\fill}

  You have spent the past six months working on a state of the art large
  language model, and finally you are ready to put it to the test. You just
  make a few adjustments (mostly adding comments and clarifying names),
  before you run the model and then decide to go on a vacation to France for
  two weeks.

  \pause
  \vspace{\fill}

  When you return from your vacation, you discover that your model failed
  with:
  {\small
  \begin{lstlisting}[style=15150code, numbers=none]
    NameError: name 'modle' is not defined. Did you mean: 'model'?
  \end{lstlisting}
  }

  \pause
  \vspace{\fill}

  This can actually happen.
\end{frame}


\begin{frame}[fragile]
  \frametitle{The Cost of Mistakes}

  What's the point? \textbf{Programmers make mistakes}.

  \pause
  \vspace{\fill}

  There are many programmers in the world. Programs that make these
  kinds of silly, one-off mistakes happen millions of times in a single
  day. We need to build tools, compilers, and programming languages that
  can make sure that these errors do not make it to real applications,
  because the cost of doing so is too high.

  \pause
  \vspace{\fill}

  Remember Tony Hoare's billion-dollar mistake. We are talking about
  fighting a war upon which rests not only billions upon billions of
  dollars across every conceivable industry, but upon which rests the
  security and continued operation of our society.
\end{frame}

\begin{frame}[fragile]
  \frametitle{Battling Software}

  How can we fix these kinds of mistakes? How do we make sure that,
  for the prodigious, gargantuan, and overflowing deluge of software
  that is pumped out every day, it is as safe and as correct as possible?
  The alternative is a reality that is horrifying to contemplate.
  \footnote{I highly recommend the book \textit{This Is How They Tell Me
  the World Ends: The Cyberweapons Arms Race} by Nicole Perlroth.}

  \pause
  \vspace{\fill}

  Software is eating the world.

  \pause
  \vspace{\fill}

  It's time to bite back.
\end{frame}

\sectionSlide{2}{Program Analysis}

\begin{frame}[fragile]
  \frametitle{What is Program Analysis?}

  \defBox{}{\term{Program analysis} is the art of discovering undesirable
  behaviors in programs, usually by automated, programmatic means.}

  \pause
  \vspace{\fill}

  These undesirable behaviors may include correctness, performance,
  security, and legibility. Ultimately, it encompasses any property
  which is worth testing, of a program.

  \pause
  \vspace{\fill}

  In essence, program analysis entails \textbf{writing programs to
  analyze programs}.
\end{frame}


\begin{frame}[fragile]
  \frametitle{Flavors of Program Analysis: Dynamic}

  Program analysis generally comes in one of two flavors:

  \pause
  \vspace{\fill}

  \defBox{}{\term{Dynamic program analysis} has to do with figuring
  out program behavior by observing its behavior at run-time.}

  \pause
  \vspace{\fill}

  This can include things like \term{fuzzing}, which involves running
  the program on a wide range of random inputs, \term{profiling}, which
  involves measuring the run-time of a program on some inputs, and even
  the simple act of writing tests.

  \pause
  \vspace{\fill}

  Dynamic program analysis is useful, and goes straight to the source
  in terms of the program's actual behavior, but it is limited in some
  other ways. Notably, if the target program loops forever, or
  crashes, then dynamic program analysis will do the same.

  \pause
  \vspace{\fill}

  More concerning is that dynamic program analysis will generally have
  computational cost equal to that of the program being analyzed. This
  means that a program which takes a very long time to run will take
  a very long time to test. We don't always have time for that.

\end{frame}

\begin{frame}[fragile]
  \frametitle{Flavors of Program Analysis: Static}

  \defBox{}{\term{Static program analysis} concerns ascertaining properties
  of programs \textbf{without ever running the program}.}

  \pause
  \vspace{\fill}

  Because as we saw in the last lecture, programs are trees, this means
  that static program analysis really just takes the form of recursive
  functions on trees.

  \pause
  \vspace{\fill}

  This will be our focus for today. Specific applications of this analysis include:
  \pause
  \begin{itemize}
    \item \term{static application security testing} (or SAST), which is
    the process of applying static program analysis to code for security purposes \pause
    \item syntax highlighting, which looks at a (possibly incomplete) program
    and tries to color it, as its being written \pause
    \item autoformatting, which looks at a program and tries to make it adhere to
    a certain stylistic convention \pause
    \item type-checking, which looks at a program and ascertains what type its
    constituent parts have (if any)
  \end{itemize}
\end{frame}

\begin{frame}[fragile]
  \frametitle{A Minor Issue}

  This is the mission we have ahead of us. Before we can dive into more
  technical details, however, we have one small issue before us:

  \pause
  \vspace{\fill}

  \textbf{Program analysis is inherently impossible}.

  \pause
  \vspace{\fill}

  {\color{blue}\href{https://en.wikipedia.org/wiki/Rice%27s_theorem}{Rice's Theorem}} is a mathematical theorem in computability theory which states:
  \begin{quote}
    All non-trivial semantic properties of programs are undecidable.
  \end{quote}

  \pause
  \vspace{\fill}

  In English:
  \begin{quote}
   It is impossible to definitively answer yes or no for any property of a
   program's behavior, in a finite amount of time.
  \end{quote}
\end{frame}

\begin{frame}[fragile]
  \frametitle{The Halting Problem}

  This is a straightforward corollary of the
  {\color{blue}\href{https://en.wikipedia.org/wiki/Halting_problem}{Halting Problem}},
  which essentially just states that it is impossible to write a program to tell
  if a program loops forever or not.\footnote{Note that these claims of impossibility
  are \textit{in general}. For instance, I can look at the program \code{fun loop x = loop x}
  with my eyes and tell you that it loops forever, but we cannot write a program
  which does that for \textit{every} program.} The reason why this is impossible
  come out of asking a simple question: what should the following function return?

  \pause
  \begin{codeblock}
    fun not_halts () =
      if halts(not_halts) then loop () else ()
  \end{codeblock}

  \pause
  \vspace{\fill}

  This ends up producing a paradox, in much the same way as the
  {\color{blue}\href{https://en.wikipedia.org/wiki/Liar_paradox}{liar's paradox}}.

  \pause
  \vspace{\fill}

  Because any program can loop forever, this ends up tainting every other possible
  question that program analysis could answer, meaning that all of them are inherently
  impossible.

  \pause
  \vspace{\fill}

  Well, that sucks.
\end{frame}

\begin{frame}[fragile]
  \frametitle{Battling Impossibility}

  So then, with this knowledge, is this lecture over?

  \pause
  \vspace{\fill}

  No, because it only sucks if you're a quitter.

  \pause
  \vspace{\fill}

  Recall our idea of \term{strengthening the implementation} or
  \term{weakening the specification}, for solving some problem, meaning that
  we can either put some elbow grease in and make our program more powerful, or
  lower our expectations.

  \pause
  \vspace{\fill}

  Well, this is a mathematical truth, so it's not a skill issue in terms of our
  ability to implement. That's not the problem here.

  \pause
  \vspace{\fill}

  So let's lower our expectations.

  \pause
  \begin{quote}
    it is impossible to definitively answer yes or no for any property of a
    program's behavior, in a finite amount of time.
  \end{quote}

  \pause
  \vspace{\fill}

  The main innovation out of program analysis is -- \textit{it's only impossible if
  you insist on being right all the time}.
\end{frame}

\begin{frame}[fragile]
  \frametitle{Compilers and Program Analysis}

  Compilers and program analysis are dual in a certain sort of sense, in that
  a compiler is itself a kind of program analysis. It follows much the same
  process, by analyzing the source text and producing answers, albeit while
  also producing an executable file.

  \pause
  \vspace{\fill}

  The main thing to realize is that \textbf{a compiler can never be wrong}.
  As stated in the previous lecture, the day that our compilers are untrustworthy
  is the day that programming becomes impossible.

  \pause
  \vspace{\fill}

  Program analysis suffers from no such thing. Our goal will be to implement
  analyses which always complete within a finite amount of time, albeit with
  the caveat that sometimes they might be inaccurate or incomplete, or throw their
  hands up and say "I don't know".
\end{frame}

\begin{frame}[fragile]
  \frametitle{Full Employment}

  A funny corollary of this sentiment is something called the
  {\color{blue}\href{https://en.wikipedia.org/wiki/Full-employment_theorem}{full-employment theorem}},
  which essentially states that there will always be jobs in program analysis and
  compiler-writing, i.e. employment is always ensured.

  \pause
  \vspace{\fill}

  This is because, due to the fact that the task is inherently impossible, it's
  always possible to write a better analysis that covers more cases, or a
  compiler which can produce better binaries. You just need more casework.
\end{frame}

\begin{frame}[fragile]
  \frametitle{Program Analysis via Casework}

  For instance, no one is stopping you from doing this:
  \pause
  \begin{codeblock}
    fun halts (program : string) : bool option =
      case program of
        "val x = 1" => SOME true
      | "val x = 2" => SOME true
      | "val x = 3" => SOME true
      | "fun loop x = loop x val _ = loop ()" => SOME false
      (* add more cases here! *)
      | _ => NONE
  \end{codeblock}

  \pause
  \vspace{\fill}

  Far from the cutting edge, but it works. A team of monkeys at typewriters
  could eventually churn out a more effective \code{halts} function than
  exists anywhere else.

  \pause
  \vspace{\fill}

  This might seem demoralizing, but this is somehow actually a motivational
  statement that I will always have a job.
\end{frame}

\begin{frame}[fragile]
  \frametitle{Resolution}

  So, this is the scope of the task in front of us.

  \pause
  \vspace{\fill}

  We have finite resources and finite time to solve a problem which is
  impossible to solve.

  \pause
  \vspace{\fill}

  And still, software is eating the world. The cost of failing is too high.

  \pause
  \vspace{\fill}

  Time to put in some elbow grease and get to work.
\end{frame}

\sectionSlide{3}{Doing the Impossible}

\begin{frame}[fragile]
  \frametitle{Type Checking}

  Consider a more specific example of program analysis, namely that of
  type-checking a program.

  \pause
  \vspace{\fill}

  Now, type-checking always terminates, and it always returns a correct
  answer. If the type-checker says an expression has some type, then we
  can trust that answer, and if it says our program is ill-typed, then
  indeed we messed up somewhere.

  \pause
  \vspace{\fill}

  However, consider the problem of typing the following function:
  \begin{codeblock}
    val div : int * int -> int
  \end{codeblock}
\end{frame}

\begin{frame}[fragile]
  \frametitle{A More Accurate \code{div}}

  For the \code{div} function, we assign it a type such that it can
  take in any two integers. This seems reasonable, with the caveat that
  if we pass it \code{0} in the second parameter, it will crash on us!

  \pause
  \vspace{\fill}

  Well, this is a dynamic error. We want to catch such things before running
  the program. Can we assign \code{div} a type, such that it will prevent us
  from ever running code which passes \code{0} to \code{div}?

  \pause
  \vspace{\fill}

  There are two properties that we would desire of such a type:
  \begin{itemize}
    \item we can ascertain it in a finite amount of time \pause
    \item it says that a usage of \code{div} is ill-typed if and only if
    we pass \code{0}, or a non-\code{int}, to it
  \end{itemize}

  \pause
  \vspace{\fill}

  It is impossible to have both of these things at the same time.
\end{frame}

\begin{frame}[fragile]
  \frametitle{Sweaty Palms}

  \begin{center}
    \includegraphics[scale=0.3]{sweaty}
  \end{center}
\end{frame}

\begin{frame}[fragile]
  \frametitle{Dependent Types and Type A Analysis}

  There is a class of languages which have a more sophisticated type
  system than Standard ML, called \term{dependently typed} languages,
  where you can actually express this.

  \pause
  \vspace{\fill}

  The issue is that type-checking in such languages can loop forever,
  meaning we would get the second property, but not the first. We will
  call this \term{Type A program analysis}.

  \pause
  \vspace{\fill}

  So the other track will be to get the first property, but not the second.
  We will have to accept being wrong sometimes, and either rejecting
  programs which do not divide by zero, or accepting programs which do.
  We will call this \term{Type B program analysis}.
\end{frame}

\begin{frame}[fragile]
  \frametitle{An Analogy for Type B Analysis}

  Consider the following analogy.

  \pause
  \vspace{\fill}

  You are in charge of security at an airport.

  \pause
  \vspace{\fill}

  You are acutely aware of the fact that in the early 2000s, a man tried to set off
  plastic explosives concealed in his shoes, during a transatlantic flight
  from France to Florida.

  \pause
  \vspace{\fill}

  The issue is that you don't have a good way of telling whether an arbitrary
  individual might have explosives in their shoes, or not.

  \pause
  \vspace{\fill}

  So, how do you minimize the chances?

  \pause
  \vspace{\fill}

  This is the story of why everyone needs to take off their shoes in the airport.
\end{frame}

\begin{frame}[fragile]
  \frametitle{A Rough Approximation}

  The parable of this story is that, while it may be difficult or impossible to
  gather an \textit{exact} answer (who has explosives in their shoes), it's easy
  to obtain an \textit{approximative} answer: assume that \textit{everyone} has
  explosives in their shoes.

  \pause
  \vspace{\fill}

  So just scan everybody's shoes. Problem solved.

  \pause
  \vspace{\fill}

  So how can we tell which programs contain an unsafe call to \code{"div"}? Well,
  if we don't mind being wrong sometimes...

  \pause
  \vspace{\fill}

  \begin{codeblock}
    fun containsUnsafeDiv (prog : string) =
      stringContains (prog, "div")
  \end{codeblock}
\end{frame}

\begin{frame}[fragile]
  \frametitle{Tradeoffs}

  Which outcome is more preferred? Well, it turns out the answer is
  "neither of them".

  \pause
  \vspace{\fill}

  We are basically saying that, to be able to reject all programs which
  might divide by zero, we can either accept infinitely looping
  compile times, or we can reject every program which contains a \code{div}.

  \pause
  \vspace{\fill}

  Now, with more sophisticated techniques, we can do a little bit better
  than rejecting every program containing a \code{div}. But not by much.

  \pause
  \vspace{\fill}

  It turns out, in practice, the right solution will be to simply not
  care so much about the division by zero case. It's not worth the
  trade-offs.
\end{frame}

\begin{frame}[fragile]
  \frametitle{Downscoping and Type C Analysis}

  We said it was impossible to have the virtues of Type A and Type B analysis at the
  same time. That's true, if we fix the problem statement. We might say that
  \term{Type C program analysis} is to both terminate and be correct, but at the cost of
  simplifying the problem we are trying to solve.

  \pause
  \vspace{\fill}

  Type-checking is usually an example of a Type C analysis. So, thus we end up
  with not being able to statically catch division by zero errors. For the
  question of "does this program divide by zero?", we decided the answer is "we
  don't care".

  \pause
  \vspace{\fill}

  What about the question of "does this program divide by a non-integer"?

  \pause
  \vspace{\fill}

  It turns out, this is perfectly solvable in a terminating manner. The
  reason why this is OK is that "non-integer" is an approximative query --
  "zero" is specific.
\end{frame}

\begin{frame}[fragile]
  \frametitle{Recap}

  So let's recap for a second:
  \pause
  \begin{itemize}
    \item we would like to answer specific questions about programs, which are
    impossible to do in general. \pause
    \item We need to give up one of guaranteed termination, perfect accuracy,
    or solving that exact problem. Types A, B, and C analysis correspond to
    giving up each of these things, respectively.
  \end{itemize}

  \pause
  \vspace{\fill}

  Specific examples include:
  \begin{itemize}
    \item dependent typechecking is a Type A analysis (can loop forever) \pause
    \item rejecting all programs with \code{div} is a Type B analysis (rejects valid programs) \pause
    \item regular typechecking is a Type C analysis (give up catching divide by zero)
  \end{itemize}

  \pause
  \vspace{\fill}

  For most practical program analysis tools, looping forever is not an option.\footnote{We
  might call this "doubly impossible".} So for our purposes, we are generally
  interested in Type B and Type C analysis.
\end{frame}

% \begin{frame}[fragile]
%   \frametitle{Flavors of Program Analysis}

%   An easy way to remember program analysis is to envision it as three different kinds of people.

%   Type A analysis is personified by an old person (they might take forever to answer)

%   Type B analysis is personified by a flat earther (they sometimes spout nonsense)

%   Type C analysis is personified by a literal child (they can only answer simple questions)
% \end{frame}

\begin{frame}[fragile]
  \frametitle{On Incorrectness}

  But, we still have significant problems left to answer.

  \pause
  \vspace{\fill}

  Type-checking is only in the Type C category because of decades of work by type theorists
  and language designers, to figure out what buckets of questions can be answered tractably
  by machines, and to what extent.\footnote{Though somehow, many modern languages fall decades
  behind still in that respect}

  \pause
  \vspace{\fill}

  For many other problems, such as code reachability, vulnerability to outside
  attackers, and unsafe behavior, we have no such guarantees. We fall squarely into
  Type B, Type A is not an option.

  \pause
  \vspace{\fill}

  So we need to accept being wrong sometimes. Let's see how.
\end{frame}

\sectionSlide{4}{Dataflow Analysis}

\begin{frame}[fragile]
  \frametitle{Approximation}

  A classic technique used in program analysis to obtain \textit{approximative}
  \footnote{Life hack: you can successfully replace "incorrect" with
  "approximative" in so many different places that it's hilarious.} answers
  in a finite amount of time is called \term{dataflow analysis}.

  \pause
  \vspace{\fill}

  Before I can define it to you, I must give you an analogy.

\end{frame}

\begin{frame}[fragile]
  \frametitle{An Analogy for Dataflow Analysis}

  Suppose you have a query you would like to solve on programs, which has many
  possible answers. Further suppose that the number of possible answers is finite.

  \pause
  \vspace{\fill}

  Suppose that you line them all up, one next to the other.

  \pause
  \vspace{\fill}

  $$ans_1, ans_2 ... ans_n$$

  \pause
  \vspace{\fill}

  Program analysis is hard because information can change a lot, infinitely much
  in fact, over the course of a program's run-time. You might pick an answer $n$,
  then move to answer $n - 2$, then move to a different answer $k$ altogether. It's
  possible to jump all around, in the limit of the program's execution.

  \pause
  \vspace{\fill}

  An observation can be made that, if you can order your answer in a way such that,
  over the course of your analysis, you only ever change your answer by moving
  right, you will always eventually terminate.
\end{frame}

\begin{frame}[fragile]
  \frametitle{Monotonicity}

  This is a roundabout way of describing what is known as a \term{monotonic function},
  which is a function which always "increases", according to some proper notion of
  "increases". In this case, our monotonic function also has an upper limit, i.e.
  a point beyond which it can no longer grow.

  \pause
  \vspace{\fill}

  For dataflow analysis, we will make use of this kind of analysis to iterate over
  our control-flow graph, constantly updating our answer, but only in a way that
  "increases", and eventually caps out. If we can do that, then we will
  guarantee that we will terminate.

  \pause
  \vspace{\fill}

  This also usually makes our answers sometimes wrong, though.
\end{frame}

\begin{frame}[fragile]
  \frametitle{A Dataflow Example}

  \begin{center}
    \begin{minipage}{0.65\textwidth}
      \raggedright
      For instance, consider the following control-flow graph:

      \vspace{10pt}

      We would like to perform an analysis known as \term{constant propagation}
      on it, by noting which variables are set to be constant.

      \vspace{10pt}

      How do we do this? We simply march forward through the CFG, and noting
      down which variables are constant as we see them, starting with the
      empty set.
    \end{minipage}
    \begin{minipage}{0.3\textwidth}
      \centering
      \begin{tikzpicture}[node distance=0.8in]
        \node[box, text width=1in] (N1) {
          \code{x <- 1}\strut\\
          \code{y <- 2}\strut
        };

        \node[box, below of=N1] (N2) {
          \code{x <- x + 2}
        };

        \node[box, below of=N2] (N3) {
          \code{ret x}
        };

        \draw[-stealth] (N1) -- (N2);
        \draw[-stealth] (N2.south west) to [out = 210, in=150, looseness=2] (N2.north west);
        \draw[-stealth] (N2) -- (N3);
      \end{tikzpicture}
    \end{minipage}
  \end{center}

\end{frame}

\begin{frame}[fragile]
  \frametitle{A Dataflow Example}

  \begin{center}
    \begin{minipage}{0.55\textwidth}
      \raggedright
      So for instance, first we traverse the entering block,
      by simply penciling in \code{x} and \code{y} as we see
      them get assigned to constants.

      \vspace{10pt}

      Once we finish, we now have the \term{out-set} for the
      first block, which we can then use to determine the other
      blocks.
    \end{minipage}
    \hfill
    \begin{minipage}{0.4\textwidth}
      \centering
      \begin{tikzpicture}[node distance=0.8in]
        \node[box, text width=1in, hlbg] (N1) {
          \code{x <- 1}\strut\\
          \code{y <- 2}\strut
        };

        \node[above of=N1, node distance=0.5cm, xshift=1.2in] {$\{\}$};
        \node[xshift=1.2in] at (N1) {$\{\code{x} \mapsto \code{1}\}$};
        \node[below of=N1, node distance=0.5cm, xshift=1.2in] {$\{\code{x} \mapsto \code{1}, \code{y} \mapsto \code{2}\}$};

        \node[box, below of=N1] (N2) {
          \code{x <- x + 2}
        };

        \node[box, below of=N2] (N3) {
          \code{ret x}
        };

        \draw[-stealth] (N1) -- (N2);
        \draw[-stealth] (N2.south west) to [out = 210, in=150, looseness=2] (N2.north west);
        \draw[-stealth] (N2) -- (N3);
      \end{tikzpicture}
    \end{minipage}
  \end{center}

\end{frame}

\begin{frame}[fragile]
  \frametitle{A Dataflow Example}

  \begin{center}
    \begin{minipage}{0.55\textwidth}
      \raggedright
      After following the highlighted edge, we end up at the second block. Since
      we have some information about what variables are constant, we can carry that
      information here.

      \vspace{10pt}

      Then, we see that \code{x} is incremented by two, and thus must be constant
      at \code{3} at the conclusion of the block.

    \end{minipage}
    \hfill
    \begin{minipage}{0.4\textwidth}
      \centering
      \begin{tikzpicture}[node distance=0.8in]
        \node[box, text width=1in] (N1) {
          \code{x <- 1}\strut\\
          \code{y <- 2}\strut
        };

        \node[above of=N1, node distance=0.5cm, xshift=1.2in] {$\{\}$};
        \node[xshift=1.2in] at (N1) {$\{\code{x} \mapsto \code{1}\}$};
        \node[below of=N1, node distance=0.5cm, xshift=1.2in] {$\{\code{x} \mapsto \code{1}, \code{y} \mapsto \code{2}\}$};

        \node[box, below of=N1, hlbg] (N2) {
          \code{x <- x + 2}
        };

        \node[above of=N2, node distance=0.5cm, xshift=1.2in] {$\{\code{x} \mapsto \code{1}, \code{y} \mapsto \code{2}\}$};
        \node[below of=N2, node distance=0.5cm, xshift=1.2in] {$\{\code{x} \mapsto \code{3}, \code{y} \mapsto \code{2}\}$};

        \node[box, below of=N2] (N3) {
          \code{ret x}
        };

        \draw[-stealth, hledge] (N1) -- (N2);
        \draw[-stealth] (N2.south west) to [out = 210, in=150, looseness=2] (N2.north west);
        \draw[-stealth] (N2) -- (N3);
      \end{tikzpicture}
    \end{minipage}
  \end{center}

\end{frame}

\begin{frame}[fragile]
  \frametitle{A Dataflow Example}

  \begin{center}
    \begin{minipage}{0.55\textwidth}
      \raggedright

      But, now we need to follow the self-loop! Something weird happens here.

      \vspace{10pt}

      There are two conflicting out-sets that are going in to the second block.
      One is the one we just computed,
      $\{\code{x} \mapsto \code{3}, \code{y} \mapsto \code{1}\}$, from the
      output of the second block itself. The other is
      $\{\code{x} \mapsto \code{1}, \code{y} \mapsto \code{1}\}$, from the
      original out-set from the first block.

      \vspace{10pt}

      This means we have a conflict. \code{x} is constant, but at two
      different values, coming in to the second block.
    \end{minipage}
    \hfill
    \begin{minipage}{0.4\textwidth}
      \centering
      \begin{tikzpicture}[node distance=0.8in]
        \node[box, text width=1in] (N1) {
          \code{x <- 1}\strut\\
          \code{y <- 2}\strut
        };

        \node[above of=N1, node distance=0.5cm, xshift=1.2in] {$\{\}$};
        \node[xshift=1.2in] at (N1) {$\{\code{x} \mapsto \code{1}\}$};
        \node[below of=N1, node distance=0.5cm, xshift=1.2in] {$\{{\color{hlset}\code{x} \mapsto \code{1}}, \code{y} \mapsto \code{2}\}$};

        \node[box, below of=N1, hlbg] (N2) {
          \code{x <- x + 2}
        };

        \node[above of=N2, node distance=0.5cm, xshift=1.2in] {$\{\code{x} \mapsto \code{1}, \code{y} \mapsto \code{2}\}$};
        \node[below of=N2, node distance=0.5cm, xshift=1.2in] {$\{{\color{hlset}\code{x} \mapsto \code{3}}, \code{y} \mapsto \code{2}\}$};

        \node[box, below of=N2] (N3) {
          \code{ret x}
        };

        \draw[-stealth] (N1) -- (N2);
        \draw[-stealth, hledge] (N2.south west) to [out = 210, in=150, looseness=2] (N2.north west);
        \draw[-stealth] (N2) -- (N3);
      \end{tikzpicture}
    \end{minipage}
  \end{center}
\end{frame}

\begin{frame}[fragile]
  \frametitle{A Dataflow Example}

  \begin{center}
    \begin{minipage}{0.55\textwidth}
      \raggedright

      This must mean that \code{x} is not constant after all.

      \vspace{10pt}

      So we set the value of \code{x} to instead be $\top$, which means
      "not constant". Note that this is different than \code{x} not having
      a value, which denotes not knowing if it is constant or not.
    \end{minipage}
    \hfill
    \begin{minipage}{0.4\textwidth}
      \centering
      \begin{tikzpicture}[node distance=0.8in]
        \node[box, text width=1in] (N1) {
          \code{x <- 1}\strut\\
          \code{y <- 2}\strut
        };

        \node[above of=N1, node distance=0.5cm, xshift=1.2in] {$\{\}$};
        \node[xshift=1.2in] at (N1) {$\{\code{x} \mapsto \code{1}\}$};
        \node[below of=N1, node distance=0.5cm, xshift=1.2in] {$\{\code{x} \mapsto \code{1}, \code{y} \mapsto \code{2}\}$};

        \node[box, below of=N1, hlbg] (N2) {
          \code{x <- x + 2}
        };

        \node[above of=N2, node distance=0.5cm, xshift=1.2in] {$\{\code{x} \mapsto {\color{hlset}\top}, \code{y} \mapsto \code{2}\}$};
        \node[below of=N2, node distance=0.5cm, xshift=1.2in] {$\{\code{x} \mapsto {\color{hlset}\top}, \code{y} \mapsto \code{2}\}$};

        \node[box, below of=N2] (N3) {
          \code{ret x}
        };

        \draw[-stealth] (N1) -- (N2);
        \draw[-stealth] (N2.south west) to [out = 210, in=150, looseness=2] (N2.north west);
        \draw[-stealth] (N2) -- (N3);
      \end{tikzpicture}
    \end{minipage}
  \end{center}
\end{frame}

\begin{frame}[fragile]
  \frametitle{A Dataflow Example}

  \begin{center}
    \begin{minipage}{0.55\textwidth}
      \raggedright

      Then, once we are assured that everything looks good, we can proceed
      to the final block, where we observe that we return \code{x} at a
      non-constant value.

      \vspace{10pt}

      This means that we cannot optimize the return value of this function
      after all.

      \vspace{10pt}

      Different story if we had returned \code{y}!
    \end{minipage}
    \hfill
    \begin{minipage}{0.4\textwidth}
      \centering
      \begin{tikzpicture}[node distance=0.8in]
        \node[box, text width=1in] (N1) {
          \code{x <- 1}\strut\\
          \code{y <- 2}\strut
        };

        \node[above of=N1, node distance=0.5cm, xshift=1.2in] {$\{\}$};
        \node[xshift=1.2in] at (N1) {$\{\code{x} \mapsto \code{1}\}$};
        \node[below of=N1, node distance=0.5cm, xshift=1.2in] {$\{\code{x} \mapsto \code{1}, \code{y} \mapsto \code{2}\}$};

        \node[box, below of=N1] (N2) {
          \code{x <- x + 2}
        };

        \node[above of=N2, node distance=0.5cm, xshift=1.2in] {$\{\code{x} \mapsto \top, \code{y} \mapsto \code{2}\}$};
        \node[below of=N2, node distance=0.5cm, xshift=1.2in] {$\{\code{x} \mapsto \top, \code{y} \mapsto \code{2}\}$};

        \node[box, below of=N2, hlbg] (N3) {
          \code{ret x}
        };

        \node[above of=N3, node distance=0.5cm, xshift=0.9in] {$\{\code{x} \mapsto \top, \code{y} \mapsto \code{2}\}$};
        \node[below of=N3, node distance=0.5cm, xshift=0.9in] {$\{\code{x} \mapsto \top, \code{y} \mapsto \code{2}\}$};

        \draw[-stealth] (N1) -- (N2);
        \draw[-stealth] (N2.south west) to [out = 210, in=150, looseness=2] (N2.north west);
        \draw[-stealth] (N2) -- (N3);
      \end{tikzpicture}
    \end{minipage}
  \end{center}
\end{frame}

\begin{frame}[fragile]
  \frametitle{Generality of Dataflow}

  This is a contrived example, but the really interesting thing is that
  dataflow analysis works for \textit{any} control-flow graph.

  \pause
  \vspace{\fill}

  The process seemed somewhat silly, as we could determine with our eyes that
  \code{x} was non-constant, but for very complicated control-flow graphs this
  is not an obvious fact at all. Programmatically, we can still run this same
  analysis, however.

  \pause
  \vspace{\fill}

  This analysis is also guaranteed to terminate, due to the monotonic reasons
  we stated earlier. The actual reason for this is that dataflow analysis
  strictly traverses up a \term{lattice}.
\end{frame}

\begin{frame}[fragile]
  \frametitle{Constant Definitions Lattice}

  \begin{center}
    \begin{tikzpicture}
      \node[bhex, fill=bgOrange] (T) {$\{\code{x} \mapsto \top, \code{y} \mapsto \top \}$};
      \node[bhex, fill=bgRed, below of=T, node distance=2.5in] (B) {$\{\}$};

      \node[bhex, below of=T, node distance=0.625in, xshift=-1in] (U1)
        {$\{\code{x} \mapsto n, \code{y} \mapsto \top\}$};
      \node[bhex, fill=bgPurple, below of=T, node distance=1.25in] (U2)
        {$\{\code{x} \mapsto n, \code{y} \mapsto n'\}$};
      \node[bhex, below of=T, node distance=0.625in, xshift=1in] (U3)
        {$\{\code{x} \mapsto \top, \code{y} \mapsto n' \}$};

      \node[bhex, above of=B, node distance=1in, xshift=-1.5in] (L1)
        {$\{\code{y} \mapsto \top\}$};
      \node[bhex, above of=B, node distance=0.625in, xshift=-0.5in] (L2)
        {$\{\code{x} \mapsto n\}$};
      \node[bhex, above of=B, node distance=0.625in, xshift=0.5in] (L3)
        {$\{\code{y} \mapsto n'\}$};
      \node[bhex, above of=B, node distance=1in, xshift=1.5in] (L4)
        {$\{\code{x} \mapsto \top \}$};

      \draw[-stealth] (B) -- (L1);
      \draw[-stealth] (B) -- (L2);
      \draw[-stealth] (B) -- (L3);
      \draw[-stealth] (B) -- (L4);

      \draw[-stealth] (L1) -- (U1);

      \draw[-stealth] (L4) -- (U3);

      \draw[-stealth] (L2) -- (U2);

      \draw[-stealth] (L3) -- (U2);

      \draw[-stealth] (U1) -- (T);
      \draw[-stealth] (U3) -- (T);

      \draw[-stealth] (U2) -- (U1);
      \draw[-stealth] (U2) -- (U3);

      \draw[-stealth] (U2) -- (U1);
      \draw[-stealth] (U2) -- (U3);

      \draw[-stealth] (L3) to [out=150, in=-10] (L1);
      \draw[-stealth] (L2) to [out=30, in=190] (L4);

    \end{tikzpicture}
  \end{center}
\end{frame}


\begin{frame}[fragile]
  \frametitle{Climbing the Lattice}

  The diagram looks scary, but the key thing is just that it
  assigns each variable a value of either no value, any constant $n$ or
  $n'$, or $\top$, which means "not constant".

  \pause
  \vspace{\fill}

  Edges go from sets to ones which have either added a new variable at
  a constant, or that have upgraded a variable from a constant to the
  not-a-constant symbol $\top$. This represents the gaining of information,
  of either a variable being declared as a constant, or a variable being
  discovered as not-a-constant.
  \pause
  \begin{itemize}
    \item
      For instance, we started at $\{\}$, and then
      went to $\{ \code{x} \mapsto \code{1} \}$ when we read \code{x <- 1}.
      \pause
    \item
      Or, we went from
        $\{\code{x} \mapsto \code{1}, \code{y} \mapsto \code{2}\}$
      to
        $\{\code{x} \mapsto \top, \code{y} \mapsto \code{2}\}$
      upon seeing that \code{x} was set as two different constants, along
      two different paths to the block.
  \end{itemize}
\end{frame}



\begin{frame}[fragile]
  \frametitle{Monotonicity in Constant Definitions}

  The main thing to take away here is that \textit{every arrow goes up}.
  We talked earlier about putting answers on a number line, which is
  represented in terms of the lattice's height, here. No matter what
  edge you pick, you go up some amount, which means we must terminate.

  \pause
  \vspace{\fill}

  We can't go down, because it's impossible to update your worldview to
  either remove a variable from the set, or to set a variable from not-constant
  to constant. Once you know something about a variable, you can't go back.
\end{frame}

\begin{frame}[fragile]
  \frametitle{Approximation of Dataflow}
  \begin{center}
    \begin{minipage}{0.65\textwidth}
      \raggedright
      I mentioned earlier that this analysis is necessarily wrong, however.

      \vspace{10pt}

      The reason comes out of the fact that the control-flow graph is just how
      the execution of the program \textit{might} go. In reality, it's quite
      possible that at runtime, we never enter the self-loop, meaning that
      \code{x} really is constant.

      \vspace{10pt}

      But, without running the program, we have no way of knowing, so we
      assume that \code{x} \textit{is} updated at some point. This makes
      our knowledge necessarily possibly wrong, but a good approximation.
    \end{minipage}
    \begin{minipage}{0.3\textwidth}
      \centering
      \begin{tikzpicture}[node distance=0.8in]
        \node[box, text width=1in] (N1) {
          \code{x <- 1}\strut\\
          \code{y <- 2}\strut
        };

        \node[box, below of=N1] (N2) {
          \code{x <- x + 2}
        };

        \node[box, below of=N2] (N3) {
          \code{ret x}
        };

        \draw[-stealth] (N1) -- (N2);
        \draw[-stealth] (N2.south west) to [out = 210, in=150, looseness=2] (N2.north west);
        \draw[-stealth] (N2) -- (N3);
      \end{tikzpicture}
    \end{minipage}
  \end{center}
\end{frame}

\begin{frame}[fragile]
  \frametitle{On Dataflow Analysis}

  This was a really simple example that I hoped you might be able to understand.

  \pause
  \vspace{\fill}

  Dataflow analysis in general is a very powerful technique, however, and admits
  many other analyses, many of which are quite useful. These include:
  \pause
  \begin{itemize}
    \item \term{available expressions} - is there a definition of this exact
    expression already at this program point? useful in optimizing away
    redundant computations. \pause
    \item \term{reaching definitions} - what definitions of a variable can
    reach a given program point? useful in building use-def chains (i.e.
    "goto definition" in IDEs) \pause
    \item \term{liveness analysis} - what variables might actually be needed
    in the future? useful in eliminating dead code. \pause
    \item \term{taint tracking} - can data from undesirable sources reach
    some sensitive program point? \textbf{very} useful in security applications.
  \end{itemize}

  \pause
  \vspace{\fill}

  The rabbit hole goes deep. This one is simple, but this is a bread-and-butter
  technique in program analysis.
\end{frame}

\begin{frame}[fragile]
  \frametitle{Dataflow Analysis and Functional Programming}

  Dataflow analysis, unfortunately, has almost nothing to do with functional
  programming.

  \pause
  \vspace{\fill}

  I've been avoiding showing you code for this part because it's both pretty
  complex, and there aren't extreme benefits to doing it in SML.

  \pause
  \vspace{\fill}

  Dataflow analysis can be considered a more "imperative" approach to
  program analysis, because it occurs at the lower level of the control-flow
  graph and the abstract assembly instructions. What if we want to stay at
  the level of the AST, where all the nice semantics and program structure live?

  \pause
  \vspace{\fill}

  It turns out, this will be a very effective approach to program analysis.

  \pause
  \vspace{\fill}

  This is the story of Semgrep.
\end{frame}

\sectionSlide{5}{Semgrep}

% \begin{frame}[fragile]
%   \frametitle{Flavors of Program Analysis}

%   It is difficult to give an accurate account of the state of cybersecurity within a
%   single lecture (while talking about other things), but the main important thing to
%   realize is that security is often the least concern of many developers.

%   Developers want to write code, and are judged upon, promoted due to, and attached
%   to the code that they write. This only makes the ultimate goal, the security of
%   software throughout all sectors, a hard one to accomplish.

%   There are engineers specifically hired to ensure the security of certain company's
%   code, there are departments dedicated towards that same practice. The cost of a
%   security breach is high, and nobody wants to be the next Yahoo.

%   What does that mean? It means that we are on an uphill slope of a slope which
%   is already upside-down.
% \end{frame}

% \begin{frame}[fragile]
%   \frametitle{Flavors of Program Analysis}

%   In the cybersecurity space, there are two big important aspects to any
%   code-scanning security tool (or \term{SAST tool}).

%   Firstly, it should terminate within a reasonable amount of time. Code is
%   usually scanned on each PR\footnote<2->{Meaning \term{pull request}, or a
%   potential contribution to a software project.}, and needs to fit within a
%   developer's normal workflow, meaning a longer scan time than a few minutes is
%   completely unreasonable.

%   Secondly, it should not be too noisy. Security tools are going to produce
%   matches that are wrong -- either false positives or false negatives. That's
%   a part of the job. But they can't be wrong \textit{too} often, or that
%   also marks an undesirable tool.
% \end{frame}

% \begin{frame}[fragile]
%   \frametitle{Flavors of Program Analysis}

%   To that end, security becomes an interesting game where the goal of a SAST
%   tool is to work, but almost invisibly. Nobody likes a noisy tool. When a
%   SAST tool pipes up, it had better be for a good reason.

%   \defBox{}{We call a reported error or region of interest from a code-scanning
%   tool a \term{finding}.}

%   But another component is that a SAST tool should try to work
%   \textit{with the developer}. It should be clear why a finding occurred,
%   how it can be fixed, and it should get in the developer's way as little
%   as possible. That's the game of security -- be functional, but frictionless.
% \end{frame}

\begin{frame}[fragile]
  \frametitle{Analysis at the ASTs}

  Before we delve deeper, it's worth developing: why should we stay at the
  AST level, as opposed to abstract assembly? Isn't that the opposite of what
  a compiler does? Shouldn't closer to the source be better?

  \pause
  \vspace{\fill}

  Before I can answer that question, it's worth noting some things about
  security and software engineering.

  \pause
  \vspace{\fill}

  \textbf{In the absence of external pressure, these things do not generally go together.}

  \pause
  \vspace{\fill}

  Software engineers are paid to write code, are judged on their code, and spend
  large amounts of time thinking about their code. What they are not necessarily,
  however, is educated in the ways of safety. So this tends to take a back seat.
\end{frame}

\begin{frame}[fragile]
  \frametitle{Being Wrong}

  For a compiler, being wrong is a hellish scenario that is too horrifying
  to even contemplate.

  \pause
  \vspace{\fill}

  For a program analysis tool, being wrong is Tuesday.

  \pause
  \vspace{\fill}

  Still, to the developers who are receiving security notifications on their
  pull requests, as well as the security engineers that are constantly on the
  lookout for big vulnerabilities, it matters \textbf{how wrong} the tool
  usually is.

  \pause
  \vspace{\fill}

  The easier way to make change is not to preach at someone the right
  way to do things, but to make them want to do things the right way. A
  SAST tool needs to be fruitful, but frictionless.

  \pause
  \vspace{\fill}

  Why am I mentioning this?
\end{frame}

\begin{frame}[fragile]
  \frametitle{Deduplicating Efforts}

  We want to be able to provide as little friction as possible, while producing
  the best possible results. In the case of Semgrep, \textbf{staying primarily
  at the language level is an enormous labor saving device}.

  \pause
  \vspace{\fill}

  We talked earlier in the compilers lecture about how a compiler will have a
  type \code{token list} and a type \code{ast}, which it converts from and to
  during parsing.

  \pause
  \vspace{\fill}

  For a program analysis tool for Python, they may need to do this
  parsing process from a token list to a Python AST. For a program analysis
  tool for C, they may need to do the same for a C AST, which is a different
  type.

  \pause
  \vspace{\fill}

  This can lead to massive duplication of efforts.
\end{frame}

\begin{frame}[fragile]
  \frametitle{A Pipeline for Program Analysis}

  So the picture looks like this:

  \begin{center}
    \begin{tikzpicture}
      \node[box] (N1) {Java code\strut};
      \node[box, right of=N1, node distance=1.5in] (N2) {Python code\strut};
      \node[box, right of=N2, node distance=1.5in] (N3) {C code\strut};

      \node[box, below of=N1, node distance=0.6in, fill=bgPurple] (M1) {Java AST};
      \node[box, below of=N2, node distance=0.6in, fill=bgPurple] (M2) {Python AST};
      \node[box, below of=N3, node distance=0.6in, fill=bgPurple] (M3) {C AST};

      \node[box, below of=M1, node distance=0.6in, fill=bgBlue] (L1) {Java analysis};
      \node[box, below of=M2, node distance=0.6in, fill=bgBlue] (L2) {Python analysis};
      \node[box, below of=M3, node distance=0.6in, fill=bgBlue] (L3) {C analysis};

      \draw[-stealth] (N1) -- (M1);
      \draw[-stealth] (N2) -- (M2);
      \draw[-stealth] (N3) -- (M3);

      \draw[-stealth] (M1) -- (L1);
      \draw[-stealth] (M2) -- (L2);
      \draw[-stealth] (M3) -- (L3);
    \end{tikzpicture}
  \end{center}

  \pause
  \vspace{\fill}

  This introduces a massive amount of boilerplate.
\end{frame}

\begin{frame}[fragile]
  \frametitle{The Essence of Programming Languages}

  For as much as people complain about programming languages\footnote{As have I.},
  after some time with them you begin to realize that most of them look quite similar.

  \pause
  \vspace{\fill}

  For instance, many languages have for loops, and many languages have exception
  raising and handling, and almost every language has functions, ways to define
  variables, and ways to call functions.

  \pause
  \vspace{\fill}

  Now, every language has its own idiosyncrasies which set them apart in miniscule ways,
  which makes writing something like a universal compiler a pipe dream. But what
  if we're working in an application where being wrong in small ways doesn't really
  matter?

  \pause
  \vspace{\fill}

  For program analysis tools, being wrong is Tuesday.
\end{frame}

\begin{frame}[fragile]
  \frametitle{A Pipeline for Semgrep Analysis}

  Semgrep is a code-scanning tool for over 24 languages.

  \pause
  \vspace{\fill}

  The way that it achieves this is that it parses every single language to the
  same type, called the Generic AST, which is a smörgåsbord union of all
  the language features in pretty much every language.

  \pause
  \vspace{\fill}

  This means that its pipeline looks remarkably simpler:

  \begin{center}
    \begin{tikzpicture}
      \node[box] (N1) {Java code\strut};
      \node[box, right of=N1, node distance=1.5in] (N2) {Python code\strut};
      \node[box, right of=N2, node distance=1.5in] (N3) {C code\strut};

      \node[hex, below of=N2, node distance=0.6in, fill=bgRed] (M) {Generic AST};

      \node[box, below of=M, node distance=0.6in, fill=bgBlue] (L) {Analysis};

      \draw[-stealth] (N1) -- (M);
      \draw[-stealth] (N2) -- (M);
      \draw[-stealth] (N3) -- (M);

      \draw[-stealth] (M) -- (L);
    \end{tikzpicture}
  \end{center}
\end{frame}

\begin{frame}[fragile]
  \frametitle{Picking Battles}

  We see that staying at the AST level saves us work, which can be wrong at
  times due to imperfect translation, but ultimately doesn't matter for a
  program analysis tool. More importantly, it allows for cleaner code to
  be written, which translates to a better tool.

  \pause
  \vspace{\fill}

  When solving an impossible problem, you have to be pragmatic. Picking
  battles that you can win is essential.

  \pause
  \vspace{\fill}

  Semgrep takes a more "syntactic" approach to program analysis that turns
  out to work very well in practice. Now, let's talk about what it does.
\end{frame}

\begin{frame}[fragile]
  \frametitle{Source-Level Errors}

  Many program errors aren't purely due to what happens
  at the level of assembly. Many of them are apparent from the source code,
  because they were written by a developer who was writing source code.

  \pause
  \vspace{\fill}

  For instance, consider the following Python code, with a function that
  has a default argument \code{to} set to \code{[]}:
  \begin{pythoncodeblock}
    def append_to(element, to=[]):
      to.append(element)
      return to
  \end{pythoncodeblock}

  \vspace{\fill}
  \pause

  What do you think this does?
\end{frame}

\begin{frame}[fragile]
  \frametitle{Default Dead Arguments}

  It turns out that the first time you call \code{append_to(2)}, you will
  get \code{[2]}. That's cool.

  \pause
  \vspace{\fill}

  The second time you call \code{append_to(2)}, you will get \code{[2, 2]}.

  \pause
  \vspace{\fill}

  Uh oh.
\end{frame}

\begin{frame}[fragile]
  \frametitle{Errors at the Source}

  It turns out this is due to the fact that Python default arguments are
  instantiated at function definition time, and so if it is mutated, that
  mutation persists through each call.

  \pause
  \vspace{\fill}

  This is an error that would be very easy to catch in code review. Unfortunately,
  there are tens of millions of developers in the world, and even a 99\%
  review success rate has bad implications.

  \pause
  \vspace{\fill}

  So at the end of the day, we might not \textit{need} to descend further down
  than the AST. Let's see how we can fix such bugs while staying above ground.
\end{frame}

\begin{frame}[fragile]
  \frametitle{Forms of Program Analysis}

  It turns out that program analysis can take many forms. We might want to use
  it for security purposes, we might want to use it for style checking, and we
  might want to use it to automate code review for things like the error we just
  saw.

  \pause
  \vspace{\fill}

  A common theme in program analysis is simply finding certain things in your
  source code. This might be something like an unused variable, a type error,
  or even a function call which might lead to a security vulnerability. We
  call such regions of interest a \term{finding}.

  \pause
  \vspace{\fill}

  Semgrep is a program analysis tool\footnote{And company.} that is mainly aimed
  being customizable to all of these use cases. This means that while that while
  some tools find security vulnerabilities, and some tools find style errors,
  the answer to what Semgrep finds is \textit{whatever you want}.
\end{frame}

\begin{frame}[fragile]
  \frametitle{A Semantic Grep}

  {\color{blue}\href{https://semgrep.dev/}{Semgrep}} stands for \textit{semantic grep}.

  \pause
  \vspace{\fill}

  If you've used the tool \code{grep} before, you know that it is really just a
  no-nonsense precursor to what is now Ctrl-F in most browsers.\footnote{Such as Arc,
  the browser that I am currently using.}

  \pause
  \vspace{\fill}

  But, it's limited in some ways. Suppose we wanted to find all instances of
  using the function \code{print} to debug, before we put the code up for
  review. What do you think is going to happen when we search for every instance
  of the literal substring "print"?

  \pause
  \vspace{\fill}

  The answer: \textbf{We are going to be absolutely inundated with results.} We'll
  find the string \code{print} if it occurs in comments, if it occurs in literal
  strings, and even if it's part of a larger method name. That doesn't make it what
  we were looking for, though!

  \pause
  \vspace{\fill}

  In other words, \code{grep} doesn't understand the \textit{meaning} of what we're
  looking for, it's just looking for sequences of characters. Semgrep does.
\end{frame}

\begin{frame}[fragile]
  \frametitle{Tree Search and Semgrep}

  How does Semgrep understand the semantics of the program? Recall that
  the literal text in a program just serve as a proxy that indicates the underlying
  tree of the program, which contains its real meaning.

  \pause
  \vspace{\fill}

  Semgrep doesn't do text search, it does \textit{tree search}. This means that
  instead of searching for a sequence of characters within a given program, it
  searches for a \textit{matching subtree} within a program's AST.

  \pause
  \vspace{\fill}

  This means that a user might specify a particular \term{pattern AST}, which refers
  to the AST that they would like to search for, and then Semgrep will search the
  \term{target AST} to find a subtree which matches it.

  \pause
  \vspace{\fill}

  The key innovation of Semgrep is that, although it does AST matching, the user
  doesn't need to be able to construct one, or even know what it is. Pattern ASTs
  are derived from patterns written \textit{in the source language}, meaning
  that they look like code.
\end{frame}

\begin{frame}[fragile]
  \frametitle{Matching, Take 1}


  \begin{center}
    \begin{minipage}[t][0.1\textheight][t]{\textwidth}
      {\color{blue}\href{https://semgrep.dev/playground/s/BBJ5}{So, our query might look like:}}
    \end{minipage}
    \begin{minipage}[t][0.6\textheight][t]{\textwidth}
      \begin{minipage}{0.38\textwidth}
        \centering
        \begin{tikzpicture}
          \node[hex] (A) {\code[keywordstyle=\color{black}]{print}\strut};
        \end{tikzpicture}
      \end{minipage}
      \begin{minipage}{0.58\textwidth}
        \centering
        \begin{tikzpicture}
          \node[box] (N1) {\code{ValDec}}
            child{node[circ, fill=bgRed, xshift=-0.3in] (N2) {\code{()}}}
            child{node[bbox, xshift=0.3in] (N3) {\code{FuncCall}}
              child{node[hex, xshift=-1cm] (N4) {\code[keywordstyle=\color{black}]{print}\strut}}
              child{node[bbox, xshift=1cm] (N5) {\code{"let's print!"}\strut}}
            };
        \end{tikzpicture}
      \end{minipage}
    \end{minipage}
    \begin{minipage}[t][0.3\textheight][t]{\textwidth}
      \begin{minipage}{0.38\textwidth}
        \centering
        \begin{codeblock}
          print
        \end{codeblock}
      \end{minipage}
      \begin{minipage}{0.58\textwidth}
        \begin{codeblock}
          val () = print "let's print!"
        \end{codeblock}
      \end{minipage}
    \end{minipage}
  \end{center}
\end{frame}

\begin{frame}[fragile]
  \frametitle{Matching, Take 1}

  \begin{center}
    \begin{minipage}[t][0.1\textheight][t]{\textwidth}
      {\color{blue}\href{https://semgrep.dev/playground/s/BBJ5}{So, our query might look like:}}
    \end{minipage}
    \begin{minipage}[t][0.6\textheight][t]{\textwidth}
      \begin{minipage}{0.38\textwidth}
        \centering
        \begin{tikzpicture}
          \node[hex] (A) {\code[keywordstyle=\color{black}]{print}\strut};
        \end{tikzpicture}
      \end{minipage}
      \begin{minipage}{0.58\textwidth}
        \centering
        \begin{tikzpicture}
          \node[box] (N1) {\code{ValDec}}
            child{node[circ, fill=bgRed, xshift=-0.3in] (N2) {\code{()}}}
            child{node[bbox, xshift=0.3in] (N3) {\code{FuncCall}}
              child{node[hex, xshift=-1cm, highlight] (N4) {\code[keywordstyle=\color{black}]{print}\strut}}
              child{node[bbox, xshift=1cm] (N5) {\code{"let's print!"}\strut}}
            };
        \end{tikzpicture}
      \end{minipage}
    \end{minipage}
    \begin{minipage}[t][0.3\textheight][t]{\textwidth}
      \begin{minipage}{0.38\textwidth}
        \centering
        \begin{codeblock}
          print
        \end{codeblock}
      \end{minipage}
      \begin{minipage}{0.58\textwidth}
        \begin{codeblock}
          val () = `print` "let's print!"
        \end{codeblock}
      \end{minipage}
    \end{minipage}
  \end{center}
\end{frame}

\begin{frame}[fragile]
  \frametitle{Syntax Aware}

  Notably, however, we avoid matching the instance of the literal string
  of "print" inside of the string that is being passed to its namesake function.

  \pause
  \vspace{\fill}

  This is because we are only matching at the granularity of nodes of the AST,
  not of the constituent text. Regular expressions, for instance, would be fooled
  by instances of "print" within comments or strings.

  \pause
  \vspace{\fill}

  When it comes to minimizing instances of program analysis tools being flat-out
  wrong, this is an invaluable help. Most program analysis tools are black-box
  applications that perform some magic to produce matches, without necessarily
  being understandable. The algorithm of Semgrep is really quite simple, and
  still manages to avoid false positives.
\end{frame}


\begin{frame}[fragile]
  \frametitle{Matching, Take 2}

  \begin{center}
    \begin{minipage}[t][0.21\textheight][t]{\textwidth}
      \raggedright
      {\color{blue}\href{https://semgrep.dev/playground/s/07Ob}{We can do a little more advanced of a query, too.}} Suppose that we were
      interested in finding all of the examples of a call to the function
      \code{ref}.

      \vspace{5pt}

      Well, we might write a pattern like \code{ref $\,\,\dollar$X}, which uses a
      \textit{metavariable} to bind to any sub-AST.
    \end{minipage}
    \begin{minipage}[t][0.515\textheight][t]{\textwidth}
      \begin{minipage}{0.48\textwidth}
        \centering
        \begin{tikzpicture}
          \node[bbox] (N1) {\code{FuncCall}}
            child{node[bbox] (N2) {\code[keywordstyle=\color{black}]{ref}}}
            child{node[unknown] (N3) {\texttt{\$}\code{X}}}
          ;
        \end{tikzpicture}
      \end{minipage}
      \begin{minipage}{0.48\textwidth}
        \centering
        \begin{tikzpicture}
          \node[box] (N1) {\code{ValDec}}
            child{node[circ, fill=bgRed, xshift=-0.3in] (N2) {\code{r}}}
            child{node[bbox, xshift=0.3in] (N3) {\code{FuncCall}}
              child{node[hex, xshift=-1cm] (N4) {\code[keywordstyle=\color{black}]{ref}}}
              child{node[bbox, xshift=1cm] (N5) {\code{2}}}
            };
        \end{tikzpicture}
      \end{minipage}
    \end{minipage}
    \begin{minipage}[t][0.3\textheight][t]{\textwidth}
      \begin{minipage}{0.48\textwidth}
        \centering
        \begin{codeblock}
          ref $\dollar$X
        \end{codeblock}
      \end{minipage}
      \begin{minipage}{0.48\textwidth}
        \begin{codeblock}
          val r = ref 2
        \end{codeblock}
      \end{minipage}
    \end{minipage}
  \end{center}
\end{frame}


\begin{frame}[fragile]
  \frametitle{Matching, Take 2}

  \begin{center}
    \begin{minipage}[t][0.21\textheight][t]{\textwidth}
      \raggedright
      {\color{blue}\href{https://semgrep.dev/playground/s/07Ob}{Here, we bind the metavariable}} \code{$\dollar$X} to the node of \code{2},
      and then the entire pattern succeeds at matching the highlighted sub-tree
      of the target.
    \end{minipage}
    \begin{minipage}[t][0.515\textheight][t]{\textwidth}
      \begin{minipage}{0.48\textwidth}
        \centering
        \begin{tikzpicture}
          \node[bbox] (N1) {\code{FuncCall}}
            child{node[bbox] (N2) {\code[keywordstyle=\color{black}]{ref}}}
            child{node[unknown] (N3) {\texttt{\$}\code{X}}}
          ;
        \end{tikzpicture}
      \end{minipage}
      \begin{minipage}{0.48\textwidth}
        \centering
        \begin{tikzpicture}
          \node[box] (N1) {\code{ValDec}}
            child{node[circ, fill=bgRed, xshift=-0.3in] (N2) {\code{r}}}
            child{node[bbox, xshift=0.3in, highlight] (N3) {\code{FuncCall}}
              child{node[hex, xshift=-1cm, highlight] (N4) {\code[keywordstyle=\color{black}]{ref}}}
              child{node[bbox, xshift=1cm, highlight] (N5) {\code{2}}}
            };
        \end{tikzpicture}
      \end{minipage}
    \end{minipage}
    \begin{minipage}[t][0.3\textheight][t]{\textwidth}
      \begin{minipage}{0.48\textwidth}
        \centering
        \begin{codeblock}
          ref $\dollar$X
        \end{codeblock}
      \end{minipage}
      \begin{minipage}{0.48\textwidth}
        \begin{codeblock}
          val r = `ref 2`
        \end{codeblock}
      \end{minipage}
    \end{minipage}
  \end{center}
\end{frame}


\begin{frame}[fragile]
  \frametitle{Matching, Take 3}

  \begin{center}
    \begin{minipage}[t][0.21\textheight][t]{\textwidth}
      \raggedright
      {\color{blue}\href{https://semgrep.dev/playground/s/YxEX}{We can also express more complicated queries.}} What if we want to find instances
      of a useless comparison, such as testing if an expression is equal to itself?
    \end{minipage}
    \begin{minipage}[t][0.515\textheight][t]{\textwidth}
      \begin{minipage}{0.48\textwidth}
        \centering
        \begin{tikzpicture}
          \node[circ] (N1) {\code{=}}
            child{node[unknown] (N2) {\code{$\dollar$X}}}
            child{node[unknown] (N3) {\code{$\dollar$X}}}
          ;
        \end{tikzpicture}
      \end{minipage}
      \begin{minipage}{0.48\textwidth}
        \centering
        \begin{tikzpicture}[level distance=1cm]
          \node[box] (N1) {\code{ValDec}}
            child{node[hex, xshift=-0.3in] (N2) {\code{b}}}
            child{node[circ] (N3) {\code{=}}
              child{node[bbox, xshift=-0.35in] (N4) {\code{FuncCall}}
                child{node[hex] (N5) {\code{f}}}
                child{node[bbox] (N6) {\code{2}}}
              }
              child{node[bbox, xshift=0.35in] (N7) {\code{FuncCall}}
                child{node[hex] (N8) {\code{f}}}
                child{node[bbox] (N9) {\code{2}}}
              }
            }
          ;
        \end{tikzpicture}
      \end{minipage}
    \end{minipage}
    \begin{minipage}[t][0.3\textheight][t]{\textwidth}
      \begin{minipage}{0.48\textwidth}
        \centering
        \begin{codeblock}
          $\dollar$X = $\dollar$X
        \end{codeblock}
      \end{minipage}
      \begin{minipage}{0.48\textwidth}
        \begin{codeblock}
          val b = (f 2 = f 2)
        \end{codeblock}
      \end{minipage}
    \end{minipage}
  \end{center}
\end{frame}

\begin{frame}[fragile]
  \frametitle{Matching, Take 3}

  \begin{center}
    \begin{minipage}[t][0.21\textheight][t]{\textwidth}
      \raggedright
      {\color{blue}\href{https://semgrep.dev/playground/s/YxEX}{Here, because we reuse the metavariable}} \code{$\dollar$X} twice,
      the two instances \term{unify}, meaning that the ASTs that each
      binds to must be the same. In this case, that works.
    \end{minipage}
    \begin{minipage}[t][0.515\textheight][t]{\textwidth}
      \begin{minipage}{0.48\textwidth}
        \centering
        \begin{tikzpicture}
          \node[circ] (N1) {\code{=}}
            child{node[unknown] (N2) {\code{$\dollar$X}}}
            child{node[unknown] (N3) {\code{$\dollar$X}}}
          ;
        \end{tikzpicture}
      \end{minipage}
      \begin{minipage}{0.48\textwidth}
        \centering
        \begin{tikzpicture}[level distance=1cm]
          \node[box] (N1) {\code{ValDec}}
            child{node[hex, xshift=-0.3in] (N2) {\code{b}}}
            child{node[circ, highlight] (N3) {\code{=}}
              child{node[bbox, xshift=-0.35in, highlight] (N4) {\code{FuncCall}}
                child{node[hex, highlight] (N5) {\code{f}}}
                child{node[bbox, highlight] (N6) {\code{2}}}
              }
              child{node[bbox, xshift=0.35in, highlight] (N7) {\code{FuncCall}}
                child{node[hex, highlight] (N8) {\code{f}}}
                child{node[bbox, highlight] (N9) {\code{2}}}
              }
            }
          ;
        \end{tikzpicture}
      \end{minipage}
    \end{minipage}
    \begin{minipage}[t][0.3\textheight][t]{\textwidth}
      \begin{minipage}{0.48\textwidth}
        \centering
        \begin{codeblock}
          $\dollar$X = $\dollar$X
        \end{codeblock}
      \end{minipage}
      \begin{minipage}{0.48\textwidth}
        \begin{codeblock}
          val b = `(f 2 = f 2)`
        \end{codeblock}
      \end{minipage}
    \end{minipage}
  \end{center}
\end{frame}

\begin{frame}[fragile]
  \frametitle{Matching, Take 3}

  \begin{center}
    \begin{minipage}[t][0.21\textheight][t]{\textwidth}
      \raggedright
      {\color{blue}\href{https://semgrep.dev/playground/s/67Q6}{But in this case}}, there would be no match, because the {\color{blue}blue} and
      {\color{magenta}magenta} subtrees do not match.
    \end{minipage}
    \begin{minipage}[t][0.515\textheight][t]{\textwidth}
      \begin{minipage}{0.48\textwidth}
        \centering
        \begin{tikzpicture}
          \node[circ] (N1) {\code{=}}
            child{node[unknown] (N2) {\code{$\dollar$X}}}
            child{node[unknown] (N3) {\code{$\dollar$X}}}
          ;
        \end{tikzpicture}
      \end{minipage}
      \begin{minipage}{0.48\textwidth}
        \centering
        \begin{tikzpicture}[level distance=1cm]
          \node[box] (N1) {\code{ValDec}}
            child{node[hex, xshift=-0.3in] (N2) {\code{b}}}
            child{node[circ] (N3) {\code{=}}
              child{node[bbox, xshift=-0.35in, highlight2] (N4) {\code{FuncCall}}
                child{node[hex, highlight2] (N5) {\code{f}}}
                child{node[bbox, highlight2] (N6) {\code{2}}}
              }
              child{node[bbox, xshift=0.35in, highlight3] (N7) {\code{FuncCall}}
                child{node[hex, highlight3] (N8) {\code{f}}}
                child{node[bbox, highlight3] (N9) {\code{2}}}
              }
            }
          ;
        \end{tikzpicture}
      \end{minipage}
    \end{minipage}
    \begin{minipage}[t][0.3\textheight][t]{\textwidth}
      \begin{minipage}{0.48\textwidth}
        \centering
        \begin{codeblock}
          $\dollar$X = $\dollar$X
        \end{codeblock}
      \end{minipage}
      \begin{minipage}{0.48\textwidth}
        \begin{codeblock}
          val b = (f 2 = g 2)
        \end{codeblock}
      \end{minipage}
    \end{minipage}
  \end{center}
\end{frame}

\begin{frame}[fragile]
  \frametitle{A 150 Analysis}

  Let's apply this to a very concrete example that we saw earlier this
  semester. Recall that appending a singleton to the end of the list is
  an anti-pattern that generally indicates you are doing something inefficient.

  \pause
  \vspace{\fill}

  We can write a Semgrep query which checks for precisely that!

  \begin{codeblock}
    $\dollar$E1 @ [$\dollar$E2]
  \end{codeblock}
\end{frame}
\begin{frame}[fragile]
  \frametitle{A 150 Analysis}

  \begin{center}
    \begin{minipage}[t][0.1\textheight][t]{\textwidth}
      \raggedright

      {\color{blue}\href{https://semgrep.dev/playground/s/oXpN}{Let's see it in action:}}
    \end{minipage}
    \begin{minipage}[t][0.5\textheight][t]{\textwidth}
      \begin{minipage}{0.48\textwidth}
        \centering
        \begin{tikzpicture}
          \node[circ] (N3) {\code{@}}
            child{node[unknown, xshift=-0.35in] (N4) {\code{$\dollar$E1}}}
            child{node[bbox, xshift=0.35in] (N7) {\code{List}}
              child{node[unknown] (N8) {\code{$\dollar$E2}}}
            }
          ;
        \end{tikzpicture}
      \end{minipage}
      \begin{minipage}{0.48\textwidth}
        \centering
        \begin{tikzpicture}[level distance=1cm]
          \node[] (N1) {\textellipsis}
            child[missing]
            child{node[circ, highlight] (N3) {\code{@}}
              child{node[bbox, xshift=-0.35in, highlight] (N4) {\code{FuncCall}}
                child{node[hex, highlight] (N5) {\code{rev}}}
                child{node[hex, highlight] (N6) {\code{xs}}}
              }
              child{node[bbox, xshift=0.35in, highlight] (N7) {\code{List}}
                child{node[hex, highlight] (N8) {\code{x}}}
              }
            }
          ;
        \end{tikzpicture}
      \end{minipage}
    \end{minipage}
    \begin{minipage}[t][0.3\textheight][t]{\textwidth}
      \begin{minipage}{0.48\textwidth}
        \centering
        \begin{codeblock}
          $\dollar$E1 @ [$\dollar$E2]
        \end{codeblock}
      \end{minipage}
      \begin{minipage}{0.48\textwidth}
        \begin{codeblock}
          fun rev [] = []
            | rev (x::xs) =
                `rev xs @ [x]`
        \end{codeblock}
      \end{minipage}
    \end{minipage}
  \end{center}
\end{frame}

\begin{frame}[fragile]
  \frametitle{Matching, Generically}

  Because all programs parse to the same AST type, we only need a single
  matching function of type:

  \begin{codeblock}
    val match_exprs : expr * expr -> finding list
  \end{codeblock}

  which can do matching for Java, for C, for OCaml, for Python, and many
  other languages besides.

  \pause
  \vspace{\fill}

  For the simple example we saw earlier, of matching the Python function with a
  list as a default argument,
  {\color{blue}\href{https://semgrep.dev/playground/s/JkjR}{we can write a very
  simple Semgrep rule.}}
\end{frame}

\begin{frame}[fragile]
  \frametitle{Semgrep and Semantics}

  When it comes to program analysis, tree matching isn't necessarily enough.
  Semgrep's main matching capability is at its core more of a Type C analysis,
  but sometimes we want to be able to find more semantic bugs!

  \pause
  \vspace{\fill}

  When it comes to tracking the flow of data, Semgrep uses the theory of
  dataflow analysis as developed before to implement \term{taint tracking},
  which allows you to specify certain sources of bad data, and sinks where that
  data should not flow into, during the program's execution.

  {\color{blue}\href{https://semgrep.dev/playground/s/D1wq}{Here's an example of taint tracking with Semgrep.}}

  \pause
  \vspace{\fill}

  It can also perform constant propagation much in the same way that we did
  earlier:
  {\color{blue}\href{https://semgrep.dev/playground/s/4APL}{Here's an example.}}

  \pause
  \vspace{\fill}

  Semgrep is aware of the semantics of the programming language that it scans,
  meaning that it is not just a simple matching process. It's a
  syntactically-based analysis that is supplemented by deeper semantic analysis.
\end{frame}

\sectionSlide{6}{Conclusions}

\begin{frame}[fragile]
  \frametitle{Program Analysis at Large}

  Program analysis is an extremely important field which aims to both
  enforce that the code we write is safe, as well as supplement the process
  of writing code by providing programmers with the information they need to
  execute smoothly. Even in the face of mathematical impossibility, it
  prevails.

  \pause
  \vspace{\fill}

  Syntax highlighting, IDE warnings, type-checkers, and autoformatters are all
  among the niceties that programmers enjoy, due to advances made in analyzing
  and better making information available about programs.
\end{frame}

\begin{frame}[fragile]
  \frametitle{Recursive Programs, Recursive Solutions}

  In a world where there are tens of millions of developers, and billions upon
  billions of lines of code, it's simply not possible to cognize in our human
  brains that volume of code. Things will slip through the cracks. Catastrophic
  things might happen.

  \pause
  \vspace{\fill}

  But, programs are ultimately just recursive entities, and recursion is just a
  technique for fitting an unlimited amount of things into your brain.

\end{frame}

\begin{frame}[fragile]
  \frametitle{Program Analysis and Impact}

  There is something to be said about making an impact with your work.

  \pause
  \vspace{\fill}

  One of the things I value a lot is the ability to tackle interesting problems.
  Another is in being able to use elegant foundations (functional programming),
  and another is in being able to ultimately make a real impact.

  \pause
  \vspace{\fill}

  Program analysis is my trinity of all three. There's nothing else in the world
  like it.
\end{frame}

% \begin{frame}[plain]
% 	\begin{center} Thank you! \end{center}

% 	\begin{center}
%     {\color{blue} \href{https://docs.google.com/forms/d/e/1FAIpQLSeLERtwFZFWx4hh44nWLjEBvcSEr_YKGohhypyVd_0wnwoGzg/viewform?usp=sf_link}{Post-lecture survey:}} \\
%     \vspace{5pt}
%     % TODO
%     \includegraphics[scale=0.035]{qr_aug3} \\
%   \end{center}
% \end{frame}

\thankyou

\end{document}
