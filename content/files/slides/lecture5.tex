% Jacob Neumann

% DOCUMENT CLASS AND PACKAGE USE
    \documentclass[aspectratio=169]{beamer}
 
    % Establish the colorlambda boolean, to control whether the lambda is solid color (true), or the same as the picture (false)
    \newif\ifcolorlambda
    \colorlambdafalse % DEFAULT: false
    
    % Use auxcolor for syntax highlighting
    \newif\ifuseaux
    \useauxfalse % DEFAULT: false
   
    % Color settings
    \useauxtrue
    
    \newcommand{\auxColor}{ED6ABE}     % the color of note boxes and stuff
    \newcommand{\presentColor}{FFABE1} % the primary color of the slide borders
    \newcommand{\bgColor}{fef7ff}      % the color of the background of the slide
    \newcommand{\darkBg}{8b98ad}
    \newcommand{\lambdaColor}{\auxColor}
  
    \colorlambdatrue

    \usepackage{comment} % comment blocks
    \usepackage{soul} % strikethrough
    \usepackage{listings} % code
    \usepackage{makecell}

    \setbeamertemplate{itemize items}[circle]
    % \setbeameroption{show notes on second screen=right}

    \usepackage{lectureSlides}
    %%%%%%%%%%%%%%%%%%%%%%%%%%%%%%%%%%%%%%%%%| <----- Don't make the title any longer than this
    \title{Trees} % TODO
    \subtitle{Awesome slides with an awesome subtitle} % TODO
    \date{01 January 2020} % TODO
    \author{Brandon Wu} % TODO

    \graphicspath{ {./img/} }
    % DONT FORGET TO PUT [fragile] on frames with codeblocks, specs, etc.
        %\begin{frame}[fragile]
        %\begin{codeblock}
        %fun fact 0 = 1
        %  | fact n = n * fact(n-1)
        %\end{codeblock}
        %\end{frame}

    % INCLUDING codefile:
        % 1. In some file under code/NN (where NN is the lecture id num), include:
    %       (* FRAGMENT KK *)
    %           <CONTENT>
    %       (* END KK *)
    
    %    Remember to not put anything on the same line as the FRAGMENT or END comment, as that won't be included. KK here is some (not-zero-padded) integer. Note that you MUST have fragments 0,1,...,KK-1 defined in this manner in order for fragment KK to be properly extracted.
        %  2. On the slide where you want code fragment K
                % \smlFrag[color]{KK}
        %     where 'color' is some color string (defaults to 'white'. Don't use presentColor.
    %  3. If you want to offset the line numbers (e.g. have them start at line 5 instead of 1), use
                % \smlFragOffset[color]{KK}{5}

\begin{document}

% Make it so ./mkWeb works correctly
\ifweb
    \renewcommand{\pause}{}
\fi

\setbeamertemplate{itemize items}[circle]

% SOLID COLOR TITLE (see SETTINGS.sty)
{
\begin{frame}[plain]
    \colorlambdatrue
    \titlepage
\end{frame}
}

\begin{frame}[fragile]
  \frametitle{Last time}

  Last time, we learned about the technique of \term{structural induction} for lists, and
  used it to prove some theorems about functions on lists.

  \vspace{\fill}

  We also learned about \term{tail recursion}, and how it can help make code more efficient. We
  used \term{accumulators} to facilitate the implementation. We also implemented some useful
  functions on lists.
\end{frame}

\begin{frame}[fragile]
  \frametitle{Lesson Plan}

  \tableofcontents
\end{frame}

\sectionSlide{1}{Trees}

\begin{frame}[fragile]
  \frametitle{Data Structures}

  In computer science, there are only a few fundamental data structures, from which everything
  else pretty much is derived from.

  \vspace{\fill}

  One of them is lists, which we've already covered. The other is \term{trees}.

  \vspace{\fill}

  \defBox{}{\, A \term{binary tree} is a data structure of nodes, where each node may have up
  to two children that are also binary trees.}

  \vspace{\fill}

  They look something like this:
  image
\end{frame}

\begin{frame}[fragile]
  \frametitle{Trees}

  SML doesn't have an in-built notion of trees, the same way that it does lists, but we can
  define our own. We can use a \term{datatype declaration} to achieve this.

  \vspace{\fill}

  \defBox{}{\, A \term{datatype declaration} is a declaration in SML, which declares a new
  type.}

  \vspace{\fill}

  A datatype declaration for binary trees on integers would look something like this:
  \begin{codeblock}
    datatype tree = Empty | Node of tree * int * tree
  \end{codeblock}
\end{frame}

\begin{frame}[fragile]
  \frametitle{Tree Constructors}

  \begin{codeblock}
    datatype tree = Empty | Node of tree * int * tree
  \end{codeblock}
  
  \vspace{\fill}

  This datatype declaration does two things. Firstly, it declares a new type, named \code{tree}.
  Secondly, it provides that type with two \term{constructors}, in the same way as the constructors 
  that we saw earlier for lists.

  \vspace{\fill}

  This declaration says that:
  \begin{itemize}
    \item \code{Empty : tree}
    \item \code{Node (L, x, R) : tree} if and only if \code{L : tree}, \code{x : int}, and \code{R : tree}.
  \end{itemize}

  \vspace{\fill}

  In essence, it's saying that a tree only takes two forms, \code{Empty} or \code{Node (L, x, R)}. A tree
  is one of those two things, and \textbf{no more}.
\end{frame}

\begin{frame}[fragile]
  \frametitle{Tree Examples}

  Let's see some examples of trees, and how they would be pictured visually:

  \code{Empty : tree}

  \code{Node(Empty, 150, Empty) : tree}

  \code{Node(Node(Empty, 1, Empty), 5, Node(Empty, 0, Empty)) : tree}

  \code{Node(Node(Empty, 1, Empty), 5, Empty) : tree}
\end{frame}

\begin{frame}[fragile]
  \frametitle{Constructors}

  The syntax is important to get used to.

  \vspace{\fill}

  Of our two constructors for the \code{tree} datatype, only one takes an argument. The other is
  \code{Empty}, which is called a \textit{constant constructor}. It requires no arguments in order
  to be a value of type \code{tree}.

  \vspace{\fill}

  The constructor that takes an argument is \code{Node}, which takes in a tuple of type
  \code{tree * int * tree}. This type uses the \code{tree} type we are in the middle of defining!

  \vspace{\fill}

  This is allowed, and called a \term{recursive type}. Datatypes are allowed to be recursive,
  in essence saying that the \code{tree} type can be built out of other \code{tree}s, with the
  \code{Node} constructor.
\end{frame}

\begin{frame}[fragile]
  \frametitle{Tree Functions}

  We can write functions on trees using pattern matching, in the same way that we
  write functions on lists.

  For instance, let's write a function which sums all the nodes of a tree:

  \begin{codeblock}
    fun treesum (Empty : tree) : int = 0
      | treesum (Node (L, x, R)) = treesum L + x + treesum R
  \end{codeblock}

  When writing recursive functions on trees, we need two recursive calls instead of one
  for lists!
\end{frame}

\begin{frame}[fragile]
  \frametitle{Getting In-\code{depth}}

  Let's write a simple function for computing the depth of a tree.

  \vspace{\fill}

  \begin{codeblock}
    fun max (x : int, y : int) : int =
      if x < y then x else y

    fun depth (Empty : tree) : int = 0
      | depth (Node (L, x, R)) = 1 + max (depth L, depth R) 
  \end{codeblock}

  \note {
    Why did I not write it by just comparing depth L and depth R?
  }
\end{frame}

\begin{frame}[fragile]
  \frametitle{Lists and Trees}

  We mentioned that lists and trees are some of the most important
  data structures in computer science. Trees are nice, but it's not
  so easy to do some operations on them, as it is on lists.

  For instance, how should we easily insert an element into a tree?

\end{frame}


\sectionSlide{2}{Structural Induction (Trees)}

\begin{frame}[fragile]
  \frametitle{Proving Theorems on Trees}

  Now that we've introduced trees, we are still interested in proving
  mathematical guarantees on our code!

  We will be able to do this for trees in the same flavor as we were
  able to do for lists.
  
  In the same way that natural numbers are built
  from other natural numbers, and lists are made of other lists, trees 
  are made from other trees! This means that they admit a principle of
  structure induction. 
\end{frame}

\begin{frame}[fragile]
  \frametitle{Structural Induction (Trees)}

  \defBox{\, The principle of \term{structural induction on trees} is as follows:}

  \vspace{\fill}

  Let $P$ be a theorem on values \code{v : tree}. We would
  like to show that, for all \code{v : tree}, $P(\code{v})$ holds.

  \vspace{\fill}

  It suffices to show that:
  \begin{itemize}
    \item $P(\code{Empty})$ holds
    \item Assuming that $P(\code{L})$ and $P(\code{R})$ hold, for some \code{L, R : tree}, show that
    $P(\code{Node(L, x, R)})$ holds, for an arbitrary \code{x : int}.
  \end{itemize}

  \vspace{\fill}

  \noteBox{\, This means that, when proving a theorem on trees, you get two 
  inductive hypotheses!}

\end{frame}

\begin{frame}[fragile]
  \frametitle{\code{depth} is Total}

  Let's prove that the \code{depth} function we wrote earlier is total.

  \vspace{\fill}

  \lemmaBox{}{\, \code{max} is total}

  \vspace{\fill}

  We proceed by \term{structural induction} on \code{T : tree}.

  \vspace{\fill}

  \bcBox{\, \code{T = Empty}}
  \begin{align*}
    \code{depth Empty} &\eeq 0 & \text{(clause 1 of \code{depth})}
  \end{align*}
\end{frame}

\begin{frame}[fragile]
  \frametitle{\code{depth} is Total}

  \ihBox{1}{\, Case: \code{T = L}, for some \code{L : tree}. Assume that \code{depth L} $\hookrightarrow$ \code{v1}.}
  \ihBox{2}{\, Case: \code{T = R}, for some \code{L : tree}. Assume that \code{depth R} $\hookrightarrow$ \code{v2}.}

  \vspace{\fill}

  \isBox{}{\, Case: \code{T = Node(L, x, R)}, for some \code{x : int}. Let's show that \code{depth (Node (L, x, R))} $\hookrightarrow$ \code{v}.}
  \begin{align*}
    \code{depth T} &\eeq \code{1 + max (depth L, depth R)} & \text{(clause 2 of \code{depth})} \\
                          &\eeq \code{1 + max (v1, v2)}         & \text{(inductive hypotheses)} \\
                          &\eeq \code{1 + v3}            & \text{(totality of \code{max})} \\
                          &\eeq \code{v}            & \text{(totality of \code{+})}
  \end{align*}
\end{frame}

\sectionSlide{3}{Datatypes}

\begin{frame}[fragile]
  \frametitle{Back to Lists}

  Now that we've seen trees, let's return to lists for a bit.

  We've seen that we can extract the first element of a list easily by
  using pattern matching. It's not as clear what to do, however, to obtain
  the last element in the list!

  Let's write a recursive \code{last} function to achieve this.

  \begin{codeblock}
    fun last ([] : int list) : int = (* ??? *)
  \end{codeblock}

  We find that we immediately encounter a problem, however.
\end{frame}

\begin{frame}[fragile]
  \frametitle{Slide to the Left}

  Before we even write the recursive case, we have to write the case for
  the empty list.

  \vspace{\fill}

  What is the last element in the empty list? We want to return an
  \code{int}, so it could be \code{0}, or \code{1}, or something arbitrary.

  \vspace{\fill}

  The bigger issue is that it simply doesn't make sense to return an integer
  for this function. We want to return something else, to distinguish the
  cases of "found 0" and "found nothing".
\end{frame}

\begin{frame}[fragile]
  \frametitle{An Optional Type}

  There are quite often times where we want to write a function, where some inputs
  do not have a well-defined answer.

  We've seen this with the \code{div} and \code{fact} functions, where \code{0} and 
  negative numbers cause an exception to be raised, and infinite looping, respectively.

  Sometimes, we don't want such things to happen, however. We would prefer to return
  a value, which is a more predictable and safe behavior.

  To facilitate this, we have the \code{option} \term{type constructor}. 
\end{frame}

\begin{frame}[fragile]
  \frametitle{Type Constructors}

  A \term{type constructor} is something which makes a type out of other types. 

  We were brief about this previously, but that is exactly what \code{list} is!
  Out of the types \code{int}, \code{bool}, and \code{int list}, we can make
  the types \code{int list}, \code{bool list}, and \code{int list list}.

  Similarly, we will be able to construct the types \code{int option}, \code{string option},
  and \code{int list option}.
\end{frame}

\begin{frame}[fragile]
  \frametitle{The \code{option} Type Constructor}

  \defBox{}{\, For any type \code{t}, there is a type \code{t option}, which describes
  a value that is \textit{possibly} a value of type \code{t}}.
  
  \vspace{\fill}

  The type \code{t option} has the following constructors:
  \begin{itemize}
    \item \code{NONE}, which is a constant constructor of no arguments
    \item \code{SOME : t -> t option}, which is a constructor that takes
    a single argument of type \code{t}.
  \end{itemize}

  \vspace{\fill}

  So for instance, here are some examples of options:
  \begin{itemize}
    \item \code{SOME 5 : int option}
    \item \code{SOME [] : int list option}
    \item \code{NONE : int option}
    \item \code{NONE : bool option}
  \end{itemize}
\end{frame}

\begin{frame}[fragile]
  \frametitle{Get Thee To A \code{NONE}-ery}

  Let's rewrite \code{last} with this knowledge!

  \vspace{\fill}

  \begin{codeblock}
    fun last ([] : int list) : int option = NONE
      | last [x] = SOME x
      | last (x::xs) = last xs 
  \end{codeblock}

  \vspace{\fill}

  We could have raised an exception, but this behavior gives more
  agency to the caller of the function, in case they want to do something
  when the list has no output value.
\end{frame}


\begin{frame}[fragile]
  \frametitle{Datatypes}

  So far, we've seen a few examples of type which have different \term{variants},
  or kinds of constructors that make up the values of the type.

  \vspace{\fill}

  Lists, for instance, can be \code{[]} or \code{x::xs}, trees can be \code{Empty}
  or \code{Node(L, x, R)}, and options can be \code{NONE} or \code{SOME x}.

  \vspace{\fill}

  These are all known as \term{variant types}, or \term{sum types}, and can be 
  defined using the \code{datatype} keyword! They define all the forms that
  values of a type can take, and \textit{no more}.

  \vspace{\fill}
  
  \begin{codeblock}
    datatype int_option = NONE | SOME of int

    datatype int_list = [] | :: of int * int list
  \end{codeblock}
\end{frame}

\begin{frame}[fragile]
  \frametitle{Framing the Problem}

  Defining types that fit the shape of the problem is one of the strongest
  aspects of a functional programming language. These types are known as 
  \term{algebraic datatypes}.

  \vspace{\fill}

  For instance, suppose we are interested in an ordering function on integers, 
  which has type \code{int * int -> order}. What should we define the type of
  \code{order} to be?

  \vspace{\fill}

  There are three possibilities. The first is less than the second, they are
  equal, or the first is greater than the second. \code{bool} will not suffice
  here!
\end{frame}

\begin{frame}[fragile]
  \frametitle{The \code{order} type}

  We could use the description of their relationship as the output value.
  We can use a \term{type alias} to alias the name \code{order} to be the
  same as the name \code{string}.

  \vspace{\fill}

  Then, we define our \code{compare} function:

  \vspace{\fill}

  \begin{codeblock}
    type order = string 

    fun compare (x : int, y : int) : order = 
      if x < y then
        "less"
      else if x = y then
        "equal" 
      else
        "greater" 
  \end{codeblock}
\end{frame}

\begin{frame}[fragile]
  \frametitle{A Fragile Approach}

  This approach is needlessly fragile, however. What does the code look like
  for a consumer of this function?

  \vspace{\fill}

  \begin{codeblock}
    case compare (x, y) of
      "LESS" => (* code for the less case *)
    | "EQUAL" => (* code for the equal case *) 
    | "GREATER" => (* code for the greater case *)
    | _ => (* shouldn't be possible??? *)
  \end{codeblock}

  \vspace{\fill}

  When casing on the result of \code{compare}, the compiler doesn't know that
  any case other than \code{"less"}, \code{"equal"}, and \code{"greater"} is
  impossible! This necessitates a redundant extra case.
  
  \vspace{\fill}

  What's another issue with the above code?

  \note {
    It's all wrong! I intentionally misspelled the strings, so actually the code
    we wrote would always enter the "impossible" case.
  }
\end{frame}

\begin{frame}[fragile]
  \frametitle{A Fragile Approach}

  The point is that, although it is \textit{possible} to have \code{order} be the
  same as \code{string}, it comes with a great deal of flaws.

  \vspace{\fill}

  Programming isn't about the possibility of the solution, it's about finding the
  \textit{best} solution. It's an inherently linguistic process, and we're interested
  in having good grammatical structure.

  \vspace{\fill}

  We can use datatype declarations to solve this problem in a better way.
\end{frame}

\begin{frame}[fragile]
  \frametitle{The \code{order} datatype}

  Let's define our own \code{datatype} that fully captures the cases that
  we are interested in.

  \vspace{\fill}

  \begin{codeblock}
    datatype order = LESS | EQUAL | GREATER

    fun compare (x : int, y : int) : order = 
      if x < y then
        LESS
      else if x = y then
        EQUAL 
      else
        GREATER 
  \end{codeblock}
\end{frame}

\begin{frame}[fragile]
  \frametitle{Communication, Better}

  Now, downstream consumers of this function can write the following code:

  \vspace{\fill}

  \begin{codeblock}
    case compare (x, y) of
      LESS => (* code for the less case *)
    | EQUAL => (* code for the equal case *) 
    | GREATER => (* code for the greater case *)
  \end{codeblock}

  \vspace{\fill}

  There's no need to have an extra redundant case, and now the compiler will
  warn if you misspell one of the variants.

  \vspace{\fill}

  This leads to much more polished, streamlined code! Remember the mantra,
  \term{Types Guide Structure}. We use the power of types to structure our
  solution to a problem.
\end{frame}


\begin{frame}[plain]
	\begin{center} Thank you! \end{center}
\end{frame}


\end{document}

