% Jacob Neumann

% DOCUMENT CLASS AND PACKAGE USE
    \documentclass[aspectratio=169]{beamer}
 
    % Establish the colorlambda boolean, to control whether the lambda is solid color (true), or the same as the picture (false)
    \newif\ifcolorlambda
    \colorlambdafalse % DEFAULT: false
    
    % Use auxcolor for syntax highlighting
    \newif\ifuseaux
    \useauxfalse % DEFAULT: false
   
    % Color settings
    \useauxtrue
    
    \newcommand{\auxColor}{25a2db}     % the color of note boxes and stuff
    \newcommand{\presentColor}{FF68DE} % the primary color of the slide borders
    \newcommand{\bgColor}{e6fcff}      % the color of the background of the slide
    \newcommand{\darkBg}{8b98ad}
    \newcommand{\lambdaColor}{\auxColor}
  
    \colorlambdatrue

    \usepackage{comment} % comment blocks
    \usepackage{soul} % strikethrough
    \usepackage{listings} % code
    \usepackage{makecell}
    \usepackage{transparent}

    \setbeamertemplate{itemize items}[circle]
    % \setbeameroption{show notes on second screen=right}

    \usepackage{lectureSlides}
    %%%%%%%%%%%%%%%%%%%%%%%%%%%%%%%%%%%%%%%%%| <----- Don't make the title any longer than this
    \title{Combinators and Staging} % TODO
    \subtitle{Taking higher-order to the next level} % TODO
    \date{15 June 2023} % TODO
    \author{Brandon Wu} % TODO

    \graphicspath{ {./img/} }
    % DONT FORGET TO PUT [fragile] on frames with codeblocks, specs, etc.
        %\begin{frame}[fragile]
        %\begin{codeblock}
        %fun fact 0 = 1
        %  | fact n = n * fact(n-1)
        %\end{codeblock}
        %\end{frame}

    % INCLUDING codefile:
        % 1. In some file under code/NN (where NN is the lecture id num), include:
    %       (* FRAGMENT KK *)
    %           <CONTENT>
    %       (* END KK *)
    
    %    Remember to not put anything on the same line as the FRAGMENT or END comment, as that won't be included. KK here is some (not-zero-padded) integer. Note that you MUST have fragments 0,1,...,KK-1 defined in this manner in order for fragment KK to be properly extracted.
        %  2. On the slide where you want code fragment K
                % \smlFrag[color]{KK}
        %     where 'color' is some color string (defaults to 'white'. Don't use presentColor.
    %  3. If you want to offset the line numbers (e.g. have them start at line 5 instead of 1), use
                % \smlFragOffset[color]{KK}{5}

\begin{document}

% Make it so ./mkWeb works correctly
\ifweb
    \renewcommand{\pause}{}
\fi

\setbeamertemplate{itemize items}[circle]

% SOLID COLOR TITLE (see SETTINGS.sty)
{
\begin{frame}[plain]
    \colorlambdatrue
    \titlepage
\end{frame}
}

\menti{6911 0749}

\begin{frame}[fragile]
  \frametitle{Lesson Plan}

  \tableofcontents
\end{frame}


\begin{frame}[fragile]
  \frametitle{Last time}

  Last time, we went over \term{higher-order functions}, which are functions which
  can take in functions as input and return other functions.

  \pause
  \vspace{\fill}

  We learned that \term{curried} functions take in multiple arguments at separate
  times, by taking them in one-by-one, and returning functions which take in the
  rest.

  \pause
  \vspace{\fill}

  We also learned about the menagerie of HOFs that we will use in this class, 
  being \code{map}, \code{filter}, \code{o}, \code{foldl}, \code{foldr}. 
\end{frame}

\sectionSlide{1}{Staging}

\begin{frame}[fragile]
  \frametitle{Currying}

  What is the advantage of currying?

  \pause
  \vspace{\fill}

  Before, we talked about how we can use currying to define hierarchies of functions.
  
  \pause
  \vspace{\fill}

  We can instantiate functions we'd ordinarily have to write recursive function definitions
  for, by using HOFs like \code{map} as templates, and producing expressions such as
  \code{map Int.toString}.

  \pause
  \vspace{\fill}

  Currying is then important because each curried argument represents a \textit{specialization}
  of a function!

  \pause
  \vspace{\fill}
  
  \begin{itemize}
    \item \code{foldr} - general template for accumulating in a list \pause
    \item \code{foldr (op+)} - general template for summing elements in a list, plus a value \pause
    \item \code{foldr (op+) 0} - function for finding sum of a list
  \end{itemize}
\end{frame}

\begin{frame}[fragile]
  \frametitle{Trivial Totality}

  A \code{fun fact} is that syntactically sugared curried functions are trivially total\footnotemark.  

  \pause
  \vspace{\fill}

  Consider the definition of a function like \code{map}:

  \pause
  \vspace{5pt}

  \begin{codeblock}
    fun map f [] = []
      | map f (x::xs) = f x :: map f xs
  \end{codeblock}

  \pause
  \vspace{\fill}

  Is \code{map} total? I claim -- yes!

  \pause
  \vspace{\fill}

  But now, we also know that \code{map f [1, 2, 3]} should be extensionally
  equivalent to \code{[f 1, f 2, f 3]}, and \code{f} might not be a total function.
  What gives?

  \pause
  \vspace{\fill}

  Indeed, if we evaluate \code{map (fn x => x div 0) [1, 2, 3]}, we get a
  raised exception \code{Div}. What gives?

  \footnotetext[1]{An even more fun fact is that there were three alliterations in that sentence.}
\end{frame}

\begin{frame}[fragile]
  \frametitle{Trivial Totality}

  \customBox{Key Fact}{\, We said that \code{map} was total -- this does not necessarily say anything about
  the totality of \code{map f}! These are \textit{two different functions}.}

  \pause
  \vspace{\fill}

  Consider the definition of \code{map}, which can be desugared to the following:
  \begin{codeblock}
    val map = fn f => fn [] => (* ... *) | x::xs => (* ... *)
  \end{codeblock}

  \pause
  \vspace{\fill}

  It is easy to see that for any function value \code{f : t1 -> t2}, \code{map f} immediately
  evaluates to a lambda expression, which takes in a list and evaluates to either
  case of \code{map}. 

  \pause
  \vspace{\fill}

  This is an important conceptual distinction to keep in your mind. So we say that
  a curried function like \code{map} is "trivially total".
\end{frame}

\begin{frame}[fragile]
  \frametitle{Trivial Totality?}

  A question remains then - is every value \code{t1 -> t2 -> t3} total?

  \pause
  \vspace{\fill}
  
  The answer: \textit{no!} Just because \code{map} and friends are, doesn't mean
  all such values of curried type are. For instance, take the following example:

  \pause
  \begin{codeblock}
    fun loop () = loop () 

    fun f x = 
      let
        val x = loop ()
      in
        fn y => 0 
      end
  \end{codeblock}

  \pause
  \vspace{\fill}

  This function does not immediately return a lambda expression upon being given a
  value. It actually immediately evaluates \code{loop ()}, an infinite loop.
\end{frame}

\begin{frame}[fragile]
  \frametitle{A Carnegie Mellon Example}

  Suppose that you are building a booth.

  \pause
  \vspace{\fill}

  It's the day of move-on, and you still haven't finished painting the wall boards. Your
  good friend stays behind to get them painted, and you refuse to move the rest of 
  the booth to Midway until they're done.

  \pause
  \vspace{\fill}

  That is silly.
\end{frame}

\begin{frame}[fragile]
  \frametitle{The Logistics of Booth Building}

  When building a booth, painting the walls is necessary, but comes much after 
  other steps, like setting up the floorboards, constructing the walls, and 
  moving the wood to Midway in the first place! 

  \pause
  \vspace{\fill}

  The point: It's absurd to wait on something completely unrelated, when you
  could do the work now with what you have!
\end{frame}



\begin{frame}[fragile]
  \frametitle{Staging}

  We refer to this as \term{staging}.

  \pause
  \vspace{\fill}
  
  \defBox{\, We use the term \term{staging} to describe the act of deliberately placing
  computations at certain points with respect to receiving curried arguments}.

  \pause
  \vspace{\fill}

  So instead of saving all computations for when all the curried arguments are received,
  we can instead move some computations to when only the \textit{necessary} arguments
  have been received.

  \pause
  \vspace{\fill}

  Suppose we have the following function:
  \begin{codeblock}
    val f = fn x => fn y => x + y
  \end{codeblock}

  \pause
  \vspace{\fill}

  Can we move the expression \code{x + y} any earlier in the curried function? The 
  answer is \textit{no}, because \code{x + y} depends on both arguments!
\end{frame}

\begin{frame}[fragile]
  \frametitle{A Code Example}

  This comes up all the time!

  \pause
  \vspace{\fill}

  Consider the following artificial code example:

  \begin{codeblock}
    fun mystery x y =
      let
        val res = horrible_computation x
      in
        res + y
      end 
  \end{codeblock}

  \pause
  \vspace{\fill}

  \code{horrible_computation} takes 3 years to evaluate.
\end{frame}

\begin{frame}[fragile]
  \frametitle{A Code Example}

  \begin{codeblock}
    fun mystery x y =
      let
        val res = horrible_computation x
      in
        res + y
      end 
  \end{codeblock}

  \pause
  \vspace{\fill}

  3 years is kind of a long time. Suppose we're interested in evaluating the following:

  \begin{codeblock}
    val res1 = mystery 2 4
    val res2 = mystery 1 2
    val res3 = mystery 2 5
  \end{codeblock}

  \pause
  \vspace{\fill}

  This code takes 9 years in total, to run. But it doesn't need to!
\end{frame}

\begin{frame}[fragile]
  \frametitle{Speed Gains}

  Something we notice about \code{mystery} is that
  \code{horrible_computation} doesn't actually depend on \code{y}.

  \pause
  \vspace{\fill}

  So we can rewrite it as:
  \begin{codeblock}
    fun mystery2 x `=`
      let
        val res = horrible_computation x
      in
        `fn y =>` res + y
      end 
  \end{codeblock}

  \pause
  \vspace{\fill}

  Now instead of returning a lambda which accepts \code{y}, computes the horrible 
  computation, and then returns, we first compute the horrible computation!
\end{frame}

\begin{frame}[fragile]
  \frametitle{Speed Gains}

  This lets us write:

  \pause
  \begin{codeblock}
    val f = mystery2 2
    val g = mystery2 1
    val res1 = f 4
    val res2 = g 2
    val res3 = f 5
  \end{codeblock}

  \pause
  \vspace{\fill}

  We can't avoid the cost of computing \code{mystery2} twice, but 6 years isn't so long.
\end{frame}

\begin{frame}[fragile]
  \frametitle{Benefits of Staging}

  Usually, the order in which computations happen doesn't matter, due to extensional
  equivalence. We might care about things which go beyond simple extensional equivalence
  however, such as expensive computations, or things that break extensional equivalence,
  such as side effects or mutability.

  \pause
  \vspace{\fill}
  
  Later this semester, we will see how side effects can make correct knowledge of staging
  even more essential.
\end{frame}

\sectionSlide{2}{Cost Analysis of HOFs}

\begin{frame}[fragile]
  \frametitle{Cost Analysis of HOFs}

  We've written some higher-order functions at this point, and we're fairly convinced
  of their correctness, but what can we say about their efficiency?

  \pause
  \vspace{\fill}

  Take \code{map} for example.

  \pause
  \vspace{\fill}

  \begin{codeblock}
    fun map (f : 'a -> 'b) ([] : 'a list) : 'b list = []
      | map f (x::xs) = f x :: map f xs
  \end{codeblock}

  \pause
  \vspace{\fill}

  From first glance, it looks like the recurrence is a standard recursion on a list,
  and thus comes out to a bound of $O(n)$. When analyzing the recursive case, a question
  comes to mind, however -- what is the work of \code{f}?
\end{frame}

\begin{frame}[fragile]
  \frametitle{The Cost of Code}

  \ptmt 

  Because HOFs like \code{map} are code which is parameterized on other code, 
  the run-time cost of functions like \code{map} is also parameterized by 
  the cost of the input function!

  \pause
  \vspace{\fill}
  
  This is different than before HOFs, when we only had to deal with being passed
  values that didn't have any notion of cost associated with them -- they just were.

  \pause
  \vspace{\fill}

  In this case, we would say that the cost of \code{map} is $O(n)$ in the number of
  calls to \code{f}, but we can't really say anything better than that.
\end{frame}

\quizBreak{NEWFOUNDLAND}

\sectionSlide{3}{HOFs and Trees}
\begin{frame}[fragile]
  \frametitle{Mapping and Folding}
  
  \tgs

  We've so far seen mapping and folding on lists. These aren't notions that are 
  specific to lists, however\footnotemark.

  \pause
  \vspace{\fill}

  Suppose we have a polymorphic tree type, as we've defined before, such as:
  \begin{codeblock}
    datatype 'a tree = Empty | Node of 'a tree * 'a * 'a tree
  \end{codeblock}

  \pause
  \vspace{\fill}

  We are interested in a \code{map} function which transforms every element of the
  tree, and a \code{fold} function which combines the elements of the tree in a 
  particular order. How can we define these functions?

  \footnotetext[2]{Indeed, "almost every" datatype admits a concept of mapping and
  folding. Proper treatment of this is outside the scope of this course, however.}
\end{frame}

\begin{frame}[fragile]
  \frametitle{Tree Mapping}

  The type signature for \code{treemap} will look similar to \code{map}:

  \pause
  \vspace{\fill}

  \spec
    {treemap}
    {('a -> 'b) -> 'a tree -> 'b tree}
    {\code{f is total}}
    {\code{treemap f T} evaluates to \code{T} with \code{f} called on
    each element}

  \pause
  \vspace{\fill}

  \begin{codeblock}
    fun treemap f Empty = Empty
      | treemap f (Node (L, x, R)) = 
          Node (treemap f L, f x, treemap f R)
  \end{codeblock}
\end{frame}

\begin{frame}[fragile]
  \frametitle{Tree Folding}

  What about folding? We need to first pick a particular traversal order. Let's
  go with \term{inorder} traversal, which is the more intuitive "left-to-right"
  traversal.

  \pause
  \vspace{\fill}

  \spec
    {treefoldl}
    {('a * 'b -> 'b) -> 'b -> 'a tree -> 'b}
    {\code{true}}
    {\code{treefold f z T} $\eeq$ \code{foldl f z (inord T)}}

  \pause
  \vspace{\fill}

  We see that the types of \code{foldl} and \code{treefoldl} look very
  similar.
\end{frame}

\begin{frame}[fragile]
  \frametitle{Tree Folding}

  Now, let's implement it.

  \pause
  \vspace{\fill}

  \begin{codeblock}
    fun treefoldl f z Empty = z
      | treefoldl f z (Node (L, x, R)) =
          let
            val left_folded = treefoldl f z L
          in
            treefoldl f (f (x, left_folded)) R
          end
  \end{codeblock}
\end{frame}

\begin{frame}[fragile]
  \frametitle{Tree Folding: Example}

  {\color{blue}\href{https://asciinema.org/a/vb4tp7vH1aFMNHKabkex1TsJH}{Let's see tree folding in action (click me!)}}

\end{frame}

\begin{frame}[fragile]
  \frametitle{Tree Search}

  Using HOFs, we can also encapsulate the design pattern for a generic function which
  performs a search on trees.

  \pause
  \vspace{\fill}

  To facilitate this, similarly to how \code{sort} took in a generic comparison function,
  our \code{search} function will take in an arbitrary \term{predicate} on elements, which
  returns a boolean on whether the element is what we are meant to search for. 

  \pause
  \vspace{\fill}

  \spec
    {search}
    {('a -> bool) -> 'a tree -> 'a option}
    {\code{p} is total}
    {\code{search p T} evaluates to the first element in \code{T} that satisfies \code{p}, in
    inorder traversal}
\end{frame}

\begin{frame}[fragile]
  \frametitle{Tree Search}

  \rprs

  \begin{codeblock}
    fun search p Empty = NONE
      | search p (Node (L, x, R)) = 
        case search p L of
          NONE => 
            if p x then SOME x 
            else search p R 
        | SOME res => SOME res
  \end{codeblock}

  \vspace{\fill}

  This search function follows the same sort of logic as the original
  \code{inord} function we defined -- it obeys the left-root-right ordering.
\end{frame}

\sectionSlide{4}{Simplifying Programming}

\begin{frame}[fragile]
  \frametitle{Towards Simpler Code}

  We've talked a lot about using HOFs to simplify our language. We obtain expressive 
  functions from generalized templates of program logic, which is an improvement that
  is almost linguistic in nature.

  \pause
  \vspace{\fill}

  It's not quite, since we're really writing code to write other code. But there are
  other useful applications of HOFs, which allows us to write literally simpler code.

  \pause
  \vspace{\fill}

  For instance, suppose we had a very nested function application. It would look
  something like:

  \begin{codeblock}
    foo (bar (qux (baz x)))
  \end{codeblock}
\end{frame}

\begin{frame}[fragile]
  \frametitle{The Pipe Operator}

  Such a nested expression is rather displeasing to the eye, as well as being 
  generally annoying to deal with, due to the parentheses. It also requires being
  read "inside out", since \code{baz} comes after \code{foo}, but is evaluated first!

  \pause
  \vspace{\fill}

  To that end, we can define the \code{|>}\footnotemark operator, which allows us to
  \textit{reverse} the order of function application. It is defined as: 

  \pause
  \begin{codeblock}
    infix |> 

    fun x |> f = f x 
  \end{codeblock}

  \pause
  \vspace{\fill}

  Thus, we could rewrite the first example as:

  \begin{codeblock}
    x |> baz |> qux |> bar |> foo
  \end{codeblock}

  \footnotetext[3]{Pronounced "pipe"}
\end{frame}

\begin{frame}[fragile]
  \frametitle{The Pipe Operator}

  This might seem weird to look at at first, but this is extremely useful for
  producing legible code, that reads like simple instructions.

  \pause
  \vspace{\fill}

  For instance, take the following recipe:
  \begin{itemize}
    \item Heat oven to 400
    \item Insert tray of mozzarella sticks 
    \item Wait two hours
    \item Remove charred remains 
  \end{itemize}
\end{frame}

\begin{frame}[fragile]
  \frametitle{The Pipe Operator}

  Well, let's see what happens if we try to codify these instructions as 
  actual SML code.

  \pause
  \vspace{\fill}

  For instance, take the following recipe:
  \begin{codeblock}
    heat oven 400
    |> insert trayOfMozzarellaSticks
    |> wait 2 
    |> remove
  \end{codeblock}
\end{frame}

\begin{frame}[fragile]
  \frametitle{A Student Problem}

  We can think of the pipe operator as stringing together operations, in the same way
  that we might construct a pipe out of components with compatible ends.

  \pause
  \vspace{\fill}

  Let's take a look at another problem which might benefit from usage of the 
  pipe operator. Suppose we would like to write: 

  \pause
  \vspace{\fill}

  \spec 
    {findStudentGrade}
    {(string * string) -> string -> int}
    {\code{true}}
    {\code{findStudentGrade (student, assignment) file} looks up the grade of \code{student}
    on assignment \code{assigment} in the grades file \code{file}}

  \pause
  \vspace{\fill}

  And then, suppose we have the following helpers: 
  \begin{itemize} 
    \item \code{readFile : string -> string}, which reads in the contents of a file
    \item \code{parseGrades : string -> grades}, which tries to read a string as a grade sheet 
    \item \code{lookup : (string * string) -> grades -> int}, which tries to look up a student's
    grade on a given assignment 
  \end{itemize}
\end{frame}

\begin{frame}[fragile]
  \frametitle{A Pipeable Problem?}

  It looks like we can just straightforwardly use \code{|>}!

  \pause
  \vspace{\fill}

  \begin{codeblock}
    fun findStudentGrade (student, assignment) grades_file =
      grades_file
      |> readFile
      |> parseGrades 
      |> lookup (student, assignment)
  \end{codeblock}

  \pause
  \vspace{\fill}

  Except, that might not actually be true. The pipe operator works because each 
  of those operations can be applied as a function. But what if some of the 
  functions we want to pipe are fallible? In other words, they return 
  \code{t option}, for some type \code{t}?
\end{frame}

\begin{frame}[fragile]
  \frametitle{Whatever Can Go Wrong...}

  We see that our helpers defined previously have a host of problems that 
  might cause them to want to return \code{option} types!

  \pause
  \vspace{\fill}

  \begin{itemize} 
    \item \code{readFile : string -> string option}, because the file might not exist 
    \item \code{parseGrades : string -> grades option}, because the grades might not be in the
    correct format to be parsed 
    \item \code{lookup : (string * string) -> grades -> int}, because the student might not exist 
  \end{itemize}

  \pause
  \vspace{\fill}

  Let's rewrite the code to accommodate this.
\end{frame}


\begin{frame}[fragile]
  \frametitle{Reframed with Options}

  So let's try doing this example again, except now all our functions return options.

  \pause
  \vspace{\fill}

  \begin{codeblock}
    fun findStudentGrade (student, assignment) grades_file =
      case readFile grades_file of
        NONE => NONE
      | SOME file_content => 
        (case parseGrades file_content of 
          NONE => NONE
        | SOME grades =>
          (case lookup (student, assignment) grades of 
            NONE => NONE 
          | SOME grade => SOME grade))
  \end{codeblock}

  \pause
  \vspace{\fill}

  This is disgusting.
\end{frame}

\begin{frame}[fragile]
  \frametitle{An Excess of Casing}

  In the previous example, we had to insert a \code{case} expression every single 
  time that we wanted to unpack the result of a step. On some level, this is expected,
  since we otherwise have no way of dispatching on what exactly was returned, but the
  redundancy is in the \code{NONE} case.

  \pause
  \vspace{\fill}

  When \code{SOME} is returned, we proceed as normal, but in every single case where we
  receive \code{NONE}, we just return \code{NONE}. This is \textit{boilerplate logic},
  because it just bloats the code, and doesn't add anything substantive to the interesting
  behavior of the function.

  \pause
  \vspace{\fill}

  Fortunately, \code{option} is something called a \term{monad}.

  \pause
  \vspace{\fill}

  \defBox{\, A \term{monad} is a particular kind of type constructor that supports some
  operations that obey certain mathematical laws.}
\end{frame}

\begin{frame}[fragile]
  \frametitle{Enter the Monad}

  \tgs

  It's not actually super important what a monad is, but the main idea is that we can
  write a single function, \code{bind}:

  \pause
  \begin{codeblock}
    fun bind (x : 'a option) (f : 'a -> 'b option) = 
      case x of
        NONE => NONE
      | SOME res => f res
  \end{codeblock}

  \pause
  \vspace{\fill}

  This kind of looks like the boilerplate logic around each one of our earlier steps!

  \pause
  \vspace{\fill}

  What this function does is take a function which is supposed to operate on a non-optional
  value, which might fail. It then passes an optional value into it by handling the 
  \code{NONE} case explicitly, as we did earlier. This takes care of the casing for us,
  so that we don't have to!
\end{frame}

\begin{frame}[fragile]
  \frametitle{Piping with \code{bind}}

  Now, we can write:

  \pause
  \begin{codeblock}
    fun findStudentGrade (student, assignment) grades_file =
      bind (readFile grades_file) (fn contents =>
      bind (parseGrades contents) (fn grades =>
      lookup (student, assignment) grades))
  \end{codeblock}

  \pause
  \vspace{\fill}

  Now, much more readable!
\end{frame}

\begin{frame}[fragile]
  \frametitle{Infix \code{bind}}

  Some enthusiasts also enjoy defining \code{bind} as an infix operator.

  \pause
  \vspace{\fill}
  
  Traditionally, this is named \code{>>=}.\footnotemark

  \pause
  \vspace{\fill}

  They define: 
  \begin{codeblock}
    infix >>=

    fun x >>= f = bind (x, f)
  \end{codeblock}

  \pause
  \vspace{\fill}

  to get:
  \begin{codeblock}
    fun findStudentGrade (student, assignment) grades_file =
      readFile grades_file >>= (fn contents =>
      parseGrades contents >>= (fn grades =>
      lookup (student, assignment)))
  \end{codeblock}

  \footnotetext[4]{This is big in the Haskell community.}
\end{frame}

\begin{frame}[fragile]
  \frametitle{The Linguistics of Functional Programming}

  Using operators like \code{>>=} and \code{|>}, which are really just
  higher-order functions, we can achieve vastly more readable code, which
  leverages simple principles to simplify program logic that would 
  otherwise bloat a program.

  \pause
  \vspace{\fill}

  It seems like a little, but code readability is really important when 
  maintaining a codebase! We posited on the first day that functional 
  programming is a refinement on our ability to communicate, and this is 
  a concrete example of how small language features can provide almost 
  linguistic benefits.
\end{frame}

\sectionSlide{5}{Case Study: Staging (Bonus)}

\begin{frame}[fragile]
  \frametitle{Benefits of Staging}

  Let's look at a concrete instance of a problem that could be staged, for
  performance benefits.

  \pause
  \vspace{\fill}

  Consider a function \code{nth_largest}, which has the following specification: 

  \pause
  \vspace{\fill}

  \spec
    {nth_largest}
    {int list -> int -> int}
    {\code{i >= 0}}
    {\code{nth_largest L i} evaluates to the \code{i}th largest element in \code{L}}
\end{frame}

\begin{frame}[fragile]
  \frametitle{A Naive Approach}

  Here's a simple way that we might implement \code{nth_largest}. Assume that
  we have the \code{sort} function that we defined earlier. 

  \pause
  \vspace{\fill}

  \begin{codeblock}
    fun nth_largest L i =
      let
        val sorted = sort (Int.compare, L)
      in
        List.nth (L, i) 
      end
  \end{codeblock}

  where \code{List.nth} is the function which gets the \code{n}th element of
  a list.

  \pause
  \vspace{\fill}

  Simple enough!
\end{frame}

\begin{frame}[fragile]
  \frametitle{Cost Analysis}

  But wait, what's the complexity of this function? We know that our previous
  work bound for sorting was $O(n \log n)$, when we implemented merge sort, and
  the standard library's \code{List.nth} function is $O(\code{i})$, where \code{i}
  is the index that we are trying to access.\footnotemark

  \pause
  \vspace{\fill}

  So then if we ran \code{nth_largest L i} on varying values of \code{i}, and 
  we did it $k$ times, we would incur $k \cdot n \log n$ in cost!

  \pause
  \vspace{\fill}

  This is primarily because on \textit{every call} to \code{nth_largest}, we 
  sort the list. This is massively wasted effort! Can we do better?

  \footnotetext[4]{This is because \code{List.nth} just pops off the first 
  \code{i} elements, until it reaches the \code{i}th one. There's no magic.}
\end{frame}

\begin{frame}[fragile]
  \frametitle{Doing Just Fine}

  We can do better.

  \pause
  \vspace{\fill}

  We notice that there is no data dependency between sorting the list and 
  the second argument, \code{i}. This means that the computation of
  \code{sort (Int.compare, L)} can be moved up!

  \pause
  \vspace{\fill}

  \begin{codeblock}
    fun nth_largest L = 
      let 
        val sorted = sort (Int.compare, L)
      in
        fn i => List.nth (sorted, i) 
      end
  \end{codeblock}
\end{frame}

\begin{frame}[fragile]
  \frametitle{A Staged Experience}

  Now, instead of calling \code{nth_largest L i} a bunch of times, we can first
  call \code{nth_largest L}, and obtain a \textit{staged} version of 
  \code{nth_largest}, which is particular to the list \code{L}. We can then query
  it as many times as we like, with only a linear cost each time.

  \pause
  \vspace{\fill}

  \begin{codeblock}
    val staged_nth_largest = nth_largest L (* O(n log n) *)

    (* each of these are O(n) *)
    val first_largest = staged_nth_largest 0
    val second_largest = staged_nth_largest 1
    val third_largest = staged_nth_largest 2
  \end{codeblock}
\end{frame}

\begin{frame}[fragile]
  \frametitle{The Effect}

  Does it matter, at the end of the day? By sorting the list at all, we still
  incur a $O(n \log n)$ cost, so maybe we haven't actually saved much.

  \pause
  \vspace{\fill}

  If we were only planning on doing a constant number of queries, it won't actually
  asymptotically matter, in the long run.

  \pause
  \vspace{\fill}

  This particular situation isn't one that will be asymptotically improved, but
  it's still \textit{better}. Avoiding a potentially large amount of duplicate
  sorts is still a big deal. It's important to be mindful of the computations
  you're doing, and see opportunities for improvement. 
\end{frame}

\begin{frame}[plain]
	\begin{center} Thank you! \end{center}

	\begin{center} 
    {\color{blue} \href{https://docs.google.com/forms/d/e/1FAIpQLScxujBV8sYyq4j2tyh4isfQZrNmN9X--LQ0FI6ebP_jriVEzQ/viewform?usp=sf_link}{Post-lecture survey:}} \\
    \vspace{5pt}
    \includegraphics[scale=0.035]{qr_june15}
  \end{center}
\end{frame}


\end{document}

